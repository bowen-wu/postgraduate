\documentclass[a4paper,12pt]{article}
\usepackage{xeCJK}          % 中文支持
\usepackage{fontspec}       % 英文/数学字体
\usepackage{amsmath, amssymb} % 数学公式
\usepackage{graphicx}       % 插入图片
\usepackage{hyperref}       % 目录超链接
\usepackage{geometry}       % 页面布局
\usepackage{bm}             % 粗体
\usepackage[table]{xcolor}  % 颜色
\usepackage{tabularx}       % 表格环境
\usepackage{tikz}           % TikZ 绘制主对角线斜线
\usepackage{tcolorbox}
\geometry{left=3cm,right=3cm,top=3cm,bottom=3cm}

% 抽离颜色和尺寸参数
\newcommand{\analysisTitleColor}{green!50!black}
\newcommand{\analysisBackColor}{white}
\newcommand{\analysisBoxRule}{0.8pt}
\newcommand{\analysisArc}{3pt}
\newcommand{\analysisPadding}{6pt}

% 定义 tcolorbox
\newtcolorbox{analysisbox}{
    title=解析,
    colback=\analysisBackColor,
    colframe=\analysisTitleColor,
    boxrule=\analysisBoxRule,
    arc=\analysisArc,
    left=\analysisPadding,
    right=\analysisPadding,
    top=4pt,
    bottom=4pt
}

% =========================
% 字体设置
% =========================
\setmainfont{Times New Roman}
\setsansfont{Helvetica Neue}
\setmonofont{Menlo}
\setCJKmainfont{PingFang SC}

% =========================
% 图形路径(可调整)
% =========================
\graphicspath{{./assets/}}

% =========================
% 文档开始
% =========================
\begin{document}

%    \title{Template}
%    \author{Bowen}
%    \date{\today}
%    \maketitle

% =========================

    \section{特征值与特征向量}

    \subsection{基础概念}

    \begin{enumerate}
        \item 设$A$是$n$阶矩阵,如果存在一个数$\lambda$及非零的$n$维列向量$\alpha$使得$A\alpha = \lambda\alpha$成立,则称$\lambda$是矩阵$A$的一个{\color[rgb]{0.2, 0.6, 0.3}{特征值}},称非零向量$\alpha$是矩阵$A$属于特征值$\lambda$的一个{\color[rgb]{0.2, 0.6, 0.3}{特征向量}}
        \item 设$A = [a_{ij}]$为一个$n$阶矩阵,则行列式
        \[
            |A - \lambda E| = \begin{vmatrix}
                                  a_{11} - \lambda & a_{12}           & \cdots & a_{1n}           \\
                                  a_{21}           & a_{22} - \lambda & \cdots & a_{2n}           \\
                                  \vdots           & \vdots           & \ddots & \vdots           \\
                                  a_{n1}           & a_{n2}           & \cdots & a_{nn} - \lambda
            \end{vmatrix}
        \]
        称为矩阵$A$的{\color[rgb]{0.2, 0.6, 0.3}{特征多项式}},$|A - \lambda E| = 0$称为$A$的{\color[rgb]{0.2, 0.6, 0.3}{特征方程}}
        \item 设$A$和$B$都是$n$阶矩阵,如果存在可逆矩阵$P$,使得$P^{-1}AP = B$则称矩阵$A$和$B$相似,记作$A \sim B$。相似具有如下性质
        \begin{itemize}
            \item 反身性: $A \sim A$
            \item 对称性: $A \sim B \Leftrightarrow B \sim A$
            \item 传递性: $A \sim B, B \sim C \Rightarrow A \sim C$
        \end{itemize}
        \item 如果$A$能与\textbf{对角矩阵}相似,则称$A${\color[rgb]{0.2, 0.6, 0.3}{可对角化}}
        \item 设 $\mathbf{A} = (a_{ij})_{n \times n}$ 为 $n$ 阶方阵,称 $\mathbf{A}$ 的\textbf{迹(trace)}为主对角线上元素之和,记作:$\operatorname{tr}(\mathbf{A}) = \sum_{i=1}^{n} a_{ii}$
    \end{enumerate}

    \subsection{运算}

    \begin{enumerate}
        \item 设矩阵 $A$ 的特征值为 $\lambda$,对应的特征向量为 $\bm{\alpha}$,即:$A\bm{\alpha} = \lambda \bm{\alpha}$则
        \renewcommand{\arraystretch}{2.2} % 调整行距
        \begin{center}
            \setlength{\tabcolsep}{12pt} % 调整列间距
            \begin{tabular}{c|cccccccc}
                \hline
                \rowcolor{gray!10}
                $A$           & $kA + E$       & $A + kE$      & $A^{-1}$             & $A^{*}$                & $A^{T}$           & $A^{n}$       & $P^{-1}AP$          & $f(A)$       \\
                \hline
                $\lambda$     & $k\lambda + 1$ & $\lambda + k$ & $\dfrac{1}{\lambda}$ & $\dfrac{|A|}{\lambda}$     & $\lambda$ & $\lambda^n$ & $\lambda$  & $f(\lambda)$ \\
                \hline
                $\bm{\alpha}$ & $\bm{\alpha}$  & $\bm{\alpha}$ & $\bm{\alpha}$        & $\bm{\alpha}$          & $\bm{\alpha}^{T}$ & $\bm{\alpha}$ & $P^{-1}\bm{\alpha}$ & $\alpha$  \\
                \hline
            \end{tabular}
        \end{center}
    \end{enumerate}

    \subsection{定理}

    \begin{enumerate}
        \item 如果$\alpha_1, \alpha_2, \dots, \alpha_t$都是矩阵$A$的属于特征值$\lambda$的特征向量,那么当$k_1\alpha_1 + k_2\alpha_2 + \dots + k_t\alpha_t$非零时,$k_1\alpha_1 + k_2\alpha_2 + \dots + k_t\alpha_t$仍是矩阵$A$属于特征值$\lambda$的特征向量
        \item 若$\alpha_1, \alpha_2$是矩阵$A$不同特征值的特征向量,则$\alpha_1 + \alpha_2$不是$A$的特征向量
        \item 设$A$是$n$阶矩阵,$\lambda_1, \lambda_2, \dots, \lambda_n$是矩阵$A$的特征值,则
        \begin{align*}
            &\Rightarrow\; \sum \lambda_i = \sum a_{ii}  \\
            &\Rightarrow\; |A| =\lambda_1\lambda_2\dots\lambda_n \\
            &\Rightarrow\; |A - \lambda E| = (\lambda_1 - \lambda)(\lambda_2 - \lambda)\cdots(\lambda_n - \lambda) \\
            &\Rightarrow\; |A - aE| = (\lambda_1 - a)(\lambda_2 - a)\cdots(\lambda_n - a) \\
            &\Rightarrow\; |A^{-1} + kE| = (\frac{1}{\lambda_1} + k)(\frac{1}{\lambda_2} + k)\cdots(\frac{1}{\lambda_n} + k)
        \end{align*}
        \item 如果$\lambda_1, \lambda_2, \dots, \lambda_m$是矩阵$A$的互不相同的特征值,$\alpha_1, \alpha_2, \dots, \alpha_m$分别是与之对应的特征向量,则$\alpha_1, \alpha_2, \dots, \alpha_m$线性无关
        \item 如果$A$是$n$阶矩阵,$\lambda_i$是$A$的$m$重特征值,则属于$\lambda_i$的线性无关的特征向量的个数不超过$m$个
        \item 如果$n$阶矩阵$A$与$B$相似,则
        \begin{align*}
            &{\color{red}{\Rightarrow}}\; \lambda_A = \lambda_B  \\
            &{\color{red}{\Rightarrow}}\; A - \lambda E = B - \lambda E  \\
            &{\color{red}{\Rightarrow}}\; |A| = |B| \\
            &{\color{red}{\Rightarrow}}\; r(A) = r(B) \\
            &{\color{red}{\Rightarrow}}\; tr_A = tr_B
        \end{align*}
        \item 单位矩阵只和自身相似
        \item $n$阶方阵$A$可相似对角化的充分必要条件是$A$有$n$个线性无关的特征向量
        \item $n$ 阶矩阵 $A$ 可相似对角化的充分必要条件是对于 $A$ 的每个特征值,其线性无关的特征向量的个数恰好等于该特征值的重数。即若 $A \sim \Lambda$,则
        \begin{align*}
            &\Leftrightarrow\; \lambda_i \text{ 是 } A \text{ 的 } n_i \text{ 重特征值,且 } \lambda_i \text{ 有 } n_i \text{ 个 {\color{red}{线性无关}}的特征向量}({\color{red}{n - r(A)}}) \\
            &\Leftrightarrow\; r(A - \lambda_i E) = n - n_i,\quad \lambda_i \text{ 为 } n_i \text{ 重特征值}
        \end{align*}
        \item 若 $n$ 阶矩阵 $A$ 有 $n$ 个不同的特征值 $\lambda_1, \lambda_2, \dots, \lambda_n$,则 $A$ 可相似对角化,且有:

        \[
            A \sim
            \begin{bmatrix}
                \lambda_1 &           &        &           \\
                & \lambda_2 &        &           \\
                &           & \ddots &           \\
                &           &        & \lambda_n
            \end{bmatrix}
            = \Lambda
        \]

        其中存在可逆矩阵 $P$ 使得:
        \[
            P^{-1} A P = \Lambda, \quad
            P = [\,\bm{\alpha}_1,\, \bm{\alpha}_2,\, \dots,\, \bm{\alpha}_n\,],
        \]
        其中 $\bm{\alpha}_i$ 是与特征值 $\lambda_i$ 对应的特征向量
        \item 实对称矩阵$A$的不同特征值$\lambda_1, \lambda_2$所对应的特征向量$\alpha_1, \alpha_2$比\textbf{正交}
        \item 实对称矩阵$A$的特征值都是实数
        \item $n$阶实对称矩阵$A$必可对角化,且总存在正交阵$Q$,使得
        \[
            Q^{-1}AQ = Q^{T}AQ = \begin{bmatrix}
                                     \lambda_1 &           &        &           \\
                                     & \lambda_2 &        &           \\
                                     &           & \ddots &           \\
                                     &           &        & \lambda_n
            \end{bmatrix}
        \]
        其中$\lambda_1, \lambda_2, \dots, \lambda_n$是$A$的特征值
        \item 上三角矩阵、下三角矩阵、对角矩阵的特征值就是矩阵主对角线上的元素
    \end{enumerate}

    \subsection{公式}

    \begin{enumerate}
        \item $A_1 \sim B_1, A_2 \sim B_2$,则
        \[
            \begin{bmatrix}
                A_1 &     \\
                & A_2
            \end{bmatrix} \sim \begin{bmatrix}
                                   B_1 &     \\
                                   & B_2
            \end{bmatrix}
        \]

    \end{enumerate}

    \subsection{方法步骤}

    \begin{enumerate}
        \item Schmidt 正交规范化方法: 如果$\bm{\alpha}_1, \bm{\alpha}_2, \dots, \bm{\alpha}_n$线性无关,令
        \[
            \begin{aligned}
                \bm{\beta}_1 &= \bm{\alpha}_1, \\[6pt]
                \bm{\beta}_2 &= \bm{\alpha}_2 - \frac{(\bm{\alpha}_2, \bm{\beta}_1)}{(\bm{\beta}_1, \bm{\beta}_1)} \bm{\beta}_1, \\[6pt]
                \bm{\beta}_3 &= \bm{\alpha}_3 - \frac{(\bm{\alpha}_3, \bm{\beta}_1)}{(\bm{\beta}_1, \bm{\beta}_1)} \bm{\beta}_1
                - \frac{(\bm{\alpha}_3, \bm{\beta}_2)}{(\bm{\beta}_2, \bm{\beta}_2)} \bm{\beta}_2, \\[6pt]
            \end{aligned}
        \]
        那么$\beta_1, \beta_2, \beta_3$两两正交,称为{\color[rgb]{0.2, 0.6, 0.3}{正交向量组}},将其单位化,有
        \[
            \bm{\gamma}_i = \frac{\bm{\beta}_i}{\|\bm{\beta}_i\|}, \quad i = 1, 2, \dots, n.
        \]
        则$\bm{\alpha}_1, \bm{\alpha}_2, \dots, \bm{\alpha}_n$到$\bm{\gamma}_1, \bm{\gamma}_2, \bm{\gamma}_3$这一过程称为{\color[rgb]{0.2, 0.6, 0.3}{Schmidt正交规范化}}
        \item 证明$A \sim B \Rightarrow A \sim \lambda \text{且} \lambda \sim B \Rightarrow A \sim B$
        \item 证明$A$可相似对角化$\Rightarrow A \sim B \text{且} B \sim \lambda \Rightarrow A \sim \lambda$即$A$可相似对角化
        \item 由$A\alpha = \lambda\alpha, \alpha \neq 0$有$(\lambda E - A)\alpha = 0$,即$\alpha$是齐次线性方程组$(\lambda E - A)\alpha = 0$的非零解
        \begin{enumerate}
            \item 先由$|A - \lambda E| = 0$求矩阵$A$的特征值$\lambda_i(\text{共}n\text{个,含重根})$
            \item 再由$(A - \lambda E)\alpha = 0$求基础解系,即矩阵$A$属于特征值$\lambda_i$的线性无关的特征向量
        \end{enumerate}
        \item 判断$A$与$B$相似
        \begin{enumerate}
            \item 如果不满足以下条件则不相似
            \begin{itemize}
                \item $\lambda_A = \lambda_B$
                \item $A - \lambda E = B - \lambda E$
                \item $|A| = |B|$
                \item $r(A) = r(B)$
                \item $tr_A = tr_B$
            \end{itemize}
            \item $A$与$B$都可对角化
        \end{enumerate}
        \item 若已知两个特征向量,求第三个特征向量时,可根据正交条件构造方程组求解。即
        \[
            \bm{\alpha}_3 \perp \bm{\alpha}_1, \quad
            \bm{\alpha}_3 \perp \bm{\alpha}_2.
        \]
        \item 若已知一个特征向量,且另两个特征值相同(即存在重根),则可设一般形式的特征向量,并利用正交条件建立方程组,求得一组线性无关的特征向量作为基础解系
        \item 对于{\color[rgb]{0.2, 0.6, 0.3}{实对称矩阵$A$}}求正交矩阵$Q$或正交对角矩阵的时候需要对特征向量改造
        \begin{itemize}
            \item 特征值不同 \Rightarrow \text{单位化}
            \item 特征值有重根
            \begin{itemize}
                \item 特征向量正交 \Rightarrow \text{单位化}
                \item 特征向量不正交 \Rightarrow \text{Schmidt正交规范化,即正交化 + 单位化}
                \item 故: 计算重根特征值的特征向量时直接给出正交的特征向量
            \end{itemize}
        \end{itemize}
        \item 矩阵
        \[
            \begin{bmatrix}
                1 & 0 & 0 \\
                2 & 3 & 0 \\
                4 & 5 & 6
            \end{bmatrix}
        \]特征值从主对角线上看出来,为$1, 3, 6$且特征值6的特征向量是$(0, 0, 1)^T$
        \item 求实对称矩阵 $A$ 的特征值(解方程 $|A - \lambda E| = 0$)的方法:
        \begin{enumerate}
            \item 通过行列式初等变换,将副对角线元素化为 0(对角化处理)
            \item 对列进行操作,例如 $C_3 - C_1$,简化行列式
            \item 对行进行操作,例如 $R_1 + R_3$,进一步化简
        \end{enumerate}
        \item 用正交矩阵将实对称矩阵$A$化为对角矩阵的步骤
        \begin{enumerate}
            \item 处理一些未知参数
            \item 求矩阵$A$的特征值
            \item 求矩阵$A$的特征向量
            \item 单位化,当特征值有重根时,{\color[rgb]{0.2, 0.6, 0.3}{可能}}还要Schmidt正交化
            \item 构造正交矩阵$P$,使得$P^{-1}AP = \Lambda$,其中$P$与$\Lambda$次序要协调一致
        \end{enumerate}
    \end{enumerate}

    \subsection{条件转换思路}

    \begin{enumerate}
        \item 实对称矩阵
        \begin{align*}
            &\Leftrightarrow\; \text{必与对角矩阵相似}  \\
            &\Leftrightarrow\; \text{可用正交矩阵对角化}  \\
            &\Leftrightarrow\; \text{不同特征值的特征向量必正交}  \\
            &\Leftrightarrow\; \text{特征值必是实数}  \\
            &\Leftrightarrow\; k\text{重特征值必是实数必有}k\text{个线性无关的特征向量}
        \end{align*}
        \item $A \sim B$
        \begin{align*}
            &\Rightarrow\; A^n = PB^{n}P^{-1} \\
            &{\color{red}{\Rightarrow}}\; \lambda_A = \lambda_B  \\
            &{\color{red}{\Rightarrow}}\; A - \lambda E = B - \lambda E  \\
            &{\color{red}{\Rightarrow}}\; |A| = |B| \\
            &{\color{red}{\Rightarrow}}\; r(A) = r(B) \\
            &{\color{red}{\Rightarrow}}\; tr_A = tr_B \\
            &\Rightarrow\; A + kE \sim B + kE  \\
            &\Rightarrow\; (A + kE)^n \sim (B + kE)^n  \\
            &\Rightarrow\; f(A) \sim f(B) \\
            &\Rightarrow\; |A + kE| = |B + kE|  \\
            &\Rightarrow\; r(A + kE) = r(B + kE)  \\
            &\Rightarrow\; A^n \sim B^n \\
            &\Rightarrow\; A^T \sim B^T \\
            &\Rightarrow\; A^* \sim B^* \\
            &\Rightarrow\; A^{-1} \sim B^{-1} \\
            &\Rightarrow\; \text{若 } A\bm{\alpha}_A = \lambda \bm{\alpha}_A,
            \text{ 则 } B(P^{-1}\bm{\alpha}_A) = \lambda (P^{-1}\bm{\alpha}_A)
            \Rightarrow {\color{red}{\bm{\alpha}_B = P^{-1}\bm{\alpha}_A}} \\
            &\Rightarrow\; \text{若 } B\bm{\alpha}_B = \lambda \bm{\alpha}_B,
            \text{ 则 } A(P\bm{\alpha}_B) = \lambda (P\bm{\alpha}_B)
            \Rightarrow {\color{red}{\bm{\alpha}_A = P\bm{\alpha}_B}}
        \end{align*}
        \item $P_1^{-1}AP_1 = B, P_2^{-1}BP_2 = C \Rightarrow P^{-1}AP = C, \text{其中}P = P_{1}P_2$
        \item $r(A) < n \Rightarrow 0 \text{是} A \text{的特征值}$
        \item $A$是$n$阶矩阵,若 $r(A) = 1$,则
        \begin{align*}
            &\Leftrightarrow\; \text{矩阵 $A$ 的行向量组线性相关,且秩为 1}  \\
            &\Rightarrow\; |A - \lambda E| = \lambda^n - \sum a_{ii}\lambda^{n-1}  \\
            &\Rightarrow\; \lambda_1 = \lambda_2 = \dots = \lambda_{n-1} = 0  \\
            &\Rightarrow\; \lambda_n = \operatorname{tr}(A) = a_{11} + a_{22} + \dots + a_{nn}
        \end{align*}
        \item 求矩阵$A$中参数
        \begin{itemize}
            \item 已知特征向量$\alpha$
            \begin{enumerate}
                \item 构造方程组
                \item $A\alpha = \lambda \alpha$
                \item 解$A$中变量和$\lambda$
            \end{enumerate}
            \item 相似于对角矩阵(可以对角化)
            \begin{enumerate}
                \item 需要有$n$个线性无关的特征向量
                \item 若有重根,则$n - r(A - \lambda E) = \text{重根个数}$
                \item $|A - \lambda E| = 0$计算特征值和特征向量
            \end{enumerate}
            \item 特征值$\lambda \Rightarrow |A - \lambda E| = 0$
        \end{itemize}
        \item 抽象矩阵思考:
        \begin{itemize}
            \item 线性相关 + 线性方程组
            \item 秩
        \end{itemize}
        \item $A^2 = A \Rightarrow A$的特征值只能取$1$或$0$
    \end{enumerate}

    \subsection{理解}

    \begin{enumerate}
        \item 在求参数的问题中,可以{\color[rgb]{0.2, 0.6, 0.3}{由特征向量可构造方程组}}
        \item 特征向量 = $k$基础解系,其中$k$为非零常数
        \item $a \neq 0$,矩阵 $A$ 如下
        \[
            A =
            \begin{bmatrix}
                1 & a & a & a \\
                a & 1 & a & a \\
                a & a & 1 & a \\
                a & a & a & 1
            \end{bmatrix}

        \]
        \begin{analysisbox}
            \[
                A = (1-a)E + a J_4,
            \]
            其中 $J_4$ 是 $4 \times 4$ 全 $1$ 矩阵。已知 $J_4$ 的特征值为 $4,0,0,0$
            \[
                \because J_4 \bm{\alpha}_1 = 4 \bm{\alpha}_1 \\
            \]
            \[
                \therefore \text{特征值}4\text{的特征向量是}(1,1,1,1)^T
            \]
            对于 $\bm{\alpha}_2, \bm{\alpha}_3, \bm{\alpha}_4$,它们满足 $J_4 \bm{\alpha}_i = 0$,因为四个分量和为零,因此它们属于零特征值的特征空间,故取标准基
            \[
                \bm{\alpha}_2 = (1,-1,0,0)^T, \\
                \bm{\alpha}_3 = (1,0,-1,0)^T, \\
                \bm{\alpha}_4 = (1,0,0,-1)^T
            \]

            利用线性性质求 $A$ 的特征值:
            \[
                \begin{aligned}
                    A \bm{\alpha}_1 &= (1-a) \bm{\alpha}_1 + a J_4 \bm{\alpha}_1 = (1-a)\bm{\alpha}_1 + a \cdot 4 \bm{\alpha}_1 = (1+3a) \bm{\alpha}_1, \\
                    A \bm{\alpha}_2 &= (1-a)\bm{\alpha}_2 + a J_4 \bm{\alpha}_2 = (1-a)\bm{\alpha}_2 + a \cdot 0 = (1-a)\bm{\alpha}_2, \\
                    A \bm{\alpha}_3 &= (1-a)\bm{\alpha}_3 + a J_4 \bm{\alpha}_3 = (1-a)\bm{\alpha}_3, \\
                    A \bm{\alpha}_4 &= (1-a)\bm{\alpha}_4 + a J_4 \bm{\alpha}_4 = (1-a)\bm{\alpha}_4.
                \end{aligned}
            \]

            \[
                \text{故 $A$ 的特征值为 } 1+3a, 1-a, 1-a, 1-a, \quad \text{对应特征向量为 } \bm{\alpha}_1, \bm{\alpha}_2, \bm{\alpha}_3, \bm{\alpha}_4.
            \]
        \end{analysisbox}
    \end{enumerate}

\end{document}
