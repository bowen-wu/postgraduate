\documentclass[a4paper,12pt]{article}
\usepackage{xeCJK}          % 中文支持
\usepackage{fontspec}       % 英文/数学字体
\usepackage{amsmath, amssymb} % 数学公式
\usepackage{graphicx}       % 插入图片
\usepackage{hyperref}       % 目录超链接
\usepackage{geometry}       % 页面布局
\usepackage{bm}             % 粗体
\usepackage{xcolor}         % 颜色
\usepackage{tabularx}       %表格环境
\usepackage{tikz}           % TikZ 绘制主对角线斜线
\geometry{left=3cm,right=3cm,top=3cm,bottom=3cm}

% =========================
% 字体设置
% =========================
\setmainfont{Times New Roman}
\setsansfont{Helvetica Neue}
\setmonofont{Menlo}
\setCJKmainfont{PingFang SC}

% =========================
% 图形路径(可调整)
% =========================
\graphicspath{{./assets/}}

% =========================
% 文档开始
% =========================
\begin{document}

%    \title{Template}
%    \author{Bowen}
%    \date{\today}
%    \maketitle

% =========================

    \section{二次型}

    \subsection{视频}

    \begin{enumerate}
        \item 二次型矩阵: 平方项对角线,混合项的系数被2除,放在对应位置上
        \item 二次型的矩阵是唯一的,
        \item 二次型矩阵 \Leftrightarrow \text{实对称矩阵} \quad \text{\color{red}{一一对应}}
        \item 坐标变换关注{\color{red}{系数行列式不为0(可逆矩阵)}},有无数多个坐标变换
    \end{enumerate}

    \subsection{基础概念}

    \begin{enumerate}
    \end{enumerate}

    \subsection{定理}

    \begin{enumerate}
    \end{enumerate}

    \subsection{运算}

    \begin{enumerate}

    \end{enumerate}

    \subsection{公式}

    \begin{enumerate}

    \end{enumerate}

    \subsection{方法步骤}

    \begin{enumerate}

    \end{enumerate}

    \subsection{条件转换思路}

    \begin{enumerate}

    \end{enumerate}

\end{document}
