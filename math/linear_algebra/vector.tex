\documentclass[a4paper,12pt]{article}
\usepackage{xeCJK}          % 中文支持
\usepackage{fontspec}       % 英文/数学字体
\usepackage{amsmath, amssymb} % 数学公式
\usepackage{graphicx}       % 插入图片
\usepackage{hyperref}       % 目录超链接
\usepackage{geometry}       % 页面布局
\usepackage{bm}             % 粗体
\usepackage{xcolor}         % 颜色
\usepackage{tabularx}       % 表格环境
\usepackage{tikz}           % TikZ 绘制主对角线斜线
\usepackage{tcolorbox}
\usepackage{xstring}
\geometry{left=3cm,right=3cm,top=3cm,bottom=3cm}

% 抽离颜色和尺寸参数
\newcommand{\analysisTitleColor}{green!50!black}
\newcommand{\analysisBackColor}{white}
\newcommand{\analysisBoxRule}{0.8pt}
\newcommand{\analysisArc}{3pt}
\newcommand{\analysisPadding}{6pt}

% 定义 tcolorbox
\newtcolorbox{analysisbox}[1][]{
    title=\IfStrEq{#1}{}{\textbf{解析}}{#1}, % 如果传参为空则使用“解析”
    colback=\analysisBackColor,
    colframe=\analysisTitleColor,
    boxrule=\analysisBoxRule,
    arc=\analysisArc,
    left=\analysisPadding,
    right=\analysisPadding,
    top=4pt,
    bottom=4pt
}

% =========================
% 字体设置
% =========================
\setmainfont{Times New Roman}
\setsansfont{Helvetica Neue}
\setmonofont{Menlo}
\setCJKmainfont{PingFang SC}

% =========================
% 图形路径(可调整)
% =========================
\graphicspath{{./assets/}}

% =========================
% 文档开始
% =========================
\begin{document}

%    \title{向量}
%    \author{Bowen}
%    \date{\today}
%    \maketitle

% =========================

    \section{向量}

    \subsection{基本概念}

    \begin{enumerate}
        \item $n$个数$a_1, a_2, \dots, a_n$所组成的有序数组$\alpha = (a_1, a_2, \dots, a_n)^T$或$\alpha = (a_1, a_2, \dots, a_n)$称为$n$维向量,其中$a_1, a_2, \dots, a_n$称为向量$\alpha$的分量(或坐标),前一个表示式称为列向量,后者称为行向量
        \item 对$n$维向量$\alpha_1, \alpha_2, \dots, \alpha_s$,如果存在不全为零的数$k$,使得
        \[
            k_{1}\alpha_1 + k_{2}\alpha_2 + \dots + k_{s}\alpha_s = 0
        \]
        则称向量组$\alpha_1, \alpha_2, \dots, \alpha_s$线性相关,否则,称向量组$\alpha_1, \alpha_2, \dots, \alpha_s$线性无关
        \begin{enumerate}
            \item 有零向量
            \item 两向量成比例
            \item $n+1$个$n$维向量
        \end{enumerate}
        \item 向量组$\alpha_1 = (a_{11}, a_{21}, \dots, a_{r1})^T, \alpha_2 = (a_{12}, a_{22}, \dots, a_{r2})^T, \dots, \alpha_m = (a_{1m}, a_{2m}, \dots, a_{rm})^T$及向量组$\widetilde{\alpha_1} = (a_{11}, a_{21}, \dots, a_{s1})^T, \widetilde{\alpha_2} = (a_{12}, a_{22}, \dots, a_{s2})^T, \dots, \widetilde{\alpha_m} = (a_{1m}, a_{2m}, \dots, a_{sm})^T$,其中 $s \le r$,则称$\widetilde{\alpha_1}, \widetilde{\alpha_2}, \dots, \widetilde{\alpha_m}$为向量组$\alpha_1, \alpha_2, \dots, \alpha_m$的延伸组(或称$\alpha_1, \alpha_2, \dots, \alpha_m$是$\widetilde{\alpha_1}, \widetilde{\alpha_2}, \dots, \widetilde{\alpha_m}$的缩短组
        \item 对$n$维向量$\alpha_1, \alpha_2, \dots, \alpha_s$和$\beta$,若存在实数$k_1, k_2, \dots, k_n$,使得
        \[
            k_{1}\alpha_1 + k_{2}\alpha_2 + \dots + k_{s}\alpha_s = \beta
        \]
        则称$\beta$是$\alpha_1, \alpha_2, \dots, \alpha_s$的线性组合,或者说$\beta$可由$\alpha_1, \alpha_2, \dots, \alpha_s$\textbf{线性表出(示)}
        \item 设有两个$n$维向量组$(I)\alpha_1, \alpha_2, \dots, \alpha_s$;$(II)\beta_1, \beta_2, \dots, \beta_t$,如果$(I)$中每个向量$\alpha_i(i = 1,2,\dots,s)$都可由$(II)$中的向量$\beta_1, \beta_2, \dots, \beta_t$线性表出,则称向量组$(I)$可由向量组$(II)$线性表出
        \item 如果$(I)(II)$这两个向量组可以互相线性表出,则称这两个\textbf{向量组等价}
        \item 在向量组$\alpha_1, \alpha_2, \dots, \alpha_s$中,若存在$r$个向量$\alpha_{i_1}, \alpha_{i_2}, \dots, \alpha_{i_r}$线性相关,再加进任一向量$a_j(j = 1,2,\dots,\s)$,向量组$\alpha_{i_1}, \alpha_{i_2}, \dots, \alpha_{i_r}, \alpha_j$就线性相关,则称$\alpha_{i_1}, \alpha_{i_2}, \dots, \alpha_{i_r}$是向量组$\alpha_1, \alpha_2, \dots, \alpha_s$的一个\textbf{极大线性无关组}
        \begin{itemize}
            \item 极大线性无关组可以表示向量组中任一向量
            \item 极大线性无关组{\color[rgb]{0.2, 0.6, 0.3}{不唯一}},但其内的向量个数一致,即向量组的秩
        \end{itemize}
        \item 向量组$\alpha_1, \alpha_2, \dots, \alpha_s$的极大线性无关组中所含向量的个数$r$称为这个向量组的秩
        \begin{itemize}
            \item $r(\alpha_1, \alpha_2, \dots, \alpha_{n-1}) \le r(\alpha_1, \alpha_2, \dots, \alpha_{n-1}, \alpha_n)$
            \item $r(\alpha_1, \alpha_2, \dots, \alpha_n) \le n$
        \end{itemize}
        \item 初等行变换不会改变列向量组的线性相关性,也不会改变它们之间的线性组合系数
        \item 线性表示具有传递性
    \end{enumerate}

    \subsection{定理}

    \begin{enumerate}
        \item 向量组$\alpha_1, \alpha_2, \dots, \alpha_s${\color[rgb]{0.2, 0.6, 0.3}{线性相关}}
        \begin{align*}
            &\Leftrightarrow\; \text{其次线性方程组}[\alpha_1, \alpha_2, \dots, \alpha_s][x_1, x_2, \dots, x_n]^T = 0\text{有非零解} \\
            &\Leftrightarrow\; r(\alpha_1, \alpha_2, \dots, \alpha_s) < s, s\text{表示}\textbf{未知数的个数}\text{或}\textbf{向量个数} \\
            &\Leftrightarrow\; \color{red}{|\alpha_1, \alpha_2, \dots, \alpha_n| = 0}
        \end{align*}
        \item 向量组$\alpha_1, \alpha_2, \dots, \alpha_s${\color{red}{线性无关}}
        \begin{align*}
            &\Leftrightarrow\; \text{其次线性方程组}[\alpha_1, \alpha_2, \dots, \alpha_s][x_1, x_2, \dots, x_n]^T = 0\text{只有零解} \\
            &\Leftrightarrow\; r(\alpha_1, \alpha_2, \dots, \alpha_s) = s, s\text{表示}\textbf{未知数的个数}\text{或}\textbf{向量个数} \\
            &\Leftrightarrow\; \color{red}{|\alpha_1, \alpha_2, \dots, \alpha_n| \neq 0}
        \end{align*}
        \item 向量组的秩$r(\alpha_1, \alpha_2, \dots, \alpha_s) = s \Leftrightarrow \alpha_1, \alpha_2, \dots, \alpha_s$线性无关
        \item $n+1$个$n$维向量一定线性相关
        \item 任何{\color{red}{部分}}组$\alpha_1, \alpha_2, \dots, \alpha_r$相关 \Rightarrow \text{{\color[rgb]{0.2, 0.6, 0.3}{整体}}组}$\alpha_1, \alpha_2, \dots, \alpha_r, \dots \alpha_s$相关
        \item {\color{red}{整体}}组$\alpha_1, \alpha_2, \dots, \alpha_r, \dots \alpha_s$无关 \Rightarrow \text{{\color[rgb]{0.2, 0.6, 0.3}{部分}}组}$\alpha_1, \alpha_2, \dots, \alpha_r$无关
        \item $\alpha_1, \alpha_2, \dots, \alpha_n$ 线性无关 \Rightarrow \text{延伸组}$\widetilde{\mathbf{\alpha_1}}, \widetilde{\mathbf{\alpha_2}}, \dots, \widetilde{\mathbf{\alpha_n}}$线性无关
        \item $\widetilde{\mathbf{\alpha_1}}, \widetilde{\mathbf{\alpha_2}}, \dots, \widetilde{\mathbf{\alpha_n}}$线性相关 \Rightarrow \text{缩短组}$\alpha_1, \alpha_2, \dots, \alpha_n$ 线性相关
        \item 向量$\beta$可由向量组$\alpha_1, \alpha_2, \dots, \alpha_s$线性表出
        \begin{align*}
            &\Leftrightarrow\; \text{非齐次线性方程组} [\alpha_1, \alpha_2, \dots, \alpha_s][\alpha_1, \alpha_2, \dots, \alpha_s]^T = \beta \text{有解} \\
            &\Leftrightarrow\; \text{秩} r[\alpha_1, \alpha_2, \dots, \alpha_s] = r[\alpha_1, \alpha_2, \dots, \alpha_s, \beta]
        \end{align*}
        \item 如果$\alpha_1, \alpha_2, \dots, \alpha_s (s \ge 2)$线性相关,则其中\textbf{必有一个向量}可用其余向量线性表出;反之,若有一个向量可用其余的$s-1$个向量线性表出,则这$s$个向量必线性相关
        \item 如果$\alpha_1, \alpha_2, \dots, \alpha_s$线性无关,$\alpha_1, \alpha_2, \dots, \alpha_s, \beta$线性相关,则$\beta$可由$\alpha_1, \alpha_2, \dots, \alpha_s$线性表出,且\textbf{表示法唯一}
        \item 如果$\alpha_1, \alpha_2, \dots, \alpha_s$可由向量组$\beta_1, \beta_2, \dots, \beta_t$线性表出,且$s > t$,那么$\alpha_1, \alpha_2, \dots, \alpha_s$线性相关。即{\color{red}{\textbf{如果多数向量能用少数向量线性表出,那么多数向量一定线性相关}}}
        \item 如果$\alpha_1, \alpha_2, \dots, \alpha_s$线性无关,且它可由可由$\beta_1, \beta_2, \dots, \beta_t$线性表出,则$s \le t$
        \item 设$\alpha_1, \alpha_2, \dots, \alpha_s$可由$\beta_1, \beta_2, \dots, \beta_t$线性表出,则$r(\alpha_1, \alpha_2, \dots, \alpha_s) \le r(\beta_1, \beta_2, \dots, \beta_t)$
        \item 如果$(I)(II)$是两个等价的向量组,则$r(I) = r(II)$
        \item 如果$r(A) = r$,则$A$中有$r$个线性无关的列向量,而其他列向量都是这$r$个线性无关列向量的线性组合,也就是$r(A) = A$的列秩
        \item 一般地,$r(A) = A \text{的列秩} = A \text{的行秩}$
        \item $A$是$m \times n$矩阵,则$Ax = 0$的解向量组的秩为$\mathbf{n - r(A)}$
    \end{enumerate}

    \subsection{运算}

    \begin{enumerate}
        \item 设$n$维向量$\alpha = (a_1, a_2, \dots, a_n)^T, \beta= (b_1, b_2, \dots, b_n)^T$,则
        \begin{enumerate}
            \item $\alpha + \beta = (a_1 + b_1, a_2 + b_2, \dots, a_n + b_n)^T$
            \item $k\alpha = (ka_1, ka_2, \dots, ka_n)^T$
            \item $0\alpha = 0$
            \item $(\alpha, \beta) = \alpha^T\beta = \beta^T\alpha = a_{1}b_{1} + a_{2}b_{2} + \dots + a_{n}b_{n}$
            \item $\alpha + \beta = \beta + \alpha$
            \item $(\alpha + \beta) + \gamma = \alpha + (\beta + \gamma)$
            \item $\alpha + 0 = \alpha $
            \item $\alpha + (-\alpha) = 0$
            \item $1\alpha = \alpha$
            \item $k(l\alpha) = (kl)\alpha$
            \item $k(\alpha + \beta) = k\alpha + k\beta$
            \item $(k + l)\alpha = k\alpha + l\alpha$
        \end{enumerate}
    \end{enumerate}

    \subsection{公式}

    \begin{enumerate}

    \end{enumerate}

    \subsection{方法步骤}

    \begin{enumerate}
        \item $\beta$可由$\alpha_1, \alpha_2, \alpha_3$,求表达式???
        \item 判断多个向量是否线性相关
        \begin{itemize}
            \item 含有零向量 $\mathbf{0}$ $\Rightarrow$ 线性相关
            \item 两个向量成比例 $\Rightarrow$ 线性相关
            \item 存在关系 $\alpha_1 + \alpha_2 = \alpha_3$(定义) $\Rightarrow$ 线性相关
        \end{itemize}
        \item 线性无关的判定与证明:\; 若向量的坐标没有给出,通常用\textbf{定义法} 或 \textbf{秩的理论} 或 \textbf{反证法}
        \begin{enumerate}
            \item \textbf{定义法}证$\alpha_1, \alpha_2, \dots, \alpha_s$线性无关
            \begin{enumerate}
                \item 设$k_{1}\alpha_1 + k_{2}\alpha_2 + \dots + k_{s}\alpha_s = 0$
                \item \Downarrow \text{恒等变形(\textbf{同乘}: 看条件 + 构造条件 或 重组)}
                \item $k_1 = 0, k_2 = 0, \dots, k_s = 0$
            \end{enumerate}
            \item \textbf{秩}证$\alpha_1, \alpha_2, \dots, \alpha_s$线性无关
            \begin{enumerate}
                \item \Leftrightarrow $[\alpha_1, \alpha_2, \dots, \alpha_s][x_1, x_2, \dots, x_s]^T = 0$只有零解
                \item \Leftrightarrow \text{秩}$r(\alpha_1, \alpha_2, \dots, \alpha_s) = s$
                \begin{itemize}
                    \item $r(A) = A\text{的列秩} = A\text{的行秩}$
                    \item $r(AB) \le r(A) \text{且} r(AB) \le r(B)$
                    \item 若$A$可逆,则$r(AB) = r(BA) = r(B)$
                    \item 若$A$是$m \times n$矩阵,且$r(A) = n$,则$r(AB) = r(B)$
                    \item 若$A$是$m \times n$矩阵,$B$是$n \times s$矩阵,且$AB = \mathbf{O}$,则$r(A) + r(B) \le n$
                \end{itemize}
            \end{enumerate}
            \item \textbf{反证法}
            \item 线性方程组$Ax = 0$
            \begin{itemize}
                \item 只有零解 \Leftrightarrow \text{线性无关}
                \item 有非零解 \Leftrightarrow \text{线性相关}
            \end{itemize}
            \item 若是$n$个$n$维向量
            \begin{itemize}
                \item $|\alpha_1, \alpha_2, \dots, \alpha_n| = 0 \Leftrightarrow$ 相关
                \item $|\alpha_1, \alpha_2, \dots, \alpha_n| \neq 0 \Leftrightarrow$ 无关
            \end{itemize}
        \end{enumerate}
        \item 判断能否线性表出
        \begin{enumerate}
            \item $\text{若向量坐标具体} \Rightarrow \text{非齐次线性方程组是否有解}$
            \begin{itemize}
                \item $\text{有解} \Rightarrow \text{能线性表出}$
                \item $\text{无解} \Rightarrow \text{不能线性表出}$
            \end{itemize}
            \item $\text{若向量坐标没有} \Rightarrow \text{线性相关 或 秩}$
            \begin{itemize}
                \item $r(\alpha_1, \alpha_2, \dots, \alpha_n) = r(\alpha_1, \alpha_2, \dots, \alpha_n, \beta) \Rightarrow$ 线性表出
                \item $r(\alpha_1, \alpha_2, \dots, \alpha_n) \neq r(\alpha_1, \alpha_2, \dots, \alpha_n, \beta) \Rightarrow$ 不能线性表出
                \item 线性相关
            \end{itemize}
        \end{enumerate}
        \item 求向量组的秩
        \begin{itemize}
            \item 设向量组$(\alpha_1, \alpha_2, \dots, \alpha_s)$的矩阵为 $A = [\alpha_1, \alpha_2, \dots, \alpha_s]$ 对$A$作初等行变换得到行最简形矩阵$A'$,则
            \[
                r(A') = r(A) = \text{向量组}(\alpha_1, \alpha_2, \dots, \alpha_s)\text{的秩}
            \]
            \item 若存在$r$阶子式不为零,则$r$为矩阵的秩,对应的$r$个向量构成一组最大线性无关组
        \end{itemize}
        \item 分析、讨论一个向量组$\beta_1, \beta_2, \beta_3$能够由$\alpha_1, \alpha_2, \alpha_3$线性表出
        \begin{enumerate}
            \item 定义: 是否存在系数$c_1, c_2, c_3$使得
            \[
                \beta_i = c_{1i}\alpha_1 + c_{2}\alpha_2 + c_{3i}\alpha_3
            \]对每一个$i = 1, 2, 3$都成立
            \item 向量组$B(\beta_1, \beta_2, \beta_3)$能由$A(\alpha_1, \alpha_2, \alpha_3)$线性表出
            \begin{align*}
                &\Leftrightarrow\; B\text{中的每个向量都是}A\text{的线性组合} \\
                &\Leftrightarrow\; r(A) = r(A, B) \text{加上}B\text{中的向量后秩不变,说明这些向量能有}A\text{线性表出} \\
                &\Leftrightarrow\; \text{若}B\text{线性无关} \Rightarrow A\text{线性无关} \\
                &\Leftrightarrow\; \text{若}B\text{线性相关} \Rightarrow A\text{线性相关}
            \end{align*}
        \end{enumerate}
        \item 求极大线性无关组,并将其余向量用极大线性无关组线性表出
        \begin{enumerate}
            \item 初等行变换
            \item 若有参数,分情况讨论
            \item 主元就是极大线性无关组
            \item 线性表示为对应向量的系数
            \begin{analysisbox}[Case:]
                \[
                    \begin{bmatrix}
                        1 & 0 & -1 & 2  \\
                        0 & 1 & -2 & -1 \\
                        0 & 0 & 0  & 0  \\
                        0 & 0 & 0  & 0
                    \end{bmatrix}
                \]
                极大线性无关组$\alpha_1, \alpha_2$ \\
                $\alpha_3 = -\alpha_1 - 2\alpha_2$ \\
                $\alpha_4 = 2\alpha_1 - \alpha_2$
            \end{analysisbox}
        \end{enumerate}
    \end{enumerate}

    \subsection{条件转换思路}

    \begin{enumerate}
        \item $\text{向量组}(I)(II)\text{等价}$
        \begin{align*}
            &\Leftrightarrow\; (I)\text{可由}(II)\text{线性表出}\text{且}(II)\text{可由}(I)\text{线性表出} \\
            &\Leftrightarrow\; r(I) = r(II) = r(I, II) \\
            &\Leftrightarrow\; r(I) = r(II)\text{且}(I)\text{可由}(II)\text{线性表出}
        \end{align*}
        \item $\alpha_1, \alpha_2, \alpha_3$可以表示任意一个三维向量
        \begin{align*}
            &\Leftrightarrow\; \text{三者线性无关}  \\
            &\Leftrightarrow\; \text{三者是一组基底}  \\
            &\Leftrightarrow\; (a, b, c)^T\text{能由}\alpha_1, \alpha_2, \alpha_3\text{线性表示}
        \end{align*}
        \item 如果$\gamma = (a, b, c)^T$(任意向量)不能由$(\alpha_1, \alpha_2, \alpha_3)$线性表出
        \begin{align*}
            &\Leftrightarrow\; \alpha_1, \alpha_2, \alpha_3\text{不可表示任意一个三维向量}  \\
            &\Leftrightarrow\; \alpha_1, \alpha_2, \alpha_3\text{线性相关}  \\
            &\Leftrightarrow\; |\alpha_1, \alpha_2, \alpha_3| = 0
        \end{align*}
        \item $\alpha_1, \alpha_2, \alpha_3\text{如果线性相关} \;\Rightarrow\; \text{构成一个平面或一条直线}$
        \item $\alpha_1, \alpha_2, \alpha_3$线性无关,$\beta_1$可由$\alpha_1, \alpha_2, \alpha_3$线性表示,$\beta_2$不能由$\alpha_1, \alpha_2, \alpha_3$线性表示,则
        \begin{enumerate}
            \item $\alpha_1, \alpha_2, \alpha_3, \beta_1$线性相关
            \item $\alpha_1, \alpha_2, \alpha_3, \beta_2$线性无关
            \item 对于任意实数$k$,有$\alpha_1, \alpha_2, \alpha_3, k\beta_1 + \beta_2$线性无关
            \item 对于任意实数$k$,$\alpha_1, \alpha_2, \alpha_3, \beta_1 + k\beta_2$
            \begin{itemize}
                \item $k = 0$ \Rightarrow \text{相关}
                \item $k \neq 0$ \Rightarrow \text{无关}
            \end{itemize}
        \end{enumerate}
        \item 向量组$\alpha_1, \alpha_2, \alpha_3$可由向量组$\beta_1, \beta_2, \beta_3$线性表示 \;\Rightarrow\; $r(\alpha_1, \alpha_2, \alpha_3) \le r(\beta_1, \beta_2, \beta_3)$
        \item $\beta$可由$\alpha_1, \alpha_2, \alpha_3$线性表出,且表示法{\color{red}{不}}唯一
        \begin{align*}
            &\Leftrightarrow\; Ax = \beta\text{有无穷多解} \\
            &\Leftrightarrow\; r(\alpha_1, \alpha_2, \alpha_3) = r(\alpha_1, \alpha_2, \alpha_3, \beta) < 3 \\
            &\Rightarrow\; \alpha_1, \alpha_2, \alpha_3\text{线性相关}
        \end{align*}
        \item $r(\alpha_1, \alpha_2, \dots, \alpha_n) \neq r(\alpha_1, \alpha_2, \dots, \alpha_n, \beta)$
        \begin{align*}
            &\Leftrightarrow\; Ax = \beta\text{无解} \\
            &\Rightarrow\; \beta\text{不能由}\alpha_1, \alpha_2, \alpha_3\text{线性表出}
        \end{align*}
    \end{enumerate}

    \subsection{理解}

    \begin{enumerate}
        \item $\text{线性相关无关} \;\Rightarrow\; Ax = 0\text{齐次方程组非零解的问题}$
        \item $\text{线性表示} \;\Rightarrow\; Ax = \beta(\text{其中}A = [\alpha_1, \alpha_2, \dots, \alpha_n])\text{非齐次方程组解的问题}$
        \begin{itemize}
            \item 非齐次方程组的解向量 $x$ 即为线性表示的系数
            \item 对于 $n$ 维向量,有 $n$ 个方程,未知数个数等于向量个数
            \item 根据秩的不同,线性表示有三种情况:
            \begin{enumerate}
                \item \textbf{唯一表示:}$\; r(A) = r(A,\beta) = n$,$(\alpha_1, \alpha_2, \dots, \alpha_n)$向量组线性无关
                \item \textbf{无穷多表示:}$\; r(A) = r(A,\beta) < n$,$(\alpha_1, \alpha_2, \dots, \alpha_n)$向量组线性相关
                \item \textbf{无表示:}$\; r(A) < r(A,\beta)$,$\beta$ 不能由 $(\alpha_1,\dots,\alpha_n)$ 线性表示
            \end{enumerate}
        \end{itemize}
        \item 设向量组$\alpha_1 = (1,0,1)^T, \alpha_2 = (0,1,1)^T, \alpha_3 = (1,3,5)^T$不能由向量组$\beta_1 = (1,1,1)^T, \beta_2 = (1,2,3)^T, \beta_3 = (3,4,a)^T$线性表示。求$a$的值并将$\beta_1, \beta_2, \beta_3$用$\alpha_1, \alpha_2, \alpha_3$线性表出
        \begin{analysisbox}
            \[
                \because |\alpha_1, \alpha_2, \alpha_3| = 1 \neq 0
            \]
            又因为$\alpha_1, \alpha_2, \alpha_3$不能由$\beta_1, \beta_2, \beta_3$线性表示, \\
            4个三维向量必线性相关,若$\beta_1, \beta_2, \beta_3$线性无关,那么$\alpha_1, \alpha_2, \alpha_3$必可由$\beta_1, \beta_2, \beta_3$线性表出 \\
            所以$\beta_1, \beta_2, \beta_3$线性相关 \\
            即$|\beta_1, \beta_2, \beta_3| = 0$ \\
            故$a = 5$ \\
            对$[\alpha_1, \alpha_2, \alpha_3 | \beta_1, \beta_2, \beta_3]$作初等变换,有
            \[
                \left[
                    \begin{array}{ccc|ccc}
                        1 & 0 & 1 & 1 & 1 & 3 \\
                        0 & 1 & 3 & 1 & 2 & 4 \\
                        1 & 1 & 5 & 1 & 3 & 5
                    \end{array}\right] \rightarrow \left[\begin{array}{ccc|ccc}
                                                             1 & 0 & 0 & 2  & 1 & 5  \\
                                                             0 & 1 & 0 & 4  & 2 & 10 \\
                                                             0 & 0 & 1 & -1 & 0 & -2
                \end{array}\right]
            \]

            所以$\beta_1 = 2\alpha_1 + 4\alpha_2 - \alpha_3, \beta_2 = \alpha_1 + 2\alpha_2, \beta_3 = 5\alpha_1 + 10\alpha_2 - 2\alpha_3$
        \end{analysisbox}
    \end{enumerate}

\end{document}
