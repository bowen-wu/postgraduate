\documentclass[a4paper,12pt]{article}
\usepackage{xeCJK}          % 中文支持
\usepackage{fontspec}       % 英文/数学字体
\usepackage{amsmath, amssymb} % 数学公式
\usepackage{graphicx}       % 插入图片
\usepackage{hyperref}       % 目录超链接
\usepackage{geometry}       % 页面布局
\usepackage{bm}             % 粗体
\usepackage{xcolor}         % 颜色
\usepackage{tabularx}       %表格环境
\geometry{left=3cm,right=3cm,top=3cm,bottom=3cm}

% =========================
% 字体设置
% =========================
\setmainfont{Times New Roman}
\setsansfont{Helvetica Neue}
\setmonofont{Menlo}
\setCJKmainfont{PingFang SC}

% =========================
% 图形路径(可调整)
% =========================
\graphicspath{{./assets/}}

% =========================
% 文档开始
% =========================
\begin{document}

%    \title{向量}
%    \author{Bowen}
%    \date{\today}
%    \maketitle

% =========================

    \section{向量}

    \subsection{基本概念}

    \begin{enumerate}
        \item $n$个数$a_1, a_2, \dots, a_n$所组成的有序数组$\alpha = (a_1, a_2, \dots, a_n)^T$或$\alpha = (a_1, a_2, \dots, a_n)$称为$n$维向量,其中$a_1, a_2, \dots, a_n$称为向量$\alpha$的分量(或坐标),前一个表示式称为列向量,后者称为行向量
        \item 对$n$维向量$\alpha_1, \alpha_2, \dots, \alpha_s$,如果存在不全为零的数$k$,使得
        \[
            k_{1}\alpha_1 + k_{2}\alpha_2 + \dots + k_{s}\alpha_s = 0
        \]
        则称向量组$\alpha_1, \alpha_2, \dots, \alpha_s$线性相关,否则,称向量组$\alpha_1, \alpha_2, \dots, \alpha_s$线性无关
        \begin{enumerate}
            \item 有零向量
            \item 两向量成比例
            \item $n+1$个$n$维向量
        \end{enumerate}
        \item TODO: 向量组$\alpha_1 = (a_{11}, a_{21}, \dots, a_{r1})^T, \alpha_2 = (a_{12}, a_{22}, \dots, a_{r2})^T, \dots, \alpha_m = (a_{1m}, a_{2m}, \dots, a_{rm})^T$及向量组$\alpha_1 = (a_{11}, a_{21}, \dots, a_{r1})^T, \alpha_2 = (a_{12}, a_{22}, \dots, a_{r2})^T, \dots, \alpha_m = (a_{1m}, a_{2m}, \dots, a_{rm})^T$
    \end{enumerate}

    \subsection{定理}

    \begin{enumerate}
        \item 向量组$\alpha_1, \alpha_2, \dots, \alpha_s$线性相关
        \begin{enumerate}
            \item \Leftrightarrow 其次线性方程组$[\alpha_1, \alpha_2, \dots, \alpha_s][x_1, x_2, \dots, x_n]^T = 0$有非零解
            \item \Leftrightarrow 向量组的秩$r(\alpha_1, \alpha_2, \dots, \alpha_s) < s$,$s$表示\textbf{未知数的个数}或\textbf{向量个数}
            \item \Leftrightarrow 若向量组是{\color{red}{方阵}}($n$个$n$维向量),则$|\alpha_1, \alpha_2, \dots, \alpha_n| = 0$
        \end{enumerate}
        \item $n+1$个$n$维向量一定线性相关
        \item 任何{\color{red}{部分}}组$\alpha_1, \alpha_2, \dots, \alpha_r$相关 \Rightarrow \text{{\color{green}{整体}}组}$\alpha_1, \alpha_2, \dots, \alpha_r, \dots \alpha_s$相关
        \item {\color{red}{整体}}组$\alpha_1, \alpha_2, \dots, \alpha_r, \dots \alpha_s$无关 \Rightarrow \text{{\color{green}{部分}}组}$\alpha_1, \alpha_2, \dots, \alpha_r$无关
        \item $\alpha_1, \alpha_2, \dots, \alpha_n$ 线性无关 \Rightarrow \text{延伸组}$\widetilde{\mathbf{\alpha_1}}, \widetilde{\mathbf{\alpha_2}}, \dots, \widetilde{\mathbf{\alpha_n}}$线性无关
        \item $\widetilde{\mathbf{\alpha_1}}, \widetilde{\mathbf{\alpha_2}}, \dots, \widetilde{\mathbf{\alpha_n}}$线性相关 \Rightarrow \text{缩短组}$\alpha_1, \alpha_2, \dots, \alpha_n$ 线性相关


    \end{enumerate}

    \subsection{运算}

    \begin{enumerate}
        \item 设$n$维向量$\alpha = (a_1, a_2, \dots, a_n)^T, \beta= (b_1, b_2, \dots, b_n)^T$,则
        \begin{enumerate}
            \item $\alpha + \beta = (a_1 + b_1, a_2 + b_2, \dots, a_n + b_n)^T$
            \item $k\alpha = (ka_1, ka_2, \dots, ka_n)^T$
            \item $0\alpha = 0$
            \item $(\alpha, \beta) = \alpha^T\beta = \beta^T\alpha = a_{1}b_{1} + a_{2}b_{2} + \dots + a_{n}b_{n}$
            \item $\alpha + \beta = \beta + \alpha$
            \item $(\alpha + \beta) + \gamma = \alpha + (\beta + \gamma)$
            \item $\alpha + 0 = \alpha $
            \item $\alpha + (-\alpha) = 0$
            \item $1\alpha = \alpha$
            \item $k(l\alpha) = (kl)\alpha$
            \item $k(\alpha + \beta) = k\alpha + k\beta$
            \item $(k + l)\alpha = k\alpha + l\alpha$
        \end{enumerate}
        \item
    \end{enumerate}

    \subsection{条件转换思路}

    \begin{enumerate}
        \item
    \end{enumerate}
\end{document}
