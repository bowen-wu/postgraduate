\documentclass[a4paper,12pt]{article}
\usepackage{xeCJK}          % 中文支持
\usepackage{fontspec}       % 英文/数学字体
\usepackage{amsmath, amssymb} % 数学公式
\usepackage{graphicx}       % 插入图片
\usepackage{hyperref}       % 目录超链接
\usepackage{geometry}       % 页面布局
\usepackage{bm}             % 粗体
\usepackage{xcolor}         % 颜色
\usepackage{tabularx}
\geometry{left=3cm,right=3cm,top=3cm,bottom=3cm}

% =========================
% 字体设置
% =========================
\setmainfont{Times New Roman}
\setsansfont{Helvetica Neue}
\setmonofont{Menlo}
\setCJKmainfont{PingFang SC}

% =========================
% 图形路径(可调整)
% =========================
\graphicspath{{./assets/}}

% =========================
% 文档开始
% =========================
\begin{document}

%    \title{线性代数}
%    \author{Bowen}
%    \date{\today}
%    \maketitle

% =========================

    \section{矩阵}

    \subsection{基础概念}

    \begin{enumerate}
        \item $m$行$n$列表格称为$m \times n$矩阵,当 $m = n$ 时,矩阵$A$称为$n$阶矩阵或\textbf{$n$阶方阵}
        \item 如果一个矩阵的所有元素都是$0$,则称这个矩阵是\textbf{零矩阵},可简记为$O$
        \item 两个$m \times n$型矩阵$A = [a_{ij}]$,$B = [b_{ij}]$,如果对应的元素都相等,即$a_{ij} = b_{ij}(i = 1,2,\dots,m; j = 1,2,\dots,n)$,则称矩阵$A$与$B$相等,记作$A = B$
        \item $n$阶方阵$A = [a_{ij}]_{n \times n}$的元素所构成的行列式称为$n$阶方阵$A$的行列式,记作$|A|$或$\det A$
        \item 把矩阵$A$的行换成同序数的列得到一个新矩阵,称为矩阵$A$的\textbf{转置矩阵},记作$A^T$
        \item 如果方阵$A$满足$A^T = A$,则称$A$是\textbf{对称矩阵}
        \item $n$阶方阵$A = [a_{ij}]_{n \times n}$,行列式$|A|$的每个元素$a_{ij}$的代数余子式$A_{ij}$所构成的如下矩阵
        \[
            A^* =
            \begin{bmatrix}
                A_{\color{red}{11}} & A_{\textcolor[rgb]{0.2, 0.6, 0.3}{21}} & \dots & A_{\textcolor{blue}{n1}} \\
                A_{\color{red}{12}} & A_{\textcolor[rgb]{0.2, 0.6, 0.3}{22}} & \dots & A_{\textcolor{blue}{n2}} \\
                \vdots              & \vdots                                 &       & \vdots                   \\
                A_{\color{red}{1n}} & A_{\textcolor[rgb]{0.2, 0.6, 0.3}{2n}} & \dots & A_{\textcolor{blue}{nn}}
            \end{bmatrix}
        \]
        称为矩阵$A$的\textbf{伴随矩阵}
        \item $n$阶方阵$A = [a_{ij}]_{n \times n}$,如果存在$n$阶方阵$B$使得$AB = BA = E$(单位矩阵)成立,则称$A$是\textbf{可逆矩阵}或\textbf{非奇异矩阵},$B$是$A$的逆矩阵
        \item 对$m \times n$矩阵,下列三种变换
        \begin{enumerate}
            \item 用非零常数$k$乘矩阵的某一行(列)
            \item 互换矩阵某两行(列)的位置
            \item 把某行(列)的$k$倍加至另一行(列)
        \end{enumerate}
        称为矩阵的\textbf{初等行(列)变换},统称为矩阵的\textbf{初等变换}
        \item 如果矩阵$A$经过有限次初等变换变成矩阵$B$,则称矩阵$A$与矩阵$B$\textbf{等价},记作$A \overset{\sim}{=} B$
        \item 单位矩阵经过一次初等变换等到的矩阵称为\textbf{初等矩阵}
        \begin{enumerate}
            \item $E_{i}(k)$ 单位矩阵第$i$行乘以常数$k$
            \item $E_{ij}$ 单位矩阵互换$i,j$行
            \item $E_{ij}(k)$ 单位矩阵第$j$行的$k$倍加至第$i$行
        \end{enumerate}
        \item 设 $A$ 是 $n$ 阶矩阵,满足 $AA^{T} = A^{T}A = E$,称 $A$ 是 \textbf{正交矩阵}:
        \item 设$\alpha = (a_1, a_2, \dots, a_n)^{T}$,$\beta = (b_1, b_2, \dots, b_n)^{T}$。向量内积:$(\alpha, \beta) = \alpha^{T}\beta = \beta^{T}\alpha = a_{1}b_{1} + a_{2}b_{2} + \dots + a_{n}b_{n}$
        \begin{align*}
            &\Leftrightarrow\; a_{1}^2 + a_{2}^2 + \dots + a_{n}^2 = 1 \\
            &\Leftrightarrow\; a_ \\
            &\Leftrightarrow\; A^{T} = A^{-1} \\
            &\Leftrightarrow\; A \text{ 的行(列)向量两两正交(单位向量)} \\
            &\Leftrightarrow\; A \text{ 的每个行(列)向量长度均为1} \\
            &\Leftrightarrow\; A \text{ 的行(列)向量平方和为 1} \\
            &{\color{red}{\Rightarrow}}\; |A|^{2} = 1 \;\;\Leftrightarrow\;\; |A| = 1 \text{ 或 } |A| = -1
        \end{align*}
    \end{enumerate}

    \subsection{定理}

    \begin{enumerate}
        \item 若$A$是可逆矩阵,则矩阵$A$的逆矩阵\textbf{唯一},记为$A^{-1}$
        \item $n$ 阶矩阵$A$可逆
        \begin{align*}
            &\Leftrightarrow\; |A| = 0  \\
            &\Leftrightarrow\; r(A) = n  \\
            &\Leftrightarrow\; A \text{的列(行)向量组线性无关}  \\
            &\Leftrightarrow\; A = P_{1}P_{2}\dots P_{s}, P_{i}(i = 1,2,\dots,s)\text{是初等矩阵}  \\
            &\Leftrightarrow\; A \text{与单位矩阵等价}  \\
            &\Leftrightarrow\; 0\text{不是矩阵} A \text{的特征值}  \\
        \end{align*}
        \item 若$A$是$n$阶矩阵,且满足$AB = E$,则必有$BA = E$
        \item 用初等矩阵$P$左(右)乘矩阵$A$,其结果$PA$($AP$)就是对矩阵$A$作一次相应的初等行(列)变换
        \item 初等矩阵均可逆,其逆矩阵是同类型的初等矩阵,即

        \renewcommand{\arraystretch}{1.2}  % 行高放大 1.2 倍
        \begin{tabularx}{\textwidth}{l c >{\raggedright\arraybackslash}X}
            倍乘 & $E_i^{-1}(k) = E_i(1/k)$      & 第 $i$ 行(或列)乘以非零常数 $k$ 的逆矩阵是第 $i$ 行(或列)乘以 $1/k$                     \\
            倍加 & $E_{ij}^{-1}(k) = E_{ij}(-k)$ & 第 $i$ 行(或列)加上 $k$ 倍第 $j$ 行(或列)的逆矩阵是第 $i$ 行(或列)加上 $-k$ 倍第 $j$ 行(或列) \\
            互换 & $E_{ij}^{-1} = E_{ij}$        & 交换第 $i$ 行(或列)和第 $j$ 行(或列)的逆矩阵是其本身                                  \\
        \end{tabularx}
        \item 矩阵$A$与$B$\textbf{等价}的充分必要条件是存在可逆矩阵$P$与$Q$,使$PAQ = B$
        \item 秩$r(A) = A \text{的列秩} = A \text{的行秩}$
        \item 矩阵经初等变换后秩不变
    \end{enumerate}

    \subsection{运算}

    \begin{enumerate}
        \item 设$A = [a_{ij}]$,$B = [b_{ij}]$是两个$m \times n$矩阵,则$m \times n$矩阵$C = [c_{ij}] = [a_{ij} + b_{ij}]$称为矩阵$A$与$B$的和,记作$A + B = C$
        \item 设$A = [a_{ij}]$是$m \times n$矩阵,$k$是一个常数,则$m \times n$矩阵$[ka_{ij}]$称为数$k$与矩阵$A$的\textbf{数乘},记作$kA$
        \item 设$A, B, C, O$都是$m \times n$矩阵,$k, l$是常数,则矩阵的加法和数乘运算满足:
        \begin{enumerate}
            \item $A + B = B + A$
            \item $(A + B) + C = A + (B + C)$
            \item $A + O = A$
            \item $A + (-A) = O$
            \item $1A = A$
            \item $k(lA) = (kl)A$
            \item $k(A + B) = kA + kB$
            \item $(k + l)A = kA + lA$
        \end{enumerate}
        \item 设$A = [a_{ij}]$是$m \times n$矩阵,$B = [b_{ij}]$是$n \times s$矩阵,那么$m \times s$矩阵$C = [c_{ij}]$,其中
        \[
            c_{ij} = a_{i1}b_{1j} + a_{i2}b_{2j} + \dots + a_{in}b_{nj} = \sum_{k=1}^{n} a_{ik}b_{kj}
        \]
        称为$A$与$B$的\textbf{乘积},记作$C = AB$
        \item 矩阵乘法有下列法则:
        \begin{enumerate}
            \item $A(BC) = (AB)C$
            \item $A(B + C) = AB + AC$
            \item $(A + B)C = AC + BC$
            \item $(kA)(lB) = klAB$
            \item $AE = EA = A$
            \item $OA = AO = O$
        \end{enumerate}
        \item 设$A$是$n$阶矩阵,$k$是正整数,
        \begin{enumerate}
            \item $A$的$k$次方幂$A^k = A \cdot A \dots A$($k$个$A$)
            \item $\mathbf{A^0 = E}$
            \item $A^k \cdot A^l = A^{k+l}$
            \item $(A^k)^l = A^{kl}$
        \end{enumerate}
        \item
        \[
            \begin{bmatrix}
                A_1 & A_2 \\
                A_3 & A_4
            \end{bmatrix}
            + \begin{bmatrix}
                  B_1 & B_2 \\
                  B_3 & B_4
            \end{bmatrix}
            = \begin{bmatrix}
                  A_1 + B_1 & A_2 + B_2 \\
                  A_3 + B_3 & A_4 + B_4
            \end{bmatrix}
        \]
        \item
        \[
            \begin{bmatrix}
                A & B \\
                C & D
            \end{bmatrix}
            \begin{bmatrix}
                X & Y \\
                Z & W
            \end{bmatrix}
            = \begin{bmatrix}
                  AX + BZ & AY + BW \\
                  CX + DZ & CY + DW
            \end{bmatrix}
        \]
        \item
        \[
            \begin{bmatrix}
                A & B \\
                C & D
            \end{bmatrix}^{T}
            = \begin{bmatrix}
                  A^T & B^T \\
                  C^T & D^T
            \end{bmatrix}
        \]
        \item $(A + B)^2 = (A + B)(A + B) = A^2 + {\color{red}{AB}} + {\color{red}{BA}} + B^2 \;\mathbf{\neq}\; A^2 + 2AB + B^2$
        \item $\color[rgb]{0.2, 0.6, 0.3}{\mathbf{(A + E)^2 = A^2 + 2A + E}}$
        \begin{enumerate}
            \item $E - A^3 = (E - A)(E + A + A^2)$
            \item $E + A^3 = (E + A)(E - A + A^2)$
            \item $AB - 2B - 4A = 0$ \Leftrightarrow $(A - 2E)(B - 4E) = 8E$
        \end{enumerate}
        \item 设$\alpha$和$\beta$都是列向量,则
        \begin{enumerate}
            \item $\alpha\beta^T = (\beta\alpha^T)^T$,两者都是$n$阶矩阵(互为转置)
            \item $\alpha^T\beta = \beta^T\alpha$ 是一个\textbf{数}
            \item
            \[
                \alpha\alpha^T =
                \begin{bmatrix}
                    a_1^2   & a_1 a_2 & a_1 a_3 & \dots  & a_1 a_n \\
                    a_1 a_2 & a_2^2   & a_2 a_3 & \dots  & a_2 a_n \\
                    a_1 a_3 & a_2 a_3 & a_3^2   & \dots  & a_3 a_n \\
                    \vdots  & \vdots  & \vdots  & \ddots & \vdots  \\
                    a_1 a_n & a_2 a_n & a_3 a_n & \dots  & a_n^2
                \end{bmatrix} \text{(对称矩阵)}
            \]
            \item
            \[
                \alpha^T \alpha = a_1^2 + a_2^2 + \dots + a_n^2 = \sum_{k=1}^{n} a_k^2 \quad \text{(平方和)}
            \]
        \end{enumerate}
    \end{enumerate}

    \subsection{公式}

    \subsubsection{转置}

    \begin{enumerate}
        \item $(A^T)^T = A$
        \item $(A + B)^T = A^T + B^T$
        \item $(A - B)^T = A^T - B^T$
        \item $(kA)^T = kA^T$
        \item $(AB)^T = B^{T}A^T $
    \end{enumerate}

    \subsubsection{可逆}

    \begin{enumerate}
        \item $(A^{-1})^{-1} = A$
        \item $(kA)^{-1} = \frac{1}{k}A^{-1}(k \neq 0)$
        \item $(AB)^{-1} = B^{-1}A^{-1}$
        \item $(ABC)^{-1} = C^{-1}B^{-1}A^{-1}$
        \item $(A^n)^{-1} = (A^{-1})^n$
        \item $(A^{-1})^T = (A^T)^{-1}$
        \item $|A^{-1}| = \frac{1}{|A|}$
        \item $A^{-1} = \frac{1}{|A|}A^{*}$
    \end{enumerate}

    \subsubsection{伴随}

    \begin{enumerate}
        \item $AA^{*} = A^{*}A = |A|E$
        \item $A^{*} = |A|A^{-1}$
        \item $|A^*| = |A|^{n-1}$
        \item $(A^*)^{-1} = (A^{-1})^* = \frac{1}{|A|}A$
        \item $(AB)^* = B^{*}A^*$
        \item $(A^*)^T = (A^T)^*$
        \item $(kA)^* = k^{n-1}A^*$
        \item $(A^*)^* = |A|^{n-2}A$
        \begin{enumerate}
            \item 若$A$不可逆($|A| = 0$),则
            \begin{enumerate}
                \item 且$n \ge 3$时,$(A^*)^* = O$
                \item 且$n = 2$时,$(A^*)^* = A$
            \end{enumerate}
        \end{enumerate}
        \item
        \begin{flalign*}
            r(A^*) &=
            \begin{cases}
                n, & \text{如果 } r(A) = n, \\
                1, & \text{如果 } r(A) = n - 1, \\
                0, & \text{如果 } r(A) < n - 1
            \end{cases} &&
        \end{flalign*}
        \item 设$A = \begin{bmatrix}
                         a & b \\
                         c & d
        \end{bmatrix}$(二阶矩阵),则$A^* = \begin{bmatrix}
                                              d  & -b \\
                                              -c & a
        \end{bmatrix}$。\textbf{主对调,副变号}
    \end{enumerate}

    \subsubsection{秩}

    \begin{enumerate}
        \item $r(A) = r(A^T) = r(A^{T}A) = r(AA^{T})$
        \item 当$k \neq 0$时,$r(kA) = r(A)$
        \item $r(A + B) \le r(A, B) \le r(A) + r(B)$
        \item $A$是$m \times n$矩阵,$B$是$n \times s$矩阵,则
        \begin{enumerate}
            \item $r(AB) \le r(A) \land r(AB) \le r(B)$,即 $r(AB) \le \min(r(A), r(B))$
            \item $r(A) + r(B) - n \le r(AB)$
            \item 且$AB = O$,则
            \begin{enumerate}
                \item $r(A) + r(B) \le n$
                \item $B$ 的列向量是齐次方程组 $Ax = 0$ 的解
                \begin{itemize}
                    \item 按列分块,有
                    \[
                        B = [b_1, b_2, \dots, b_s], \\
                        AB = A[b_1, b_2, \dots, b_s] = [Ab_1, Ab_2, \dots, Ab_s] = [0, 0, \dots, 0]
                    \]
                    因此
                    \[
                        Ab_i = 0, \quad i = 1, 2, \dots, s.
                    \]
                \end{itemize}
            \end{enumerate}
            \item 且$AB = C$,则
            \begin{enumerate}
                \item {\color[rgb]{0.2, 0.6, 0.3}{矩阵$AB$的行向量$\alpha_1, \alpha_2, \dots, \alpha_n$可由$B$的行向量$\beta_1, \beta_2, \dots, \beta_n$线性表出}}
                \begin{itemize}
                    \item 对 $B$,$C$ 按列分块,有
                    \[
                        \begin{bmatrix}
                            a_{11} & a_{12} & \dots & a_{1n} \\
                            a_{21} & a_{22} & \dots & a_{2n} \\
                            \vdots & \vdots &       & \vdots \\
                            a_{n1} & a_{n2} & \dots & a_{nn} \\
                        \end{bmatrix}
                        \begin{bmatrix}
                            \beta_1 \\ \beta_2 \\ \vdots \\ \beta_n
                        \end{bmatrix}
                        =
                        \begin{bmatrix}
                            \alpha_1 \\ \alpha_2 \\ \vdots \\ \alpha_n
                        \end{bmatrix}
                    \]
                    即
                    \[
                        \begin{cases}
                            a_{11}\beta_1 + \dots + a_{1n}\beta_n &= \alpha_1, \\
                            a_{21}\beta_1 + \dots + a_{2n}\beta_n &= \alpha_2, \\
                            \vdots & \\
                            a_{n1}\beta_1 + \dots + a_{nn}\beta_n &= \alpha_n
                        \end{cases}
                    \]
                \end{itemize}
                \item {\color[rgb]{0.2, 0.6, 0.3}{矩阵$AB$的列向量可由$A$的列向量线性表出}}
            \end{enumerate}
        \end{enumerate}
        \item 若$A$可逆,则$r(AB) = r(B) = r(BA)$
        \item 若$A$列满秩,则$r(AB) = r(B)$
        \item 若$A$行满秩,则$r(AB) = r(A)$
        \item $A$是$m \times n$矩阵,$B$是$n \times s$矩阵,$C$是$s \times t$矩阵,则
        \[
            r(AB) + r(BC) \le r(ABC) + r(B)
        \]
        \item
        \[
            r\!\begin{bmatrix}
                   A & O \\
                   O & B
            \end{bmatrix}
            = r\!\begin{bmatrix}
                     O & A \\
                     B & O
            \end{bmatrix}
            = r(A) + r(B)
        \]
        \item
        \[
            r\!\begin{bmatrix}
                   A & O \\
                   C & B
            \end{bmatrix}
            \ge r(A) + r(B)
        \]
        \item 若$A \sim B$,则
        \begin{enumerate}
            \item $r(A) = r(B)$
            \item $r(A + kE) = r(B + kE)$
        \end{enumerate}
    \end{enumerate}

    \subsubsection{分块矩阵}

    \begin{enumerate}
        \item 若$B$,$C$分别是$m$阶与$n$阶矩阵,则
        \[
            \begin{bmatrix}
                B & O \\
                O & C
            \end{bmatrix}^n
            = \begin{bmatrix}
                  B^n & O   \\
                  O   & C^n
            \end{bmatrix}
        \]
        \item 若$B$,$C$分别是$m$阶与$n$阶\textbf{可逆}矩阵,则
        \begin{enumerate}
            \item
            \[
                \begin{bmatrix}
                    B & O \\
                    O & C
                \end{bmatrix}^{-1}
                = \begin{bmatrix}
                      B^{-1} & O      \\
                      O      & C^{-1}
                \end{bmatrix}
            \]
            \item
            \[
                \begin{bmatrix}
                    O & B \\
                    C & O
                \end{bmatrix}^{-1}
                = \begin{bmatrix}
                      O      & C^{-1} \\
                      B^{-1} & O
                \end{bmatrix}
            \]
        \end{enumerate}
        \item $A$是$m \times n$矩阵,$B$是$n \times s$矩阵,且
        \item $A$是$m \times n$矩阵,$B$是$n \times s$矩阵,$C$是$s \times t$矩阵,则
    \end{enumerate}

    \subsubsection{特殊矩阵$n$次方}
    \begin{enumerate}
        \item 若$r(A) = 1$,则
        \begin{enumerate}
            \item $A$可分解为一个列向量与一个行向量的乘积
            \item $A^2 = lA$其中$l = \sum a_{ii} = a_{11} + a_{22} + \dots + a_{nn} $
            \item $A^n = l^{n-1}A$其中$l = \sum a_{ii} = a_{11} + a_{22} + \dots + a_{nn} $
        \end{enumerate}
        \item 设 $A$ 为 $n \times n$ 上三角矩阵,主对角线为 0
        \[
            A =
            \begin{bmatrix}
                0      & a_{12} & a_{13} & \dots  & a_{1n} \\
                0      & 0      & a_{23} & \dots  & a_{2n} \\
                0      & 0      & 0      & \dots  & a_{3n} \\
                \vdots & \vdots & \vdots & \ddots & \vdots \\
                0      & 0      & 0      & \dots  & 0
            \end{bmatrix}.
        \]

        则:
        \[
            A^2 =
            \begin{bmatrix}
                0      & 0      & b_{13} & \dots  & b_{1n} \\
                0      & 0      & 0      & \dots  & b_{2n} \\
                0      & 0      & 0      & \dots  & b_{3n} \\
                \vdots & \vdots & \vdots & \ddots & \vdots \\
                0      & 0      & 0      & \dots  & 0
            \end{bmatrix}, \quad
            A^3 =
            \begin{bmatrix}
                0      & 0      & 0      & c_{14} & \dots  & c_{1n} \\
                0      & 0      & 0      & 0      & \dots  & c_{2n} \\
                0      & 0      & 0      & 0      & \dots  & c_{3n} \\
                \vdots & \vdots & \vdots & \vdots & \ddots & \vdots \\
                0      & 0      & 0      & 0      & \dots  & 0
            \end{bmatrix}
        \]
        \[
            A^n = 0, \quad A^k = 0 \; \text{当} k \ge n
        \]
        \item 若$B = P^{-1}AP$,则$B^2 = P^{-1}A^{2}P$,即
        \begin{enumerate}
            \item $B^n = P^{-1}A^{n}P$
            \item $A^n = PB^{n}P^{-1}$
        \end{enumerate}
    \end{enumerate}

    \subsection{条件转换思路}

    \begin{enumerate}
        \item 设 $\mathbf{A}$ 是 $m \times n$ 矩阵,$\mathbf{B}$ 是 $n \times s$ 矩阵,若 $\mathbf{AB} = \mathbf{O}$,则
        \begin{enumerate}
            \item $\mathbf{B}$ 的列向量是其次方程组 $\mathbf{Ax} = 0$ 的解
            \item $r(\mathbf{A}) + r(\mathbf{B}) \le n$
        \end{enumerate}
    \end{enumerate}

\end{document}
