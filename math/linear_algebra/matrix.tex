\documentclass[a4paper,12pt]{article}
\usepackage{xeCJK}          % 中文支持
\usepackage{fontspec}       % 英文/数学字体
\usepackage{amsmath, amssymb} % 数学公式
\usepackage{graphicx}       % 插入图片
\usepackage{hyperref}       % 目录超链接
\usepackage{geometry}       % 页面布局
\usepackage{bm}             % 粗体
\usepackage[table]{xcolor}  % 颜色
\usepackage{tabularx}       % 表格环境
\usepackage{tikz}           % TikZ 绘制主对角线斜线
\usepackage{tcolorbox}
\geometry{left=3cm,right=3cm,top=3cm,bottom=3cm}

% 抽离颜色和尺寸参数
\newcommand{\analysisTitleColor}{green!50!black}
\newcommand{\analysisBackColor}{white}
\newcommand{\analysisBoxRule}{0.8pt}
\newcommand{\analysisArc}{3pt}
\newcommand{\analysisPadding}{6pt}

% 定义 tcolorbox
\newtcolorbox{analysisbox}{
    title=解析,
    colback=\analysisBackColor,
    colframe=\analysisTitleColor,
    boxrule=\analysisBoxRule,
    arc=\analysisArc,
    left=\analysisPadding,
    right=\analysisPadding,
    top=4pt,
    bottom=4pt
}

% =========================
% 字体设置
% =========================
\setmainfont{Times New Roman}
\setsansfont{Helvetica Neue}
\setmonofont{Menlo}
\setCJKmainfont{PingFang SC}

% =========================
% 图形路径(可调整)
% =========================
\graphicspath{{./assets/}}

% =========================
% 文档开始
% =========================
\begin{document}

%    \title{线性代数}
%    \author{Bowen}
%    \date{\today}
%    \maketitle

% =========================

    \section{矩阵}

    \subsection{基础概念}

    \begin{enumerate}
        \item $m$行$n$列表格称为$m \times n$矩阵,当 $m = n$ 时,矩阵$A$称为$n$阶矩阵或\textbf{$n$阶方阵}
        \item 如果一个矩阵的所有元素都是$0$,则称这个矩阵是\textbf{零矩阵},可简记为$\mathbf{O}$
        \item 如果一个方阵,所有非主对角线元素都是0,则称这个矩阵是\textbf{对角矩阵}
        \item 两个$m \times n$型矩阵$A = [a_{ij}]$,$B = [b_{ij}]$,如果对应的元素都相等,即$a_{ij} = b_{ij}(i = 1,2,\dots,m; j = 1,2,\dots,n)$,则称矩阵$A$与$B$相等,记作$A = B$
        \item $n$阶方阵$A = [a_{ij}]_{n \times n}$的元素所构成的行列式称为$n$阶方阵$A$的行列式,记作$|A|$或$\det A$
        \item 把矩阵$A$的行换成同序数的列得到一个新矩阵,称为矩阵$A$的\textbf{转置矩阵},记作$A^T$
        \item 如果方阵$A$满足$A^T = A$,则称$A$是\textbf{对称矩阵},即$a_{ij} = a_{ji}$
        \item 如果方阵$A$满足$A^T = -A$,则称$A$是\textbf{反对称矩阵},即$a_{ij} = -a_{ji}$
        \item $n$阶方阵$A = [a_{ij}]_{n \times n}$,行列式$|A|$的每个元素$a_{ij}$的代数余子式$A_{ij}$所构成的如下矩阵
        \[
            A^* =
            \begin{bmatrix}
                A_{\color{red}{11}} & A_{\textcolor[rgb]{0.2, 0.6, 0.3}{21}} & \dots & A_{\textcolor{blue}{n1}} \\
                A_{\color{red}{12}} & A_{\textcolor[rgb]{0.2, 0.6, 0.3}{22}} & \dots & A_{\textcolor{blue}{n2}} \\
                \vdots              & \vdots                                 &       & \vdots                   \\
                A_{\color{red}{1n}} & A_{\textcolor[rgb]{0.2, 0.6, 0.3}{2n}} & \dots & A_{\textcolor{blue}{nn}}
            \end{bmatrix}
        \]
        称为矩阵$A$的\textbf{伴随矩阵}
        \item 伴随矩阵是余子式矩阵的转置,即$A^* = [A_{ji}] = (A_{ij})^T$
        \item $n$阶方阵$A = [a_{ij}]_{n \times n}$,如果存在$n$阶方阵$B$使得$AB = BA = E$(单位矩阵)成立,则称$A$是\textbf{可逆矩阵}或\textbf{非奇异矩阵},$B$是$A$的逆矩阵
        \item 对$m \times n$矩阵,下列三种变换
        \begin{enumerate}
            \item 用非零常数$k$乘矩阵的某一行(列)
            \item 互换矩阵某两行(列)的位置
            \item 把某行(列)的$k$倍加至另一行(列)
        \end{enumerate}
        称为矩阵的\textbf{初等行(列)变换},统称为矩阵的\textbf{初等变换}
        \item 如果矩阵$A$经过有限次初等变换变成矩阵$B$,则称矩阵$A$与矩阵$B$\textbf{等价},记作$A \overset{\sim}{=} B$
        \item 单位矩阵经过一次初等变换等到的矩阵称为\textbf{初等矩阵}
        \begin{enumerate}
            \item $E_{i}(k)$ 单位矩阵第$i$行乘以常数$k$
            \item $E_{ij}$ 单位矩阵互换$i,j$行
            \item $E_{ij}(k)$ 单位矩阵第$j$行的$k$倍加至第$i$行
        \end{enumerate}
        \item 设$\alpha = (a_1, a_2, \dots, a_n)^{T}$,$\beta = (b_1, b_2, \dots, b_n)^{T}$。向量内积为
        \[
            (\alpha, \beta) = \alpha^{T}\beta = \beta^{T}\alpha = a_{1}b_{1} + a_{2}b_{2} + \dots + a_{n}b_{n}
        \]
        \item 向量$\alpha = (a_1, a_2, \dots, a_n)^{T}$的长度
        \[
            ||\alpha|| = \sqrt{\alpha^T\alpha} = \sqrt {a_1^2 + a_2^2 + \dots + a_n^2}
        \]
        \item 若$(\alpha, \beta) = 0$即$a_{1}b_{1} + a_{2}b_{2} + \dots + a_{n}b_{n} = 0$,则称$\alpha$与$\beta$正交,记为$\alpha \perp \beta$
        \item 设 $A$ 是 $n$ 阶矩阵,满足 $AA^{T} = A^{T}A = E$,称 $A$ 是 \textbf{正交矩阵:}
        \begin{align*}
            &\Leftrightarrow\; A^{T} = A^{-1} \\
            &\Leftrightarrow\; A \text{ 的行(列)向量两两正交(单位向量)} \\
            &\Leftrightarrow\; A \text{ 的每个行(列)向量长度均为1} \\
            &\Leftrightarrow\; A \text{ 的行(列)向量平方和为 1} \\
            &\Leftrightarrow\; a_{1}^2 + a_{2}^2 + \dots + a_{n}^2 = 1 \\
            &{\color{red}{\Rightarrow}}\; |A|^{2} = 1 \;\;\Leftrightarrow\;\; |A| = 1 \text{ 或 } |A| = -1
        \end{align*}
        \item 若 $A$ 是正交矩阵且 $A = (\alpha_1,\alpha_2,\dots,\alpha_n)$,则:
        \begin{enumerate}
            \item $\alpha_{i}^{T}\alpha_{i} = 1$
            \item $\alpha_{i}^{T}\alpha_{j} = 0 \; (i \neq j)$
        \end{enumerate}
        \item 在$m \times n$矩阵$A$中,任取$k$行与$k$列$(k \le m, k \le n)$,位于这个行与列的交叉点上的$k^2$个元素按其在原来矩阵$A$中的次序可构成一个$k$阶行列式,称其为矩阵$A$的一个$k$阶\textbf{子式}
        \item 矩阵$A$的非零子式的最高阶数称为矩阵$A$的{\color[rgb]{0.2, 0.6, 0.3}{秩}},记为$r(A)$。零矩阵的秩规定为$0$
        \item 矩阵秩的理解
        \begin{enumerate}
            \item $r(A) = r$ \Leftrightarrow A \text{中\textcolor[rgb]{0.2, 0.6, 0.3}{有}}r\text{阶子式不为}0\text{,\textcolor[rgb]{0.2, 0.6, 0.3}{任何}}r + 1\text{阶子式(若存在)必\textcolor[rgb]{0.2, 0.6, 0.3}{全}为}0
            \item $r(A) < r$ \Leftrightarrow A \text{中\textcolor[rgb]{0.2, 0.6, 0.3}{每}一个}r\text{阶子式\textcolor[rgb]{0.2, 0.6, 0.3}{全}为}0
            \item $r(A) \ge r$ \Leftrightarrow A \text{中\textcolor[rgb]{0.2, 0.6, 0.3}{有}}r\text{阶子式不为}0
            \item $r(A) = 0$ \Leftrightarrow A = \mathbf{O}
            \item $r(A) \neq \mathbf{O}$ \Leftrightarrow 1 \le r(A) \le n
            \item 若$A$是$n$阶矩阵
            \begin{itemize}
                \item $r(A) = n$ \Leftrightarrow |A| \neq 0 \Leftrightarrow A\text{可逆}
                \item $r(A) < n$ \Leftrightarrow |A| = 0 \Leftrightarrow A\text{不可逆}
            \end{itemize}
            \item 若$A$是$m \times n$阶矩阵 \Leftrightarrow r(A) \le min(m, n)
        \end{enumerate}
    \end{enumerate}

    \subsection{定理}

    \begin{enumerate}
        \item 若$A$是可逆矩阵,则矩阵$A$的逆矩阵\textbf{唯一},记为$A^{-1}$
        \item $n$ 阶矩阵$A$可逆
        \begin{align*}
            &\Leftrightarrow\; |A| \neq 0  \\
            &\Leftrightarrow\; r(A) = n  \\
            &\Leftrightarrow\; A \text{的列(行)向量组线性无关}  \\
            &\Leftrightarrow\; A = P_{1}P_{2}\dots P_{s}, P_{i}(i = 1,2,\dots,s)\text{是初等矩阵}  \\
            &\Leftrightarrow\; A \text{通过初等变换能化为单位矩阵}  \\
            &\Leftrightarrow\; A \text{与单位矩阵等价}  \\
            &\Leftrightarrow\; 0\text{不是矩阵} A \text{的特征值}  \\
            &\Leftrightarrow\; \text{齐次线性方程组} Ax = 0 \text{只有零解}  \\
        \end{align*}
        \item $n$ 阶矩阵$A${\color{red}{不可逆}}
        \begin{align*}
            &\Leftrightarrow\; |A| = 0  \\
            &\Leftrightarrow\; r(A) < n  \\
            &\Leftrightarrow\; A \text{的列(行)向量组线性相关}  \\
            &\;\Leftrightarrow\; A \text{无法表示为初等矩阵的乘积} \\
            &\Leftrightarrow\; A \text{无法通过初等变换能化为单位矩阵}  \\
            &\Leftrightarrow\; 0\text{是矩阵} A \text{的特征值}  \\
            &\Leftrightarrow\; \text{齐次线性方程组} Ax = 0 \text{有非零解}
        \end{align*}
        \item $A \overset{\sim}{=} B$
        \begin{align*}
            &\Leftrightarrow\; r(A) = r(B)  \\
            &\Leftrightarrow\; A \text{通过初等变换能化为} B  \\
            &\Rightarrow\; |A| = 0 \Leftrightarrow |B| = 0, |A| \neq 0 \Leftrightarrow |B| \neq 0. \text{即}A\;B\text{的行列式同时为0或同时不为0}  \\
        \end{align*}
        \item 若$A$是$n$阶矩阵,且满足$AB = E$,则必有$BA = E$
        \item 用初等矩阵$P$左(右)乘矩阵$A$,其结果$PA$($AP$)就是对矩阵$A$作一次相应的初等行(列)变换 \; \Rightarrow \textbf{左乘行变换,右乘列变换}
        \item 初等矩阵均可逆,其逆矩阵是同类型的初等矩阵,即

        \renewcommand{\arraystretch}{1.2}  % 行高放大 1.2 倍
        \begin{tabularx}{\textwidth}{l c >{\raggedright\arraybackslash}X}
            倍乘 & $E_i^{-1}(k) = E_i(1/k)$      & 第 $i$ 行(或列)乘以非零常数 $k$ 的逆矩阵是第 $i$ 行(或列)乘以 $1/k$                     \\
            互换 & $E_{ij}^{-1} = E_{ij}$        & 交换第 $i$ 行(或列)和第 $j$ 行(或列)的\textbf{\color{red}{逆矩阵是其本身}}            \\
            倍加 & $E_{ij}^{-1}(k) = E_{ij}(-k)$ & 第 $i$ 行(或列)加上 $k$ 倍第 $j$ 行(或列)的逆矩阵是第 $i$ 行(或列)加上 $-k$ 倍第 $j$ 行(或列) \\
        \end{tabularx}
        \item 矩阵$A$与$B$\textbf{等价}的充分必要条件是存在可逆矩阵$P$与$Q$,使$PAQ = B$
        \item 秩$r(A) = A \text{的列秩} = A \text{的行秩}$
        \item 矩阵经初等变换后秩不变
    \end{enumerate}

    \subsection{运算}

    \begin{enumerate}
        \item 设$A = [a_{ij}]$,$B = [b_{ij}]$是两个$m \times n$矩阵,则$m \times n$矩阵$C = [c_{ij}] = [a_{ij} + b_{ij}]$称为矩阵$A$与$B$的和,记作$A + B = C$
        \item 设$A = [a_{ij}]$是$m \times n$矩阵,$k$是一个常数,则$m \times n$矩阵$[ka_{ij}]$称为数$k$与矩阵$A$的\textbf{数乘},记作$kA$
        \item 设$A, B, C, \mathbf{O}$都是$m \times n$矩阵,$k, l$是常数,则矩阵的加法和数乘运算满足:
        \begin{enumerate}
            \item $A + B = B + A$
            \item $(A + B) + C = A + (B + C)$
            \item $A + \mathbf{O} = A$
            \item $A + (-A) = \mathbf{O}$
            \item $1A = A$
            \item $k(lA) = (kl)A$
            \item $(kA)^n = k^{n}A^n$
            \item $k(A + B) = kA + kB$
            \item $(k + l)A = kA + lA$
        \end{enumerate}
        \item 设$A = [a_{ij}]$是$m \times n$矩阵,$B = [b_{ij}]$是$n \times s$矩阵,那么$m \times s$矩阵$C = [c_{ij}]$,其中
        \[
            c_{ij} = a_{i1}b_{1j} + a_{i2}b_{2j} + \dots + a_{in}b_{nj} = \sum_{k=1}^{n} a_{ik}b_{kj} = \sum \text{第$i$行} \times \text{第$j$列}
        \]
        称为$A$与$B$的\textbf{乘积},记作$C = AB$
        \item 矩阵乘法有下列法则:
        \begin{enumerate}
            \item $A(BC) = (AB)C$
            \item $A(B + C) = AB + AC$
            \item $(A + B)C = AC + BC$
            \item $(kA)(lB) = klAB$
            \item $AE = EA = A$
            \item $\mathbf{O}A = A\mathbf{O} = \mathbf{O}$
        \end{enumerate}
        \item 设$A$是$n$阶矩阵,$k$是正整数,
        \begin{enumerate}
            \item $A$的$k$次方幂$A^k = A \cdot A \dots A$($k$个$A$)
            \item $\mathbf{A^0 = E}$
            \item $A^k \cdot A^l = A^{k+l}$
            \item $(A^k)^l = A^{kl}$
        \end{enumerate}
        \item
        \[
            \begin{bmatrix}
                A_1 & A_2 \\
                A_3 & A_4
            \end{bmatrix}
            + \begin{bmatrix}
                  B_1 & B_2 \\
                  B_3 & B_4
            \end{bmatrix}
            = \begin{bmatrix}
                  A_1 + B_1 & A_2 + B_2 \\
                  A_3 + B_3 & A_4 + B_4
            \end{bmatrix}
        \]
        \item
        \[
            \begin{bmatrix}
                A & B \\
                C & D
            \end{bmatrix}
            \begin{bmatrix}
                X & Y \\
                Z & W
            \end{bmatrix}
            = \begin{bmatrix}
                  AX + BZ & AY + BW \\
                  CX + DZ & CY + DW
            \end{bmatrix}
        \]
        \item
        \[
            \begin{bmatrix}
                A & B \\
                C & D
            \end{bmatrix}^{T}
            = \begin{bmatrix}
                  A^T & C^T \\
                  B^T & D^T
            \end{bmatrix}
        \]
        \item $(A + B)^2 = (A + B)(A + B) = A^2 + {\color{red}{AB}} + {\color{red}{BA}} + B^2 \;\mathbf{\neq}\; A^2 + 2AB + B^2$
        \item $\color[rgb]{0.2, 0.6, 0.3}{\mathbf{(A + E)^2 = A^2 + 2A + E}}$
        \begin{enumerate}
            \item $E - A^3 = (E - A)(E + A + A^2)$
            \item $E + A^3 = (E + A)(E - A + A^2)$
            \item $AB - 2B - 4A = 0$ \Leftrightarrow \; $(A - 2E)(B - 4E) = 8E$
        \end{enumerate}
        \item 设$\alpha$和$\beta$都是列向量,则
        \begin{enumerate}
            \item 列向量 \cdot \text{行向量:} \, $\alpha\beta^T = (\beta\alpha^T)^T$,两者都是$n$阶矩阵(互为转置)
            \item 行向量 \cdot \text{列向量:} \, $\alpha^T\beta = \beta^T\alpha$ 是一个\textbf{数}
            \item
            \[
                \alpha\alpha^T =
                \begin{bmatrix}
                    a_1^2   & a_1 a_2 & a_1 a_3 & \dots  & a_1 a_n \\
                    a_1 a_2 & a_2^2   & a_2 a_3 & \dots  & a_2 a_n \\
                    a_1 a_3 & a_2 a_3 & a_3^2   & \dots  & a_3 a_n \\
                    \vdots  & \vdots  & \vdots  & \ddots & \vdots  \\
                    a_1 a_n & a_2 a_n & a_3 a_n & \dots  & a_n^2
                \end{bmatrix} \textbf{(对称矩阵)}
            \]
            \item $\color{red}{r(\alpha\alpha^T) = 1}$
            \item $\alpha\alpha^T$特征值是{\color{red}{$||\alpha||^2, 0, 0, \dots, 0(n - 1\text{个})$}}
            \item
            \[
                \alpha^T \alpha = a_1^2 + a_2^2 + \dots + a_n^2 = \sum_{k=1}^{n} a_k^2 \quad \text{(平方和)}
            \]
        \end{enumerate}
        \item 向量
        \begin{itemize}
            \item $(\alpha, \beta) = (\beta, \alpha)$
            \item $(k\alpha, \beta) = (\alpha, k\beta) = k(\alpha, \beta)$
            \item $(\alpha + \beta, \gamma) = (\alpha, \gamma) + (\beta, \gamma)$
            \item $(\alpha, \beta + \gamma) = (\alpha, \beta) + (\alpha, \gamma)$
            \item $(\alpha, \alpha) \ge 0$
        \end{itemize}
    \end{enumerate}

    \subsection{公式}

    \subsubsection{行列式}

    \begin{enumerate}
        \item $|A^T| = |A|$
        \item $|kA| = k^{n}|A|$
        \item $|AB| = |A||B|$ \;\text{,} $|A^2| = |A|^2$
        \item \color{red}{$|A^*| = |A|^{n-1}$}
        \item $|A^{-1}| = |A|^{-1}$
    \end{enumerate}

    \subsubsection{转置}

    \begin{enumerate}
        \item $(A^T)^T = A$
        \item $(A + B)^T = A^T + B^T$
        \item $(A - B)^T = A^T - B^T$
        \item $(kA)^T = kA^T$
        \item $(AB)^T = B^{T}A^T $
        \item $(E + A)^T = E + A^T $
    \end{enumerate}

    \subsubsection{伴随}

    \begin{enumerate}
        \item $(A^*)^{-1} = (A^{-1})^* = \frac{1}{|A|}A$
        \item $AA^{*} = A^{*}A = |A|E$
        \item $A^{*} = |A|A^{-1}$
        \item $A$可逆有$|A^*| = |A|^{n-1}$
        \item $(AB)^* = B^{*}A^*$
        \item $(A^*)^T = (A^T)^*$
        \item $(kA)^* = k^{n-1}A^*$
        \item $(A^*)^* = |A|^{n-2}A$
        \begin{enumerate}
            \item 若$A$不可逆($|A| = 0$),则
            \begin{enumerate}
                \item 且$n \ge 3$时,$(A^*)^* = O$
                \item 且$n = 2$时,$(A^*)^* = A$
            \end{enumerate}
        \end{enumerate}
        \item
        \begin{flalign*}
            r(A^*) &=
            \begin{cases}
                n, & \text{如果 } r(A) = n, \\
                1, & \text{如果 } r(A) = n - 1, \\
                0, & \text{如果 } r(A) < n - 1
            \end{cases} &&
        \end{flalign*}
        \item 设$A = \begin{bmatrix}
                         a & b \\
                         c & d
        \end{bmatrix}$(二阶矩阵),则$A^* = \begin{bmatrix}
                                              d  & -b \\
                                              -c & a
        \end{bmatrix}$。\textbf{主对调,副变号}
    \end{enumerate}

    \subsubsection{可逆}

    \begin{enumerate}
        \item $(A^{-1})^{-1} = A$
        \item $(kA)^{-1} = \frac{1}{k}A^{-1}(k \neq 0)$
        \item $(AB)^{-1} = B^{-1}A^{-1}$
        \item $(ABC)^{-1} = C^{-1}B^{-1}A^{-1}$
        \item $(A^n)^{-1} = (A^{-1})^n$
        \item $(A^{-1})^T = (A^T)^{-1}$
        \item $A^{-1} = \frac{1}{|A|}A^{*}$
        \item $|A^{-1}| = \frac{1}{|A|} \Rightarrow |P^{-1}||P| = 1$
        \item
        \[
            \begin{bmatrix}
                a & b \\
                c & d
            \end{bmatrix}^{-1}
            = \frac{1}{ad - bc}\begin{bmatrix}
                                   d  & -b \\
                                   -c & a
            \end{bmatrix}
        \]
        \item
        \[
            \begin{bmatrix}
                &   & a \\
                & b &   \\
                c &   &   \\
            \end{bmatrix}^{-1}
            = \begin{bmatrix}
                  &             & \frac{1}{c} \\
                  & \frac{1}{b} &             \\
                  \frac{1}{a} &             &             \\
            \end{bmatrix}
        \]
    \end{enumerate}

    \subsubsection{秩}

    \begin{enumerate}
        \item $r(A) = r(A^T) = r(A^{T}A) = r(AA^{T})$
        \item 当$k \neq 0$时,$r(kA) = r(A)$
        \item $r(A + B) \le r(A, B) \le r(A) + r(B)$
        \item $A$是$m \times n$矩阵,$B$是$n \times s$矩阵,则
        \begin{enumerate}
            \item $r(AB) \le r(A) \text{并且} r(AB) \le r(B)$,即 $r(AB) \le \min(r(A), r(B))$
            \item $r(A) + r(B) - n \le r(AB)$
            \item $\color{red}{r(A, AB) = r(A)}$详见理解1
            \item $\color{red}{r(B, BA) = r(B)}$
            \item 且$AB = O$,则
            \begin{enumerate}
                \item $r(A) + r(B) {\color{red}{\le}} n$
                \item $B$ 的列向量是齐次方程组 $Ax = 0$ 的解
                \begin{itemize}
                    \item 按列分块,有
                    \[
                        B = [b_1, b_2, \dots, b_s], \\
                        AB = A[b_1, b_2, \dots, b_s] = [Ab_1, Ab_2, \dots, Ab_s] = [0, 0, \dots, 0]
                    \]
                    因此
                    \[
                        Ab_i = 0, \quad i = 1, 2, \dots, s.
                    \]
                \end{itemize}
            \end{enumerate}
            \item 且$AB = C$,则
            \begin{enumerate}
                \item {\color[rgb]{0.2, 0.6, 0.3}{矩阵$C(AB)$的行向量$\alpha_1, \alpha_2, \dots, \alpha_n$可由$B$的行向量$\beta_1, \beta_2, \dots, \beta_n$线性表出}}
                \begin{itemize}
                    \item 对 $B$,$C$ 按列分块,有
                    \[
                        \begin{bmatrix}
                            a_{11} & a_{12} & \dots & a_{1n} \\
                            a_{21} & a_{22} & \dots & a_{2n} \\
                            \vdots & \vdots &       & \vdots \\
                            a_{n1} & a_{n2} & \dots & a_{nn} \\
                        \end{bmatrix}
                        \begin{bmatrix}
                            \beta_1 \\ \beta_2 \\ \vdots \\ \beta_n
                        \end{bmatrix}
                        =
                        \begin{bmatrix}
                            \alpha_1 \\ \alpha_2 \\ \vdots \\ \alpha_n
                        \end{bmatrix}
                    \]
                    即
                    \[
                        \begin{cases}
                            a_{11}\beta_1 + \dots + a_{1n}\beta_n &= \alpha_1, \\
                            a_{21}\beta_1 + \dots + a_{2n}\beta_n &= \alpha_2, \\
                            \vdots & \\
                            a_{n1}\beta_1 + \dots + a_{nn}\beta_n &= \alpha_n
                        \end{cases}
                    \]
                \end{itemize}
                \item {\color[rgb]{0.2, 0.6, 0.3}{矩阵$C(AB)$的列向量可由$A$的列向量线性表出}}
            \end{enumerate}
        \end{enumerate}
        \item 若$A$可逆,则$r(AB) = r(B) = r(BA)$
        \item 若$A$列满秩,则$r(AB) = r(B)$
        \item 若$A$行满秩,则$r(AB) = r(A)$
        \item $A$是$m \times n$矩阵,$B$是$n \times s$矩阵,$C$是$s \times t$矩阵,则
        \[
            r(AB) + r(BC) \le r(ABC) + r(B)
        \]
        \item
        \[
            r\!\begin{bmatrix}
                   A & O \\
                   O & B
            \end{bmatrix}
            = r\!\begin{bmatrix}
                     O & A \\
                     B & O
            \end{bmatrix}
            = r(A) + r(B)
        \]
        \item
        \[
            r\!\begin{bmatrix}
                   A & O \\
                   C & B
            \end{bmatrix}
            \ge r(A) + r(B)
        \]
        \item 若$A \sim B$,则
        \begin{enumerate}
            \item $r(A) = r(B)$
            \item $r(A + kE) = r(B + kE)$
        \end{enumerate}
    \end{enumerate}

    \subsubsection{分块矩阵}

    \begin{enumerate}
        \item 若$B$,$C$分别是$m$阶与$n$阶矩阵,则
        \[
            \begin{bmatrix}
                B & O \\
                O & C
            \end{bmatrix}^n
            = \begin{bmatrix}
                  B^n & O   \\
                  O   & C^n
            \end{bmatrix}
        \]
        \item 若$B$,$C$分别是$m$阶与$n$阶\textbf{可逆}矩阵,则
        \begin{enumerate}
            \item
            \[
                \begin{bmatrix}
                    B & O \\
                    O & C
                \end{bmatrix}^{-1}
                = \begin{bmatrix}
                      B^{-1} & O      \\
                      O      & C^{-1}
                \end{bmatrix}
            \]
            \item
            \[
                \begin{bmatrix}
                    O & B \\
                    C & O
                \end{bmatrix}^{-1}
                = \begin{bmatrix}
                      O      & C^{-1} \\
                      B^{-1} & O
                \end{bmatrix}
            \]
        \end{enumerate}
        \item 若$B$,$C$分别是$m$阶与$n$阶\textbf{可逆}矩阵,则
        \begin{enumerate}
            \item
            \[
                \begin{bmatrix}
                    B & Z \\
                    O & C
                \end{bmatrix}^{-1}
                = \begin{bmatrix}
                      B^{-1} & \color[rgb]{0.2, 0.6, 0.3}{-B^{-1}ZC^{-1}} \\
                      O      & C^{-1}
                \end{bmatrix}
            \]
            \item
            \[
                \begin{bmatrix}
                    B & O \\
                    Z & C
                \end{bmatrix}^{-1}
                = \begin{bmatrix}
                      B^{-1}                                     & O      \\
                      \color[rgb]{0.2, 0.6, 0.3}{-C^{-1}ZB^{-1}} & C^{-1}
                \end{bmatrix}
            \]
            \item
            \[
                \begin{bmatrix}
                    Z & B \\
                    C & O
                \end{bmatrix}^{-1}
                = \begin{bmatrix}
                      O      & C^{-1}                                     \\
                      B^{-1} & \color[rgb]{0.2, 0.6, 0.3}{-B^{-1}ZC^{-1}}
                \end{bmatrix}
            \]
            \item
            \[
                \begin{bmatrix}
                    O & B \\
                    C & Z
                \end{bmatrix}^{-1}
                = \begin{bmatrix}
                      \color[rgb]{0.2, 0.6, 0.3}{-C^{-1}ZB^{-1}} & C^{-1} \\
                      B^{-1}                                     & O
                \end{bmatrix}
            \]
        \end{enumerate}
        \item 若$B$,$C$分别是$m$阶与$n$阶\textbf{可逆}矩阵,则
        \begin{enumerate}
            \item
            \[
                \begin{bmatrix}
                    B & O \\
                    O & C
                \end{bmatrix}^{*}
                = \begin{bmatrix}
                      |C|B^{*} & O        \\
                      O        & |B|C^{*}
                \end{bmatrix}
            \]
            \item
            \[
                \begin{bmatrix}
                    O & B \\
                    C & O
                \end{bmatrix}^{*}
                = (-1)^{mn}\begin{bmatrix}
                               O        & |B|C^{*} \\
                               |C|B^{*} & O
                \end{bmatrix}
            \]
            \item
            \[
                \begin{bmatrix}
                    B & Z \\
                    O & C
                \end{bmatrix}^{*}
                = \begin{bmatrix}
                      |C|B^{*} & \color[rgb]{0.2, 0.6, 0.3}{-B^{*}ZC^*} \\
                      O        & |B|C^{*}
                \end{bmatrix}
            \]
            \item
            \[
                \begin{bmatrix}
                    B & O \\
                    Z & C
                \end{bmatrix}^{*}
                = \begin{bmatrix}
                      |C|B^{*}                               & O        \\
                      \color[rgb]{0.2, 0.6, 0.3}{-C^{*}ZB^*} & |B|C^{*}
                \end{bmatrix}
            \]
            \item
            \[
                \begin{bmatrix}
                    Z & B \\
                    C & O
                \end{bmatrix}^{*}
                = (-1)^{mn}\begin{bmatrix}
                               O        & |B|C^{*}                               \\
                               |C|B^{*} & \color[rgb]{0.2, 0.6, 0.3}{-B^{*}ZC^*}
                \end{bmatrix}
            \]
            \item
            \[
                \begin{bmatrix}
                    O & B \\
                    C & Z
                \end{bmatrix}^{*}
                = (-1)^{mn}\begin{bmatrix}
                               \color[rgb]{0.2, 0.6, 0.3}{-C^{*}ZB^*} & |B|C^{*} \\
                               |C|B^{*}                               & O
                \end{bmatrix}
            \]
        \end{enumerate}
    \end{enumerate}

    \subsubsection{对角矩阵}
    \begin{enumerate}
        \item $\Lambda_1\Lambda_2 = \Lambda_2\Lambda_1$
        \item
        \[
            \begin{bmatrix}
                b_1 &     &     \\
                & b_2 &     \\
                &     & b_3
            \end{bmatrix}
            \begin{bmatrix}
                a_1 &     &     \\
                & a_2 &     \\
                &     & a_3
            \end{bmatrix}
            =
            \begin{bmatrix}
                b_1 a_1 &         &         \\
                & b_2 a_2 &         \\
                &         & b_3 a_3
            \end{bmatrix}
        \]
        \item
        \[
            \begin{bmatrix}
                a_1 &     &     \\
                & a_2 &     \\
                &     & a_3
            \end{bmatrix}^n
            = \begin{bmatrix}
                  a_{1}^n &         &         \\
                  & a_{2}^n &         \\
                  &         & a_{3}^n
            \end{bmatrix}
        \]
        \item
        \[
            \begin{bmatrix}
                a_1 &     &     \\
                & a_2 &     \\
                &     & a_3
            \end{bmatrix}^{-1}
            = \begin{bmatrix}
                  \frac{1}{a_{1}} &                 &                 \\
                  & \frac{1}{a_{2}} &                 \\
                  &                 & \frac{1}{a_{3}}
            \end{bmatrix}
        \]
    \end{enumerate}

    \subsubsection{特殊矩阵$n$次方}
    \begin{enumerate}
        \item 若$r(A) = 1$,则
        \begin{enumerate}
            \item $A$可分解为一个列向量与一个行向量的乘积
            \item $A^2 = lA$其中$l = \sum a_{ii} = a_{11} + a_{22} + \dots + a_{nn} $
            \item $A^n = l^{n-1}A$其中$l = \sum a_{ii} = a_{11} + a_{22} + \dots + a_{nn} $
        \end{enumerate}
        \item 设 $A$ 为 $n \times n$ 上三角矩阵,主对角线为 0
        \[
            A =
            \begin{bmatrix}
                0      & a_{12} & a_{13} & \dots  & a_{1n} \\
                0      & 0      & a_{23} & \dots  & a_{2n} \\
                0      & 0      & 0      & \dots  & a_{3n} \\
                \vdots & \vdots & \vdots & \ddots & \vdots \\
                0      & 0      & 0      & \dots  & 0
            \end{bmatrix}.
        \]

        则:
        \[
            A^2 =
            \begin{bmatrix}
                0      & 0      & b_{13} & \dots  & b_{1n} \\
                0      & 0      & 0      & \dots  & b_{2n} \\
                0      & 0      & 0      & \dots  & b_{3n} \\
                \vdots & \vdots & \vdots & \ddots & \vdots \\
                0      & 0      & 0      & \dots  & 0
            \end{bmatrix}, \quad
            A^3 =
            \begin{bmatrix}
                0      & 0      & 0      & c_{14} & \dots  & c_{1n} \\
                0      & 0      & 0      & 0      & \dots  & c_{2n} \\
                0      & 0      & 0      & 0      & \dots  & c_{3n} \\
                \vdots & \vdots & \vdots & \vdots & \ddots & \vdots \\
                0      & 0      & 0      & 0      & \dots  & 0
            \end{bmatrix}
        \]
        \[
            A^n = 0, \quad A^k = 0 \; \text{当} k \ge n
        \]
        \item 若$B = P^{-1}AP$,则$B^2 = P^{-1}A^{2}P$,即
        \begin{enumerate}
            \item $\mathbf{B^n = P^{-1}A^{\color{red}{n}}P}$
            \item $A^n = PB^{n}P^{-1}$
        \end{enumerate}
    \end{enumerate}

    \subsection{方法步骤}

    \begin{enumerate}
        \item 已知矩阵$A$,若下三角可逆矩阵$P$和上三角可逆矩阵$Q$,使得$PAQ$为对角矩阵,求$P, Q$
        \begin{enumerate}
            \item 标准型: 对角矩阵是特征值
            \item \textbf{初等行变换}: 对$A$做初等\textbf{行变换}化为上三角矩阵$B([A | E] -> [B | P])$得到$P$,再对$B$做\textbf{列变换}或{\color{red}{$B^T$作\textbf{行变换}}}化为对角矩阵$\Lambda([B^T | E] -> [\Lambda | Q])$得到$Q$
        \end{enumerate}
        \item 由$A^*$求$A$
        \begin{enumerate}
            \item $|A^*|$
            \item $|A^*| = |A|^{n-1}$ \Rightarrow |A|
            \item $AA^* = |A|E$ \Rightarrow A = |A|(A^*)^{-1}
        \end{enumerate}
        \item 秩求法
        \begin{itemize}
            \item $|A| \neq 0$ \Leftrightarrow r(A) = n
            \item $|A| = 0$ \Leftrightarrow r(A) < n
            \item 初等行变换矩阵秩不变
            \item 找不为$0$的子式 \;\le\; $r(A)$
            \item $A\;B$相似 \Rightarrow r(A) = r(B)
        \end{itemize}
        \item 求特殊矩阵的$n$次方
        \begin{itemize}
            \item 分块
            \item 若$r(A) = 1$,则$A^n = l^{n-1}A$,其中$l = \sum {a_{ii}} = a_{11} + a_{22} + \dots + a_{nn}$
            \item $P^{-1}AP = B$ \Rightarrow A^n = PB^{n}P^{-1}, B^n = P^{-1}A^{n}P
            \item 观察多少次幂之后是$0$,之后的都是$0$
            \item 对角矩阵的$n$次方
        \end{itemize}
        \item 求伴随矩阵$A^*$
        \begin{itemize}
            \item 定义
            \item $A^* = |A|A^{-1}$
        \end{itemize}
        \item 求可逆矩阵
        \begin{itemize}
            \item 求代数余子式$A_{ij} = (-1)^{i+j}M_{ij}$,$(kA)^{-1} = \frac{1}{k}A^{-1}(k \neq 0)$
            \item 用初等行变换
            \[
                (A \; E)
                \;\overset{\text{由上往下}}{\longrightarrow \dots \longrightarrow}\;
                (\text{上三角} ...)
                \;\overset{\text{由下往上}}{\longrightarrow \dots \longrightarrow}\;
                (\text{\\} *)
                \;\overset{\text{某行乘 }k}{\longrightarrow}\;
                \;\longrightarrow\;
                (E \; A^{-1})
            \]
            \item 分块
            \begin{align*}
                \begin{bmatrix}
                    B & O \\
                    O & C
                \end{bmatrix}^{-1}
                &=
                \begin{bmatrix}
                    B^{-1} & O      \\
                    O      & C^{-1}
                \end{bmatrix}, \\[2mm]
%
                \begin{bmatrix}
                    O & B \\
                    C & O
                \end{bmatrix}^{-1}
                &=
                \begin{bmatrix}
                    O      & C^{-1} \\
                    B^{-1} & O
                \end{bmatrix}.
            \end{align*}
        \end{itemize}
    \end{enumerate}

    \subsection{条件转换思路}

    \begin{enumerate}
        \item 设 $\mathbf{A}$ 是 $m \times n$ 矩阵,$\mathbf{B}$ 是 $n \times s$ 矩阵,若 $\mathbf{AB} = \mathbf{O}$,则
        \begin{enumerate}
            \item $\mathbf{B}$ 的列向量是齐次方程组 $\mathbf{Ax} = 0$ 的解
            \item $r(\mathbf{A}) + r(\mathbf{B}) \le n$
            \item 若 $\mathbf{A}$ 和 $\mathbf{B}$ 为方阵,则$|A| = 0$或$|B| = 0$
            \item 且$A\;B$非零,则
            \begin{align*}
                &\Rightarrow\; r(A) < n\text{且}r(B) < n \\
                &\Rightarrow\; A\text{列向量线性相关} \\
                &\Rightarrow\; B\text{行向量线性相关} \\
            \end{align*}
        \end{enumerate}
        \item 若 $a_{ij} + A_{ij} = 0$ 则:
        \[
            \begin{aligned}
                A_{ij} &= -a_{ij} \\
                A^* = (A_{ij})^T &= (-a_{ij})^T = -(a_{ij})^T = -A^T
            \end{aligned}
        \]
        \item 若 $A^* = A^T$,则$A_{ij} = a_{ij}$
        \item 矩阵$A$经过若干次初等\textbf{行变换}得到矩阵$B$,则
        \begin{enumerate}
            \item $Ax = 0$与$Bx = 0$同解
            \item $A \overset{\sim}{=} B, B = PA$
            \item $r(A) = r(B)$
        \end{enumerate}
        \item 矩阵$A$经过若干次初等\textbf{列变换}得到矩阵$B$,则
        \begin{enumerate}
            \item $A \overset{\sim}{=} B, B = AQ$
            \item $r(A) = r(B)$
        \end{enumerate}
        \item $A^* \neq 0 \Rightarrow r(A) \ge n - 1(A\text{中至少有一个}n-1\text{阶子式不为0})$
        \item 若$A, B, C$为$n$阶矩阵,且$ABC = E$,则
        \begin{align*}
            &\Rightarrow\; |A||B||C| = 1 \\
            &\Rightarrow\; A B C\text{均可逆} \\
            &\Rightarrow\; BC = A^{-1} \Rightarrow BCA = E \\
            &\Rightarrow\; AB = C^{-1} \Rightarrow CAB = E
        \end{align*}
        \item $r(A + AB) \Rightarrow \text{加法,找可逆,若}A\text{可逆,则}r(AB) = r(B)$
        \item 设 $A$ 为 $m \times n$ 矩阵,$r(A)$ 为秩
        \begin{itemize}
            \item \textbf{基本定义:} 秩 = 列向量或行向量的最大线性无关个数
            \item \textbf{行列式:} 方阵 $A$ 满秩 $\Leftrightarrow |A| \neq 0 \Leftrightarrow A$ 可逆
            \item \textbf{线性相关性:}
            \begin{itemize}
                \item 列满秩 $\Rightarrow$ 列向量线性无关
                \item 行满秩 $\Rightarrow$ 行向量线性无关
            \end{itemize}
            \item \textbf{齐次方程:} $Ax = 0$
            \begin{itemize}
                \item 唯一解 $\Leftrightarrow r(A) = n$
                \item 非零解 $\Leftrightarrow r(A) < n$, 解空间维数 $n - r(A)$
            \end{itemize}
            \item \textbf{矩阵运算:}
            \begin{itemize}
                \item $r(AB) \le \min(r(A), r(B))$
                \item $r(A+B) \le r(A) + r(B)$
            \end{itemize}
            \item \textbf{逆矩阵:} 列满秩 → 左逆,行满秩 → 右逆;方阵满秩 → 可逆
            \item \textbf{特征值:} 方阵秩 < n $\Rightarrow$ 0 是特征值;方阵满秩 $\Rightarrow$ 0 不是特征值
        \end{itemize}
        \item $A$为$n$阶矩阵,$A$各行元素之和都为$0$,则
        \begin{itemize}
            \item 列向量都是$1$是$Ax = 0$的解
            \item $A$的行向量线性相关
            \item $r(A) < n$
        \end{itemize}
    \end{enumerate}

    \subsection{理解}

    \begin{enumerate}
        \item 设$A, B$为$n$阶矩阵,记$(X Y)$表示分块矩阵,则$r(A, AB) == r(A)$
        \begin{enumerate}
            \item 记$AB = C$,对$A, C$ 按列分块有
            \[
                [\alpha_1, \alpha_2, \dots, \alpha_n]
                \begin{bmatrix}
                    b_{11} & b_{12} & \dots & b_{1n} \\
                    b_{21} & b_{22} & \dots & b_{2n} \\
                    \vdots & \vdots &       & \vdots \\
                    b_{n1} & b_{n2} & \dots & b_{nn} \\
                \end{bmatrix}
                = [\beta_1, \beta_2, \dots, \beta_n]
            \]
            即$\beta_1, \beta_2, \dots, \beta_n$可由$\alpha_1, \alpha_2, \dots, \alpha_n$线性表出,矩阵的秩就是列向量组的秩,故
            \[
                r(A, AB) = r(\alpha_1, \alpha_2, \dots, \alpha_n, \beta_1, \beta_2, \dots, \beta_n) = r(\alpha_1, \alpha_2, \dots, \alpha_n) = r(A)
            \]
        \end{enumerate}
        \item $A$是$m \times n$矩阵,$B$是$n \times m$矩阵,$E$为$m$阶单位矩阵,若$AB = E$,则{\color[rgb]{0.2, 0.6, 0.3}{$r(A) = m, r(B) = m$}}
        \begin{analysisbox}
            已知 $AB = E_m$,则
            \[
                \because AB = E_m \Rightarrow r(AB) = r(E_m) = m,
            \]
            又因为
            \[
                r(AB) \le \min\{r(A), r(B)\}, \quad r(A)\le m, \ r(B)\le m,
            \]
            \[
                \therefore r(A)=m,\quad r(B)=m.
            \]
        \end{analysisbox}
        \item 矩阵相乘 \Leftrightarrow \text{两个数的{\color{red}{积}}相{\color{red}{加}}}
        \item 秩
        \begin{enumerate}
            \item 给定一个矩阵$A$,它的秩就是矩阵中\textbf{线性无关的行(或列)的最大个数}
            \item 秩 == 矩阵所包含的“独立信息量”
            \item Case: 如果你有10行数据,但其中5行其实是由另外5行“复制”或“线性组合”出来的,那么这些重复的信息是“冗余的”,真正“独立”的信息只有5行 \Rightarrow r(A) = 5
            \item 解线性方程组: 判断方程有没有解、是不是唯一解
            \begin{enumerate}
                \item 方程 $Ax = b$有解 \Leftrightarrow r(A) = r(A | b) = r(\bar{A})
                \item 唯一解 \Leftrightarrow r(A) = r(\bar{A}) = \text{变量数}
                \item 多解 \Leftrightarrow r(A) = r(\bar{A}) < \text{变量数}
            \end{enumerate}
            \item 判断向量独立性: 度量“向量空间中有多少个独立方向”
            \begin{enumerate}
                \item 如果列向量组成的矩阵$r(A) =$ 列数 \Rightarrow \text{向量组线性无关}
                \item 否则线性相关
            \end{enumerate}
            \item 维度的桥梁: 刻画了“变换的本质效果”
            \begin{enumerate}
                \item 秩本质上就是矩阵对应线性映射的像空间(列空间)的维数
                \item 这告诉我们,线性变换把空间压缩到了几维。
                \item Case: $3 \times 3$矩阵$A$
                \begin{itemize}
                    \item $r(A) = 3$ \Rightarrow \text{保留三维空间的全部信息(可能只是旋转或缩放)}
                    \item $r(A) = 2$ \Rightarrow \text{把三维空间压缩成二维平面}
                    \item $r(A) = 1$ \Rightarrow \text{压缩成一条直线}
                    \item $r(A) = 0$ \Rightarrow \text{全部压缩成原点}
                \end{itemize}
            \end{enumerate}
            \item 与行列式、可逆性关系: 可逆性的根本判据
            \begin{enumerate}
                \item 如果$n$阶矩阵$A$,$r(A) = n$,那么$A$可逆,$|A| \neq 0$
                \item 如果$r(A) < n$,矩阵$A$不可逆
            \end{enumerate}
        \end{enumerate}
    \end{enumerate}

\end{document}
