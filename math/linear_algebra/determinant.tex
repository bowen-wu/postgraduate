\documentclass[a4paper,12pt]{article}
\usepackage{xeCJK}          % 中文支持
\usepackage{fontspec}       % 英文/数学字体
\usepackage{amsmath, amssymb} % 数学公式
\usepackage{graphicx}       % 插入图片
\usepackage{hyperref}       % 目录超链接
\usepackage{geometry}       % 页面布局
\usepackage{bm}             % 粗体
\usepackage{xcolor}         % 颜色
\geometry{left=3cm,right=3cm,top=3cm,bottom=3cm}

% =========================
% 字体设置
% =========================
\setmainfont{Times New Roman}
\setsansfont{Helvetica Neue}
\setmonofont{Menlo}
\setCJKmainfont{PingFang SC}

% =========================
% 图形路径(可调整)
% =========================
\graphicspath{{./assets/}}

% =========================
% 文档开始
% =========================
\begin{document}

%    \title{线性代数}
%    \author{Bowen}
%    \date{\today}
%    \maketitle

% =========================


    \section{行列式}

    \subsection{性质}

    \begin{enumerate}
        \item 经转置行列式的值不变,即 $|A^T| = |A|$
        \item 某行元素全为 $0 \;\Rightarrow\;$ 行列式的值为 0
        \item 两行相等 $\Rightarrow$ 行列式的值为 0
        \item 两行成比例 $\Rightarrow$ 行列式的值为 0
        \item 某行(列)有公因数 $k$,可把 $k$ 提到行列式外
        \item 两行互换,行列式变号
        \item \textbf{某行} 所有元素都是两个数的和,则可写成两个行列式之和
        \item 某行的 $k$ 倍加至另一行,行列式的值不变
    \end{enumerate}

    \subsection{基础}

    \subsubsection{完全展开式}

    \begin{align*}
        \begin{vmatrix}
            a_{11} & a_{12} & \dots & a_{1n} \\
            a_{21} & a_{22} & \dots & a_{2n} \\
            \vdots & \vdots &       & \vdots \\
            a_{n1} & a_{n2} & \dots & a_{nn}
        \end{vmatrix}
        &= \sum_{j_1\,j_2\dots j_{n}} (-1)^{\tau(j_1\,j_2\dots j_n)} \mathbf{a_{1j_1}a_{2j_2}\dots a_{nj_n}} \\
        &= \sum_{\sigma \in S_n} (-1)^{\tau(\sigma)} \prod_{i=1}^{n} a_{i, \sigma(i)}
    \end{align*}

    \subsubsection{余子式 \& 代数余子式}

    在 $n$ 阶行列式中,划去元素 $a_{ij}$ 所在的第 $i$ 行、第 $j$ 列,由剩下的元素按原来的排法构成一个 $(n-1)$ 阶行列式,称为 $a_{ij}$ 的 \textbf{余子式},记为$M_{ij}$;称$(-1)^{i+j}M_{ij}$为$a_{ij}$的代数余子式,记为$A_{ij}$,即

    \[
        A_{ij} = (-1)^{i+j}M_{ij}
    \]

    \subsection{定理}

    \subsubsection{展开公式}

    \begin{enumerate}
        \item $n$阶行列式等于它的任意一行(列)的所有元素与他们各自对应的代数余子式的乘积之和,即
        \begin{align*}
            |A| &= a_{k1}A_{k1} + a_{k2}A_{k2} + \dots + a_{kn}A_{kn} (k=1,2,\dots,n)  & \text{行} \\
            &= a_{1k}A_{1k} + a_{2k}A_{2k} + \dots + a_{nk}A_{nk} (k=1,2,\dots,n)  & \text{列}
        \end{align*}
        \item 任意一行(列)的所有元素与其他行的代数余子式乘积之和为0,即
        \begin{align*}
            a_{i1}A_{k1} + a_{i2}A_{k2} + \dots + a_{in}A_{kn} = 0 &= 0  \quad  (i \neq k \text{且} i,k=1,2,\dots,n)   \\
            a_{1j}A_{1k} + a_{2j}A_{2k} + \dots + a_{nj}A_{nk} = 0 &= 0  \quad  (j \neq k \text{且} j,k=1,2,\dots,n)
        \end{align*}
    \end{enumerate}

    \subsubsection{乘法公式}
    设$\mathbf{A}$,$\mathbf{B}$都是$n$阶方阵,则
    \[
        |\mathbf{AB}| = |\mathbf{A}| \cdot |\mathbf{B}|
    \]

    \subsection{公式}

    \subsubsection{上(下)三角形}

    \begin{enumerate}
        \item 主对角线三角形
        \begin{align*}
            \begin{vmatrix}
                a_{11} & a_{12} & \dots  & a_{1n} \\
                & a_{22} & \dots  & a_{2n} \\
                &        & \ddots & \vdots \\
                &        &        & a_{nn}
            \end{vmatrix}
            = \begin{vmatrix}
                  a_{11} &        &        &        \\
                  a_{21} & a_{22} &        &        \\
                  \vdots & \vdots & \ddots &        \\
                  a_{n1} & a_{n2} & \dots  & a_{nn}
            \end{vmatrix}
            = a_{11}a_{22}\dots a_{nn}
        \end{align*}
        \item 副对角线三角形
        \begin{align*}
            \begin{vmatrix}
                a_{11} & a_{12} & \dots & a_{1,n-1} & a_{1n} \\
                a_{21} & a_{22} & \dots & a_{2,n-2} & 0      \\
                \vdots & \vdots &       & \vdots    & \vdots \\
                a_{n1} & 0      & \dots & 0         & 0
            \end{vmatrix}
            = \begin{vmatrix}
                  0      & \dots & 0         & a_{1n} \\
                  0      & \dots & a_{2,n-1} & a_{2n} \\
                  \vdots &       & \vdots    & \vdots \\
                  a_{n1} & \dots & a_{n,n-1} & a_{nn}
            \end{vmatrix}
            = (-1)^{\frac{n(n-1)}{2}}a_{1n}a_{2,n-2}\dots a_{n1}
        \end{align*}
    \end{enumerate}

    \subsubsection{拉普拉斯展开式}

    \begin{enumerate}
        \item 主对角线
        \[
            \renewcommand{\arraystretch}{1} % 缩小行距,避免竖线过高
            \setlength{\arraycolsep}{3pt}     % 调整列距
            \begin{vmatrix}
                \mathbf{A} & *          \\
                \mathbf{O} & \mathbf{B}
            \end{vmatrix}
            = \begin{vmatrix}
                  \mathbf{A} & \mathbf{O} \\
                  *          & \mathbf{B}
            \end{vmatrix}
            = |\mathbf{A}| \cdot |\mathbf{B}|
        \]
        \item 副对角线
        \[
            \renewcommand{\arraystretch}{1} % 缩小行距,避免竖线过高
            \setlength{\arraycolsep}{3pt}     % 调整列距
            \begin{vmatrix}
                *          & \mathbf{A} \\
                \mathbf{B} & \mathbf{O}
            \end{vmatrix}
            = \begin{vmatrix}
                  \mathbf{O} & \mathbf{A} \\
                  \mathbf{B} & *
            \end{vmatrix}
            = (-1)^{mn}|\mathbf{A}| \cdot |\mathbf{B}|
        \]
        \text{$m$,$n$分别是方阵$A$,$B$的阶数}
    \end{enumerate}

    \subsubsection{范德蒙行列式}

    \[
        \begin{vmatrix}
            1           & 1           & \dots & 1           \\
            x_1         & x_2         & \dots & x_n \\
            {x_1}^{2}   & {x_2}^2     & \dots & {x_n}^2     \\
            \vdots      & \vdots      &       & \vdots \\
            {x_1}^{n-1} & {x_2}^{n-1} & \dots & {x_n}^{n-1} \\
        \end{vmatrix}
        = \prod_{1 \le j < i \le n} (x_{i} - x_{j})
    \]

    \subsubsection{特征多项式}

    \subsection{方阵行列式}

    \begin{enumerate}
        \item 若$A$是$n$阶矩阵,$A^T$是$A$的转置矩阵 \, \Rightarrow \, $|A^T| = |A|$
        \item 若$A$是$n$阶矩阵 \, \Rightarrow \, $|kA| = k^{n}|A|$
        \item 若$A$,$B$都是$n$阶矩阵 \, \Rightarrow \, $|AB| = |A||B|$,$|A^2| = |A|^2$
        \item 若$A$是$n$阶矩阵 \, \Rightarrow \, $|A^*| = |A|^{n-1}$
        \item 若$A$是$n$阶\textbf{可逆}矩阵 \, \Rightarrow \, $|A^{-1}| = |A|^{-1}$
    \end{enumerate}

    \subsection{克拉默法则}

    设有 $n$ 元线性方程组:
    \[
        \begin{cases}
            a_{11}x_1 + a_{12}x_2 + \dots + a_{1n}x_n = b_1 \\
            a_{21}x_1 + a_{22}x_2 + \dots + a_{2n}x_n = b_2 \\
            \quad \vdots \\
            a_{n1}x_1 + a_{n2}x_2 + \dots + a_{nn}x_n = b_n
        \end{cases}
    \]

    记系数矩阵为 $\mathbf{A} = (a_{ij})_{n\times n}$,则其行列式为 $|\mathbf{A}|$。若 $|\mathbf{A}| \neq 0$,则方程组有唯一解,并且第 $i$ 个未知数 $x_i$ 可由下式求得:

    \[
        x_i = \frac{|\mathbf{A}_i|}{|\mathbf{A}|}, \quad i=1,2,\dots,n
    \]

    其中 $\mathbf{A}_i$ 是将 $\mathbf{A}$ 的第 $i$ 列替换为常数列向量 $\mathbf{b} = (b_1, b_2, \dots, b_n)^T$ 后得到的矩阵,即:

    \[
        \mathbf{A}_i =
        \begin{pmatrix}
            a_{11} & \dots & b_1    & \dots & a_{1n} \\
            a_{21} & \dots & b_2    & \dots & a_{2n} \\
            \vdots &       & \vdots &       & \vdots \\
            a_{n1} & \dots & b_n    & \dots & a_{nn}
        \end{pmatrix}
    \]

    \paragraph{推论} 若齐次线性方程组:
    \[
        \begin{cases}
            a_{11}x_1 + a_{12}x_2 + \dots + a_{1n}x_n = 0 \\
            a_{21}x_1 + a_{22}x_2 + \dots + a_{2n}x_n = 0 \\
            \quad \vdots \\
            a_{n1}x_1 + a_{n2}x_2 + \dots + a_{nn}x_n = 0
        \end{cases}
    \]

    \begin{enumerate}
        \item 系数行列式$|A| \neq 0$ $\Leftrightarrow$ 方程组\textbf{只有零解}
        \item 系数行列式$|A| = 0$ $\Leftrightarrow$ 方程组\textbf{有非零解}
    \end{enumerate}


    \section{矩阵}

    \subsection{条件转换思路}

    \begin{enumerate}
        \item 设 $\mathbf{A}$ 是 $m \times n$ 矩阵,$\mathbf{B}$ 是 $n \times s$ 矩阵,若 $\mathbf{AB} = \mathbf{O}$,则
        \begin{enumerate}
            \item $\mathbf{B}$ 的列向量是其次方程组 $\mathbf{Ax} = 0$ 的解
            \item $r(\mathbf{A}) + r(\mathbf{B}) \le n$
        \end{enumerate}
    \end{enumerate}


\end{document}
