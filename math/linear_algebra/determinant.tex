\documentclass[a4paper,12pt]{article}
\usepackage{xeCJK}          % 中文支持
\usepackage{fontspec}       % 英文/数学字体
\usepackage{amsmath, amssymb} % 数学公式
\usepackage{graphicx}       % 插入图片
\usepackage{hyperref}       % 目录超链接
\usepackage{geometry}       % 页面布局
\usepackage{bm}             % 粗体
\usepackage{xcolor}         % 颜色
\usepackage{tabularx}       %表格环境
\usepackage{tikz}           % TikZ 绘制主对角线斜线
\usepackage{tcolorbox}
\geometry{left=3cm,right=3cm,top=3cm,bottom=3cm}

% 抽离颜色和尺寸参数
\newcommand{\analysisTitleColor}{green!50!black}
\newcommand{\analysisBackColor}{white}
\newcommand{\analysisBoxRule}{0.8pt}
\newcommand{\analysisArc}{3pt}
\newcommand{\analysisPadding}{6pt}

% 定义 tcolorbox
\newtcolorbox{analysisbox}{
    title=解析,
    colback=\analysisBackColor,
    colframe=\analysisTitleColor,
    boxrule=\analysisBoxRule,
    arc=\analysisArc,
    left=\analysisPadding,
    right=\analysisPadding,
    top=4pt,
    bottom=4pt
}

% =========================
% 字体设置
% =========================
\setmainfont{Times New Roman}
\setsansfont{Helvetica Neue}
\setmonofont{Menlo}
\setCJKmainfont{PingFang SC}

% =========================
% 图形路径(可调整)
% =========================
\graphicspath{{./assets/}}

% =========================
% 文档开始
% =========================
\begin{document}

%    \title{线性代数}
%    \author{Bowen}
%    \date{\today}
%    \maketitle

% =========================


    \section{行列式}

    \subsection{基础概念}
    \begin{enumerate}
        \item 由$1, 2, \dots, n$组成的有序数组程伟一个$n$阶排列,通常用$j_1, j_2, \dots, j_n$表示$n$阶排列
        \item 一个排列中,如果一个大的数排在小的数之前,就称这两个数构成\textbf{一个逆序},\textbf{一个排列的逆序总数称为这个排列的逆序数},用$\tau(j_1, j_2, \dots, j_n)$表示
        \item 如果一个排列的逆序数是偶数,则称这个排列为\textbf{偶排列},否则称为\textbf{奇排列}
    \end{enumerate}

    \subsection{性质}

    \begin{enumerate}
        \item 经转置行列式的值不变,即 $|A^T| = |A|$
        \item 某行元素全为 $0 \;\Rightarrow\;$ 行列式的值为 0
        \item 两行相等 $\Rightarrow$ 行列式的值为 0
        \item 两行成比例 $\Rightarrow$ 行列式的值为 0
        \item 某行(列)有公因数 $k$,可把 $k$ 提到行列式外
        \item 两行互换,行列式变号
        \item {\color{red}{\textbf{某行}}}所有元素都是两个数的和,则可写成两个行列式之和
        \item 某行的 $k$ 倍加至另一行,行列式的值不变
    \end{enumerate}

    \subsection{基础}

    \subsubsection{完全展开式}

    \begin{align*}
        \begin{vmatrix}
            a_{11} & a_{12} & \dots & a_{1n} \\
            a_{21} & a_{22} & \dots & a_{2n} \\
            \vdots & \vdots &       & \vdots \\
            a_{n1} & a_{n2} & \dots & a_{nn}
        \end{vmatrix}
        &= \sum_{j_1\,j_2\dots j_{n}} (-1)^{\tau(j_1\,j_2\dots j_n)} \mathbf{a_{1j_1}a_{2j_2}\dots a_{nj_n}} \\
        &= \sum_{\sigma \in S_n} (-1)^{\tau(\sigma)} \prod_{i=1}^{n} a_{i, \sigma(i)}
    \end{align*}

    \subsubsection{余子式 \& 代数余子式}

    \begin{enumerate}
        \item 在 $n$ 阶行列式中,划去元素 $a_{ij}$ 所在的第 $i$ 行、第 $j$ 列,由剩下的元素按原来的排法构成一个 $(n-1)$ 阶行列式,称为 $a_{ij}$ 的 \textbf{余子式},记为$\mathbf{M_{ij}}$;称$(-1)^{i+j}M_{ij}$为$a_{ij}$的\textbf{代数余子式},记为$\mathbf{A_{ij}}$,即
        \[
            A_{ij} = (-1)^{i+j}M_{ij}
        \]
        \item 三阶行列式的代数余子式
        \[
            A_{ij} =
            \begin{bmatrix}
                M_{11}  & -M_{12} & M_{13}  \\
                -M_{21} & M_{22}  & -M_{23} \\
                M_{31}  & -M_{32} & M_{33}
            \end{bmatrix}.
        \]
    \end{enumerate}

    \subsection{定理}

    \subsubsection{展开公式}

    \begin{enumerate}
        \item $n$阶行列式等于它的任意一行(列)的所有元素与他们各自对应的代数余子式的乘积之和,即
        \begin{align*}
            |A| &= a_{k1}A_{k1} + a_{k2}A_{k2} + \dots + a_{kn}A_{kn} (k=1,2,\dots,n)  & \text{行} \\
            &= a_{1k}A_{1k} + a_{2k}A_{2k} + \dots + a_{nk}A_{nk} (k=1,2,\dots,n)  & \text{列}
        \end{align*}
        \item 任意一行(列)的所有元素与其他行的代数余子式乘积之和为0,即
        \begin{align*}
            a_{i1}A_{k1} + a_{i2}A_{k2} + \dots + a_{in}A_{kn} = 0 &= 0  \quad  (i \neq k \text{且} i,k=1,2,\dots,n)   \\
            a_{1j}A_{1k} + a_{2j}A_{2k} + \dots + a_{nj}A_{nk} = 0 &= 0  \quad  (j \neq k \text{且} j,k=1,2,\dots,n)
        \end{align*}
    \end{enumerate}

    \subsubsection{乘法公式}
    设$\mathbf{A}$,$\mathbf{B}$都是$n$阶方阵,则
    \[
        |\mathbf{AB}| = |\mathbf{A}| \cdot |\mathbf{B}|
    \]

    \subsection{公式}

    \subsubsection{上(下)三角形}

    \begin{enumerate}
        \item 主对角线三角形
        \begin{align*}
            \begin{vmatrix}
                a_{11} & a_{12} & \dots  & a_{1n} \\
                & a_{22} & \dots  & a_{2n} \\
                &        & \ddots & \vdots \\
                &        &        & a_{nn}
            \end{vmatrix}
            = \begin{vmatrix}
                  a_{11} &        &        &        \\
                  a_{21} & a_{22} &        &        \\
                  \vdots & \vdots & \ddots &        \\
                  a_{n1} & a_{n2} & \dots  & a_{nn}
            \end{vmatrix}
            = a_{11}a_{22}\dots a_{nn}
        \end{align*}
        \item 副对角线三角形
        \begin{align*}
            \begin{vmatrix}
                a_{11} & a_{12} & \dots & a_{1,n-1} & a_{1n} \\
                a_{21} & a_{22} & \dots & a_{2,n-2} & 0      \\
                \vdots & \vdots &       & \vdots    & \vdots \\
                a_{n1} & 0      & \dots & 0         & 0
            \end{vmatrix}
            = \begin{vmatrix}
                  0      & \dots & 0         & a_{1n} \\
                  0      & \dots & a_{2,n-1} & a_{2n} \\
                  \vdots &       & \vdots    & \vdots \\
                  a_{n1} & \dots & a_{n,n-1} & a_{nn}
            \end{vmatrix}
            = (-1)^{\frac{n(n-1)}{2}}a_{1n}a_{2,n-2}\dots a_{n1}
        \end{align*}
    \end{enumerate}

    \subsubsection{拉普拉斯展开式}

    \begin{enumerate}
        \item 主对角线
        \[
            \renewcommand{\arraystretch}{1} % 缩小行距,避免竖线过高
            \setlength{\arraycolsep}{3pt}     % 调整列距
            \begin{vmatrix}
                \mathbf{A} & *          \\
                \mathbf{O} & \mathbf{B}
            \end{vmatrix}
            = \begin{vmatrix}
                  \mathbf{A} & \mathbf{O} \\
                  *          & \mathbf{B}
            \end{vmatrix}
            = |\mathbf{A}| \cdot |\mathbf{B}|
        \]
        \item 副对角线
        \[
            \renewcommand{\arraystretch}{1} % 缩小行距,避免竖线过高
            \setlength{\arraycolsep}{3pt}     % 调整列距
            \begin{vmatrix}
                *          & \mathbf{A} \\
                \mathbf{B} & \mathbf{O}
            \end{vmatrix}
            = \begin{vmatrix}
                  \mathbf{O} & \mathbf{A} \\
                  \mathbf{B} & *
            \end{vmatrix}
            = (-1)^{mn}|\mathbf{A}| \cdot |\mathbf{B}|
        \]
        \text{$m$,$n$分别是方阵$A$,$B$的阶数}
    \end{enumerate}

    \subsubsection{范德蒙行列式}

    \[
        \begin{vmatrix}
            1           & 1           & \dots & 1           \\
            x_1         & x_2         & \dots & x_n \\
            {x_1}^{2}   & {x_2}^2     & \dots & {x_n}^2     \\
            \vdots      & \vdots      &       & \vdots \\
            {x_1}^{n-1} & {x_2}^{n-1} & \dots & {x_n}^{n-1} \\
        \end{vmatrix}
        = \prod_{1 \le j < i \le n} (x_{i} - x_{j})
    \]

    \subsubsection{特征多项式}
    \begin{enumerate}
        \item 设$A = [a_{ij}]$是三阶矩阵,则$A$的特征多项式
        \[
            |A - \lambdaE| = -\lambda^3 + (a_{11} + a_{22} + a_{33})\lambda^2 - s_2 \lambda + |A|
        \]其中$s_2 =
        \begin{vmatrix}
            a_{11} & a_{12} \\
            a_{21} & a_{22}
        \end{vmatrix}
        +
        \begin{vmatrix}
            a_{11} & a_{13} \\
            a_{31} & a_{33}
        \end{vmatrix}
        +
        \begin{vmatrix}
            a_{22} & a_{23} \\
            a_{32} & a_{33}
        \end{vmatrix}.$
    \end{enumerate}

    \subsection{方阵行列式}

    \begin{enumerate}
        \item 若$A$是$n$阶矩阵,$A^T$是$A$的转置矩阵 \, \Rightarrow \, $|A^T| = |A|$
        \item 若$A$是$n$阶矩阵 \, \Rightarrow \, $|kA| = k^{n}|A|$
        \item 若$A$,$B$都是$n$阶矩阵 \, \Rightarrow \, $|AB| = |A||B|, |A^2| = |A|^2$
        \item 若$A$是$n$阶矩阵 \, \Rightarrow \, $|A^*| = |A|^{n-1}$
        \item 若$A$是$n$阶\textbf{可逆}矩阵 \, \Rightarrow \, $|A^{-1}| = |A|^{-1}$
        \item 若$A$是$n$阶矩阵,$\lambda_i(i = 1, 2, \cdots, n)$是$A$的特征值 \, \Rightarrow \, $|A| = \lambda_1 \lambda_2 \cdots \lambda_n$
        \item 若$n$阶矩阵$A$与$B$相似 \, \Rightarrow \, $|A| = |B|, |A + kE| = |B + kE|$
        \item
        \[
            \begin{vmatrix}
                A & B \\
                B & A
            \end{vmatrix}
            = \begin{vmatrix}
                  A + B & A + B \\
                  B     & A
            \end{vmatrix}
            = \begin{vmatrix}
                  A + B & 0     \\
                  B     & A - B
            \end{vmatrix}
            = |A + B| \cdot |A - B|
        \]
    \end{enumerate}

    \subsection{克拉默法则}

    设有 $n$ 元线性方程组:
    \[
        \begin{cases}
            a_{11}x_1 + a_{12}x_2 + \dots + a_{1n}x_n = b_1 \\
            a_{21}x_1 + a_{22}x_2 + \dots + a_{2n}x_n = b_2 \\
            \quad \vdots \\
            a_{n1}x_1 + a_{n2}x_2 + \dots + a_{nn}x_n = b_n
        \end{cases}
    \]

    记系数矩阵为 $\mathbf{A} = (a_{ij})_{n\times n}$,则其行列式为 $|\mathbf{A}|$。\textbf{若 $|\mathbf{A}| \neq 0$,则方程组有唯一解},并且第 $i$ 个未知数 $x_i$ 可由下式求得:

    \[
        x_i = \frac{|\mathbf{A}_i|}{|\mathbf{A}|}, \quad i=1,2,\dots,n
    \]

    其中 $\mathbf{A}_i$ 是将 $\mathbf{A}$ 的第 $i$ 列替换为常数列向量 $\mathbf{b} = (b_1, b_2, \dots, b_n)^T$ 后得到的矩阵,即:

    \[
        \mathbf{A}_i =
        \begin{pmatrix}
            a_{11} & \dots & b_1    & \dots & a_{1n} \\
            a_{21} & \dots & b_2    & \dots & a_{2n} \\
            \vdots &       & \vdots &       & \vdots \\
            a_{n1} & \dots & b_n    & \dots & a_{nn}
        \end{pmatrix}
    \]

    \paragraph{推论} 若齐次线性方程组:
    \[
        \begin{cases}
            a_{11}x_1 + a_{12}x_2 + \dots + a_{1n}x_n = 0 \\
            a_{21}x_1 + a_{22}x_2 + \dots + a_{2n}x_n = 0 \\
            \quad \vdots \\
            a_{n1}x_1 + a_{n2}x_2 + \dots + a_{nn}x_n = 0
        \end{cases}
    \]

    \begin{enumerate}
        \item 系数行列式$|A| \neq 0$ $\Leftrightarrow$ 方程组\textbf{只有零解}
        \item 系数行列式$|A| = 0$ $\Leftrightarrow$ 方程组\textbf{有非零解}
    \end{enumerate}

    \subsection{方法步骤}

    \begin{enumerate}
        \item 对于\textbf{主对角线爪型}行列式,可用\textbf{主}对角线元素将其化为上(下)三角型来计算
        \[
            \begin{bmatrix}
                1 & 1 & 1 & 1 \\
                1 & 2 & 0 & 0 \\
                1 & 0 & 3 & 0 \\
                1 & 0 & 0 & 4
            \end{bmatrix}
        \]
        \item 对于\textbf{副对角线爪型}行列式,可用\textbf{副}对角线元素将其化为\textbf{反上(下)三角型}来计算
        \[
            \begin{bmatrix}
                1 & 1 & 1 & 1 \\
                0 & 0 & 2 & 1 \\
                0 & 3 & 0 & 1 \\
                4 & 0 & 0 & 1
            \end{bmatrix}
        \]
        \item 把各行(列)均加到第一行(列)
        \item 逐行(列)相加
        \item 若有较多$0$,可考虑直接用行(列)展开公式
        \item 特殊的\textbf{三对角线}行列式
        \begin{enumerate}
            \item 三角化法: 逐行相加,构造上下三角型
            \item 递推法
            \item 归纳法
        \end{enumerate}
        \item 数学归纳法
        \begin{itemize}
            \item \textbf{普通数学归纳法(Mathematical Induction)}
            \begin{enumerate}
                \item 验证 $n = 1$ 时,命题 $f_n$ 成立;
                \item 假设 $n = k$ 时,命题 $f_n$ 成立;
                \item 证明 $n = k + 1$ 时,命题 $f_n$ 成立。
            \end{enumerate}

            \item \textbf{强(完全)归纳法(Strong Induction)}
            \begin{enumerate}
                \item 验证 $n = 1$ 和 $n = 2$ 时,命题 $f_n$ 成立;
                \item 假设当 $n < k$ 时,命题 $f_n$ 均成立;
                \item 证明 $n = k$ 时,命题 $f_n$ 成立。
            \end{enumerate}
        \end{itemize}
    \end{enumerate}

    \subsection{条件转换思路}

    \begin{enumerate}
        \item 齐次方程组$Ax = 0$有非零解
        \begin{align*}
            &\Leftrightarrow\; r(A) < n (\text{其中}n = \text{未知数的个数})  \\
            &\Leftrightarrow\; |A| = 0  \\
            &\Leftrightarrow\; A \text{的列向量组线性相关}  \\
        \end{align*}
    \end{enumerate}

    \subsection{理解}

    \begin{enumerate}
    \end{enumerate}
\end{document}
