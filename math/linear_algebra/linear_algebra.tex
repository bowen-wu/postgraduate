\documentclass[a4paper,12pt]{article}
\usepackage{xeCJK}          % 中文支持
\usepackage{fontspec}       % 英文/数学字体
\usepackage{amsmath, amssymb} % 数学公式
\usepackage{graphicx}       % 插入图片
\usepackage{hyperref}       % 目录超链接
\usepackage{geometry}       % 页面布局
\geometry{left=3cm,right=3cm,top=3cm,bottom=3cm}

% =========================
% 字体设置
% =========================
\setmainfont{Times New Roman}
\setsansfont{Helvetica Neue}
\setmonofont{Menlo}
\setCJKmainfont{PingFang SC}

% =========================
% 图形路径(可调整)
% =========================
\graphicspath{{./assets/}}

% =========================
% 文档开始
% =========================
\begin{document}

    \title{线性代数}
    \author{Bowen}
    \date{\today}
    \maketitle

% =========================
    \section{行列式}

    \subsection{基础}
    \paragraph{定义与性质}
    行列式定义:
    \[
        \det A = \sum_{\sigma \in S_n} \text{sgn}(\sigma) \prod_{i=1}^n a_{i, \sigma(i)}
    \]

    \paragraph{行列式的基本性质}
     可以列出性质,如 |A^T| = |A|

    \subsubsection{计算方法}
    \paragraph{按行/列展开}
    \paragraph{对角线法则(2×2, 3×3)}

    \subsection{克拉默法则}
    \paragraph{线性方程组解法}
    \subparagraph{唯一解情况}
    \subparagraph{无解或无穷多解情况}

    \section{矩阵}

    \subsection{初等变换}
    \paragraph{初等行变换}
    \subparagraph{交换两行}
    \subparagraph{行倍加}
    \subparagraph{行乘非零常数}

    \paragraph{初等列变换}
    \subparagraph{交换两列}
    \subparagraph{列倍加}
    \subparagraph{列乘非零常数}

    \subsubsection{伴随矩阵}
    \paragraph{伴随矩阵定义}
    \subparagraph{余子式矩阵}
    \subparagraph{转置操作}

    \subsubsection{矩阵的秩}
    \paragraph{秩的定义}
    \subparagraph{行秩与列秩}
    \subparagraph{零矩阵的秩}
    \paragraph{秩的计算方法}
    \subparagraph{初等变换法}
    \subparagraph{阶梯形矩阵法}

% =========================
% 示例公式与例题
% =========================
    \section{例题与练习}
    \subsection{行列式练习}
    \begin{enumerate}
        \item 计算矩阵 $A = \begin{bmatrix} 1 & 2 \\ 3 & 4 \end{bmatrix}$ 的行列式。
        \item 证明 $|AB| = |A||B|$。
    \end{enumerate}

    \subsection{矩阵运算练习}
    \begin{enumerate}
        \item 求矩阵 $A$ 的逆矩阵。
        \item 利用初等变换求矩阵的秩。
    \end{enumerate}

\end{document}
