\documentclass[a4paper,12pt]{article}
\usepackage{xeCJK}          % 中文支持
\usepackage{fontspec}       % 英文/数学字体
\usepackage{amsmath, amssymb} % 数学公式
\usepackage{graphicx}       % 插入图片
\usepackage{hyperref}       % 目录超链接
\usepackage{geometry}       % 页面布局
\usepackage{float}          % 更灵活的浮动体控制
\usepackage{bm}             % 粗体
\geometry{left=3cm,right=3cm,top=3cm,bottom=3cm}

% =========================
% 字体设置
% =========================
\setmainfont{Times New Roman}
\setsansfont{Helvetica Neue}
\setmonofont{Menlo}
\setCJKmainfont{PingFang SC}

\usepackage{booktabs}            % 图片插入
\graphicspath{{./assets/}}

% =========================
% 文档开始
% =========================
\begin{document}

    \title{LateX Guide}
    \author{Bowen}
    \date{\today}
    \maketitle
% =========================
% 字体测试
% =========================
    \section*{字体测试 (Font Test)}
    中文正文:这是苹方字体。
    English text: This is Times New Roman.
    \textsf{This is Helvetica Neue (sans-serif).}
    \texttt{This is Menlo (monospace).}

% =========================
% LaTex 空格
% =========================


    \section{Space}

    \begin{tabular}{ll}
        \toprule
        \textbf{命令}   & \textbf{说明}       \\
        \midrule
        \verb|\,|     & 很小的空格(thin space) \\
        \verb|\:|     & 小空格(medium space) \\
        \verb|\;|     & 厚空格(thick space)  \\
        \verb|\quad|  & 较宽的空格,相当于字母“M”的宽度 \\
        \verb|\qquad| & 更宽的空格,相当于两个 \quad \\
    \end{tabular}

% =========================
% LaTex 粗体
% =========================

    \begin{itemize}
        \item \textbf{这是粗体文字}
        \item $\mathbf{A} + \mathbf{B} = \mathbf{C}$
        \item $\bm{\alpha + \beta} = \gamma$
        \item \textbf{这里有一个数学公式: } $\mathbf{A + B = C}$
    \end{itemize}

% =========================
% LaTex 数学语法
% =========================


    \section{集合}

    \begin{table}[h]
        \centering
        \begin{tabular}{|c|c|c|}
            \hline
            符号           & LaTeX 代码             & 含义       \\
            \hline
            $\cup$       & \verb|A \cup B|      & 并集       \\
            $\cap$       & \verb|A \cap B|      & 交集       \\
            $\setminus$  & \verb|A \setminus B| & 差集       \\
            $\subset$    & \verb|A \subset B|   & 真子集      \\
            $\subseteq$  & \verb|A \subseteq B| & 子集 / 包含于 \\
            $\in$        & \verb|A \in B|       & 属于       \\
            $\emptyset$  & \verb|\emptyset|     & 空集       \\
            $\mathbb{N}$ & \verb|\mathbb{N}|    & 自然数集     \\
            $\mathbb{Z}$ & \verb|\mathbb{Z}|    & 整数集      \\
            $\mathbb{Q}$ & \verb|\mathbb{Q}|    & 有理数集     \\
            $\mathbb{R}$ & \verb|\mathbb{R}|    & 实数集      \\
            $\mathbb{C}$ & \verb|\mathbb{C}|    & 复数集      \\
            \hline
        \end{tabular}\label{tab:table}
    \end{table}


    \section{Math Grammar}

    \subsection{基本数学符号}

    \begin{tabular}{lll}
        \toprule
        \textbf{功能} & \textbf{LaTeX}     & \textbf{显示}              \\
        \midrule
        上标          & x^2                & x2x^2                    \\
        下标          & a_i                & aia_i                    \\
        上划线         & \overline{A}       & \overline{A}             \\
        否定命题        & \neg A             & \neg A                   \\
        分数          & $\frac{a}{b}$      &                          \\
        根号          & \sqrt{x}           & x\sqrt{x}                \\
        n次根         & $\sqrt[n]{x}$      &                          \\
        加减乘除        & + -  \times \div     & 加减乘除                     \\
        左乘          & \cdot              & 左乘                       \\
        等号与不等号      & = \neq \le \ge < > & =≠≤≥<>= \neq \le \ge < > \\
        省略号         & \dots              & 横向                       \\
        省略号         & \cdots             & 中点                       \\
        省略号         & \vdots             & 竖向                       \\
        省略号         & \ddots             & 斜向                       \\
        \bottomrule
    \end{tabular}

    \subsection{逻辑推理符号}

    \begin{tabular}{ll}
        \toprule
        \textbf{LaTeX 命令}                & \textbf{含义}      \\
        \midrule
        \to                              & 映射/推出            \\
        \rightarrow                      & 推出               \\
        \leftarrow                       & 从…得出             \\
        \Rightarrow                      & 蕴含 / 正推          \\
        \Leftarrow                       & 蕴含 / 逆推          \\
        \Leftrightarrow                  & 等价 / 互推          \\
        \implies                         & 蕴含(amsmath 提供)   \\
        \impliedby                       & 被蕴含(amsmath 提供)  \\
        \iff                             & 当且仅当(amsmath 提供) \\
        \vdash                           & 可推出 / 证明         \\
        \models                          & 语义蕴含 / 满足关系      \\
        \overset{\text{说明}}{\rightarrow} & 在符号上方再加文
        \bottomrule
    \end{tabular}

    \subsection{基础}
    行内公式 \\
    \( a^2 + b^2 = c^2 \) \qquad $ a^2 + b^2 = c^2 $

    独立公式:

    \[
        \int_0^{\pi} \sin(x)\, dx = 2
    \] \\
    $$ a^2 + b^2 = c^2 $$

    \subsection{上下标}

    a_i, \; a^2, \; a_i^2, \; a_{ij}

    \subsection{分数与根号}

    \begin{itemize}
        \item $\frac{a}{b}$, \; frac == fraction n. 分数;
        \item \sqrt{x}, \; square root,
        \item \sqrt[n]{x}
    \end{itemize}

    \subsection{求和与连乘}

    \sum_{i=1}^n i, \quad \prod_{i=1}^n i, \;  prod == product

    \subsection{积分与极限}

    \int_0^1 x^2\, dx, \quad

    解释: int → integral(积分符号 ∫)

    \lim_{x \to \infty} f(x)

    \subsection{行列式}

    \begin{vmatrix}
        a & b \\ c & d
    \end{vmatrix}

    \subsection{矩阵}
    \begin{bmatrix}
        a & b \\ c & d
    \end{bmatrix}

    \[
        A = \begin{bmatrix}
                1 & 2 & 3 \\
                4 & 5 & 6 \\
                7 & 8 & 9
        \end{bmatrix}
    \]

    \subsection{特殊符号}
    ∞: \infty, \\
    ≤: \leq, \\
    ≥: \geq, \\
    ≠: \neq \\
    ∈: \in, \\
    ∉: \notin, \\
    ⊆: \subseteq, \\
    ∀: \forall, \\
    ∃: \exists, \\
    ≈:\approx,

% =========================
% 希腊字母
% =========================
    \section*{希腊字母 LaTeX 速查表}

    \subsection*{小写希腊字母}
    \begin{tabular}{c c c c c c}
        \hline
        字母         & LaTeX            & 显示         & 字母       & LaTeX         & 显示       \\
        \hline
        $\alpha$   & \verb|\alpha|    & $\alpha$   & $\beta$  & \verb|\beta|  & $\beta$  \\
        $\gamma$   & \verb|\gamma|    & $\gamma$   & $\delta$ & \verb|\delta| & $\delta$ \\
        $\epsilon$ & \verb|\epsilon|  & $\epsilon$ & $\zeta$  & \verb|\zeta|  & $\zeta$  \\
        $\eta$     & \verb|\eta|      & $\eta$     & $\theta$ & \verb|\theta| & $\theta$ \\
        $\iota$    & \verb|\iota|     & $\iota$    & $\kappa$ & \verb|\kappa| & $\kappa$ \\
        $\lambda$  & \verb|\lambda|   & $\lambda$  & $\mu$    & \verb|\mu|    & $\mu$    \\
        $\nu$      & \verb|\nu|       & $\nu$      & $\xi$    & \verb|\xi|    & $\xi$    \\
        $\pi$      & \verb|\pi|       & $\pi$      & $\rho$   & \verb|\rho|   & $\rho$   \\
        $\sigma$   & \verb|\sigma|    & $\sigma$   & $\tau$   & \verb|\tau|   & $\tau$   \\
        $\upsilon$ & \verb|\upsilon|  & $\upsilon$ & $\phi$   & \verb|\phi|   & $\phi$   \\
        $\chi$     & \verb|\chi|      & $\chi$     & $\psi$   & \verb|\psi|   & $\psi$   \\
        $\omega$   & \verb|\omega|    & $\omega$   &          &               &          \\
        $\varphi$  & \varphi|\varphi| & $\varphi$  &          &               &          \\
        \hline
    \end{tabular}

    \subsection*{大写希腊字母}
    \begin{tabular}{c c c c c c}
        \hline
        字母       & LaTeX         & 显示       & 字母         & LaTeX           & 显示         \\
        \hline
        $\Gamma$ & \verb|\Gamma| & $\Gamma$ & $\Delta$   & \verb|\Delta|   & $\Delta$   \\
        $\Theta$ & \verb|\Theta| & $\Theta$ & $\Lambda$  & \verb|\Lambda|  & $\Lambda$  \\
        $\Xi$    & \verb|\Xi|    & $\Xi$    & $\Pi$      & \verb|\Pi|      & $\Pi$      \\
        $\Sigma$ & \verb|\Sigma| & $\Sigma$ & $\Upsilon$ & \verb|\Upsilon| & $\Upsilon$ \\
        $\Phi$   & \verb|\Phi|   & $\Phi$   & $\Psi$     & \verb|\Psi|     & $\Psi$     \\
        $\Omega$ & \verb|\Omega| & $\Omega$ &            &                 &            \\
        \hline
    \end{tabular}

% =========================
% 二元运算符
% =========================

    \section*{二元运算符}
    ± (plus-minus):\pm

    × (times):\times

    ÷ (division):\div

    · (dot operator):\dot

    ^ (logical AND):\wedge

% =========================
% 图形测试
% =========================

    \section*{图形测试 (Graphics Test)}
    插入示意图:

    \includegraphics[width=1\linewidth]{tense.jpeg}

\end{document}
