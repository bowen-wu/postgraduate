\documentclass[a4paper,12pt]{article}
\usepackage{xeCJK}          % 中文支持
\usepackage{fontspec}       % 英文/数学字体
\usepackage{amsmath, amssymb} % 数学公式
\usepackage{graphicx}       % 插入图片
\usepackage{hyperref}       % 目录超链接
\usepackage{geometry}       % 页面布局
\usepackage{float}          % 更灵活的浮动体控制
\geometry{left=3cm,right=3cm,top=3cm,bottom=3cm}

% =========================
% 字体设置
% =========================
\setmainfont{Times New Roman}
\setsansfont{Helvetica Neue}
\setmonofont{Menlo}
\setCJKmainfont{PingFang SC}

\usepackage{booktabs}            % 图片插入
\graphicspath{{./assets/}}

% =========================
% 文档开始
% =========================
\begin{document}

    \title{LateX Guide}
    \author{Bowen}
    \date{\today}
    \maketitle
% =========================
% 字体测试
% =========================
    \section*{字体测试 (Font Test)}
    中文正文:这是苹方字体。
    English text: This is Times New Roman.
    \textsf{This is Helvetica Neue (sans-serif).}
    \texttt{This is Menlo (monospace).}

% =========================
% LaTex 空格
% =========================


    \section{Space}

    \begin{tabular}{ll}
        \toprule
        \textbf{命令}   & \textbf{说明}       \\
        \midrule
        \verb|\,|     & 很小的空格(thin space) \\
        \verb|\:|     & 小空格(medium space) \\
        \verb|\;|     & 厚空格(thick space)  \\
        \verb|\quad|  & 较宽的空格,相当于字母“M”的宽度 \\
        \verb|\qquad| & 更宽的空格,相当于两个 \quad \\
    \end{tabular}

% =========================
% LaTex 数学语法
% =========================


    \section{Math Grammar}

    \subsection{基本数学符号}

    \begin{tabular}{lll}
        \toprule
        \textbf{功能} & \textbf{LaTeX}     & \textbf{显示}              \\
        \midrule
        上标          & x^2                & x2x^2                    \\
        下标          & a_i                & aia_i                    \\
        分数          & $\frac{a}{b}$      &                          \\
        根号          & \sqrt{x}           & x\sqrt{x}                \\
        n次根         & $\sqrt[n]{x}$      &                          \\
        加减乘除        & + -  \times \div     & 加减乘除                     \\
        左乘          & \cdot              & 左乘                       \\
        等号与不等号      & = \neq \le \ge < > & =≠≤≥<>= \neq \le \ge < > \\
        省略号         & \dots              & 横向                       \\
        省略号         & \cdots             & 中点                       \\
        省略号         & \vdots             & 竖向                       \\
        省略号         & \ddots             & 斜向                       \\
        \bottomrule
    \end{tabular}

    \subsection{基础}
    行内公式

    \( a^2 + b^2 = c^2 \) \qquad $ a^2 + b^2 = c^2 $

    独立公式:

    \[
        \int_0^{\pi} \sin(x)\, dx = 2
    \]

    $$ a^2 + b^2 = c^2 $$

    \subsection{上下标}

    a_i, \; a^2, \; a_i^2, \; a_{ij}

    \subsection{分数与根号}

    \begin{itemize}
        \item $\frac{a}{b}$, \; frac == fraction n. 分数;
        \item \sqrt{x}, \; square root,
        \item \sqrt[n]{x}
    \end{itemize}

    \subsection{求和与连乘}

    \sum_{i=1}^n i, \quad \prod_{i=1}^n i, \;  prod == product

    \subsection{积分与极限}

    \int_0^1 x^2\, dx, \quad

    解释: int → integral(积分符号 ∫)

    \lim_{x \to \infty} f(x)

    \subsection{行列式}

    \[
    \begin{vmatrix}
        a & b \\ c & d
    \end{vmatrix}
    \]

    \subsection{矩阵}

    $$
    \begin{bmatrix}
        a & b \\ c & d
    \end{bmatrix}
    $$

    \[
        A = \begin{bmatrix}
                1 & 2 & 3 \\
                4 & 5 & 6 \\
                7 & 8 & 9
        \end{bmatrix}
    \]

    \subsection{特殊符号}
    ∞: \infty, \\
    ≤: \leq, \\
    ≥: \geq, \\
    ≠: \neq \\
    ∈: \in, \\
    ∉: \notin, \\
    ⊆: \subseteq, \\
    ∀: \forall, \\
    ∃: \exists, \\
    ≈:\approx,

% =========================
% 希腊字母
% =========================
    \section*{希腊字母}
    α (alpha):\alpha, \;
    β (beta):\beta, \;
    γ: \gamma, \;
    Δ: \Delta, \;
    λ: \lambda, \;
    π (pi): \pi, \;
    σ: \sigma, \;
    Ω: \Omega, \;
    δ (delta):\delta, \;


% =========================
% 二元运算符
% =========================
    \section*{二元运算符}
    ± (plus-minus):\pm

    × (times):\times

    ÷ (division):\div

    · (dot operator):\dot

    ^ (logical AND):\wedge

% =========================
% 图形测试
% =========================
    \section*{图形测试 (Graphics Test)}
    插入示意图:

    \includegraphics[width=1\linewidth]{tense.jpeg}
% =========================
% 正文示例
% =========================

    \part{线性代数}

    \chapter{行列式}

    \section{基础}
    \subsection{定义与性质}
    \subsubsection{行列式的定义}
    \subsubsection{行列式的基本性质}

    \subsection{计算方法}
    \subsubsection{按行/列展开}
    \subsubsection{对角线法则(2×2, 3×3)}

    \section{克拉默法则}
    \subsection{线性方程组解法}
    \subsubsection{唯一解情况}
    \subsubsection{无解或无穷多解情况}

    \chapter{矩阵}

    \section{初等变换}
    \subsection{初等行变换}
    \subsubsection{交换两行}
    \subsubsection{行倍加}
    \subsubsection{行乘非零常数}

    \subsection{初等列变换}
    \subsubsection{交换两列}
    \subsubsection{列倍加}
    \subsubsection{列乘非零常数}

    \section{伴随矩阵}
    \subsection{伴随矩阵定义}
    \subsubsection{余子式矩阵}
    \subsubsection{转置操作}

    \section{矩阵的秩}
    \subsection{秩的定义}
    \subsubsection{行秩与列秩}
    \subsubsection{零矩阵的秩}
    \subsection{秩的计算方法}
    \subsubsection{初等变换法}
    \subsubsection{阶梯形矩阵法}

\end{document}
