\documentclass[a4paper,12pt]{article}
\usepackage{xeCJK}          % 中文支持
\usepackage{fontspec}       % 英文/数学字体
\usepackage{amsmath, amssymb} % 数学公式
\usepackage{graphicx}       % 插入图片
\usepackage{hyperref}       % 目录超链接
\usepackage{geometry}       % 页面布局
\usepackage{bm}             % 粗体
\usepackage{xcolor}         % 颜色
\usepackage{tabularx}       % 表格环境
\usepackage{tikz}           % TikZ 绘制主对角线斜线
\usepackage{tcolorbox}
\usepackage{xstring}
\usepackage{amsfonts}
\geometry{left=3cm,right=3cm,top=3cm,bottom=3cm}

% 抽离颜色和尺寸参数
\newcommand{\analysisTitleColor}{green!50!black}
\newcommand{\analysisBackColor}{white}
\newcommand{\analysisBoxRule}{0.8pt}
\newcommand{\analysisArc}{3pt}
\newcommand{\analysisPadding}{6pt}

% 定义 tcolorbox
\newtcolorbox{analysisbox}[1][]{
    title=\IfStrEq{#1}{}{\textbf{解析}}{#1}, % 如果传参为空则使用“解析”
    colback=\analysisBackColor,
    colframe=\analysisTitleColor,
    boxrule=\analysisBoxRule,
    arc=\analysisArc,
    left=\analysisPadding,
    right=\analysisPadding,
    top=4pt,
    bottom=4pt
}

% =========================
% 字体设置
% =========================
\setmainfont{Times New Roman}
\setsansfont{Helvetica Neue}
\setmonofont{Menlo}
\setCJKmainfont{PingFang SC}

% =========================
% 图形路径(可调整)
% =========================
\graphicspath{{./assets/}}

% =========================
% 文档开始
% =========================
\begin{document}

%    \title{Template}
%    \author{Bowen}
%    \date{\today}
%    \maketitle
%    \tableofcontents
%    \newpage


% =========================

    \section{基本逻辑}

    \subsection{基础概念}

    \begin{enumerate}
        \item 全称命题: $\forall x,$均有$A$成立
        \item 特称命题: $\exists x,$使得$A$成立
        \begin{enumerate}
            \item 全称命题和特称命题互为否定命题
            \item 全称命题否定命题: $\exists x,$使得$\overline{A}$成立
            \item 特称命题否定命题: $\forall x,$均有$\overline{A}$成立
            \item Case: $\forall x \in \mathbb{R}$均有$x^2 \ge 0$成立。否定命题: $\exists x \in \mathbb{R},$使得$x^2 < 0$成立
            \item Case: $\exists x \in \mathbb{R}$使得$x^2 \ge 0$。否定命题: $\forall x \in \mathbb{R},$均有$x^2 < 0$成立
        \end{enumerate}
        \item 蕴称命题: $A \Rightarrow B$(若$A$成立,则$B$成立)
        \begin{enumerate}
            \item 其{\color{red}{否定}}命题: $A$成立且$\overline{B}$成立
            \item \textbf{逆否命题}: 若$B$不成立则$A$不成立($\overline{B} \Rightarrow \overline{A}$)。与蕴含命题\textbf{等价}
            \item 其{\color{red}{否}}命题: $\overline{A} \Rightarrow \overline{B}$。原命题真,其否命题可真可假
        \end{enumerate}
        \item 多元命题: $\forall x, \forall y,$均有$A$成立;或$\forall x, \exists y,$使得$A$成立
        \begin{align*}
            & \lim\limits_{n \to \infty} x_n = a &\Leftrightarrow \forall \epsilon > 0, \exists N > 0, \forall n > N \Rightarrow |x_n - a| < \epsilon \\[6pt]
            & \text{其否定命题: } a \text{ 不是 } x_n \text{ 的极限 } &\Leftrightarrow \exists \epsilon > 0, \forall N > 0, \exists n > N \Rightarrow |x_n - a| \ge \epsilon
        \end{align*}
        \item 充分条件: $A \Rightarrow B$,若$A$成立,则$B$成立,称$A$是$B$的充分条件
        \item 必要条件: $A \Leftarrow B$,若$B$成立,则$A$成立,称$A$是$B$的必要条件
        \item 充分必要条件: $A \Leftrightarrow B$,若$A$成立,则$B$成立,若$B$成立,则$A$成立,称$A$和$B$互为充分必要条件
        \item 反证思路:所给出的\textbf{"否定命题"}是推理中增加的一个强有力的条件。假设其否定命题成立,推导出与已知成立的某条件矛盾,则思路完成。以下情形可用:
        \begin{enumerate}
            \item 显而易见但不易说明
            \item 与常见形式对立
        \end{enumerate}
        \item 数学归纳:在试算$n$较小时的特殊情况后,增加$n=k$是$A$成立(第一数学归纳法)或者$n<k+1$是$A$成立(第二数学归纳法)这个强有力的条件,推导$n=k+1$是$A$成立
    \end{enumerate}

    \subsection{定理}

    \begin{enumerate}
    \end{enumerate}

    \subsection{运算}

    \begin{enumerate}

    \end{enumerate}

    \subsection{公式}

    \begin{enumerate}

    \end{enumerate}

    \subsection{方法步骤}

    \begin{enumerate}

    \end{enumerate}

    \subsection{条件转换思路}

    \begin{enumerate}

    \end{enumerate}

    \subsection{理解}

    \begin{enumerate}
    \end{enumerate}

\end{document}
