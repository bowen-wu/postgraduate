\documentclass[a4paper,12pt]{article}
\usepackage{xeCJK}          % 中文支持
\usepackage{fontspec}       % 英文/数学字体
\usepackage{amsmath, amssymb} % 数学公式
\usepackage{graphicx}       % 插入图片
\usepackage{hyperref}       % 目录超链接
\usepackage{geometry}       % 页面布局
\usepackage{bm}             % 粗体
\usepackage{xcolor}         % 颜色
\usepackage{tabularx}       % 表格环境
\usepackage{tikz}           % TikZ 绘制主对角线斜线
\usepackage{tcolorbox}
\usepackage{xstring}
\usepackage{pgfplots}
\usepackage{enumitem}
\usepackage{ulem}
\pgfplotsset{compat=1.18}
\geometry{left=3cm,right=3cm,top=3cm,bottom=3cm}

% 抽离颜色和尺寸参数
\newcommand{\analysisTitleColor}{green!50!black}
\newcommand{\analysisBackColor}{white}
\newcommand{\analysisBoxRule}{0.8pt}
\newcommand{\analysisArc}{3pt}
\newcommand{\analysisPadding}{6pt}

% 定义 tcolorbox
\newtcolorbox{analysisbox}[1][]{
    title=\IfStrEq{#1}{}{\textbf{解析}}{#1}, % 如果传参为空则使用“解析”
    colback=\analysisBackColor,
    colframe=\analysisTitleColor,
    boxrule=\analysisBoxRule,
    arc=\analysisArc,
    left=\analysisPadding,
    right=\analysisPadding,
    top=4pt,
    bottom=4pt
}
\newlist{circlenum}{enumerate}{1}
\setlist[circlenum]{
    label=\ding{\numexpr171+\arabic*},
    leftmargin=2.2em,
    itemsep=0.4em
}

\newcommand{\blueuline}[1]{{\color{blue}\uline{\color{black}{#1}}}}

% =========================
% 字体设置
% =========================
\setmainfont{Times New Roman}
\setsansfont{Helvetica Neue}
\setmonofont{Menlo}
\setCJKmainfont{PingFang SC}

% =========================
% 图形路径(可调整)
% =========================
\graphicspath{{./assets/}}

% =========================
% 文档开始
% =========================
\begin{document}

%    \title{Template}
%    \author{Bowen}
%    \date{\today}
%    \maketitle
%    \tableofcontents
%    \newpage


% =========================

    \section{一元函数微分学的应用 - 几何应用}

    \subsection{极值的定义}
    \begin{enumerate}
        \item 对于函数 $f(x)$,若存在点 $x_0$ 的 {\color{red}{某个邻域}},
        使得在该邻域内任意一点 $x$,均有
        \[
            f(x) \le f(x_0)\quad (\text{或 } f(x) \ge f(x_0))
        \]
        成立,则称点 $x_0$ 为 $f(x)$ 的
        \textbf{极大值点}(或 \textbf{极小值点}),
        $f(x_0)$ 为 $f(x)$ 的极大值(或极小值)。

        \item 极值是一个\textbf{局部}的概念。

        \item 极值要求点 $x_0$ 的左右邻域均有定义,
        \textbf{端点处不讨论极值,间断点不可能是极值点}。
    \end{enumerate}

    \subsection{极值的定义}

    \subsubsection{单调性的判别}
    \begin{enumerate}
        \item 设函数 $y = f(x)$ 在 $[a,b]$ 上连续,在 $(a,b)$ 内可导。

        \begin{circlenum}
            \item 若在 $(a,b)$ 内有 $f'(x) \ge 0$,
            且等号仅{\blueuline{在有限个点处成立}},则函数 $y = f(x)$ 在 $[a,b]$ 上
                {\color{red}{严格单调增加}}。

            \item 若在 $(a,b)$ 内有 $f'(x) \le 0$,
            且等号仅 {\blueuline{在有限个点处成立}},则函数 $y = f(x)$ 在 $[a,b]$ 上
                {\color{red}{严格单调减少}}

            \item 导数为$0$仅能说明在某点处的函数值变化\textbf{充分小},而不能说明没变化
        \end{circlenum}
    \end{enumerate}

    \subsection{基础概念}

    \begin{enumerate}
    \end{enumerate}

    \subsection{结论}

    \begin{enumerate}
    \end{enumerate}

    \subsection{定理}

    \begin{enumerate}
    \end{enumerate}

    \subsection{运算}

    \begin{enumerate}

    \end{enumerate}

    \subsection{公式}

    \begin{enumerate}

    \end{enumerate}

    \subsection{方法总结}

    \begin{enumerate}

    \end{enumerate}

    \subsection{条件转换思路}

    \begin{enumerate}

    \end{enumerate}

    \subsection{理解}

    \begin{enumerate}
    \end{enumerate}

\end{document}
