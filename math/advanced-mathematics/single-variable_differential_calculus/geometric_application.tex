\documentclass[a4paper,12pt]{article}
\usepackage{xeCJK}          % 中文支持
\usepackage{fontspec}       % 英文/数学字体
\usepackage{amsmath, amssymb} % 数学公式
\usepackage{graphicx}       % 插入图片
\usepackage{hyperref}       % 目录超链接
\usepackage{geometry}       % 页面布局
\usepackage{bm}             % 粗体
\usepackage{xcolor}         % 颜色
\usepackage{tabularx}       % 表格环境
\usepackage{tikz}           % TikZ 绘制主对角线斜线
\usepackage{ctex}           % 中文(强烈建议)
\usepackage{tcolorbox}
\usepackage{xstring}
\usepackage{pgfplots}
\usepackage{enumitem}
\usepackage{pifont}
\usepackage{ulem}
\pgfplotsset{compat=1.18}
\geometry{left=3cm,right=3cm,top=3cm,bottom=3cm}
\usetikzlibrary{arrows.meta, positioning, shapes.geometric}

\tikzstyle{startstop} = [rectangle, rounded corners,
minimum width=2.5cm, minimum height=0.9cm, draw=black]

\tikzstyle{process} = [rectangle,
minimum width=3.6cm, minimum height=1cm, draw=black]

\tikzstyle{decision} = [diamond, aspect=2,
minimum width=4cm, minimum height=1.5cm, draw=black]

\tikzstyle{arrow} = [->, thick]


% 抽离颜色和尺寸参数
\newcommand{\analysisTitleColor}{green!50!black}
\newcommand{\analysisBackColor}{white}
\newcommand{\analysisBoxRule}{0.8pt}
\newcommand{\analysisArc}{3pt}
\newcommand{\analysisPadding}{6pt}

% 定义 tcolorbox
\newtcolorbox{analysisbox}[1][]{
    title=\IfStrEq{#1}{}{\textbf{解析}}{#1}, % 如果传参为空则使用“解析”
    colback=\analysisBackColor,
    colframe=\analysisTitleColor,
    boxrule=\analysisBoxRule,
    arc=\analysisArc,
    left=\analysisPadding,
    right=\analysisPadding,
    top=4pt,
    bottom=4pt
}

\newcommand{\blueuline}[1]{{\color{blue}\uline{\color{black}{#1}}}}

\newlist{circlenum}{enumerate}{1}
\setlist[circlenum]{
    label=\ding{\numexpr171+\arabic*},
    leftmargin=2.2em,
    itemsep=0.6em,      % item 之间的距离(主要)
    topsep=0.6em,       % 列表与上下正文的距离
    parsep=0.3em,       % item 内段落的间距
    partopsep=0.3em     % 列表前后额外间距
}

% =========================
% 字体设置
% =========================
\setmainfont{Times New Roman}
\setsansfont{Helvetica Neue}
\setmonofont{Menlo}
\setCJKmainfont{PingFang SC}

% =========================
% 图形路径(可调整)
% =========================
\graphicspath{{./assets/}}

% =========================
% 文档开始
% =========================
\begin{document}

%    \title{Template}
%    \author{Bowen}
%    \date{\today}
%    \maketitle
%    \tableofcontents
%    \newpage


% =========================

    \section{一元函数微分学的应用 - 几何应用}

    \subsection{极值的定义}
    \begin{enumerate}
        \item 对于函数 $f(x)$,若存在点 $x_0$ 的 {\color{red}{某个邻域}},
        使得在该邻域内任意一点 $x$,均有
        \[
            f(x) \le f(x_0)\quad (\text{或 } f(x) \ge f(x_0))
        \]
        成立,则称点 $x_0$ 为 $f(x)$ 的
        \textbf{极大值点}(或 \textbf{极小值点}),
        $f(x_0)$ 为 $f(x)$ 的极大值(或极小值)。

        \item 极值是一个\textbf{局部}的概念。

        \item 极值要求点 $x_0$ 的左右邻域均有定义,
        \textbf{端点处不讨论极值,间断点不可能是极值点}。
    \end{enumerate}

    \subsection{单调性与极值的判别}

    \subsubsection{单调性的判别}
    \begin{enumerate}
        \item 设函数 $y = f(x)$ 在 $[a,b]$ 上连续,在 $(a,b)$ 内可导。

        \begin{circlenum}
            \item 若在 $(a,b)$ 内有 $f'(x) \ge 0$,
            且等号仅{\blueuline{在有限个点处成立}},则函数 $y = f(x)$ 在 $[a,b]$ 上
                {\color{red}{严格单调增加}}。

            \item 若在 $(a,b)$ 内有 $f'(x) \le 0$,
            且等号仅 {\blueuline{在有限个点处成立}},则函数 $y = f(x)$ 在 $[a,b]$ 上
                {\color{red}{严格单调减少}}

            \item 导数为$0$仅能说明在某点处的函数值变化\textbf{充分小},而不能说明没变化
        \end{circlenum}
    \end{enumerate}

    \subsubsection{一阶可导点是极值点的必要条件}

    \begin{enumerate}
        \item 设$f(x)$在$x = x_0$处可导,且在点$x_0$处取得极值,则必有$f'(x_0) = 0$
        \item 若$x = x_0$为曲线$y = f(x)$的极值点,则只有以下两种情况
        \begin{circlenum}
            \item 驻点: $f'(x_0) = 0$
            \item 不可导点: $f'(x_0)$不存在。如$y = |x|$在$(0, 0)$处的情形
        \end{circlenum}
        \item 找极值的两种情况: 驻点 | 不可导点
    \end{enumerate}

    \subsubsection{判别极值的第一充分条件}
    \begin{theorem}
        设函数 $f(x)$ 在 $x=x_0$ 处连续,且在 $x_0$ 的某去心邻域
        $U(x_0,\delta)\ (\delta>0)$ 内可导,则:

        \begin{circlenum}
            \item 若当 $x\in(x_0-\delta,x_0)$ 时,$f'(x)<0$,而当 $x\in(x_0,x_0+\delta)$ 时,$f'(x)>0$,则 $f(x)$ 在 $x=x_0$ 处取得极小值

            \item 若当 $x\in(x_0-\delta,x_0)$ 时,$f'(x)>0$,而当 $x\in(x_0,x_0+\delta)$ 时,$f'(x)<0$,则 $f(x)$ 在 $x=x_0$ 处取得极大值

            \item 若 $f'(x)$ 在 $(x_0-\delta,x_0)$ 与 $(x_0,x_0+\delta)$ 内不变号,则点 $x_0$ 不是极值点

            \item $f(x)$在$x = x_0$处不一定可导,可能出现角点
        \end{circlenum}
    \end{theorem}

    \subsubsection{判别极值的第二充分条件}

    \begin{theorem}
        设函数 $f(x)$ 在 $x=x_0$ 处二阶可导,且 $f'(x_0)=0$。
        \begin{circlenum}
            \item 若 $f''(x_0)<0$,则 $f(x)$ 在 $x=x_0$ 处取得极大值
            \item 若 $f''(x_0)>0$,则 $f(x)$ 在 $x=x_0$ 处取得极小值
            \item 若 $f''(x_0)=0$,则该判别法失效,需借助高阶导数或其他方法判断。如$f(x) = x^4$在$x_0 = 0$处是极小值, $f(x) = x^3$在$x_0 = 0$处是拐点,不是极值点
        \end{circlenum}
    \end{theorem}

    \subsubsection{判别极值的第三充分条件}

    \begin{theorem}
        设函数 $f(x)$ 在 $x=x_0$ 处 $n$ 阶可导,且$f^{(m)}(x_0)=0 (m=1,2,\ldots,n-1), f^{(n)}(x_0)\neq 0 \ (n\ge 2)$,则:
        \begin{circlenum}
            \item 若 $n$ 为偶数且 $f^{(n)}(x_0)<0$,则 $f(x)$ 在 $x=x_0$ 处取得极大值
            \item 若 $n$ 为偶数且 $f^{(n)}(x_0)>0$,则 $f(x)$ 在 $x=x_0$ 处取得极小值
        \end{circlenum}
        当 $n$ 为奇数时,点 $x_0$ 不是极值点。
    \end{theorem}

    \subsection{凹凸性与拐点的概念}

    \subsubsection{凹凸性的定义}
    \begin{enumerate}
        \item \begin{definition}
                  设函数 $f(x)$ 在区间 $I$ 上连续。
                  \begin{enumerate}
                      \item 若对 $I$ 上任意不同的两点 $x_1,x_2$,恒有
                      \[
                          f\!\left(\frac{x_1+x_2}{2}\right) < \frac{f(x_1)+f(x_2)}{2},
                      \]
                      则称函数 $y=f(x)$ 在区间 $I$ 上的图形是\textbf{凹的}(或称凹弧)

                      \item 若对 $I$ 上任意不同的两点 $x_1,x_2$,恒有
                      \[
                          f\!\left(\frac{x_1+x_2}{2}\right) > \frac{f(x_1)+f(x_2)}{2},
                      \]
                      则称函数 $y=f(x)$ 在区间 $I$ 上的图形是\textbf{凸的}(或称凸弧)。
                      \item 广义化:
                      \[
                          f(\lambda_1 x_1 + \lambda_2 x_2)
                          \begin{cases}
                              < \lambda_1 f(x_1) + \lambda_2 f(x_2), & \text{凹函数},\\
                              > \lambda_1 f(x_1) + \lambda_2 f(x_2), & \text{凸函数},
                          \end{cases}
                      \]
                      其中 $0 < \lambda_1,\lambda_2 < 1$,且 $\lambda_1+\lambda_2=1$。
                  \end{enumerate}
        \end{definition}
        \item 设函数 $f(x)$ 在区间 $[a,b]$ 上连续,在 $(a,b)$ 内可导。

        \begin{itemize}
            \item 若对任意 $x_0\in(a,b)$ 及任意 $x\in(a,b)$ 且 $x\neq x_0$,恒有
            \[
                f(x) < f(x_0) + f'(x_0)(x - x_0),
            \]
            即除切点外,曲线严格位于其切线的下方,
            则称函数 $y=f(x)$ 在 $[a,b]$ 上是\textbf{凹的}。

            \item 若对任意 $x_0\in(a,b)$ 及任意 $x\in(a,b)$ 且 $x\neq x_0$,恒有
            \[
                f(x) > f(x_0) + f'(x_0)(x - x_0),
            \]
            即除切点外,曲线严格位于其切线的上方,
            则称函数 $y=f(x)$ 在 $[a,b]$ 上是\textbf{凸的}。
        \end{itemize}
    \end{enumerate}

    \subsubsection{拐点的定义}

    连续曲线的凹弧与凸弧的分界点称为该曲线的\textbf{拐点}。

    \begin{circlenum}
        \item 间断点不可能为拐点,拐点处函数\textbf{必须连续}。

        \item 判别拐点时,只需判断凹凸性的变化,\textbf{凹与凸不分先后}。

        \item 极值点只写横坐标 $x=x_0$;而拐点应写$\bigl(x_0,\; f(x_0)\bigr)$,拐点在曲线上
    \end{circlenum}

    \subsection{凹凸性与拐点的判别}

    \subsubsection{判别凹凸性}

    设函数 $f(x)$ 在区间 $I$ 上二阶可导,则:

    \begin{circlenum}
        \item 若在 $I$ 上 $f''(x) > 0$,则 $f(x)$ 在 $I$ 上的图形是\textbf{凹的}(凹弧)。

        \item 若在 $I$ 上 $f''(x) < 0$,则 $f(x)$ 在 $I$ 上的图形是\textbf{凸的}(凸弧)。
    \end{circlenum}

    \subsubsection{二阶可导点是拐点的必要条件}

    设 $f''(x_0)$ 存在,且点 $(x_0,\,f(x_0))$ 为曲线的拐点,则必有 $f''(x_0)=0$。若点 $(x_0,\,f(x_0))$ 是曲线 $y = f(x)$ 的拐点,则只有以下两种情况

    \begin{enumerate}
        \item 二阶导数存在必为$0$: $f''(x_0)=0$。如 $y=x^3$ 在 $(0,0)$ 处的情形
        \item 二阶导数不存在的点也有可能是拐点: $f''(x_0)$ 不存在。如 $y=\sqrt[3]{x}$ 在 $(0,0)$ 处的情形
    \end{enumerate}

    \subsubsection{判别拐点的第一充分条件}

    设函数 $f(x)$ 在点 $x=x_0$ 处连续,在点 $ x = x_0$ 的某去心邻域$U(x_0,\delta)\ (\delta>0)$ 内二阶导数存在,且在 $x_0$ 的左、右邻域内$f''(x)$ 变号(无论是由正变负,还是由负变正),则点 $(x_0,f(x_0))$ 为曲线的拐点

    \subsubsection{判别拐点的第二充分条件}

    设函数 $f(x)$ 在 $x=x_0$ 处三阶可导,且 $f''(x_0)=0$,$f'''(x_0)\neq 0$, 则点 $(x_0,f(x_0))$ 为曲线的拐点

    \subsubsection{判别拐点的第三充分条件}

    设函数 $f(x)$ 在 $x=x_0$ 处 $n$ 阶可导,且 $f^{(m)}(x_0)=0\ (m=2,3,\dots,n-1)$,$f^{(n)}(x_0)\neq 0$,则当 $n$ 为奇数时,点 $(x_0,f(x_0))$ 为曲线的拐点。

    \subsection{极值点与拐点的重要结论}

    \begin{circlenum}
        \item 曲线的可导点{\blueuline{不可同时为极值点和拐点}};曲线的不可导点可{\blueuline{同时为极值点和拐点}}
        \item 设多项式函数$f(x) = (x-a)^{n}g(x)(n>1)$,且$g(a) \neq 0$,则当$n$为{\color{red}{偶数}}时,$x=a$是$f(x)$的{\color{red}{极值点}};当$n$为{\color{red}{奇数}}时,$x=a$是$f(x)$的{\color{red}{拐点}}
        \item 设多项式函数 $f(x)= (x-a_1)^{n_1}(x-a_2)^{n_2}\cdots(x-a_k)^{n_k}$,其中 $n_i$ 为正整数,$a_i$ 为实数,且 $a_i\ (i=1,2,\dots,k)$ 两两不等。记$k_1$为 $n_i=1$ 的个数,$k_2$ 为满足 $n_i>1$ 且 $n_i$ 为偶数的个数,$k_3$ 为满足 $n_i>1$ 且 $n_i$ 为奇数的个数。则 $f(x)$ 的{\color{red}{极值点}}个数为$\color{red}{k_1 + 2k_2 + k_3 - 1}$,{\color{blue}{拐点}}个数为$\color{blue}{k_1 + 2k_2 + 3k_3 - 2}$
    \end{circlenum}

    \subsection{渐近线}

    当曲线上的点{\color{red}{远离原点}}时,曲线与某直线充分靠近,则称该直线为曲线的渐近线

    \subsubsection{铅直渐近线}

    若$\lim\limits_{x\to x_0^{+}} f(x) = \infty$(或$\lim\limits_{x\to x_0^{-}} f(x) = \infty$),则$x=x_0$为一条铅直渐近线。$x_0$的可能情况为:
    \begin{enumerate}
        \item 函数的无定义点。
        \emph{e.g.}:函数 $y=\tan x$ 在 $x=\dfrac{\pi}{2}$ 处无定义。

        \item 函数定义区间的端点。
        \emph{e.g.}:函数 $y=\ln x\ (x>0)$ 在 $x\to 0^+$ 处。

        \item 分段函数的分段点。
        \emph{e.g.}:
        \[
            f(x)=
            \begin{cases}
                1, & x \ge 0,\\[4pt]
                \dfrac{1}{x}, & x < 0,
            \end{cases}
        \]
        在 $x=0$ 处为分段点。
    \end{enumerate}

    \subsubsection{水平渐近线}

    \begin{enumerate}
        \item 若 $\lim\limits_{x\to +\infty} f(x)=y_1$,则 $y=y_1$ 为函数 $y=f(x)$ 的一条水平渐近线;
        \item 若 $\lim\limits_{x\to -\infty} f(x)=y_2$,则 $y=y_2$ 为函数 $y=f(x)$ 的一条水平渐近线;
        \item 若 $\lim\limits_{x\to +\infty} f(x) = \lim\limits_{x\to -\infty} f(x) = y_0$,则 $y=y_0$ 为函数 $y=f(x)$ 的一条水平渐近线
    \end{enumerate}

    \subsubsection{斜渐近线}

    \begin{enumerate}
        \item 若 $\lim\limits_{x\to +\infty}\dfrac{f(x)}{x}=a_1\,(a_1\neq 0)$,
        且 $\lim\limits_{x\to +\infty}\bigl[f(x)-a_1x\bigr]=b_1$,
        则直线 $y=a_1x+b_1$ 是曲线 $y=f(x)$ 的一条斜渐近线。

        \item 若 $\lim\limits_{x\to -\infty}\dfrac{f(x)}{x}=a_2\,(a_2\neq 0)$,
        且 $\lim\limits_{x\to -\infty}\bigl[f(x)-a_2x\bigr]=b_2$,
        则直线 $y=a_2x+b_2$ 是曲线 $y=f(x)$ 的一条斜渐近线。

        \item 若 $\lim\limits_{x\to +\infty}\dfrac{f(x)}{x}
        =\lim\limits_{x\to -\infty}\dfrac{f(x)}{x}=a\,(a\neq 0)$,
        且 $\lim\limits_{x\to +\infty}\bigl[f(x)-ax\bigr]
        =\lim\limits_{x\to -\infty}\bigl[f(x)-ax\bigr]=b$,
        则直线 $y=ax+b$ 是曲线 $y=f(x)$ 的一条斜渐近线。
    \end{enumerate}

    \subsubsection{方法总结}

    \begin{enumerate}
        \item 寻找渐近线顺序:铅直渐近线、水平渐近线、斜渐近线
        \item 求斜渐近线时,$a$与$b$均应求出来才可能,仅求出$a$不能确定有斜渐近线
        \item $\color{red}{\bigstar}$ 曲线与渐近线可能会有交点。\emph{e.g.} $y = \frac{\sin x}{x}$
    \end{enumerate}

    \begin{center}
        \begin{tikzpicture}
            [
            scale=0.75,
            transform shape,
            node distance=1.6cm,
            every node/.style={font=\footnotesize},
            startstop/.style={
                rectangle, rounded corners,
                draw, align=center,
                minimum width=2.6cm, minimum height=0.7cm
            },
            process/.style={
                rectangle,
                draw, align=center,
                minimum width=3.6cm, minimum height=0.8cm
            },
            decision/.style={
                diamond,
                draw, align=center,
                aspect=2.2,
                inner sep=1pt
            },
            arrow/.style={->, thick}
            ]

            \node (f) [startstop] {$y=f(x)$};

            \node (find) [process, below=of f]
            {找无定义点、端点、分段点 $x_0$};

            \node (vertcheck) [decision, below=of find]
            {$\lim\limits_{x\to x_0} f(x)=\pm\infty$};

            \node (vertyes) [process, below=of vertcheck]
            {铅直渐近线 $x=x_0$};

            \node (vertno) [process, left=2.2cm of vertcheck]
            {无铅直渐近线};

            \node (horizcheck) [decision, right=2.2cm of f]
            {$\lim\limits_{x\to\infty} f(x)=A$};

            \node (horizyes) [process, right=2.2cm of horizcheck]
            {水平渐近线 $y=A$};

            \node (slopecheck) [decision, below=of horizcheck]
            {$\lim\limits_{x\to\infty}\dfrac{f(x)}{x}=a\ (a\neq0)$};

            \node (slopeno) [process, right=2.2cm of slopecheck]
            {无斜渐近线};

            \node (bcheck) [process, below=of slopecheck]
            {$b=\lim\limits_{x\to\infty}[f(x)-ax]$};

            \node (bno) [process, right=2.2cm of bcheck]
            {无斜渐近线};

            \node (byes) [process, below=of bcheck]
            {斜渐近线 $y=ax+b$};

            \draw [arrow] (f) -- node[ left=3pt, draw=red, circle, text=red, inner sep=1.2pt, font=\small ]{1} (find);
            \draw [arrow] (find) -- (vertcheck);
            \draw [arrow] (vertcheck) -- node[right]{是} (vertyes);
            \draw [arrow] (vertcheck) -- node[above]{否} (vertno);

            \draw [arrow] (f) -- node[ above=3pt, draw=red, circle, text=red, inner sep=1.2pt, font=\small ]{2} (horizcheck);
            \draw [arrow] (horizcheck) -- node[above]{是} (horizyes);
            \draw [arrow] (horizcheck) --
            node[right]{否}
            node[right=16pt,
            draw=red, circle, text=red,
            inner sep=1.2pt, font=\small]{3}
            node[right=28pt]{\color{red}{A = $\infty$}}
            (slopecheck);

            \draw [arrow] (slopecheck) -- node[above]{否} (slopeno);
            \draw [arrow] (slopecheck) -- node[right]{是} (bcheck);

            \draw [arrow] (bcheck) -- node[above]{不存在} (bno);
            \draw [arrow] (bcheck) -- node[right]{存在} (byes);

        \end{tikzpicture}
    \end{center}

    \subsection{最值或取值范围}

    \subsubsection{最值的定义}

    设 $x_0$ 为函数 $f(x)$ 定义域内的一点。若对于 $f(x)$ 的定义域内任意一点 $x$,均有
    \[
        f(x) \leq f(x_0)\quad (\text{或 } f(x) \geq f(x_0))
    \]
    成立,则称 $f(x_0)$ 为函数 $f(x)$ 的最大值(或最小值)。

    \subsubsection{求区间$[a, b]$上连续函数$f(x)$的最大值$M$和最小值$m$}

    \begin{circlenum}
        \item 求出函数 $f(x)$ 在 $(a,b)$ 内的可疑点——驻点与不可导点,并求出这些可疑点处的函数值;
        \item 求出端点处的函数值 $f(a)$ 和 $f(b)$;
        \item 比较以上所求得的所有函数值,其中最大者为 $f(x)$ 在 $[a,b]$ 上的最大值 $M$,最小者为 $f(x)$ 在 $[a,b]$ 上的最小值 $m$
    \end{circlenum}

    \subsubsection{求区间$(a, b)$内连续函数$f(x)$的最值或取值范围}

    \begin{circlenum}
        \item 求出函数 $f(x)$ 在 $(a,b)$ 内的可疑点——驻点与不可导点,并求出这些可疑点处的函数值;
        \item 求区间 $(a,b)$ 两端的单侧极限:若 $a,b$ 为有限常数,则求 $\lim_{x\to a^+} f(x)$ 与 $\lim_{x\to b^-} f(x)$;若 $a=-\infty$,则求 $\lim_{x\to -\infty} f(x)$;若 $b=+\infty$,则求 $\lim_{x\to +\infty} f(x)$,记以上所求左端极限为 $A$,右端极限为 $B$;
        \item 比较前两步所得结果,确定函数的最值或取值范围。
    \end{circlenum}

    \subsection{作函数图像}

    \subsubsection{给出函数 $f(x)$,作图的一般步骤}

    \begin{circlenum}
        \item 确定定义域,考查函数是否具有奇偶性、周期性,并合理利用图像变换;
        \item 利用导数工具:一阶导数确定函数的单调区间与极值点,二阶导数确定曲线的凹凸区间与拐点;
        \item 考查函数的渐近线;
        \item 作出函数图像
    \end{circlenum}

    \subsubsection{常用平面图像}

    \begin{enumerate}
        \item 心形线
        \begin{tikzpicture}
            \begin{axis}[
                axis equal,
                axis lines=middle,
                xmin=-2.5, xmax=2.5,
                ymin=-2.5, ymax=2.5,
                samples=600,
                xtick=\empty, ytick=\empty, % 去掉坐标刻度
                legend style={
                    at={(0.02,0.98)},
                    anchor=north west,
                    xshift=120pt,
                    yshift=0pt,
                    draw=none
                }
            ]
                % r = 1 - cos θ
                \addplot[thick,domain=0:6.283]
                ({(1 - cos(deg(x)))*cos(deg(x))}, {(1 - cos(deg(x)))*sin(deg(x))});
                \addlegendentry{ $r = 1 - \cos\theta$ }
            \end{axis}
        \end{tikzpicture}

        \item 心形线
        \begin{tikzpicture}
            \begin{axis}[
                axis equal,
                axis lines=middle,
                xmin=-2.5, xmax=2.5,
                ymin=-2.5, ymax=2.5,
                samples=600,
                xtick=\empty, ytick=\empty,
                legend style={
                    at={(0.02,0.98)},
                    anchor=north west,
                    xshift=120pt,
                    yshift=0pt,
                    draw=none
                }
            ]
                % r = 1 + cos θ
                \addplot[thick,domain=0:6.283]
                ({(1 + cos(deg(x)))*cos(deg(x))}, {(1 + cos(deg(x)))*sin(deg(x))});
                \addlegendentry{ $r = 1 + \cos\theta$ }
            \end{axis}
        \end{tikzpicture}

        \item 心形线
        \begin{tikzpicture}
            \begin{axis}[
                axis equal,
                axis lines=middle,
                xmin=-2.5, xmax=2.5,
                ymin=-2.5, ymax=2.5,
                samples=600,
                xtick=\empty, ytick=\empty,
                legend style={
                    at={(0.02,0.98)},
                    anchor=north west,
                    xshift=120pt,
                    yshift=0pt,
                    draw=none
                }
            ]
                % r = 1 - sin θ
                \addplot[thick,domain=0:6.283]
                ({(1 - sin(deg(x)))*cos(deg(x))}, {(1 - sin(deg(x)))*sin(deg(x))});
                \addlegendentry{ $r = 1 - \sin\theta$ }
            \end{axis}
        \end{tikzpicture}

        \item 心形线
        \begin{tikzpicture}
            \begin{axis}[
                axis equal,
                axis lines=middle,
                xmin=-2.5, xmax=2.5,
                ymin=-2.5, ymax=2.5,
                samples=600,
                xtick=\empty, ytick=\empty,
                legend style={
                    at={(0.02,0.98)},
                    anchor=north west,
                    xshift=120pt,
                    yshift=0pt,
                    draw=none
                }
            ]
                % r = 1 + sin θ
                \addplot[thick,domain=0:6.283]
                ({(1 + sin(deg(x)))*cos(deg(x))}, {(1 + sin(deg(x)))*sin(deg(x))});
                \addlegendentry{ $r = 1 + \sin\theta$ }
            \end{axis}
        \end{tikzpicture}
        \item 摆线(平摆线)TODO
        \begin{tikzpicture}
            \begin{axis}
                [
                axis equal,
                axis lines=middle,
                samples=400,
                xlabel={$x$},
                ylabel={$y$}
                ]
                \addplot[thick,domain=0:6.28]
                ({x - sin(deg(x))},{1 - cos(deg(x))});
            \end{axis}
        \end{tikzpicture}

        \item 星形线
    \end{enumerate}

    \subsubsection{直角坐标系的观点画极坐标的图}

    \begin{enumerate}
        \item 直角坐标系的观点下,视$\theta$为$x$,$\gamma$为$y$,即可画出图像
        \item 用{\color{red}{描点法(看变化趋势)}}可画出其在极坐标下的图像
    \end{enumerate}

    \subsection{曲率及曲率半径}

    设函数 $y(x)$ 二阶可导,则曲线 $y=y(x)$ 在点 $(x,\,y(x))$ 处的{\blueuline{曲率(表示曲线弯曲程度)}}公式为
    \[
        k=\frac{|y''|}{\left[1+\bigl(y'\bigr)^2\right]^{\tfrac{3}{2}}}.
    \]

    曲率半径的计算公式为
    \[
        R=\frac{1}{k}
        =\frac{\left[1+\bigl(y'\bigr)^2\right]^{\tfrac{3}{2}}}{|y''|}. \; y'' \neq 0
    \]

        {\color{blue}{弯曲程度越大,曲率越大,曲率圆的半径越小}}

    \subsection{结论}

    \begin{enumerate}
        \item 如$\alpha$是$f(x) = 0$的$m(\geq 1)$重根,则$\alpha$是$f'(x)$的$m-1$重根
        \item 如果 $f(x)$ 在区间$I$上有最值点 $x_0$,并且此最值点 $x_0$ 不是区间 $I$ 的端点而是$I$内部的点,那么此 $x_0$必是$f(x)$的一个极值点
    \end{enumerate}

    \subsection{条件转换思路}

    \begin{enumerate}
        \item $f(x)\cdot f'(x)$ => $\{[f(x)]^2\}' = 2f(x) \cdot f'(x)$
    \end{enumerate}

\end{document}
