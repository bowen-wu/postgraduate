\documentclass[a4paper,12pt]{article}
\usepackage{xeCJK}          % 中文支持
\usepackage{fontspec}       % 英文/数学字体
\usepackage{amsmath, amssymb} % 数学公式
\usepackage{graphicx}       % 插入图片
\usepackage{hyperref}       % 目录超链接
\usepackage{geometry}       % 页面布局
\usepackage{bm}             % 粗体
\usepackage{xcolor}         % 颜色
\usepackage{tabularx}       % 表格环境
\usepackage{tikz}           % TikZ 绘制主对角线斜线
\usepackage{tcolorbox}
\usepackage{xstring}
\usepackage{pgfplots}
\usepackage{enumitem}
\usepackage{pifont}
\usepackage{ulem}
\pgfplotsset{compat=1.18}
\geometry{left=3cm,right=3cm,top=3cm,bottom=3cm}

% 抽离颜色和尺寸参数
\newcommand{\analysisTitleColor}{green!50!black}
\newcommand{\analysisBackColor}{white}
\newcommand{\analysisBoxRule}{0.8pt}
\newcommand{\analysisArc}{3pt}
\newcommand{\analysisPadding}{6pt}

% 定义 tcolorbox
\newtcolorbox{analysisbox}[1][]{
    title=\IfStrEq{#1}{}{\textbf{解析}}{#1}, % 如果传参为空则使用“解析”
    colback=\analysisBackColor,
    colframe=\analysisTitleColor,
    boxrule=\analysisBoxRule,
    arc=\analysisArc,
    left=\analysisPadding,
    right=\analysisPadding,
    top=4pt,
    bottom=4pt
}

\newcommand{\blueuline}[1]{{\color{blue}\uline{\color{black}{#1}}}}

\newlist{circlenum}{enumerate}{1}
\setlist[circlenum]{
    label=\ding{\numexpr171+\arabic*},
    leftmargin=2.2em,
    itemsep=0.4em
}

% =========================
% 字体设置
% =========================
\setmainfont{Times New Roman}
\setsansfont{Helvetica Neue}
\setmonofont{Menlo}
\setCJKmainfont{PingFang SC}

% =========================
% 图形路径(可调整)
% =========================
\graphicspath{{./assets/}}

% =========================
% 文档开始
% =========================
\begin{document}

%    \title{Template}
%    \author{Bowen}
%    \date{\today}
%    \maketitle
%    \tableofcontents
%    \newpage


% =========================

    \section{一元函数微分学的应用 - 几何应用}

    \subsection{极值的定义}
    \begin{enumerate}
        \item 对于函数 $f(x)$,若存在点 $x_0$ 的 {\color{red}{某个邻域}},
        使得在该邻域内任意一点 $x$,均有
        \[
            f(x) \le f(x_0)\quad (\text{或 } f(x) \ge f(x_0))
        \]
        成立,则称点 $x_0$ 为 $f(x)$ 的
        \textbf{极大值点}(或 \textbf{极小值点}),
        $f(x_0)$ 为 $f(x)$ 的极大值(或极小值)。

        \item 极值是一个\textbf{局部}的概念。

        \item 极值要求点 $x_0$ 的左右邻域均有定义,
        \textbf{端点处不讨论极值,间断点不可能是极值点}。
    \end{enumerate}

    \subsection{单调性与极值的判别}

    \subsubsection{单调性的判别}
    \begin{enumerate}
        \item 设函数 $y = f(x)$ 在 $[a,b]$ 上连续,在 $(a,b)$ 内可导。

        \begin{circlenum}
            \item 若在 $(a,b)$ 内有 $f'(x) \ge 0$,
            且等号仅{\blueuline{在有限个点处成立}},则函数 $y = f(x)$ 在 $[a,b]$ 上
                {\color{red}{严格单调增加}}。

            \item 若在 $(a,b)$ 内有 $f'(x) \le 0$,
            且等号仅 {\blueuline{在有限个点处成立}},则函数 $y = f(x)$ 在 $[a,b]$ 上
                {\color{red}{严格单调减少}}

            \item 导数为$0$仅能说明在某点处的函数值变化\textbf{充分小},而不能说明没变化
        \end{circlenum}
    \end{enumerate}

    \subsubsection{一阶可导点是极值点的必要条件}

    \begin{enumerate}
        \item 设$f(x)$在$x = x_0$处可导,且在点$x_0$处取得极值,则必有$f'(x_0) = 0$
        \item 若$x = x_0$为曲线$y = f(x)$的极值点,则只有以下两种情况
        \begin{circlenum}
            \item 驻点: $f'(x_0) = 0$
            \item 不可导点: $f'(x_0)$不存在。如$y = |x|$在$(0, 0)$处的情形
        \end{circlenum}
        \item 找极值的两种情况: 驻点 | 不可导点
    \end{enumerate}

    \subsubsection{判别极值的第一充分条件}
    \begin{theorem}
        设函数 $f(x)$ 在 $x=x_0$ 处连续,且在 $x_0$ 的某去心邻域
        $U(x_0,\delta)\ (\delta>0)$ 内可导,则:

        \begin{circlenum}
            \item 若当 $x\in(x_0-\delta,x_0)$ 时,$f'(x)<0$,而当 $x\in(x_0,x_0+\delta)$ 时,$f'(x)>0$,则 $f(x)$ 在 $x=x_0$ 处取得极小值

            \item 若当 $x\in(x_0-\delta,x_0)$ 时,$f'(x)>0$,而当 $x\in(x_0,x_0+\delta)$ 时,$f'(x)<0$,则 $f(x)$ 在 $x=x_0$ 处取得极大值

            \item 若 $f'(x)$ 在 $(x_0-\delta,x_0)$ 与 $(x_0,x_0+\delta)$ 内不变号,则点 $x_0$ 不是极值点

            \item $f(x)$在$x = x_0$处不一定可导,可能出现角点
        \end{circlenum}
    \end{theorem}

    \subsubsection{判别极值的第二充分条件}

    \begin{theorem}
        设函数 $f(x)$ 在 $x=x_0$ 处二阶可导,且 $f'(x_0)=0$。
        \begin{circlenum}
            \item 若 $f''(x_0)<0$,则 $f(x)$ 在 $x=x_0$ 处取得极大值
            \item 若 $f''(x_0)>0$,则 $f(x)$ 在 $x=x_0$ 处取得极小值
            \item 若 $f''(x_0)=0$,则该判别法失效,需借助高阶导数或其他方法判断。如$f(x) = x^4$在$x_0 = 0$处是极小值, $f(x) = x^3$在$x_0 = 0$处是拐点,不是极值点
        \end{circlenum}
    \end{theorem}

    \subsubsection{判别极值的第三充分条件}

    \begin{theorem}
        设函数 $f(x)$ 在 $x=x_0$ 处 $n$ 阶可导,且$f^{(m)}(x_0)=0 (m=1,2,\ldots,n-1), f^{(n)}(x_0)\neq 0 \ (n\ge 2)$,则:
        \begin{circlenum}
            \item 若 $n$ 为偶数且 $f^{(n)}(x_0)<0$,则 $f(x)$ 在 $x=x_0$ 处取得极大值
            \item 若 $n$ 为偶数且 $f^{(n)}(x_0)>0$,则 $f(x)$ 在 $x=x_0$ 处取得极小值
        \end{circlenum}
        当 $n$ 为奇数时,点 $x_0$ 不是极值点。
    \end{theorem}

    \subsection{凹凸性与拐点的概念}

    \subsubsection{凹凸性的定义}
    \begin{enumerate}
        \item \begin{definition}
                  设函数 $f(x)$ 在区间 $I$ 上连续。
                  \begin{enumerate}
                      \item 若对 $I$ 上任意不同的两点 $x_1,x_2$,恒有
                      \[
                          f\!\left(\frac{x_1+x_2}{2}\right) < \frac{f(x_1)+f(x_2)}{2},
                      \]
                      则称函数 $y=f(x)$ 在区间 $I$ 上的图形是\textbf{凹的}(或称凹弧)

                      \item 若对 $I$ 上任意不同的两点 $x_1,x_2$,恒有
                      \[
                          f\!\left(\frac{x_1+x_2}{2}\right) > \frac{f(x_1)+f(x_2)}{2},
                      \]
                      则称函数 $y=f(x)$ 在区间 $I$ 上的图形是\textbf{凸的}(或称凸弧)。
                      \item 广义化:
                      \[
                          f(\lambda_1 x_1 + \lambda_2 x_2)
                          \begin{cases}
                              < \lambda_1 f(x_1) + \lambda_2 f(x_2), & \text{凹函数},\\
                              > \lambda_1 f(x_1) + \lambda_2 f(x_2), & \text{凸函数},
                          \end{cases}
                      \]
                      其中 $0 < \lambda_1,\lambda_2 < 1$,且 $\lambda_1+\lambda_2=1$。
                  \end{enumerate}
        \end{definition}
        \item 设函数 $f(x)$ 在区间 $[a,b]$ 上连续,在 $(a,b)$ 内可导。

        \begin{itemize}
            \item 若对任意 $x_0\in(a,b)$ 及任意 $x\in(a,b)$ 且 $x\neq x_0$,恒有
            \[
                f(x) < f(x_0) + f'(x_0)(x - x_0),
            \]
            即除切点外,曲线严格位于其切线的下方,
            则称函数 $y=f(x)$ 在 $[a,b]$ 上是\textbf{凹的}。

            \item 若对任意 $x_0\in(a,b)$ 及任意 $x\in(a,b)$ 且 $x\neq x_0$,恒有
            \[
                f(x) > f(x_0) + f'(x_0)(x - x_0),
            \]
            即除切点外,曲线严格位于其切线的上方,
            则称函数 $y=f(x)$ 在 $[a,b]$ 上是\textbf{凸的}。
        \end{itemize}
    \end{enumerate}

    \subsubsection{拐点的定义}

    连续曲线的凹弧与凸弧的分界点称为该曲线的\textbf{拐点}。

    \begin{circlenum}
        \item 间断点不可能为拐点,拐点处函数\textbf{必须连续}。

        \item 判别拐点时,只需判断凹凸性的变化,\textbf{凹与凸不分先后}。

        \item 极值点只写横坐标 $x=x_0$;而拐点应写$\bigl(x_0,\; f(x_0)\bigr)$,拐点在曲线上
    \end{circlenum}

    \subsection{凹凸性与拐点的判别}

    \subsubsection{判别凹凸性}

    设函数 $f(x)$ 在区间 $I$ 上二阶可导,则:

    \begin{circlenum}
        \item 若在 $I$ 上 $f''(x) > 0$,则 $f(x)$ 在 $I$ 上的图形是\textbf{凹的}(凹弧)。

        \item 若在 $I$ 上 $f''(x) < 0$,则 $f(x)$ 在 $I$ 上的图形是\textbf{凸的}(凸弧)。
    \end{circlenum}

    \subsubsection{二阶可导点是拐点的必要条件}

    设 $f''(x_0)$ 存在,且点 $(x_0,\,f(x_0))$ 为曲线的拐点,则必有 $f''(x_0)=0$。若点 $(x_0,\,f(x_0))$ 是曲线 $y = f(x)$ 的拐点,则只有以下两种情况

    \begin{enumerate}
        \item 二阶导数存在必为$0$: $f''(x_0)=0$。如 $y=x^3$ 在 $(0,0)$ 处的情形
        \item 二阶导数不存在的点也有可能是拐点: $f''(x_0)$ 不存在。如 $y=\sqrt[3]{x}$ 在 $(0,0)$ 处的情形
    \end{enumerate}

    \subsubsection{判别拐点的第一充分条件}

    设函数 $f(x)$ 在点 $x=x_0$ 处连续,在点 $ x = x_0$ 的某去心邻域$U(x_0,\delta)\ (\delta>0)$ 内二阶导数存在,且在 $x_0$ 的左、右邻域内$f''(x)$ 变号(无论是由正变负,还是由负变正),则点 $(x_0,f(x_0))$ 为曲线的拐点

    \subsubsection{判别拐点的第二充分条件}

    设函数 $f(x)$ 在 $x=x_0$ 处三阶可导,且 $f''(x_0)=0$,$f'''(x_0)\neq 0$, 则点 $(x_0,f(x_0))$ 为曲线的拐点

    \subsubsection{判别拐点的第三充分条件}

    设函数 $f(x)$ 在 $x=x_0$ 处 $n$ 阶可导,且 $f^{(m)}(x_0)=0\ (m=2,3,\dots,n-1)$,$f^{(n)}(x_0)\neq 0$,则当 $n$ 为奇数时,点 $(x_0,f(x_0))$ 为曲线的拐点。

    \subsection{基础概念}

    \begin{enumerate}
    \end{enumerate}

    \subsection{结论}

    \begin{enumerate}
    \end{enumerate}

    \subsection{定理}

    \begin{enumerate}
    \end{enumerate}

    \subsection{运算}

    \begin{enumerate}

    \end{enumerate}

    \subsection{公式}

    \begin{enumerate}

    \end{enumerate}

    \subsection{方法总结}

    \begin{enumerate}

    \end{enumerate}

    \subsection{条件转换思路}

    \begin{enumerate}
        \item $f(x)\cdot f'(x)$ => $\{[f(x)]^2\}' = 2f(x) \cdot f'(x)$

    \end{enumerate}

    \subsection{理解}

    \begin{enumerate}
    \end{enumerate}

\end{document}
