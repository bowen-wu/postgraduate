\documentclass[a4paper,12pt]{article}
\usepackage{xeCJK}          % 中文支持
\usepackage{fontspec}       % 英文/数学字体
\usepackage{amsmath, amssymb} % 数学公式
\usepackage{graphicx}       % 插入图片
\usepackage{hyperref}       % 目录超链接
\usepackage{geometry}       % 页面布局
\usepackage{bm}             % 粗体
\usepackage{xcolor}         % 颜色
\usepackage{tabularx}       % 表格环境
\usepackage{tikz}           % TikZ 绘制主对角线斜线
\usepackage{tcolorbox}
\usepackage{xstring}
\usepackage{pgfplots}
\usepackage{enumitem}
\usepackage{pifont}
\usepackage{ulem}
\pgfplotsset{compat=1.18}
\geometry{left=3cm,right=3cm,top=3cm,bottom=3cm}

% 抽离颜色和尺寸参数
\newcommand{\analysisTitleColor}{green!50!black}
\newcommand{\analysisBackColor}{white}
\newcommand{\analysisBoxRule}{0.8pt}
\newcommand{\analysisArc}{3pt}
\newcommand{\analysisPadding}{6pt}

% 定义 tcolorbox
\newtcolorbox{analysisbox}[1][]{
    title=\IfStrEq{#1}{}{\textbf{解析}}{#1}, % 如果传参为空则使用“解析”
    colback=\analysisBackColor,
    colframe=\analysisTitleColor,
    boxrule=\analysisBoxRule,
    arc=\analysisArc,
    left=\analysisPadding,
    right=\analysisPadding,
    top=4pt,
    bottom=4pt
}

\newlist{circlenum}{enumerate}{1}
\setlist[circlenum]{
    label=\ding{\numexpr171+\arabic*},
    leftmargin=2.2em,
    itemsep=0.6em,      % item 之间的距离(主要)
    topsep=0.6em,       % 列表与上下正文的距离
    parsep=0.3em,       % item 内段落的间距
    partopsep=0.3em     % 列表前后额外间距
}

\newcommand{\blueuline}[1]{{\color{blue}\uline{\color{black}{#1}}}}

% =========================
% 字体设置
% =========================
\setmainfont{Times New Roman}
\setsansfont{Helvetica Neue}
\setmonofont{Menlo}
\setCJKmainfont{PingFang SC}

% =========================
% 图形路径(可调整)
% =========================
\graphicspath{{./assets/}}

% =========================
% 文档开始
% =========================
\begin{document}

%    \title{Template}
%    \author{Bowen}
%    \date{\today}
%    \maketitle
%    \tableofcontents
%    \newpage


% =========================

    \section{一元函数微分学的应用 - 中值定理}

    \subsection{涉及函数的中值定理}

    \subsubsection{有界与最值定理}

    设 $f(x)$ 在 $[a, b]$上连续,则 $m \leq f(x) \leq M$,其中$m, M$分别为$f(x)$在$[a, b]$上的最小值和最大值

    \subsubsection{介值定理}

    设 $f(x)$ 在 $[a, b]$上连续,当$m \leq \mu \leq M$时,存在$\xi \in [a, b]$,使得$f(\xi) = \mu$

    \subsubsection{平均值定理}

    设 $f(x)$ 在 $[a, b]$上连续,当$a < x_1 < x_2 < \cdots < x_n < b$时,在$[x_1, x_n]$内至少存在一点$\xi$,使得
    \[
        f(\xi) = \frac{f(x_1) + f(x_2) + \cdots + f(x_n)}{n}
    \]

    \subsubsection{零点定理(介质定理的特例)}

    设 $f(x)$ 在 $[a, b]$上连续,当$f(a)\cdot f(b) < 0$时,存在$\xi \in (a, b)$,使得$f(\xi) = 0$
    \begin{itemize}
        \item 推广的零点定理: 若$f(x)$在$(a, b)$内连续,$\lim\limits_{x \to a^+} f(x) = \alpha, \lim\limits_{x \to b^-} f(x) = \beta$,且$\alpha \cdot \beta < 0$,则$f(x) = 0$在$(a, b)$内至少存在一个根
    \end{itemize}

    \subsection{涉及导数(微分)的中值定理}

    \subsubsection{费马定理}

    设函数 $f(x)$ 在点 $x_0$ 处满足
    \[
        \begin{cases}
            \text{(1) 可导(左右导数存在且相等)},\\
            \text{(2) 取极值 },
        \end{cases}
    \]
    则
    \[
        f'(x_0)=0.
    \]

    \subsubsection{导数零点定理}

    $f(x)$在$[a, b]$上可导,当$f'_{+}(a) \cdot f'_{-}(b) < 0$时,存在$\xi \in (a, b)$,使得$f'(\xi) = 0$
    \\

    \subsubsection{罗尔定理}

    设$f(x)$满足
    \[
        \begin{cases}
            \text{(1) 在} [a, b] \text{上连续},\\
            \text{(2) 在} (a, b) \text{内可导},\\
            \text{(3) }f(a) = f(b),
        \end{cases}
    \]
    则存在 $\xi \in (a, b)$,使得$f'(\xi) = 0$
    \begin{enumerate}
        \item 推广的罗尔定理: 设$f(x)$在$(a, b)$内可导,$\lim\limits_{x \to a^+} f(x) = \lim\limits_{x \to b^-} f(x) = A$,则在$(a, b)$内至少存在一点$\xi$,使得$f'(\xi) = 0$
        \item 罗尔定理的使用需要构造{\color{red}{辅助函数}},其方法总结如下:
        \begin{itemize}
            \item 乘积求导公式 $(uv)' = u'v + uv'$ 的逆用
            \begin{circlenum}
                \item $2f(x)\cdot f'(x) = [f^2(x)]' = [f(x)f(x)]' = F'(x)$
                \item $[f'(x)]^2 + f(x)f''(x) = [f(x)\cdot f'(x)]' = F'(x)$
                \item {\setlength{\baselineskip}{1.4\baselineskip} $[f'(x) + f(x)\varphi'(x)]e^{\varphi(x)} = f'(x)e^{\varphi(x)} + f(x)e^{\varphi(x)}\cdot \varphi'(x) = [f(x)e^{\varphi(x)}]'$ $\Rightarrow$见到$f'(x) + f(x)\varphi'(x)$,令$F(x) = f(x)e^{\varphi(x)}$。如:
                \begin{enumerate}
                    \item $\varphi(x) = x \Rightarrow$ 见到$f'(x) + f(x)$,令$F(x) = f(x)e^x$
                    \item $\varphi(x) = -x \Rightarrow$ 见到$f'(x) - f(x)$,令$F(x) = f(x)e^{-x}$
                    \item $\varphi(x) = kx \Rightarrow$ 见到$f'(x) + kf(x)$,令$F(x) = f(x)e^{kx}$
                \end{enumerate}
                \item $(uv)'' = u''v + 2u'v' + uv''$
            \end{circlenum}
            \item 商的求导公式$(\displaystyle\frac{u}{v})' = \displaystyle\frac{u'v - uv'}{v^2}$
            \begin{circlenum}
                \item $\displaystyle\frac{f'(x)x - f(x)}{x^2} = [\displaystyle\frac{f(x)}{x}]'$ \\ $\Rightarrow$见到$f'(x)x - f(x), x \neq 0$,令$F(x) = \displaystyle\frac{f(x)}{x}$
                \item $\displaystyle\frac{f''(x)f(x) - [f'(x)]^2}{f^{2}(x)} = [\displaystyle\frac{f'(x)}{f(x)}]'$ \\ $\Rightarrow$见到$f''(x)f(x) - [f'(x)]^2, f(x) \neq 0$,令$F(x) = \displaystyle\frac{f'(x)}{f(x)}$
                \item $\displaystyle\frac{f'(x)}{f(x)} = [\ln f(x)]'$
                \item {\setlength{\baselineskip}{1.4\baselineskip} $\displaystyle\frac{f''(x)f(x) - [f'(x)]^2}{f^2(x)} = [\displaystyle\frac{f'(x)}{f(x)}]' = [\ln f(x)]''$ \\ $\Rightarrow$见到 $f''(x)f(x) - [f'(x)]^2, f(x) > 0$,可考虑令$F(x) = \ln f(x)$
            \end{circlenum}
        \end{itemize}
    \end{enumerate}

    \subsubsection{拉格朗日中值定理}

    设函数 $f(x)$ 满足
    \[
        \begin{cases}
            \text{(1) 在} [a, b] \text{上连续},\\
            \text{(2) 在} (a, b) \text{内可导}
        \end{cases}
    \]
    则存在$\xi \in (a, b)$,使得
    \[
        f(b) - f(a) = f'(\xi)(b - a)
    \]
    或者写成
    \[
        f'(\xi) = \frac{f(b) - f(a)}{b - a}
    \]

    \begin{enumerate}
        \item 见到$f(a) - f(b)$与$f$与$f'$的关系,一般想到用拉格朗日中值
        \item 拉格朗日中值定理的作用是{\color{red}{用导函数的值来控制函数值的增减}}
    \end{enumerate}

    \subsubsection{柯西中值定理}

    设函数 $f(x), g(x)$ 满足
    \[
        \begin{cases}
            \text{(1) 在} [a, b] \text{上连续},\\
            \text{(2) 在} (a, b) \text{内可导},\\
            \text{(3) } g'(x) \neq 0 \\
        \end{cases}
    \]
    则存在$\xi \in (a, b)$,使得
    \[
        \frac{f(b) - f(a)}{g(b) - g(a)} = \frac{f'(\xi)}{g'(\xi)}
    \]

    \begin{enumerate}
        \item $f(x), g(x)$往往考察一个具体函数,一个抽象函数
    \end{enumerate}

    \subsubsection{泰勒公式(微分中值定理)}

    \begin{enumerate}
        \item 带拉格朗日余项的 $n$ 阶泰勒公式:设函数 $f(x)$ 在{\blueuline{点 $x_0$ 的某邻域内 $n+1$ 阶导数存在}},则对该邻域内任意点 $x$,有
        \[
            f(x)
            = f(x_0) + f'(x_0)(x-x_0)
            + \frac{f''(x_0)}{2!}(x-x_0)^2
            + \cdots
            + \frac{1}{n!}f^{(n)}(x_0)(x-x_0)^n
            + R_n(x),
        \]
        其中拉格朗日余项
        \[
            R_n(x)
            = \frac{f^{(n+1)}(\xi)}{(n+1)!}(x-x_0)^{n+1},
            \quad \color{red}{\xi \in (x_0, x)}.
        \]

        \item 带佩亚诺余项的 $n$ 阶泰勒公式:设函数 $f(x)$ 在{\blueuline{点 $x_0$ 处 $n$ 阶可导}},则存在$x_0$的一个邻域,对于该邻域内的任意点 $x$,有
        \[
            f(x)
            = f(x_0) + f'(x_0)(x-x_0)
            + \frac{f''(x_0)}{2!}(x-x_0)^2
            + \cdots
            + \frac{f^{(n)}(x_0)}{n!}(x-x_0)^n
            + o\!\left((x-x_0)^n\right).
        \]
        \item 当 $x_0 = 0$时泰勒公式称为\textbf{麦克劳林公式}
        \begin{enumerate}
            \item \[
                      f(x)
                      = f(0)
                      + \frac{f'(0)}{1!}x
                      + \frac{f''(0)}{2!}x^2
                      + \cdots
                      + \frac{f^{(n)}(0)}{n!}x^n
                      + \frac{f^{(n+1)}(\xi)}{(n+1)!}x^{\,n+1}, \quad \xi \in (0, x)
            \]
            \item \[
                      f(x)
                      = f(0)
                      + \frac{f'(0)}{1!}x
                      + \frac{f''(0)}{2!}x^2
                      + \cdots
                      + \frac{f^{(n)}(0)}{n!}x^n
                      + o(x^n).
            \]
            \item 重要函数的麦克劳林展开式
            \begin{circlenum}
                \item 指数函数
                \[
                    e^x
                    = 1 + x + \frac{x^2}{2!} + \cdots + \frac{x^n}{n!} + o(x^n)
                \]

                \item 正弦函数
                \[
                    \sin x
                    = x - \frac{x^3}{3!} + \cdots + (-1)^n \frac{x^{2n+1}}{(2n+1)!}
                    + o\!\left(x^{2n+1}\right)
                \]

                \item 余弦函数$== \sin' x$
                \[
                    \cos x
                    = 1 - \frac{x^2}{2!} + \frac{x^4}{4!} \cdots + (-1)^n \frac{x^{2n}}{(2n)!}
                    + o\!\left(x^{2n}\right)
                \]

                \item 几何级数(一)
                \[
                    \frac{1}{1 - x}
                    = 1 + x + x^2 + \cdots + x^n + o(x^n)
                \]

                \item 几何级数(二)$== [\ln (1+x)]'$
                \[
                    \frac{1}{1 + x}
                    = 1 - x + x^2 - \cdots + (-1)^n x^n + o(x^n)
                \]

                \item 对数函数
                \[
                    \ln(1 + x)
                    = x - \frac{x^2}{2} + \frac{x^3}{3}
                    - \cdots + (-1)^{n-1} \frac{x^n}{n}
                    + o(x^n)
                \]

                \item 二项式展开
                \[
                    (1 + x)^a
                    = 1 + ax + \frac{a(a-1)}{2!}x^2
                    + \cdots +
                    \frac{a(a-1)\cdots(a-n+1)}{n!}x^n
                    + o(x^n)
                \]
            \end{circlenum}
        \end{enumerate}
        \item \noindent \textbf{说明:}
        \begin{circlenum}
            \item 带拉格朗日余项的 $n$ 阶泰勒公式{\color[rgb]{0.2, 0.6, 0.3}{适用于}}区间$[a, b]$,常在证明题中使用。如证不等式、中值等式等
            \item 带佩亚诺余项的 $n$ 阶泰勒公式{\color[rgb]{0.2, 0.6, 0.3}{适用于}}点$x = x_0$及其邻域,常用于研究点$x = x_0$处的某些结论。如求极限、判定无穷小的阶数、判定极值等
        \end{circlenum}
    \end{enumerate}

    \subsection{基础概念}

    \begin{enumerate}
    \end{enumerate}

    \subsection{结论}

    \begin{enumerate}
        \item 若$f(x)$在$(a, b)$内可导且$f'(x)$有界,则$f(x)$在$(a, b)$内有界
    \end{enumerate}

    \subsection{定理}

    \begin{enumerate}
    \end{enumerate}

    \subsection{运算}

    \begin{enumerate}

    \end{enumerate}

    \subsection{公式}

    \begin{enumerate}

    \end{enumerate}

    \subsection{方法总结}

    \begin{enumerate}
        \item 解题方法
        \begin{circlenum}
            \item 找定义式、关系式、约束式
            \item $\color{red}{\bigstar}$做一至两步逆运算
            \item 联想{\color{red}{经典}}形式
            \item 恒等变形
            \begin{itemize}
                \item $a = a - 0$
                \item $a = a + b - b$
                \item $e - 1 = e^1 - e^0$
            \end{itemize}
            \item 翻译数学名词
        \end{circlenum}
    \end{enumerate}

    \subsection{条件转换思路}

    \begin{enumerate}
        \item 函数$f(x)$,过点$A(0, f(0))$与点$B(1, f(1))$的直线与曲线$y = f(x)$相交于点$C(c, f(c)) \Rightarrow$ 构造辅助函数 $F(x) = f(x) - \ell(x), \ell(x)$ 是$A, B, C$ 三点的直线
        \item 若证$f^{(n)}(\xi) = 0$用罗尔定理可能性大
        \item 若证$f^{(n)}(\xi) \neq 0$(不等式)用泰勒公式可能性大
        \item 遇到$f$与$f^{(n)}(n \geq 2)$的关系,考虑泰勒公式
    \end{enumerate}

    \subsection{理解}

    \begin{enumerate}
        \item $f(x)$ 在 $x_0$ 处连续:
        \[
            \begin{aligned}
                &\lim_{x \to x_0} f(x) = f(x_0)
                &&\Rightarrow f(x) - f(x_0) = o(1) \\
                &&&\Rightarrow f(x) = f(x_0) + o(1)
            \end{aligned}
        \]

        \item $f(x)$ 在 $x_0$ 处可导:
        \[
            \begin{aligned}
                &f'(x_0)
                = \lim_{x \to x_0} \frac{f(x)-f(x_0)}{x-x_0}
                &&\Rightarrow
                \frac{f(x)-f(x_0)}{x-x_0}
                = f'(x_0) + o(1) \\
                &&&\Rightarrow
                f(x) = f(x_0) + f'(x_0)(x-x_0) + o(x-x_0)
            \end{aligned}
        \]

        \item $f(x)$ 在 $x_0$ 处 $n$ 阶可导:
        \[
            \begin{aligned}
                \Rightarrow
                f(x)
                = f(x_0) + f'(x_0)(x - x_0) + \displaystyle\frac{f''(x_0)}{2!}(x - x_0) + \cdots + \displaystyle\frac{f^{(n)}(x_0)}{n!}(x - x_0)^{n} + o((x - x_0)^n)
            \end{aligned}
        \]
    \end{enumerate}


    \section{一元函数微分学的应用 - 微分等式}

    方程$f(x) = 0$的根就是函数$f(x)$的零点。从几何上讲,方程的根作为两条曲线的交点。基于此,为讨论方程的根,有时可改为讨论曲线的交点。方法步骤如下

    \begin{enumerate}
        \item 讨论{\color{red}{存在性}}
        \begin{itemize}
            \item 观察,带入特殊值(0, 1, 端点),判断$f(x)$正负
            \item 使用零点定理: 若 $f(x)$ 在 $[a, b]$上连续,且$f(a)\cdot f(b) < 0$,则$f(x) = 0$在 $(a, b)$内至少有一个根
            \item $\color{red}{\bigstar}$实系数奇次方程至少有一个实根
        \end{itemize}
        \item 讨论{\color{red}{唯一性}}
        \begin{itemize}
            \item 单调性: 若$f(x)$在$(a, b)$内单调,则$f(x) = 0$在$(a, b)$内至多有一个根
            \item 罗尔定理的推论: 若$f^{(n)}(x) = 0$至多有$k$个根,则$f(x) = 0$至多有$k + n$个根
        \end{itemize}
    \end{enumerate}


    \section{一元函数微分学的应用 - 微分不等式}

    \subsection{用函数性态(包括单调性、凹凸性和最值等)证明不等式}

    \begin{enumerate}
        \item 若 $f'(x)\ge 0,\; a<x<b$,则$f(a)\leq f(x)\leq f(b)$
        \item 若有$f''(x) \geq 0, a < x < b$,则有$f'(a) \leq f'(x) \leq f'(b)$
        \begin{circlenum}
            \item 当$f'(a) > 0$时,$f'(x) > 0$,则$f(x)$单调增加
            \item 当$f'(b) < 0$时,$f'(x) < 0$,则$f(x)$单调减少
        \end{circlenum}
        \item 设函数 $f(x)$ 在区间 $I$ 内连续,且在 $I$ 内只有一个极值点 $x_0$,则
        \[
            \begin{cases}
                \text{当 } x_0 \text{ 为极大值点时,即为 } I \text{ 内的最大值点 } (f(x_0)=M), \\
                \qquad f(x_0)\ge f(x), \quad x\in I; \\[6pt]
                \text{当 } x_0 \text{ 为极小值点时,即为 } I \text{ 内的最小值点 } (f(x_0)=m), \\
                \qquad f(x_0)\le f(x), \quad x\in I.
            \end{cases}
        \]
        \item 若有$f''(x) > 0(\text{凹}), a < x < b, f(a) = f(b) = 0$,则有$f(x) < 0$
        \item 若有$f''(x) < 0(\text{凸}), a < x < b, f(a) = f(b) = 0$,则有$f(x) > 0$
    \end{enumerate}

    \subsection{用常数变量化证明不等式}

    如果欲证的不等式中都是常数(常数不可直接求导),则可以将其中一个或者几个{\color{red}{常数变量化}},再利用上面所述的导数工具去证明

    \begin{analysisbox}[Example]
        证明
        \[
            \frac{\ln b - \ln a}{\,b - a\,}
            <
            \frac{1}{\sqrt{ab}}.
        \]

        令
        \[
            f(x)
            = \ln x - \ln a - \frac{x-a}{\sqrt{ax}}.
        \]
    \end{analysisbox}

    \subsection{用中值定理证明不等式}

    主要用拉格朗日中值定理或者泰勒公式

    \begin{analysisbox}[Example]
        证明
        \[
            \frac{2a}{a^2+b^2}
            <
            \frac{\ln b-\ln a}{\,b-a\,}.
        \]

        令 $f(x)=\ln x$,
        由拉格朗日中值定理,
        至少存在一点 $\xi\in(a,b)$,
        使得
        \[
            \frac{\ln b-\ln a}{b-a}
            = f'(\xi).
        \]
    \end{analysisbox}

    \subsection{结论}

    \begin{enumerate}
        \item 当$0 < x < \displaystyle\frac{\pi}{2}$时,$\displaystyle\frac{2x}{\pi} < \sin x < x$
    \end{enumerate}


\end{document}
