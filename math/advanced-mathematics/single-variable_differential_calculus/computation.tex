\documentclass[a4paper,12pt]{article}
\usepackage{xeCJK}          % 中文支持
\usepackage{fontspec}       % 英文/数学字体
\usepackage{amsmath, amssymb} % 数学公式
\usepackage{graphicx}       % 插入图片
\usepackage{hyperref}       % 目录超链接
\usepackage{geometry}       % 页面布局
\usepackage{bm}             % 粗体
\usepackage{xcolor}         % 颜色
\usepackage{tabularx}       % 表格环境
\usepackage{tikz}           % TikZ 绘制主对角线斜线
\usepackage{tcolorbox}
\usepackage{xstring}
\usepackage{pgfplots}
\usepackage{enumitem}
\usepackage{pifont}
\pgfplotsset{compat=1.18}
\geometry{left=3cm,right=3cm,top=3cm,bottom=3cm}

% 抽离颜色和尺寸参数
\newcommand{\analysisTitleColor}{green!50!black}
\newcommand{\analysisBackColor}{white}
\newcommand{\analysisBoxRule}{0.8pt}
\newcommand{\analysisArc}{3pt}
\newcommand{\analysisPadding}{6pt}

% 定义 tcolorbox
\newtcolorbox{analysisbox}[1][]{
    title=\IfStrEq{#1}{}{\textbf{解析}}{#1}, % 如果传参为空则使用“解析”
    colback=\analysisBackColor,
    colframe=\analysisTitleColor,
    boxrule=\analysisBoxRule,
    arc=\analysisArc,
    left=\analysisPadding,
    right=\analysisPadding,
    top=4pt,
    bottom=4pt
}

\newlist{circlenum}{enumerate}{1}
\setlist[circlenum]{
    label=\ding{\numexpr171+\arabic*},
    leftmargin=2.2em,
    itemsep=0.4em
}

% =========================
% 字体设置
% =========================
\setmainfont{Times New Roman}
\setsansfont{Helvetica Neue}
\setmonofont{Menlo}
\setCJKmainfont{PingFang SC}

% =========================
% 图形路径(可调整)
% =========================
\graphicspath{{./assets/}}

% =========================
% 文档开始
% =========================
\begin{document}

%    \title{Template}
%    \author{Bowen}
%    \date{\today}
%    \maketitle
%    \tableofcontents
%    \newpage


% =========================

    \section{一元函数微分学的计算}

    \subsection{基本求导公式}

    \begin{enumerate}
        \item $(x^a)' = a x^{a-1}\,(a \text{ 为常数})$

        \item $(a^x)' = a^x \ln a \,(a>0,\ a\neq1)$

        \item $(e^x)' = e^x$

        \item $(\log_a x)' = \displaystyle\frac{1}{x\ln a}
        \,(a>0,\ a\neq1)$

        \item $(\ln |x|)' = \displaystyle\frac{1}{x}$

        \item 三角函数求导
        \begin{itemize}
            \item $(\sin x)' = \cos x$
            \item $(\cos x)' = -\sin x$
            \item $(\tan x)' = \sec^2 x$
            \item $(\cot x)' = -\csc^2 x$
            \item $(\sec x)' = \sec x \tan x$
            \item $(\csc x)' = -\csc x \cot x$
            \item $(\arcsin x)' = \displaystyle\frac{1}{\sqrt{1-x^2}}$
            \item $(\arccos x)' = -\displaystyle\frac{1}{\sqrt{1-x^2}}$
            \item $(\arctan x)' = \displaystyle\frac{1}{1+x^2}$
            \item $(\arccot x)' = -\displaystyle\frac{1}{1+x^2}$
        \end{itemize}

        \item $\bigl[\ln (x+\sqrt{x^2+1})\bigr]'
        = \displaystyle\frac{1}{\sqrt{x^2+1}}$

        \item $\bigl[\ln (x+\sqrt{x^2-1})\bigr]'
        = \displaystyle\frac{1}{\sqrt{x^2-1}}$
    \end{enumerate}

    \subsection{四则运算}

    \begin{enumerate}
        \item $(u \pm v)' = u' \pm v'$
        \item $d[u(x) \pm v(x)] = d[u(x)] \pm d[v(x)]$

        \item $(uv)' = u'v + uv'$
        \item $d[u(x)v(x)] = u(x)d[v(x)] + v(x)d[u(x)]$

        \item $\left(\dfrac{u}{v}\right)'
        = \dfrac{u'v - uv'}{v^2}\qquad (v \neq 0)$
        \item d$\left[\dfrac{u(x)}{v(x)}\right]
        = \dfrac{v(x)d[u(x)] - u(x)d[v(x)]}{[v(x)]^2}\qquad v(x) \neq 0$
    \end{enumerate}

    \subsection{复合函数的导数与微分形式不变性}

    设$u = g(x)$在点$x$处可导,$y = f(u)$在点$u = g(x)$处可导,则
    \begin{itemize}
        \item $\{f[g(x)]\}' = f'[g(x)]g'(x)$
        \item $d\{f[g(x)]\} = f'[g(x)]g'(x)dx = f'[g(x)]d[g(x)]$
    \end{itemize}

    \subsection{分段函数的导数}

    设
    \[
        f(x)=
        \begin{cases}
            f_1(x), & x \ge x_0,\\
            f_2(x), & x < x_0,
        \end{cases}
    \]
    其中 $f_1(x)$、$f_2(x)$ 分别在 $x>x_0$ 与 $x<x_0$ 时可导,则:

    \begin{circlenum}
        \item 在分段点 $x_0$ 处,用导数定义求导:
        \[
            f_+'(x_0)=\lim_{x\to x_0^+}\frac{f_1(x)-f(x_0)}{x-x_0},\qquad
            f_-'(x_0)=\lim_{x\to x_0^-}\frac{f_2(x)-f(x_0)}{x-x_0}.
        \]
        根据 $f_+'(x_0)$ 是否等于 $f_-'(x_0)$ 来判定 $f'(x_0)$

        \item 在非分段点用导数公式求导:
        \[
            \begin{cases}
                x>x_0,\quad f'(x)=f_1'(x),\\
                x<x_0,\quad f'(x)=f_2'(x).
            \end{cases}
        \]
    \end{circlenum}

    \subsection{反函数的导数}

    \begin{enumerate}
        \item 设 $y=f(x)$ 为单调且可导函数,并且 $f'(x)\neq 0$,
        则存在反函数 $x=\varphi(y)$,且
        \[
            \frac{dx}{dy}
            =
            \frac{1}{\dfrac{dy}{dx}}
            =
            \frac{1}{f'(x)},
        \]
        即
        \[
            \varphi'(y)=\frac{1}{f'(x)}.
        \]
        \item $y = f(x), x = \varphi (y), \varphi'(y) = -\displaystyle\frac{f''(x)}{[f'(x)]^3}$
        \item $y''_{xx} = -\displaystyle\frac{x''_{yy}}{(x'_y)^3}$
        \item $x''_{yy} = -\displaystyle\frac{y''_{xx}}{(y'_x)^3}$
    \end{enumerate}

    \subsection{隐函数求导法}

    设函数$y = f(x)$是由方程$F(x, y) = 0$确定的可导函数,则
    \begin{circlenum}
        \item 方程$F(x, y) = 0$两边对自变量$x$求导,注意$y = y(x)$,即将$y$看作中间变量,得到一个关于$y'$的方程
        \item 解该方程便可求出$y'$
    \end{circlenum}

    \subsection{参数方程所确定的函数的导数}

    设函数 $y=y(x)$ 由参数方程
    \[
        \begin{cases}
            x=\varphi(t),\\
            y=\psi(t)
        \end{cases}
    \]
    所确定,其中 $t$ 为参数,且 $\varphi(t)$、$\psi(t)$ 均可导,
    并且 $\varphi'(t)\neq 0$,则

    \[
        \frac{dy}{dx}
        =
        \frac{\dfrac{dy}{dt}}{\dfrac{dx}{dt}}
        =
        \frac{\psi'(t)}{\varphi'(t)}.
    \]

    \subsection{对数求导法}

    对于\textbf{多项相乘、相除、开方、乘方}的式子,一般\textbf{先取对数再求导}。设 $y=f(x)$(其中 $f(x)>0$),则
    \begin{circlenum}
        \item 等式两边取自然对数,得
        \[
            \ln y = \ln f(x).
        \]

        \item 两边对自变量 $x$ 求导(注意 $y=y(x)$,将 $y$ 视为中间变量),得
        \[
            \frac{1}{y}\,y' = [\ln f(x)]'.
        \]
        因此
        \[
            y' = y\,\frac{f'(x)}{f(x)}.
        \]
    \end{circlenum}

    \subsection{幂指函数求导法}

    对于$u(x)^{v(x)}(u(x) > 0, \text{且}u(x) \neq 1)$,除了用上面的对数求导法,还可以先化成指数函数
    \[
        u(x)^{v(x)} = e^{v(x)\ln u(x)}
    \]
    然后求导

    \subsection{\textcolor{red}{$\bigstar$} 高阶导数}
    \begin{enumerate}
        \item 归纳法:逐次求导,探索规律,得出通式
        \item 莱布尼茨公式:设$u = u(x), v = v(x)$均$n$阶可导,则
        \begin{itemize}
            \item $(u \pm v)^{(n)} = u^{(n)} \pm v^{(n)}$
            \item ${\color{red}{\bigstar}}(uv)^{(n)} = \displaystyle\sum_{k=0}^{n}C_n^k\,u^{(k)}\,v^{(n-k)}$
            \item $C_n^k = \displaystyle\frac{n!}{k!(n-k)!}$
            \item \[
                      \begin{array}{ccccccccc}
                          & & & & 1 \\
                          & & & 1 & & 1 \\
                          & & 1 & & 2 & & 1 \\
                          & 1 & & 3 & & 3 & & 1 \\
                          1 & & 4 & & 6 & & 4 & & 1
                      \end{array}
            \]
        \end{itemize}
        \item 泰勒展开式
        \begin{circlenum}
            \item 抽象展开:任何一个在点 $x_0$ 的邻域内无穷阶可导的函数,
            都可以表示为其泰勒级数
            \[
                y=f(x)
                =
                \sum_{n=0}^{\infty}
                \frac{f^{(n)}(x_0)}{n!}(x-x_0)^n.
            \]
            特别地,当 $x_0=0$ 时,称为麦克劳林展开式:
            \[
                y=f(x)
                =
                \sum_{n=0}^{\infty}
                \frac{f^{(n)}(0)}{n!}x^n.
            \]

            \item 具体展开:当题目给出具体的无穷阶可导函数 $y=f(x)$ 时,
            可利用已知基本函数的展开式,将其展开为幂级数。

            \item 函数泰勒展开式的唯一性:无论 $f(x)$ 通过何种方法展开,
            其泰勒展开式具有唯一性。
            因此可通过比较\ding{202}、\ding{203}中幂级数的系数,
            求出 $f^{(n)}(x_0)$ 或 $f^{(n)}(0)$。
        \end{circlenum}
        \[
            \begin{aligned}
                e^x &= \sum_{n=0}^{\infty} \frac{x^n}{n!} = 1 + x + \frac{x^2}{2!} + \cdots + \frac{x^n}{n!} + \cdots, & -\infty < x < +\infty \\
                \frac{1}{1-x} &= \sum_{n=0}^{\infty} x^n = 1 + x + x^2 + \cdots + x^n + \cdots, & |x| < 1 \\
                \frac{1}{1+x} &= \sum_{n=0}^{\infty} (-1)^n x^n = 1 - x + x^2 - \cdots + (-1)^n x^n + \cdots, & |x| < 1 \\
                \ln(1+x) &= \sum_{n=1}^{\infty} (-1)^{n-1} \frac{x^n}{n} = x - \frac{x^2}{2} + \frac{x^3}{3} - \cdots + (-1)^{n-1} \frac{x^n}{n} + \cdots, & |x| < 1 \\
                \sin x &= \sum_{n=0}^{\infty} (-1)^n \frac{x^{2n+1}}{(2n+1)!} = x - \frac{x^3}{3!} + \frac{x^5}{5!} - \cdots, & -\infty < x < +\infty \\
                \cos x &= \sum_{n=0}^{\infty} (-1)^n \frac{x^{2n}}{(2n)!} = 1 - \frac{x^2}{2!} + \frac{x^4}{4!} - \cdots, & -\infty < x < +\infty \\
                (1+x)^a &= \sum_{n=0}^{\infty} \binom{a}{n} x^n = 1 + ax + \frac{a(a-1)}{2!} x^2 + \cdots, & |x| < 1 \\
                \tan x &= x + \frac{x^3}{3} + \frac{2x^5}{15} + \cdots, & |x| < \frac{\pi}{2} \\
                \arcsin x &= \sum_{n=0}^{\infty} \frac{(2n)!}{4^n (n!)^2 (2n+1)} x^{2n+1} = x + \frac{x^3}{6} + \frac{3x^5}{40} + \cdots, & |x| \le 1 \\
                \arctan x &= \sum_{n=0}^{\infty} (-1)^n \frac{x^{2n+1}}{2n+1} = x - \frac{x^3}{3} + \frac{x^5}{5} - \cdots, & |x| \le 1
            \end{aligned}
        \]
    \end{enumerate}

    \subsection{基础概念}

    \begin{enumerate}
    \end{enumerate}

    \subsection{结论}

    \begin{enumerate}
    \end{enumerate}

    \subsection{定理}

    \begin{enumerate}
    \end{enumerate}

    \subsection{运算}

    \begin{enumerate}

    \end{enumerate}

    \subsection{公式}

    \begin{enumerate}
        \item $[\ln(x + \sqrt {x^2 + a^2})]' = \displaystyle\frac{1}{\sqrt {x^2 + a^2}}$
        \item $(x\ln x)' = \ln x + 1$
        \item $(\displaystyle\frac{1}{x} \ln x)' = -\displaystyle\frac{1}{x^2} \ln x + \frac{1}{x^2}$
        \item 常用的高阶导数
        \begin{itemize}
            \item $\bigl(e^{ax+b}\bigr)^{(n)} = a^{n} e^{ax+b}.$
            \item $\bigl[\sin(ax+b)\bigr]^{(n)} = a^{n} \sin\!\left(ax+b+\frac{n\pi}{2}\right)$
            \item $\bigl[\cos(ax+b)\bigr]^{(n)} = a^{n} \cos\!\left(ax+b+\frac{n\pi}{2}\right)$
            \item $\bigl[\ln(ax+b)\bigr]^{(n)} = (-1)^{\,n-1}a^{n}\displaystyle\frac{(n-1)!}{(ax+b)^{n}}$
            \item $\left(\displaystyle\frac{1}{ax+b}\right)^{(n)} = (-1)^{\,n}a^{n}\displaystyle\frac{n!}{(ax+b)^{n+1}} \Rightarrow (\displaystyle\frac{1}{1 + x})^{(n)} = (-1)^{\,n}\displaystyle\frac{n!}{(x+1)^{n+1}}$
        \end{itemize}
    \end{enumerate}

    \subsection{方法总结}

    \begin{enumerate}
        \item 对于$g(x) = x^k f(x)$型,可以考虑用泰勒公式求$g^{(n)}(0)$
    \end{enumerate}

    \subsection{条件转换思路}

    \begin{enumerate}

    \end{enumerate}

    \subsection{理解}

    \begin{enumerate}
    \end{enumerate}

\end{document}
