\documentclass[a4paper,12pt]{article}
\usepackage{xeCJK}          % 中文支持
\usepackage{fontspec}       % 英文/数学字体
\usepackage{amsmath, amssymb} % 数学公式
\usepackage{graphicx}       % 插入图片
\usepackage{hyperref}       % 目录超链接
\usepackage{geometry}       % 页面布局
\usepackage{bm}             % 粗体
\usepackage{xcolor}         % 颜色
\usepackage{tabularx}       % 表格环境
\usepackage{tikz}           % TikZ 绘制主对角线斜线
\usepackage{tcolorbox}
\usepackage{xstring}
\usepackage{pgfplots}
\usepackage{enumitem}
\pgfplotsset{compat=1.18}
\geometry{left=3cm,right=3cm,top=3cm,bottom=3cm}

% 抽离颜色和尺寸参数
\newcommand{\analysisTitleColor}{green!50!black}
\newcommand{\analysisBackColor}{white}
\newcommand{\analysisBoxRule}{0.8pt}
\newcommand{\analysisArc}{3pt}
\newcommand{\analysisPadding}{6pt}

% 定义 tcolorbox
\newtcolorbox{analysisbox}[1][]{
    title=\IfStrEq{#1}{}{\textbf{解析}}{#1}, % 如果传参为空则使用“解析”
    colback=\analysisBackColor,
    colframe=\analysisTitleColor,
    boxrule=\analysisBoxRule,
    arc=\analysisArc,
    left=\analysisPadding,
    right=\analysisPadding,
    top=4pt,
    bottom=4pt
}

% =========================
% 字体设置
% =========================
\setmainfont{Times New Roman}
\setsansfont{Helvetica Neue}
\setmonofont{Menlo}
\setCJKmainfont{PingFang SC}

% =========================
% 图形路径(可调整)
% =========================
\graphicspath{{./assets/}}

% =========================
% 文档开始
% =========================
\begin{document}

%    \title{Template}
%    \author{Bowen}
%    \date{\today}
%    \maketitle
%    \tableofcontents
%    \newpage


% =========================

    \section{一元函数微分学的概念}

    \subsection{导数}

    \begin{enumerate}
        \item 设\( y = f(x) \)定义在区间\( I \)上,让自变量在\( x = x_0 \)处加一个增量\( \Delta x \)(可正可负),其中\( x_0 \in I, x_0 + \Delta x \in I \),则可得函数的增量\( \Delta y = f(x_0 + \Delta x) - f(x_0) \)。若函数增量\( \Delta y \)与自变量增量\( \Delta x \)的比值在\( \Delta x \to 0 \)时的极限存在,即\( \lim\limits_{\Delta x \to 0} \displaystyle\frac{\Delta y}{\Delta x} \)存在,则称函数\( y = f(x) \)在点\( x_0 \)处可导,并称这个{\color{red}{极限}}为\( y = f(x) \)在点\( x_0 \)处的{\color{red}{导数(变化率)}},记作\( f'(x_0) \),即
        \[
            \frac{dy}{dx}\bigg|_{x=x_0} = f'(x_0) = \lim\limits_{\Delta x \to 0} \frac{\Delta y}{\Delta x} = \lim\limits_{\Delta x \to 0} \frac{f(x_0 + \Delta x) - f(x_0)}{\Delta x}.
            \tag{*}
        \]
        广义化
        \[
            \lim_{\text{狗} \to 0} \frac{f(x_0 + \text{狗}) - f(x_0)}{\text{狗}}.
            \tag{**}
        \]
        令 $x_0 + \Delta x = x$,从而得到
        \[
            \text{函数式 } f'(x_0) = \lim_{x \to x_0} \frac{f(x) - f(x_0)}{x - x_0},
            \tag{***}
        \]
        \item 下面这三种提法是等价的:
        \begin{itemize}
            \item $y = f(x)$在点$x_0$处可导;
            \item $y = f(x)$在点$x_0$处导数存在;
            \item $f'(x_0) = A$($A$为有限数)。
        \end{itemize}
        \item $\color{red}{\bigstar}$ 函数在一点可导的充要条件: $f'(x_0)$存在 $\Leftrightarrow$ 其左导数$f'_-(x_0)$与右导数$f'_+(x_0)$均存在且相等
        \item 函数在一点可导的必要条件: 若$f(x)$在一点可导 $\Rightarrow$ $f(x)$在该点连续
        \item 导数的性质
        \begin{itemize}
            \item $\color{red}{\bigstar}$求导一次,奇偶性互换
            \item 若$f(x)$是可导的周期为$T$的周期函数,则$f'(x)$也是以$T$为周期的周期函数
            \item 若$f'(x) > 0$,$f(x)$单调递增;若$f'(x) < 0$,则$f(x)$单调递减
        \end{itemize}
    \end{enumerate}

    \subsection{基础概念}

    \begin{enumerate}
    \end{enumerate}

    \subsection{结论}

    \begin{enumerate}
    \end{enumerate}

    \subsection{定理}

    \begin{enumerate}
    \end{enumerate}

    \subsection{运算}

    \begin{enumerate}

    \end{enumerate}

    \subsection{公式}

    \begin{enumerate}

    \end{enumerate}

    \subsection{方法总结}

    \begin{enumerate}

    \end{enumerate}

    \subsection{条件转换思路}

    \begin{enumerate}

    \end{enumerate}

    \subsection{理解}

    \begin{enumerate}
    \end{enumerate}

\end{document}
