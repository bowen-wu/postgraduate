\documentclass[a4paper,12pt]{article}
\usepackage{xeCJK}          % 中文支持
\usepackage{fontspec}       % 英文/数学字体
\usepackage{amsmath, amssymb} % 数学公式
\usepackage{graphicx}       % 插入图片
\usepackage{hyperref}       % 目录超链接
\usepackage{geometry}       % 页面布局
\usepackage{bm}             % 粗体
\usepackage{xcolor}         % 颜色
\usepackage{tabularx}       % 表格环境
\usepackage{tikz}           % TikZ 绘制主对角线斜线
\usepackage{tcolorbox}
\usepackage{xstring}
\usepackage{pgfplots}
\usepackage{enumitem}
\pgfplotsset{compat=1.18}
\geometry{left=3cm,right=3cm,top=3cm,bottom=3cm}

% 抽离颜色和尺寸参数
\newcommand{\analysisTitleColor}{green!50!black}
\newcommand{\analysisBackColor}{white}
\newcommand{\analysisBoxRule}{0.8pt}
\newcommand{\analysisArc}{3pt}
\newcommand{\analysisPadding}{6pt}

% 定义 tcolorbox
\newtcolorbox{analysisbox}[1][]{
    title=\IfStrEq{#1}{}{\textbf{解析}}{#1}, % 如果传参为空则使用“解析”
    colback=\analysisBackColor,
    colframe=\analysisTitleColor,
    boxrule=\analysisBoxRule,
    arc=\analysisArc,
    left=\analysisPadding,
    right=\analysisPadding,
    top=4pt,
    bottom=4pt
}

% =========================
% 字体设置
% =========================
\setmainfont{Times New Roman}
\setsansfont{Helvetica Neue}
\setmonofont{Menlo}
\setCJKmainfont{PingFang SC}

% =========================
% 图形路径(可调整)
% =========================
\graphicspath{{./assets/}}

% =========================
% 文档开始
% =========================
\begin{document}

%    \title{Template}
%    \author{Bowen}
%    \date{\today}
%    \maketitle
%    \tableofcontents
%    \newpage


% =========================

    \section{一元函数微分学的概念}

    \subsection{导数}

    \begin{enumerate}
        \item 设\( y = f(x) \)定义在区间\( I \)上,让自变量在\( x = x_0 \)处加一个增量\( \Delta x \)(可正可负),其中\( x_0 \in I, x_0 + \Delta x \in I \),则可得函数的增量\( \Delta y = f(x_0 + \Delta x) - f(x_0) \)。若函数增量\( \Delta y \)与自变量增量\( \Delta x \)的比值在\( \Delta x \to 0 \)时的极限存在,即\( \lim\limits_{\Delta x \to 0} \displaystyle\frac{\Delta y}{\Delta x} \)存在,则称函数\( y = f(x) \)在点\( x_0 \)处可导,并称这个{\color{red}{极限}}为\( y = f(x) \)在点\( x_0 \)处的{\color{red}{导数(变化率)}},记作\( f'(x_0) \),即
        \[
            \frac{dy}{dx}\bigg|_{x=x_0} = f'(x_0) = \lim\limits_{\Delta x \to 0} \frac{\Delta y}{\Delta x} = \lim\limits_{\Delta x \to 0} \frac{f(x_0 + \Delta x) - f(x_0)}{\Delta x}.
            \tag{*}
        \]
        广义化
        \[
            \lim_{\text{狗} \to 0} \frac{f(x_0 + \text{狗}) - f(x_0)}{\text{狗}}.
            \tag{**}
        \]
        令 $x_0 + \Delta x = x$,从而得到
        \[
            \text{函数式 } f'(x_0) = \lim_{x \to x_0} \frac{f(x) - f(x_0)}{x - x_0},
            \tag{***}
        \]
        \item 下面这三种提法是等价的:
        \begin{itemize}
            \item $y = f(x)$在点$x_0$处可导;
            \item $y = f(x)$在点$x_0$处导数存在;
            \item $f'(x_0) = A$($A$为有限数)。
        \end{itemize}
        \item $\color{red}{\bigstar}$ 函数在一点可导的充要条件: $f'(x_0)$存在 $\Leftrightarrow$ 其左导数$f'_-(x_0)$与右导数$f'_+(x_0)$均存在且相等
        \item 函数在一点可导的必要条件: \textbf{若$f(x)$在一点可导 $\Rightarrow$ $f(x)$在该点连续},反之未必
        \item 导数的性质
        \begin{itemize}
            \item $\color{red}{\bigstar}$求导一次,奇偶性互换(导数定义)
            \item 若$f(x)$是可导的周期为$T$的周期函数,则$f'(x)$也是以$T$为周期的周期函数(导数定义)
            \item 若$f'(x) > 0$,$f(x)$单调递增;若$f'(x) < 0$,则$f(x)$单调递减
        \end{itemize}
    \end{enumerate}

    \subsection{导数的几何意义}

    \begin{enumerate}
        \item 函数 $y=f(x)$ 在点 $x_0$ 处的导数值 $f'(x_0)$,就是曲线 $y=f(x)$ 在点 $(x_0,\,y_0)$处切线的斜率 $k$,即 $k = f'(x_0)$,因此,曲线 $y=f(x)$ 在点 $(x_0,\,y_0)$ 处的切线方程为
        \[
            y - y_0 = f'(x_0)\,(x - x_0).
        \]

        该点处的法线方程为
        \[
            y - y_0 = -\frac{1}{f'(x_0)}\,(x - x_0) \; (f'(x_0) \neq 0)
        \]
        \item 切线存在不代表导数存在,但导数存在切线一定存在
        \item 若 $f_+'(x_0) \neq f_-'(x_0)$ 即在点 $x_0$ 处出现角点(或尖点),则函数 $f(x)$ 在 $x_0$ 处不可导,且不存在切线
        \item 若函数 $f(x)$ 在点 $x_0$ 处具有无穷导数,则该点处存在切线,但导数不存在
    \end{enumerate}

    \subsection{基础概念}

    \begin{enumerate}
    \end{enumerate}

    \subsection{结论}

    \begin{enumerate}
        \item 若函数 $\varphi(x)$ 在 $x=x_0$ 处连续,则函数$f(x)=|x-x_0|\,\varphi(x)$在 $x=x_0$ 处可导的充要条件是 $\varphi(x_0)=0$
        \item \item 函数 $f(x)$ 与 $|f(x)|$ 的连续性与可导性关系总结:
        \begin{item}
            \item 若 $f(x)$ 在 $x_0$ 处连续,则 $|f(x)|$ 在 $x_0$ 处连续; 反之不成立。
            \item 若 $f(x)$ 在 $x_0$ 处可导,则:
            \begin{enumerate}[label=(\alph*)]
                \item 若 $f(x_0) \neq 0$,则 $|f(x)|$ 在 $x_0$ 处{\color{red}{可导}},且
                \[
                    \left.|\,|f(x)|\,\right.'\right|_{x=x_0}
                    =
                    \begin{cases}
                        f'(x_0), & f(x_0) > 0, \\
                        -f'(x_0), & f(x_0) < 0.
                    \end{cases}
                \]

                \item 若 $f(x_0) = 0$,则:
                \[
                    \begin{cases}
                        f'(x_0) = 0
                        \;\Rightarrow\;
                        |f(x)| \text{ 在 } x_0 \text{ 处可导,且 }
                        \left.|\,|f(x)|\,\right.'\right|_{x=x_0} = 0, \\[0.5em]
                        f'(x_0) \neq 0
                        \;\Rightarrow\;
                        |f(x)| \text{ 在 } x_0 \text{ 处{\color{red}{不可导}}}.
                    \end{cases}
                \]
                \item 总结不可导即:
                \[
                    \left.
                    \begin{aligned}
                        & f(x) \text{ 在 } x_0 \text{ 处可导} \\
                        & f(x_0) = 0 \\
                        & f'(x_0) \neq 0
                    \end{aligned}
                    \right\}
                    \;\Longrightarrow\;
                    |f(x)| \text{ 在 } x_0 \text{ 处 {\color{red}{不可导}}}
                \]
            \end{enumerate}
        \end{item}
    \end{enumerate}

    \subsection{定理}

    \begin{enumerate}
    \end{enumerate}

    \subsection{运算}

    \begin{enumerate}

    \end{enumerate}

    \subsection{公式}

    \begin{enumerate}
        \item $[(e^x - 1)g(x)]' = e^xg(x) + (e^x - 1)g'(x)$
    \end{enumerate}

    \subsection{方法总结}

    \begin{enumerate}

    \end{enumerate}

    \subsection{条件转换思路}

    \begin{enumerate}

    \end{enumerate}

    \subsection{理解}

    \begin{enumerate}
        \item 对于连续或可导函数,只要$f(x) > 0$(或$f(x) < 0$),无论$f(x_0)$与$0$的距离有多小,它旁边相依相偎的$f(x)$一定$> 0$(或$< 0$)
        \item 设 $f(x)$ 在 $x=0$ 的某邻域内有定义,且 $|f(x)| \le 1-\cos x$,证明 $f(x)$ 在 $x=0$ 处连续、可导且 $f'(0)=0$
        \begin{analysisbox}[经典]
            \textbf{证明:} 分三个层次完成。

            \medskip
            \textcolor{analysisTitleColor}{\textbf{(1)求函数极限}}

            由
            \[
                0 \le |f(x)| \le 1-\cos x,
                \qquad
                \lim_{x\to 0}(1-\cos x)=0,
            \]
            根据夹逼准则,有
            \[
                \lim_{x\to 0}|f(x)| = 0
                \quad\Rightarrow\quad
                \lim_{x\to 0}f(x) = 0.
            \]

            \medskip
            \textcolor{analysisTitleColor}{\textbf{(2)求函数值,验证连续性}}

            在原不等式中令 $x=0$,得
            \[
                |f(0)| \le 1-\cos 0 = 0,
            \]
            故 $f(0)=0$。

            因此
            \[
                \lim_{x\to 0}f(x) = f(0),
            \]
            即 $f(x)$ 在 $x=0$ 处连续,并且有
            \[
                |f(x)-f(0)| = |f(x)| \le 1-\cos x.
            \]

            \medskip
            \textcolor{analysisTitleColor}{\textbf{(3)求导数(按定义)}}

            由上述不等式可得
            \[
                0 \le
                \left|
                \frac{f(x)-f(0)}{x-0}
                \right|
                \le
                \frac{1-\cos x}{|x|}.
            \]

            又因为
            \[
                \lim_{x\to 0}\frac{1-\cos x}{|x|}
                =
                \lim_{x\to 0}\frac{\frac{1}{2}x^2}{|x|}
                =0,
            \]
            再次由夹逼准则,
            \[
                \lim_{x\to 0}
                \left|
                \frac{f(x)-f(0)}{x-0}
                \right|
                =0,
            \]
            即
            \[
                \lim_{x\to 0}
                \frac{f(x)-f(0)}{x-0}
                =0,
            \]
            从而 $f'(0)=0$。

            \medskip
            \textbf{综上,}
            $f(x)$ 在 $x=0$ 处连续、可导,且 $f'(0)=0$。$\square$

        \end{analysisbox}

        \item
    \end{enumerate}


\end{document}
