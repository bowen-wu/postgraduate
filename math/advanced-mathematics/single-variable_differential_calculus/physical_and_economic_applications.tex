\documentclass[a4paper,12pt]{article}
\usepackage{xeCJK}          % 中文支持
\usepackage{fontspec}       % 英文/数学字体
\usepackage{amsmath, amssymb} % 数学公式
\usepackage{graphicx}       % 插入图片
\usepackage{hyperref}       % 目录超链接
\usepackage{geometry}       % 页面布局
\usepackage{bm}             % 粗体
\usepackage{xcolor}         % 颜色
\usepackage{tabularx}       % 表格环境
\usepackage{tikz}           % TikZ 绘制主对角线斜线
\usepackage{tcolorbox}
\usepackage{xstring}
\usepackage{pgfplots}
\usepackage{enumitem}
\usepackage{pifont}
\usepackage{ulem}
\pgfplotsset{compat=1.18}
\geometry{left=3cm,right=3cm,top=3cm,bottom=3cm}

% 抽离颜色和尺寸参数
\newcommand{\analysisTitleColor}{green!50!black}
\newcommand{\analysisBackColor}{white}
\newcommand{\analysisBoxRule}{0.8pt}
\newcommand{\analysisArc}{3pt}
\newcommand{\analysisPadding}{6pt}

% 定义 tcolorbox
\newtcolorbox{analysisbox}[1][]{
    title=\IfStrEq{#1}{}{\textbf{解析}}{#1}, % 如果传参为空则使用“解析”
    colback=\analysisBackColor,
    colframe=\analysisTitleColor,
    boxrule=\analysisBoxRule,
    arc=\analysisArc,
    left=\analysisPadding,
    right=\analysisPadding,
    top=4pt,
    bottom=4pt
}

\newlist{circlenum}{enumerate}{1}
\setlist[circlenum]{
    label=\ding{\numexpr171+\arabic*},
    leftmargin=2.2em,
    itemsep=0.6em,      % item 之间的距离(主要)
    topsep=0.6em,       % 列表与上下正文的距离
    parsep=0.3em,       % item 内段落的间距
    partopsep=0.3em     % 列表前后额外间距
}

\newcommand{\blueuline}[1]{{\color{blue}\uline{\color{black}{#1}}}}

% =========================
% 字体设置
% =========================
\setmainfont{Times New Roman}
\setsansfont{Helvetica Neue}
\setmonofont{Menlo}
\setCJKmainfont{PingFang SC}

% =========================
% 图形路径(可调整)
% =========================
\graphicspath{{./assets/}}

% =========================
% 文档开始
% =========================
\begin{document}

%    \title{Template}
%    \author{Bowen}
%    \date{\today}
%    \maketitle
%    \tableofcontents
%    \newpage


% =========================

    \section{一元函数微分学的应用 - 物理应用与经济应用}

    \subsection{物理应用}

    相关物理概念
    \begin{enumerate}
        \item 速度:位移对时间的变化率
        \item 加速度:速度对时间的变化率
        \item 牛顿第二定律($F = ma$)
    \end{enumerate}
    \begin{analysisbox}[Example]
        已知质点运动的位移 $s$ 关于时间 $t$ 的函数为$s = s(t)$,称其为质点的运动方程(位移方程),则质点的速度为
        \[
            v(t)
            = \lim_{\Delta t \to 0} \frac{\Delta s}{\Delta t}
            = \frac{\mathrm{d}s}{\mathrm{d}t}
            = s'(t).
        \]

        其加速度为
        \[
            a(t)
            = \lim_{\Delta t \to 0} \frac{\Delta v}{\Delta t}
            = \frac{\mathrm{d}v}{\mathrm{d}t}
            = v'(t)
            = s''(t).
        \]

        此外,由链式法则可得
        \[
            a(t)
            = \frac{\mathrm{d}v}{\mathrm{d}s}\cdot\frac{\mathrm{d}s}{\mathrm{d}t}
            = v\,\frac{\mathrm{d}v}{\mathrm{d}s}
            = \frac{d(\displaystyle\frac{ds}{dt})}{dt}
            = \frac{d^{2}s}{dt^2}
        \]

    \end{analysisbox}

    \subsection{相关变化率}

    研究 $\displaystyle\frac{dA}{dB} = \displaystyle\frac{dA}{dC}\cdot \displaystyle\frac{dC}{dB}$

    \begin{enumerate}
        \item 若已知$\displaystyle\frac{\mathrm{d}A}{\mathrm{d}B}, \displaystyle\frac{\mathrm{d}C}{\mathrm{d}B}$,则
        \[
            \frac{\mathrm{d}A}{\mathrm{d}C} = \frac{\displaystyle\frac{\mathrm{d}A}{\mathrm{d}B}}{\displaystyle\frac{\mathrm{d}C}{\mathrm{d}B}}
        \]
        \item 该等式建立了$\displaystyle\frac{\mathrm{d}A}{\mathrm{d}B}$与$\displaystyle\frac{\mathrm{d}C}{\mathrm{d}B}$的关系,其中 $A,B,C$ 可以扩展为许多实际物理量
        \item 若函数 $y=f(x)$ 由参数方程
        \[
            \begin{cases}
                x = x(t), \\
                y = y(t),
            \end{cases}
        \]
        所确定,则
        \[
            \frac{\mathrm{d}y}{\mathrm{d}t}
            = \frac{\mathrm{d}y}{\mathrm{d}x}\cdot \frac{\mathrm{d}x}{\mathrm{d}t}.
            = f'(x)\frac{\mathrm{d}x}{\mathrm{d}t}
        \]
        上式时,$\displaystyle\frac{\mathrm{d}y}{\mathrm{d}t}$与$\displaystyle\frac{\mathrm{d}x}{\mathrm{d}t}$由$f'(x)$联系在一起,这种相互关联的变化率称为\textbf{相关变化率}
    \end{enumerate}

    \subsection{结论}

    \begin{enumerate}
    \end{enumerate}

    \subsection{定理}

    \begin{enumerate}
    \end{enumerate}

    \subsection{运算}

    \begin{enumerate}

    \end{enumerate}

    \subsection{公式}

    \begin{enumerate}

    \end{enumerate}

    \subsection{方法总结}

    \begin{enumerate}
        \item 涉及相关变化率问题
        \begin{circlenum}
            \item 建立相关变量方程
            \item 求导找出相关变化率,进而通过已知变化率求未知变化率
        \end{circlenum}

    \end{enumerate}

    \subsection{条件转换思路}

    \begin{enumerate}

    \end{enumerate}

    \subsection{理解}

    \begin{enumerate}
    \end{enumerate}

\end{document}
