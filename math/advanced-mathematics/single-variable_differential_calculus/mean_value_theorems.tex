\documentclass[a4paper,12pt]{article}
\usepackage{xeCJK}          % 中文支持
\usepackage{fontspec}       % 英文/数学字体
\usepackage{amsmath, amssymb} % 数学公式
\usepackage{graphicx}       % 插入图片
\usepackage{hyperref}       % 目录超链接
\usepackage{geometry}       % 页面布局
\usepackage{bm}             % 粗体
\usepackage{xcolor}         % 颜色
\usepackage{tabularx}       % 表格环境
\usepackage{tikz}           % TikZ 绘制主对角线斜线
\usepackage{tcolorbox}
\usepackage{xstring}
\usepackage{pgfplots}
\usepackage{enumitem}
\usepackage{pifont}
\usepackage{ulem}
\pgfplotsset{compat=1.18}
\geometry{left=3cm,right=3cm,top=3cm,bottom=3cm}

% 抽离颜色和尺寸参数
\newcommand{\analysisTitleColor}{green!50!black}
\newcommand{\analysisBackColor}{white}
\newcommand{\analysisBoxRule}{0.8pt}
\newcommand{\analysisArc}{3pt}
\newcommand{\analysisPadding}{6pt}

% 定义 tcolorbox
\newtcolorbox{analysisbox}[1][]{
    title=\IfStrEq{#1}{}{\textbf{解析}}{#1}, % 如果传参为空则使用“解析”
    colback=\analysisBackColor,
    colframe=\analysisTitleColor,
    boxrule=\analysisBoxRule,
    arc=\analysisArc,
    left=\analysisPadding,
    right=\analysisPadding,
    top=4pt,
    bottom=4pt
}

\newlist{circlenum}{enumerate}{1}
\setlist[circlenum]{
    label=\ding{\numexpr171+\arabic*},
    leftmargin=2.2em,
    itemsep=0.4em
}

\newcommand{\blueuline}[1]{{\color{blue}\uline{\color{black}{#1}}}}

% =========================
% 字体设置
% =========================
\setmainfont{Times New Roman}
\setsansfont{Helvetica Neue}
\setmonofont{Menlo}
\setCJKmainfont{PingFang SC}

% =========================
% 图形路径(可调整)
% =========================
\graphicspath{{./assets/}}

% =========================
% 文档开始
% =========================
\begin{document}

%    \title{Template}
%    \author{Bowen}
%    \date{\today}
%    \maketitle
%    \tableofcontents
%    \newpage


% =========================

    \section{一元函数微分学的应用 - 中值定理}

    \subsection{涉及函数的中值定理}

    \subsubsection{有界与最值定理}

    设 $f(x)$ 在 $[a, b]$上连续,则 $m \leq f(x) \leq M$,其中$m, M$分别为$f(x)$在$[a, b]$上的最小值和最大值

    \subsubsection{介值定理}

    设 $f(x)$ 在 $[a, b]$上连续,当$m \leq \mu \leq M$时,存在$\xi \in [a, b]$,使得$f(\xi) = \mu$

    \subsubsection{平均值定理}

    设 $f(x)$ 在 $[a, b]$上连续,当$a < x_1 < x_2 < \cdots < x_n < b$时,在$[x_1, x_n]$内至少存在一点$\xi$,使得
    \[
        f(\xi) = \frac{f(x_1) + f(x_2) + \cdots + f(x_n)}{n}
    \]

    \subsubsection{零点定理(介质定理的特例)}

    设 $f(x)$ 在 $[a, b]$上连续,当$f(a)\cdot f(b) < 0$时,存在$\xi \in (a, b)$,使得$f(\xi) = 0$
    \begin{itemize}
        \item 推广的零点定理: 若$f(x)$在$(a, b)$内连续,$\lim\limits_{x \to a^+} f(x) = \alpha, \lim\limits_{x \to b^-} f(x) = \beta$,且$\alpha \cdot \beta < 0$,则$f(x) = 0$在$(a, b)$内至少存在一个根
    \end{itemize}

    \subsection{涉及导数(微分)的中值定理}

    \subsubsection{费马定理}

    设函数 $f(x)$ 在点 $x_0$ 处满足
    \[
        \begin{cases}
            \text{(1) 可导(左右导数存在且相等)},\\
            \text{(2) 取极值 },
        \end{cases}
    \]
    则
    \[
        f'(x_0)=0.
    \]

    \subsubsection{导数零点定理}

    $f(x)$在$[a, b]$上可导,当$f'_{+}(a) \cdot f'_{-}(b) < 0$时,存在$\xi \in (a, b)$,使得$f'(\xi) = 0$

    \subsubsection{罗尔定理}

    设$f(x)$满足
    \[
        \begin{cases}
            \text{(1) 在} [a, b] \text{上连续},\\
            \text{(2) 在} (a, b) \text{内可导},\\
            \text{(3) }f(a) = f(b),
        \end{cases}
    \]
    则存在 $\xi \in (a, b)$,使得$f'(\xi) = 0$
    \begin{itemize}
        \item 推广的罗尔定理: 设$f(x)$在$(a, b)$内可导,$\lim\limits_{x \to a^+} f(x) = \lim\limits_{x \to b^-} f(x) = A$,则在$(a, b)$内至少存在一点$\xi$,使得$f'(\xi) = 0$
        \item 罗尔定理的使用需要构造辅助函数,其方法总结如下: //TODO
    \end{itemize}

    \subsubsection{拉格朗日中值定理}

    设函数 $f(x)$ 满足
    \[
        \begin{cases}
            \text{(1) 在} [a, b] \text{上连续},\\
            \text{(2) 在} (a, b) \text{内可导}
        \end{cases}
    \]
    则存在$\xi \in (a, b)$,使得
    \[
        f(b) - f(a) = f'(\xi)(b - a)
    \]
    或者写成
    \[
        f'(\xi) = \frac{f(b) - f(a)}{b - a}
    \]

    \begin{enumerate}
        \item 见到$f(a) - f(b)$与$f$与$f'$的关系,一般想到用拉格朗日中值
        \item 拉格朗日中值的作用是{\color{red}{用导函数的值来控制函数值的增减}}
    \end{enumerate}

    \subsubsection{柯西中值定理}

    设函数 $f(x), g(x)$ 满足
    \[
        \begin{cases}
            \text{(1) 在} [a, b] \text{上连续},\\
            \text{(2) 在} (a, b) \text{内可导},\\
            \text{(3) } g'(x) \neq 0 \\
        \end{cases}
    \]
    则存在$\xi \in (a, b)$,使得
    \[
        \frac{f(b) - f(a)}{g(b) - g(a)} = \frac{f'(\xi)}{g'(\xi)}
    \]

    \begin{enumerate}
        \item $f(x), g(x)$往往考察一个具体函数,一个抽象函数
    \end{enumerate}

    \subsubsection{泰勒公式(微分中值定理)}

    \begin{enumerate}
        \item 带拉格朗日余项的 $n$ 阶泰勒公式:设函数 $f(x)$ 在{\blueuline{点 $x_0$ 的某邻域内 $n+1$ 阶导数存在}},则对该邻域内任意点 $x$,有
        \[
            f(x)
            = f(x_0) + f'(x_0)(x-x_0)
            + \frac{f''(x_0)}{2!}(x-x_0)^2
            + \cdots
            + \frac{1}{n!}f^{(n)}(x_0)(x-x_0)^n
            + R_n(x),
        \]
        其中拉格朗日余项
        \[
            R_n(x)
            = \frac{f^{(n+1)}(\xi)}{(n+1)!}(x-x_0)^{n+1},
            \quad \xi \in (x_0, x).
        \]

        \item 带佩亚诺余项的 $n$ 阶泰勒公式:设函数 $f(x)$ 在{\blueuline{点 $x_0$ 处 $n$ 阶可导}},则存在$x_0$的一个邻域,对于该邻域内的任意点 $x$,有
        \[
            f(x)
            = f(x_0) + f'(x_0)(x-x_0)
            + \frac{f''(x_0)}{2!}(x-x_0)^2
            + \cdots
            + \frac{f^{(n)}(x_0)}{n!}(x-x_0)^n
            + o\!\left((x-x_0)^n\right).
        \]
        \item 当 $x_0 = 0$时泰勒公式称为\textbf{麦克劳林公式}
        \begin{circlenum}
            \item \[
                      f(x)
                      = f(0)
                      + \frac{f'(0)}{1!}x
                      + \frac{f''(0)}{2!}x^2
                      + \cdots
                      + \frac{f^{(n)}(0)}{n!}x^n
                      + \frac{f^{(n+1)}(\xi)}{(n+1)!}x^{\,n+1}, \quad \xi \in (0, x)
            \]
            \item \[
                      f(x)
                      = f(0)
                      + \frac{f'(0)}{1!}x
                      + \frac{f''(0)}{2!}x^2
                      + \cdots
                      + \frac{f^{(n)}(0)}{n!}x^n
                      + o(x^n).
            \]
        \end{circlenum}
        \item \noindent \textbf{说明:}
        \begin{circlenum}
            \item 带拉格朗日余项的 $n$ 阶泰勒公式{\color[rgb]{0.2, 0.6, 0.3}{适用于}}区间$[a, b]$,常在证明题中使用。如证不等式、中值等式等
            \item 带佩亚诺余项的 $n$ 阶泰勒公式{\color[rgb]{0.2, 0.6, 0.3}{适用于}}点$x = x_0$及其邻域,常用于研究点$x = x_0$处的某些结论。如求极限、判定无穷小的阶数、判定极值等
        \end{circlenum}
    \end{enumerate}

    \subsection{基础概念}

    \begin{enumerate}
    \end{enumerate}

    \subsection{结论}

    \begin{enumerate}
    \end{enumerate}

    \subsection{定理}

    \begin{enumerate}
    \end{enumerate}

    \subsection{运算}

    \begin{enumerate}

    \end{enumerate}

    \subsection{公式}

    \begin{enumerate}

    \end{enumerate}

    \subsection{方法总结}

    \begin{enumerate}
        \item 解题方法
        \begin{circlenum}
            \item 找定义式、关系式、约束式
            \item $\color{red}{\bigstar}$做一至两步逆运算
            \item 联想{\color{red}{经典}}形式
            \item 恒等变形
            \begin{itemize}
                \item $a = a - 0$
                \item $a = a + b - b$
                \item $e - 1 = e^1 - e^0$
            \end{itemize}
            \item 翻译数学名词
        \end{circlenum}
    \end{enumerate}

    \subsection{条件转换思路}

    \begin{enumerate}

    \end{enumerate}

    \subsection{理解}

    \begin{enumerate}
    \end{enumerate}

\end{document}
