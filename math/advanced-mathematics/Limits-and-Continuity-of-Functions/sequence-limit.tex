\documentclass[a4paper,12pt]{article}
\usepackage{xeCJK}          % 中文支持
\usepackage{fontspec}       % 英文/数学字体
\usepackage{amsmath, amssymb} % 数学公式
\usepackage{graphicx}       % 插入图片
\usepackage{hyperref}       % 目录超链接
\usepackage{geometry}       % 页面布局
\usepackage{bm}             % 粗体
\usepackage{xcolor}         % 颜色
\usepackage{tabularx}       % 表格环境
\usepackage{tikz}           % TikZ 绘制主对角线斜线
\usepackage{tcolorbox}
\usepackage{xstring}
\geometry{left=3cm,right=3cm,top=3cm,bottom=3cm}

% 抽离颜色和尺寸参数
\newcommand{\analysisTitleColor}{green!50!black}
\newcommand{\analysisBackColor}{white}
\newcommand{\analysisBoxRule}{0.8pt}
\newcommand{\analysisArc}{3pt}
\newcommand{\analysisPadding}{6pt}

% 定义 tcolorbox
\newtcolorbox{analysisbox}[1][]{
    title=\IfStrEq{#1}{}{\textbf{解析}}{#1}, % 如果传参为空则使用“解析”
    colback=\analysisBackColor,
    colframe=\analysisTitleColor,
    boxrule=\analysisBoxRule,
    arc=\analysisArc,
    left=\analysisPadding,
    right=\analysisPadding,
    top=4pt,
    bottom=4pt
}

% =========================
% 字体设置
% =========================
\setmainfont{Times New Roman}
\setsansfont{Helvetica Neue}
\setmonofont{Menlo}
\setCJKmainfont{PingFang SC}

% =========================
% 图形路径(可调整)
% =========================
\graphicspath{{./assets/}}

% =========================
% 文档开始
% =========================
\begin{document}

%    \title{Template}
%    \author{Bowen}
%    \date{\today}
%    \maketitle
%    \tableofcontents
%    \newpage


% =========================

    \section{数列极限}

    \subsection{数列}

    \begin{enumerate}
        \item 等差数列
        \item 等比数列
    \end{enumerate}

    \subsection{基础概念}

    \begin{enumerate}
        \item 设${x_n}$为以数列,若存在常数$a$,对于任意的$\epsilon > 0$(不论它多么小),总存在正整数$N$,使得当$n > N$时,$|x_n - a| < \epsilon$,则称常数$a$是数列${x_n}$的极限,或者称数列${x_n}$收敛于$a$,记为
        \begin{gather*}
            \lim_{n \to \infty} x_n = a \\
            \forall \epsilon > 0,\ \exists N > 0,\ n > N \Rightarrow |x_n - a| < \epsilon
        \end{gather*}
        \item
    \end{enumerate}

    \subsection{定理}

    \begin{enumerate}
    \end{enumerate}

    \subsection{运算}

    \begin{enumerate}

    \end{enumerate}

    \subsection{公式}

    \begin{enumerate}

    \end{enumerate}

    \subsection{方法步骤}

    \begin{enumerate}

    \end{enumerate}

    \subsection{条件转换思路}

    \begin{enumerate}

    \end{enumerate}

    \subsection{理解}

    \begin{enumerate}
    \end{enumerate}

\end{document}
