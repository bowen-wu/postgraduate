\documentclass[a4paper,12pt]{article}
\usepackage{xeCJK}          % 中文支持
\usepackage{fontspec}       % 英文/数学字体
\usepackage{amsmath, amssymb} % 数学公式
\usepackage{graphicx}       % 插入图片
\usepackage{hyperref}       % 目录超链接
\usepackage{geometry}       % 页面布局
\usepackage{bm}             % 粗体
\usepackage{xcolor}         % 颜色
\usepackage{tabularx}       % 表格环境
\usepackage{tikz}           % TikZ 绘制主对角线斜线
\usepackage{tcolorbox}
\usepackage{xstring}
\usepackage{enumitem}
\geometry{left=3cm,right=3cm,top=3cm,bottom=3cm}

% 抽离颜色和尺寸参数
\newcommand{\analysisTitleColor}{green!50!black}
\newcommand{\analysisBackColor}{white}
\newcommand{\analysisBoxRule}{0.8pt}
\newcommand{\analysisArc}{3pt}
\newcommand{\analysisPadding}{6pt}

% 定义 tcolorbox
\newtcolorbox{analysisbox}[1][]{
    title=\IfStrEq{#1}{}{\textbf{解析}}{#1}, % 如果传参为空则使用“解析”
    colback=\analysisBackColor,
    colframe=\analysisTitleColor,
    boxrule=\analysisBoxRule,
    arc=\analysisArc,
    left=\analysisPadding,
    right=\analysisPadding,
    top=4pt,
    bottom=4pt
}

% =========================
% 字体设置
% =========================
\setmainfont{Times New Roman}
\setsansfont{Helvetica Neue}
\setmonofont{Menlo}
\setCJKmainfont{PingFang SC}

% =========================
% 图形路径(可调整)
% =========================
\graphicspath{{./assets/}}

% =========================
% 文档开始
% =========================
\begin{document}

%    \title{Template}
%    \author{Bowen}
%    \date{\today}
%    \maketitle
%    \tableofcontents
%    \newpage


% =========================

    \section{数列极限}

    \subsection{数列}

    \begin{enumerate}
        \item 等差数列
        \item 等比数列
    \end{enumerate}

    \subsection{基础概念}

    \begin{enumerate}
        \item 设${x_n}$为一数列,若存在常数$a$,对于任意的$\epsilon > 0$(不论它多么小),总存在正整数$N$,使得当$n > N$时,$|x_n - a| < \epsilon$,则称常数$a$是数列${x_n}$的极限,或者称数列${x_n}$收敛于$a$,记为
        \begin{gather*}
            \lim_{n \to \infty} x_n = a \\
            \forall \epsilon > 0,\ \exists N > 0,\ n > N \Rightarrow |x_n - a| < \epsilon
        \end{gather*}
        \item 有界数列: 若对所有正整数$n$,存在正实数$M$,有$|a_n| \leq M$,则称数列${a_n}$为有界数列。证明数列有界的方法
        \begin{itemize}
            \item 找$M$,使得$|a_n| \leq M$
            \item 放缩法
            \item 找最值
            \item 基本不等式法
        \end{itemize}
        \item 设 $\{x_n\}$ 为一数列,若存在常数$a$,对任意 $\varepsilon>0$(无论它多么小),总存在正整数 $N$,使得当 $n>N$ 时,$|x_n - a| < \varepsilon$恒成立,则称数列 $\{x_n\}$ \textbf{收敛于} $a$,或者称常数$a$是数列${x_n}$的极限,记作$\lim\limits_{n\to\infty} x_n = a$
        \item 数列收敛于$a$的速度问题:设数列$\{x_n\}, \{y_n\}$在$n\to \infty$的过程中同时趋于$a$,记$u_n = |x_n - a|, v_n = |y_n - a|$,且当$n\to \infty$时,$u_n$和$v_n$都是无穷小量。若$I = \lim\limits_{n\to\infty} \displaystyle\frac{u_n}{v_n}$存在,则有

        \begin{itemize}
            \item \textbf{高阶无穷小}:当$I = 0$时,称$u_n$是$v_n$的高阶无穷小,记作$u_n = o(v_n)$。此时$\{x_n\}$比$\{y_n\}$更快收敛于$a$
            \item \textbf{同阶无穷小}:当$I = c \neq 0$时($c$为常数),称$u_n$与$v_n$是同阶无穷小,记作$u_n = O(v_n)$。此时$\{x_n\}$与$\{y_n\}$以相同速度收敛于$a$
            \item \textbf{等价无穷小}:当$I = 1$时,称$u_n$与$v_n$是等价无穷小,记作$u_n \sim v_n$。这是同阶无穷小的特例
            \item \textbf{低阶无穷小}:当$I = \infty$时,称$u_n$是$v_n$的低阶无穷小(或$v_n$是$u_n$的高阶无穷小)。此时$\{x_n\}$比$\{y_n\}$更慢收敛于$a$
        \end{itemize}

        \begin{analysisbox}[示例]
            设$x_n = \dfrac{1}{n}$,$y_n = \dfrac{1}{\sqrt{n}}$,$\lim\limits_{n\to \infty} x_n = \lim\limits_{n\to \infty} y_n = 0$,则
            \[
                I = \lim\limits_{n\to\infty} \dfrac{\left|\dfrac{1}{n} - 0\right|}{\left|\dfrac{1}{\sqrt{n}} - 0\right|}
                = \lim\limits_{n\to\infty} \dfrac{\dfrac{1}{n}}{\dfrac{1}{\sqrt{n}}}
                = \lim\limits_{n\to\infty} \dfrac{1}{\sqrt{n}}
                = 0
            \]
            故$\dfrac{1}{n} = o\!\left(\dfrac{1}{\sqrt{n}}\right)$,即$x_n = \dfrac{1}{n}$比$y_n = \dfrac{1}{\sqrt{n}}$更快收敛于$0$。
        \end{analysisbox}
    \end{enumerate}

    \subsection{收敛数列的性质}

    \begin{enumerate}
        \item 唯一性: 给出数列$\{x_n\}$,若$\lim\limits_{n\to\infty} x_n = a$(存在),则$a$是唯一的
        \item 有界性: 若数列$\{x_n\}$极限存在,则数列$\{x_n\}$有界
        \item 保号性: 设数列 $\{x_n\}$ 收敛于 $a$,$b$ 为任意实数。

        \begin{enumerate}
            \item 若 $a>b$(或 $a<b$),
            则存在正整数 $N$,
            使得当 $n>N$ 时,
            恒有
            \[
                x_n>b \quad (\text{或 } x_n<b).
            \]

            \item 若存在正整数 $N$,
            使得当 $n>N$ 时,
            恒有
            \[
                x_n\ge b \quad (\text{或 } x_n\le b),
            \]
            且
            \[
                \lim_{n\to\infty} x_n=a,
            \]
            则必有
            \[
                a\ge b \quad (\text{或 } a\le b).
            \]
        \end{enumerate}

        其中常考情形为 $b=0$
        \item \textbf{脱帽(严格不等):} $\lim\limits_{n\to\infty} x_n > b \Rightarrow x_n > b \quad (\text{或 } \lim\limits_{n\to\infty} x_n < b \Rightarrow x_n < b)$
        \item \textbf{带帽(非严格不等):} $x_n \geq b \Rightarrow \lim\limits_{n\to\infty} x_n \geq b \quad (\text{或 } x_n \leq b \Rightarrow \lim\limits_{n\to\infty} x_n \leq b)$
    \end{enumerate}

    \subsection{定理}

    \begin{enumerate}
        \item 若数列 $\{a_n\}$ 收敛,则其任意子列 $\{a_{n_k}\}$ 也收敛,
        且
        \[
            \lim_{k\to\infty} a_{n_k}
            =
            \lim_{n\to\infty} a_n.
        \]
        \item 海涅定理(归结原则): 设 $f(x)$ 在$\mathring{U}(x_0,\delta)$内有定义,则$\lim\limits_{x\to x_0} f(x) = A$存在 $\Leftrightarrow$ 对任何$\mathring{U}(x_0,\delta)$内以$x_0$为极限的数列$\{x_n\}(x_n \neq x_0)$,极限$\lim\limits_{x\to x_0} f(x_n) = A$存在
        \begin{itemize}
            \item 当$x\to 0$时,取$x_n = \displaystyle\frac{1}{n}$,若$\lim\limits_{x\to 0} f(x) = A$,则$\lim\limits_{n\to \infty} f(\displaystyle\frac{1}{n}) = A$
            \item 当$x\to +\infty$时,取$x_n = n$,若$\lim\limits_{x\to +\infty} f(x) = A$,则$\lim\limits_{n\to \infty} f(n) = A$
            \item 当$x\to a$时,且$x_n \neq a$时,若$\lim\limits_{x\to a} f(x) = A$,则$\lim\limits_{n\to \infty} f(x_n) = A$
        \end{itemize}
        \item 单调有界准则: 若数列${x_n}$单调增加(减少)且有上界(下界),则$\lim\limits_{x\to \infty} x_n$存在
        \begin{itemize}
            \item $x_n \leq x_{n+1} \leq a$,则$\lim\limits_{x\to \infty} x_n$存在
            \item $a \leq x_{n+1} \leq x_n$,则$\lim\limits_{x\to \infty} x_n$存在
        \end{itemize}
    \end{enumerate}

    \subsection{结论}

    \begin{enumerate}
        \item $\lim\limits_{n\to\infty} a_n = A \; {\color{red}{\Rightarrow}} \lim\limits_{n\to\infty} |a_n| = |A|$
        \item $\lim\limits_{x\to x_0} f(x) = A \; {\color{red}{\Rightarrow}} \lim\limits_{x\to x_0} |f(x)| = |A|$
        \item $\lim\limits_{x\to x_0} f(x) = 0 \; {\color{red}{\Leftrightarrow}} \lim\limits_{x\to x_0} |f(x)| = 0$
        \item $\lim\limits_{n\to\infty} a_n = 0 \; {\color{red}{\Leftrightarrow}} \lim\limits_{n\to\infty} |a_n| = 0$ \\
        因此,若要证明
        \[
            \lim_{n\to\infty} a_n = 0,
        \]
        只需证明
        \[
            \lim_{n\to\infty} |a_n| = 0.
        \]

        又由于 $|a_n|\ge 0$,
        可利用夹逼准则:
        若存在数列 $\{b_n\}$,使得
        \[
            0 \le |a_n| \le b_n,
            \qquad
            \lim_{n\to\infty} b_n = 0,
        \]
        则有
        \[
            \lim_{n\to\infty} a_n = 0.
        \]
        \item $\color{red}{\bigstar}$若 $a_1,a_2,\dots,a_m \ge 0$,则
        \[
            \lim_{n\to\infty} \sqrt[n]{a_1^n + a_2^n + \cdots + a_m^n}
            = \max\{a_1,a_2,\dots,a_m\}
        \]
    \end{enumerate}

    \subsection{运算}

    \begin{enumerate}

    \end{enumerate}

    \subsection{公式}

    \begin{enumerate}

    \end{enumerate}

    \subsection{方法步骤}

    \begin{enumerate}
        \item 判断数列发散的方法
        \begin{itemize}
            \item 对于一个数列$\{a_n\}$,如果能找到一个发散的子列,则原数列一定发散
            \item 对于一个数列$\{a_n\}$,如果能找到至少两个收敛的子列$\{x_{n_k}\}$和$\{x_{n'_k}\}$,但它们收敛到不同极限,则原数列一定发散
        \end{itemize}
        \item \textbf{证明数列 $\{x_n\}$ 单调性的常用方法}
        \begin{enumerate}[label=\alph*.]
            \item \textbf{差分法或比值法}
            \[
                x_{n+1}-x_n > 0 \ (\text{或 } < 0),
                \quad \text{或} \quad
                \frac{x_{n+1}}{x_n} > 1 \ (\text{或 } < 1).
            \]

            \item \textbf{数学归纳法}
            \begin{enumerate}
                \item 验证 $n=1$ 时结论成立;
                \item 假设 $n=k$ 时结论成立;
                \item 证明 $n=k+1$ 时结论仍成立。
            \end{enumerate}

            \item \textbf{利用重要不等式}

            \item \textbf{利用递推形式 $x_{n+1}=f(x_n)(n = 1, 2, \cdots)$}

            设 $x_n \in I$,且 $f(x)$ 在区间 $I$ 上可导:
            \begin{itemize}
                \item 若 $f'(x) > 0$(即 $f(x)$ 在 $I$ 上单调增加),
                则数列 $\{x_n\}$ 单调(无法确定是单调递增还是单调递减),且
                \[
                    \begin{cases}
                        x_2 > x_1, & \text{则 } \{x_n\} \text{ 单调递增}, \\[2mm]
                        x_2 < x_1, & \text{则 } \{x_n\} \text{ 单调递减}.
                    \end{cases}
                \]
                \item 若 $f'(x) < 0$(即 $f(x)$ 在 $I$ 上单调减少),
                则数列 $\{x_n\}$ \textbf{不单调}(通常呈振荡)。
            \end{itemize}
        \end{enumerate}
    \end{enumerate}

    \subsection{条件转换思路}

    \begin{enumerate}

    \end{enumerate}

    \subsection{理解}

    \begin{enumerate}
        \item 压缩映射原理: 使用时,需写出证明过程
        \begin{itemize}
            \item 对数列${x_n}$,若存在常数$k(0 < k < 1)$,使得$|x_{n+1} - a| \leq k|x_n - a|, n = 1, 2, \cdots$,则${x_n}$收敛于$a$
            \item {\color{red}{TODO: 证}}
            \item
            设数列 $\{x_n\}$ 满足
            \[
                x_{n+1}=f(x_n), \quad n=1,2,\dots,
            \]
            其中 $f(x)$ 在 $\mathbb{R}$ 上可导,$a$ 是方程
            \[
                f(x)=x
            \]
            的唯一解。若存在常数 $k$,使得对任意 $x\in\mathbb{R}$ 都有
            \[
                |f'(x)|\le k<1,
            \]
            则数列 $\{x_n\}$ 收敛,且
            \[
                \lim_{n\to\infty} x_n = a.
            \]
            \item {\color{red}{TODO: 证}}
        \end{itemize}
        \item 极限定义中$\epsilon$的含义:$\epsilon$必须是不依赖于$n$的任意小正数。若$\epsilon$与$n$有关,相当于对收敛速度提出了额外要求,而极限定义并不限制收敛速度
    \end{enumerate}

\end{document}
