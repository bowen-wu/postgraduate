\documentclass[a4paper,12pt]{article}
\usepackage{xeCJK}          % 中文支持
\usepackage{fontspec}       % 英文/数学字体
\usepackage{amsmath, amssymb} % 数学公式
\usepackage{graphicx}       % 插入图片
\usepackage{hyperref}       % 目录超链接
\usepackage{geometry}       % 页面布局
\usepackage{bm}             % 粗体
\usepackage{xcolor}         % 颜色
\usepackage{tabularx}       % 表格环境
\usepackage{tikz}           % TikZ 绘制主对角线斜线
\usepackage{tcolorbox}
\usepackage{xstring}
\geometry{left=3cm,right=3cm,top=3cm,bottom=3cm}

% 抽离颜色和尺寸参数
\newcommand{\analysisTitleColor}{green!50!black}
\newcommand{\analysisBackColor}{white}
\newcommand{\analysisBoxRule}{0.8pt}
\newcommand{\analysisArc}{3pt}
\newcommand{\analysisPadding}{6pt}

% 定义 tcolorbox
\newtcolorbox{analysisbox}[1][]{
    title=\IfStrEq{#1}{}{\textbf{解析}}{#1}, % 如果传参为空则使用“解析”
    colback=\analysisBackColor,
    colframe=\analysisTitleColor,
    boxrule=\analysisBoxRule,
    arc=\analysisArc,
    left=\analysisPadding,
    right=\analysisPadding,
    top=4pt,
    bottom=4pt
}

% =========================
% 字体设置
% =========================
\setmainfont{Times New Roman}
\setsansfont{Helvetica Neue}
\setmonofont{Menlo}
\setCJKmainfont{PingFang SC}

% =========================
% 图形路径(可调整)
% =========================
\graphicspath{{./assets/}}

% =========================
% 文档开始
% =========================
\begin{document}

%    \title{Template}
%    \author{Bowen}
%    \date{\today}
%    \maketitle
%    \tableofcontents
%    \newpage


% =========================

    \section{数列极限}

    \subsection{数列}

    \begin{enumerate}
        \item 等差数列
        \item 等比数列
    \end{enumerate}

    \subsection{基础概念}

    \begin{enumerate}
        \item 设${x_n}$为以数列,若存在常数$a$,对于任意的$\epsilon > 0$(不论它多么小),总存在正整数$N$,使得当$n > N$时,$|x_n - a| < \epsilon$,则称常数$a$是数列${x_n}$的极限,或者称数列${x_n}$收敛于$a$,记为
        \begin{gather*}
            \lim_{n \to \infty} x_n = a \\
            \forall \epsilon > 0,\ \exists N > 0,\ n > N \Rightarrow |x_n - a| < \epsilon
        \end{gather*}
        \item 有界数列: 若对所有正整数$n$,存在正实数$M$,有$|a_n| \leq M$,则称数列${a_n}$为有界数列。证明数列有界的方法
        \begin{itemize}
            \item 找$M$,使得$|a_n| \leq M$
            \item 放缩法
            \item 找最值
            \item 基本不等式法
        \end{itemize}
        \item 设 $\{x_n\}$ 为一数列,若存在常数$a$,对任意 $\varepsilon>0$(无论它多么小),总存在正整数 $N$,使得当 $n>N$ 时,$|x_n - a| < \varepsilon$恒成立,则称数列 $\{x_n\}$ \textbf{收敛于} $a$,或者称常数$a$是数列${x_n}$的极限,记作$\lim\limits_{n\to\infty} x_n = a$
    \end{enumerate}


    \subsection{收敛数列的性质}

    \begin{enumerate}
        \item 唯一性: 给出数列$\{x_n\}$,若$\lim\limits_{n\to\infty} x_n = a$(存在),则$a$是唯一的
        \item 有界性: 若数列$\{x_n\}$极限存在,则数列$\{x_n\}$有界
        \item 保号性: 设数列 $\{x_n\}$ 收敛于 $a$,$b$ 为任意实数。

        \begin{enumerate}
            \item 若 $a>b$(或 $a<b$),
            则存在正整数 $N$,
            使得当 $n>N$ 时,
            恒有
            \[
                x_n>b \quad (\text{或 } x_n<b).
            \]

            \item 若存在正整数 $N$,
            使得当 $n>N$ 时,
            恒有
            \[
                x_n\ge b \quad (\text{或 } x_n\le b),
            \]
            且
            \[
                \lim_{n\to\infty} x_n=a,
            \]
            则必有
            \[
                a\ge b \quad (\text{或 } a\le b).
            \]
        \end{enumerate}

        其中常考情形为 $b=0$
        \item \textbf{脱帽(严格不等):} $\lim\limits_{n\to\infty} x_n > b \Rightarrow x_n > b \quad (\text{或 } \lim\limits_{n\to\infty} x_n < b \Rightarrow x_n < b)$
        \item \textbf{带帽(非严格不等):} $x_n \geq b \Rightarrow \lim\limits_{n\to\infty} x_n \geq b \quad (\text{或 } x_n \leq b \Rightarrow \lim\limits_{n\to\infty} x_n \leq b)$
    \end{enumerate}

    \subsection{定理}

    \begin{enumerate}
        \item 若数列 $\{a_n\}$ 收敛,则其任意子列 $\{a_{n_k}\}$ 也收敛,
        且
        \[
            \lim_{k\to\infty} a_{n_k}
            =
            \lim_{n\to\infty} a_n.
        \]
        \item 海涅定理(归结原则): 设 $f(x)$ 在$\mathring{U}(x_0,\delta)$内有定义,则$\lim\limits_{x\to x_0} f(x) = A$存在 $\Leftrightarrow$ 对任何$\mathring{U}(x_0,\delta)$内以$x_0$为极限的数列$\{x_n\}(x_n \neq x_0)$,极限$\lim\limits_{x\to x_0} f(x_n) = A$存在
        \begin{itemize}
            \item 当$x\to 0$时,取$x_n = \displaystyle\frac{1}{n}$,若$\lim\limits_{x\to 0} f(x) = A$,则$\lim\limits_{n\to \infty} f(\displaystyle\frac{1}{n}) = A$
            \item 当$x\to +\infty$时,取$x_n = n$,若$\lim\limits_{x\to +\infty} f(x) = A$,则$\lim\limits_{n\to \infty} f(n) = A$
            \item 当$x\to a$时,且$x_n \neq a$时,若$\lim\limits_{x\to a} f(x) = A$,则$\lim\limits_{n\to \infty} f(x_n) = A$
        \end{itemize}
    \end{enumerate}

    \subsection{结论}

    \begin{enumerate}
        \item 判断数列发散的方法
        \begin{itemize}
            \item 对于一个数列$\{a_n\}$,如果能找到一个发散的子列,则原数列一定发散
            \item 对于一个数列$\{a_n\}$,如果能找到至少两个收敛的子列$\{x_{n_k}\}$和$\{x_{n'_k}\}$,但它们收敛到不同极限,则原数列一定发散
        \end{itemize}
        \item $\lim\limits_{n\to\infty} a_n = A \; {\color{red}{\Rightarrow}} \lim\limits_{n\to\infty} |a_n| = |A|$
        \item $\lim\limits_{x\to x_0} f(x) = A \; {\color{red}{\Rightarrow}} \lim\limits_{x\to x_0} |f(x)| = |A|$
        \item $\lim\limits_{x\to x_0} f(x) = 0 \; {\color{red}{\Leftrightarrow}} \lim\limits_{x\to x_0} |f(x)| = 0$
        \item $\lim\limits_{n\to\infty} a_n = 0 \; {\color{red}{\Leftrightarrow}} \lim\limits_{n\to\infty} |a_n| = 0$ \\
        因此,若要证明
        \[
            \lim_{n\to\infty} a_n = 0,
        \]
        只需证明
        \[
            \lim_{n\to\infty} |a_n| = 0.
        \]

        又由于 $|a_n|\ge 0$,
        可利用夹逼准则:
        若存在数列 $\{b_n\}$,使得
        \[
            0 \le |a_n| \le b_n,
            \qquad
            \lim_{n\to\infty} b_n = 0,
        \]
        则有
        \[
            \lim_{n\to\infty} a_n = 0.
        \]
    \end{enumerate}

    \subsection{运算}

    \begin{enumerate}

    \end{enumerate}

    \subsection{公式}

    \begin{enumerate}

    \end{enumerate}

    \subsection{方法步骤}

    \begin{enumerate}

    \end{enumerate}

    \subsection{条件转换思路}

    \begin{enumerate}

    \end{enumerate}

    \subsection{理解}

    \begin{enumerate}
    \end{enumerate}

\end{document}
