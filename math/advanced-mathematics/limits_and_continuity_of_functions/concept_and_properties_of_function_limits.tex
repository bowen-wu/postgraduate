\documentclass[a4paper,12pt]{article}
\usepackage{xeCJK}          % 中文支持
\usepackage{fontspec}       % 英文/数学字体
\usepackage{amsmath, amssymb} % 数学公式
\usepackage{graphicx}       % 插入图片
\usepackage{hyperref}       % 目录超链接
\usepackage{geometry}       % 页面布局
\usepackage{bm}             % 粗体
\usepackage{xcolor}         % 颜色
\usepackage{tabularx}       % 表格环境
\usepackage{tikz}           % TikZ 绘制主对角线斜线
\usepackage{tcolorbox}
\usepackage{xstring}
\usepackage{pgfplots}
\pgfplotsset{compat=1.18}
\geometry{left=3cm,right=3cm,top=3cm,bottom=3cm}

% 抽离颜色和尺寸参数
\newcommand{\analysisTitleColor}{green!50!black}
\newcommand{\analysisBackColor}{white}
\newcommand{\analysisBoxRule}{0.8pt}
\newcommand{\analysisArc}{3pt}
\newcommand{\analysisPadding}{6pt}

% 定义 tcolorbox
\newtcolorbox{analysisbox}[1][]{
    title=\IfStrEq{#1}{}{\textbf{解析}}{#1}, % 如果传参为空则使用“解析”
    colback=\analysisBackColor,
    colframe=\analysisTitleColor,
    boxrule=\analysisBoxRule,
    arc=\analysisArc,
    left=\analysisPadding,
    right=\analysisPadding,
    top=4pt,
    bottom=4pt
}

% =========================
% 字体设置
% =========================
\setmainfont{Times New Roman}
\setsansfont{Helvetica Neue}
\setmonofont{Menlo}
\setCJKmainfont{PingFang SC}

% =========================
% 图形路径(可调整)
% =========================
\graphicspath{{./assets/}}

% =========================
% 文档开始
% =========================
\begin{document}

%    \title{Template}
%    \author{Bowen}
%    \date{\today}
%    \maketitle
%    \tableofcontents
%    \newpage


% =========================

    \section{函数极限的概念与性质}

    \subsection{基础概念}

    \renewcommand{\arraystretch}{1.6}
    \setlength{\tabcolsep}{6pt}

    \begin{tabular}{|c|p{3.2cm}|p{3.2cm}|p{3.2cm}|p{3.2cm}|}
        \hline
        $x \backslash f(x)$
        & $f(x)\to A$
        & $f(x)\to +\infty$
        & $f(x)\to -\infty$
        & $f(x)\to \infty$
        \\ \hline

        $x\to x_0$
        & $\begin{aligned}
               &\forall\varepsilon>0,\ \exists\delta>0,\\
               &0<|x-x_0|<\delta\\
               &\Rightarrow |f(x)-A|<\varepsilon
        \end{aligned}$
        & $\begin{aligned}
               &\forall M>0,\ \exists\delta>0,\\
               &0<|x-x_0|<\delta\\
               &\Rightarrow f(x)>M
        \end{aligned}$
        & $\begin{aligned}
               &\forall M>0,\ \exists\delta>0,\\
               &0<|x-x_0|<\delta\\
               &\Rightarrow f(x)<-M
        \end{aligned}$
        & $\begin{aligned}
               &\forall M>0,\ \exists\delta>0,\\
               &0<|x-x_0|<\delta\\
               &\Rightarrow |f(x)|>M
        \end{aligned}$
        \\ \hline

        $x\to x_0^+$
        & $\begin{aligned}
               &\forall\varepsilon>0,\ \exists\delta>0,\\
               &0<x-x_0<\delta\\
               &\Rightarrow |f(x)-A|<\varepsilon
        \end{aligned}$
        & $\begin{aligned}
               &\forall M>0,\ \exists\delta>0,\\
               &0<x-x_0<\delta\\
               &\Rightarrow f(x)>M
        \end{aligned}$
        & $\begin{aligned}
               &\forall M>0,\ \exists\delta>0,\\
               &0<x-x_0<\delta\\
               &\Rightarrow f(x)<-M
        \end{aligned}$
        & $\begin{aligned}
               &\forall M>0,\ \exists\delta>0,\\
               &0<x-x_0<\delta\\
               &\Rightarrow |f(x)|>M
        \end{aligned}$
        \\ \hline

        $x\to x_0^-$
        & $\begin{aligned}
               &\forall\varepsilon>0,\ \exists\delta>0,\\
               &0<x_0-x<\delta\\
               &\Rightarrow |f(x)-A|<\varepsilon
        \end{aligned}$
        & $\begin{aligned}
               &\forall M>0,\ \exists\delta>0,\\
               &0<x_0-x<\delta\\
               &\Rightarrow f(x)>M
        \end{aligned}$
        & $\begin{aligned}
               &\forall M>0,\ \exists\delta>0,\\
               &0<x_0-x<\delta\\
               &\Rightarrow f(x)<-M
        \end{aligned}$
        & $\begin{aligned}
               &\forall M>0,\ \exists\delta>0,\\
               &0<x_0-x<\delta\\
               &\Rightarrow |f(x)|>M
        \end{aligned}$
        \\ \hline

        $x\to +\infty$
        & $\begin{aligned}
               &\forall\varepsilon>0,\ \exists N>0,\\
               &x>N\\
               &\Rightarrow |f(x)-A|<\varepsilon
        \end{aligned}$
        & $\begin{aligned}
               &\forall M>0,\ \exists N>0,\\
               &x>N\\
               &\Rightarrow f(x)>M
        \end{aligned}$
        & $\begin{aligned}
               &\forall M>0,\ \exists N>0,\\
               &x>N\\
               &\Rightarrow f(x)<-M
        \end{aligned}$
        & $\begin{aligned}
               &\forall M>0,\ \exists N>0,\\
               &x>N\\
               &\Rightarrow |f(x)|>M
        \end{aligned}$
        \\ \hline

        $x\to -\infty$
        & $\begin{aligned}
               &\forall\varepsilon>0,\ \exists N>0,\\
               &x<-N\\
               &\Rightarrow |f(x)-A|<\varepsilon
        \end{aligned}$
        & $\begin{aligned}
               &\forall M>0,\ \exists N>0,\\
               &x<-N\\
               &\Rightarrow f(x)>M
        \end{aligned}$
        & $\begin{aligned}
               &\forall M>0,\ \exists N>0,\\
               &x<-N\\
               &\Rightarrow f(x)<-M
        \end{aligned}$
        & $\begin{aligned}
               &\forall M>0,\ \exists N>0,\\
               &x<-N\\
               &\Rightarrow |f(x)|>M
        \end{aligned}$
        \\ \hline

        $x\to \infty$
        & $\begin{aligned}
               &\forall\varepsilon>0,\ \exists N>0,\\
               &|x|>N\\
               &\Rightarrow |f(x)-A|<\varepsilon
        \end{aligned}$
        & $\begin{aligned}
               &\forall M>0,\ \exists N>0,\\
               &|x|>N\\
               &\Rightarrow f(x)>M
        \end{aligned}$
        & $\begin{aligned}
               &\forall M>0,\ \exists N>0,\\
               &|x|>N\\
               &\Rightarrow f(x)<-M
        \end{aligned}$
        & $\begin{aligned}
               &\forall M>0,\ \exists N>0,\\
               &|x|>N\\
               &\Rightarrow |f(x)|>M
        \end{aligned}$
        \\ \hline
    \end{tabular}

    \begin{enumerate}
    \end{enumerate}

    \subsection{函数极限的性质}
    \begin{enumerate}
        \item 唯一性: 如果极限$\lim\limits_{x\to{x_0}}f(x)$存在,那么极限唯一
        \begin{itemize}
            \item 函数极限存在的充要条件: \textbf{左右极限存在且相等}
        \end{itemize}
        \item 局部有界性: 如果$\lim\limits_{x\to{x_0}}f(x) = A$,则存在正常数$M$和$\delta$,使得当$0 < |x - x_0| < \delta$时,有$|f(x)| \leq M$
        \begin{itemize}
            \item 若$f(x)$在$(a, b)$内是连续函数,且$\lim\limits_{x\to a^+}f(x)$, $\lim\limits_{x\to b^-}f(x)$都存在,则$f(x)$在$(a, b)$内必定有界
        \end{itemize}
        \item $\color{red}{\bigstar}$局部保号性:
        \begin{itemize}
            \item 脱帽严格不等
            \item $\lim f > 0 \Rightarrow f > 0$
            \item $\lim f < 0 \Rightarrow f < 0$
            \item 带帽非严格不等
            \item $f \geq 0 \Rightarrow \lim f \geq 0$
            \item $f \leq 0 \Rightarrow \lim f \leq 0$
        \end{itemize}
    \end{enumerate}

    \subsection{等价无穷小}
    \begin{enumerate}
        \item $\displaystyle \lim_{x\to x_0}\frac{\alpha(x)}{\beta(x)}=\infty$,
        则 $\alpha(x)$ 是 $\beta(x)$ 的低阶无穷小
        \item $\displaystyle \lim_{x\to x_0}\frac{\alpha(x)}{\beta(x)}=0$,
        则 $\alpha(x)$ 是 $\beta(x)$ 的高阶无穷小
        \item $\displaystyle \lim_{x\to x_0}\frac{\alpha(x)}{\beta(x)}=c\neq 0$,
        则 $\alpha(x)$ 与 $\beta(x)$ 是同阶无穷小
        \item $\displaystyle \lim_{x\to x_0}\frac{\alpha(x)}{\beta(x)}=1$,
        则 $\alpha(x)\sim\beta(x)$,称为等价无穷小,记为$\alpha(x) \sim \beta(x)$
        \item 有界函数与无穷小的乘积是无穷小
        \item 当 $\color{red}{x \to 0}$ 时,常用的等价无穷小
        \begin{itemize}
            \item $x \sim \sin x$
            \item $x \sim \tan x$
            \item $x \sim \arcsin x$
            \item $x \sim \arctan x$
            \item $x \sim \ln(1+x)$
            $\Rightarrow\ \ln u \sim u - 1 \quad (u \to 1)$
            \item $x \sim e^x - 1$
            \item $x \sim \ln\!\left(x + \sqrt{x^2 + 1}\right)$
            \item $a^x - 1 \sim x \ln a \quad (a>0,\ a\neq 1)$
            $\Rightarrow\ e^{x\ln a} - 1 \sim x \ln a$
            \item $1 - \cos x \sim \displaystyle\frac{1}{2}x^2$
            $\Rightarrow\ 1 - (\cos x)^a \sim \displaystyle\frac{a}{2}x^2 \quad (a\neq 0)$
            \item $x - \ln(1 + x) \sim \displaystyle\frac{1}{2}x^2$
            \item $(1+x)^a - 1 \sim ax$
            \item $\tan x - \sin x = \tan x(1 - \cos x) \sim \displaystyle\frac{1}{2}x^3$
        \end{itemize}

        \item 当$x \to 0^+$时,$(1 + x)^{\displaystyle\frac{1}{x}} - e \sim -\displaystyle\frac{e}{2}x \Rightarrow (1 + \text{狗})^{\displaystyle\frac{1}{\text{狗}}} - e \sim -\displaystyle\frac{e}{2}\text{狗}$
        \begin{itemize}
            \item 当$n \to \infty$时,若$(1 + \displaystyle\frac{1}{n})^n - e$与$\displaystyle\frac{a}{n}$是等价无穷小,则$a = -\displaystyle\frac{e}{2}$
            \item 当$n \to \infty$时,$(1 + \displaystyle\frac{1}{n})^n - e \sim -\displaystyle\frac{e}{2}\cdot\frac{1}{n}$
        \end{itemize}
    \end{enumerate}

    \subsection{结论}

    \begin{enumerate}
        \item 极限存在必有界,有界极限不一定存在
        \item 若$\lim\limits_{x\to a^+}f(x)$存在,则存在$\delta > 0$,当$x\in(a, a + \delta)$时,$f(x)$有界
    \end{enumerate}

    \subsection{定理}

    \begin{enumerate}
    \end{enumerate}

    \subsection{运算}

    \begin{enumerate}

    \end{enumerate}

    \subsection{公式}

    \begin{enumerate}
        \item $\lim\limits_{u\to+\infty}e^u = +\infty$
        \item $\lim\limits_{u\to-\infty}e^u = 0$

    \end{enumerate}

    \subsection{方法总结}

    \begin{enumerate}
        \item 遇到$e^{\infty}$,需讨论$e^{-\infty}$和$e^{+\infty}$
        \item 分段函数在分段点处的极限要考虑左、右极限
        \item 对于 $e^{\tan x} - e^{\sin x}$,通常先提取公因式,使其化为 $e^{\Box} - 1$ 的形式。
    \end{enumerate}

    \subsection{条件转换思路}

    \begin{enumerate}

    \end{enumerate}

    \subsection{理解}

    \begin{enumerate}
        \item $\forall\varepsilon>0,\ \exists\delta>0, 0<|x-x_0|<\delta \Rightarrow |f(x)-A|<\varepsilon$,在定义中$x \neq x_0$,但$f(x)$可以等于$A$(绝对值是非负的)
    \end{enumerate}
\end{document}
