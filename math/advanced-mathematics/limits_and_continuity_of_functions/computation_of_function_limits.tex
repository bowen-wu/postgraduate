\documentclass[a4paper,12pt]{article}
\usepackage{xeCJK}          % 中文支持
\usepackage{fontspec}       % 英文/数学字体
\usepackage{amsmath, amssymb} % 数学公式
\usepackage{graphicx}       % 插入图片
\usepackage{hyperref}       % 目录超链接
\usepackage{geometry}       % 页面布局
\usepackage{bm}             % 粗体
\usepackage{xcolor}         % 颜色
\usepackage{tabularx}       % 表格环境
\usepackage{tikz}           % TikZ 绘制主对角线斜线
\usepackage{tcolorbox}
\usepackage{xstring}
\usepackage{pgfplots}
\pgfplotsset{compat=1.18}
\geometry{left=3cm,right=3cm,top=3cm,bottom=3cm}

% 抽离颜色和尺寸参数
\newcommand{\analysisTitleColor}{green!50!black}
\newcommand{\analysisBackColor}{white}
\newcommand{\analysisBoxRule}{0.8pt}
\newcommand{\analysisArc}{3pt}
\newcommand{\analysisPadding}{6pt}

% 定义 tcolorbox
\newtcolorbox{analysisbox}[1][]{
    title=\IfStrEq{#1}{}{\textbf{解析}}{#1}, % 如果传参为空则使用“解析”
    colback=\analysisBackColor,
    colframe=\analysisTitleColor,
    boxrule=\analysisBoxRule,
    arc=\analysisArc,
    left=\analysisPadding,
    right=\analysisPadding,
    top=4pt,
    bottom=4pt
}

% =========================
% 字体设置
% =========================
\setmainfont{Times New Roman}
\setsansfont{Helvetica Neue}
\setmonofont{Menlo}
\setCJKmainfont{PingFang SC}

% =========================
% 图形路径(可调整)
% =========================
\graphicspath{{./assets/}}

% =========================
% 文档开始
% =========================
\begin{document}

%    \title{Template}
%    \author{Bowen}
%    \date{\today}
%    \maketitle
%    \tableofcontents
%    \newpage


% =========================

    \section{函数极限的计算}

    \subsection{方法}

    \begin{enumerate}
        \item \textbf{极限四则运算规则}: 当$f(x)$和$g(x)$极限都存在时,函数的加减乘除的极限分别等于极限的加减乘除(除要求分母的极限不为零)
        \item \textbf{洛必达法则一($\color{red}{\dfrac{0}{0}}$ 型)}

        设当
        \[
            x \to a \quad \text{或} \quad x \to \infty
        \]
        时,函数 $f(x)$ 与 $F(x)$ 都趋于 {\color{red}{0}},即
        \[
            \lim_{x\to a} f(x) = \lim_{x\to a} F(x) = 0
            \quad \text{或} \quad
            \lim_{x\to \infty} f(x) = \lim_{x\to \infty} F(x) = 0.
        \]

        若在点 $a$ 的某去心邻域内(或当 $|x|>X$,其中 $X$ 为充分大的正数时),
        $f'(x)$ 与 $F'(x)$ 存在,且
        \[
            F'(x) \neq 0,
        \]
        并且极限
        \[
            \lim_{x\to a} \frac{f'(x)}{F'(x)}
            \quad \text{或} \quad
            \lim_{x\to \infty} \frac{f'(x)}{F'(x)}
        \]
        存在或为无穷大,

        则有
        \[
            \lim_{x\to a} \frac{f(x)}{F(x)}
            =
            \lim_{x\to a} \frac{f'(x)}{F'(x)},
            \qquad
            \lim_{x\to \infty} \frac{f(x)}{F(x)}
            =
            \lim_{x\to \infty} \frac{f'(x)}{F'(x)}.
        \]
        \item \textbf{洛必达法则二($\color{red}{\dfrac{\infty}{\infty}}$ 型)}
        设当
        \[
            x \to a \quad \text{或} \quad x \to \infty
        \]
        时,函数 $f(x)$ 与 $F(x)$ 都趋于 {\color{red}{无穷大}},即
        \[
            \lim_{x\to a} f(x) = \lim_{x\to a} F(x) = \infty
            \quad \text{或} \quad
            \lim_{x\to \infty} f(x) = \lim_{x\to \infty} F(x) = \infty.
        \]

        若在点 $a$ 的某去心邻域内(或当 $|x|>X$,其中 $X$ 为充分大的正数时),
        $f'(x)$ 与 $F'(x)$ 存在,且
        \[
            F'(x) \neq 0,
        \]
        并且极限
        \[
            \lim_{x\to a} \frac{f'(x)}{F'(x)}
            \quad \text{或} \quad
            \lim_{x\to \infty} \frac{f'(x)}{F'(x)}
        \]
        存在或为无穷大,

        则有
        \[
            \lim_{x\to a} \frac{f(x)}{F(x)}
            =
            \lim_{x\to a} \frac{f'(x)}{F'(x)},
            \qquad
            \lim_{x\to \infty} \frac{f(x)}{F(x)}
            =
            \lim_{x\to \infty} \frac{f'(x)}{F'(x)}.
        \]

        \item \textbf{泰勒公式}: 设$f(x)$在点$x = a$处$n$阶可导,则存在$x = a$的一个邻域,对于该邻域内的任一点$x$,有
        \[
            f(x) = f(a) + f'(a)(x-a) + \frac{f''(a)}{2!}(x-a)^2
            + \cdots +
            + \frac{f^{(n)}(a)}{n!}(x-a)^n + o(x^n)
        \]
        \begin{itemize}
            \item $\displaystyle \sin x = x - \frac{x^3}{3!} + o(x^3) \Rightarrow \text{狗} - \sin \text{狗} \sim \frac{\text{狗}^3}{6}, \quad \text{狗}\to 0$
            \item $\displaystyle \arcsin x = x + \frac{x^3}{6} + o(x^3) \Rightarrow \arcsin \text{狗} - \text{狗} \sim \frac{\text{狗}^3}{6}, \quad \text{狗}\to 0$
            \item $\displaystyle \cos x = 1 - \frac{x^2}{2!} + \frac{x^4}{4!} + o(x^4) \Rightarrow 1 - \cos \text{狗} \sim \frac{\text{狗}^2}{2}, \quad \text{狗}\to 0$
            \item $\displaystyle \tan x = x + \frac{1}{3}x^3 + o(x^3) \Rightarrow \tan \text{狗} - \text{狗} \sim \frac{\text{狗}^3}{3}, \quad \text{狗}\to 0$
            \item $\displaystyle \arctan x = x - \frac{x^3}{3} + o(x^3) \Rightarrow \text{狗} - \arctan \text{狗} \sim \frac{\text{狗}^3}{3}, \quad \text{狗}\to 0$
            \item $\displaystyle e^x = 1 + x + \frac{x^2}{2!} + \frac{x^3}{3!} + o(x^3) \Rightarrow e^{\text{狗}} - 1 - \text{狗} \sim \frac{\text{狗}^2}{2}, \quad \text{狗}\to 0$
            \item $\displaystyle \ln(1+x) = x - \frac{x^2}{2} + \frac{x^3}{3} + o(x^3) \Rightarrow \text{狗} - \ln(1+\text{狗}) \sim \frac{\text{狗}^2}{2}, \quad \text{狗}\to 0$
            \item $\displaystyle (1+x)^a = 1 + ax + \frac{a(a-1)}{2!}x^2 + o(x^2) \Rightarrow (1+\text{狗})^a - 1 - a\text{狗} \sim \frac{a(a-1)}{2}\text{狗}^2, \quad \text{狗}\to 0$
            \item $\displaystyle \frac{1}{1+x^2} = 1 - x^2 + x^4 + o(x^4)$
        \end{itemize}
        \item \textbf{无穷小计算(记号运算)}:设 $m,n$ 为正整数,$x\to 0$,则
        \begin{itemize}
            \item $o(x^m)\pm o(x^n)=o\!\left(x^{\min\{m,n\}}\right)$
            (对无穷小而言,次数越小,量级越大;对无穷大而言,次数越大,量级越大)

            \item $o(x^m)\cdot o(x^n)=o(x^{m+n}),\quad
            x^m\cdot o(x^n)=o(x^{m+n})$

            \item $o(kx^m)=o(x^m),\quad
            k\,o(x^m)=o(x^m),\quad k\neq 0\text{ 为常数}$
        \end{itemize}

        \item \textbf{泰勒公式应用时的展开原则:}
        \begin{itemize}
            \item $\displaystyle\frac{A}{B}$型,适用"上下同阶"原则: 如果分母(或分子)是$x$的$k$次幂,则应把分子(或分母)展开到$x$的$k$次幂
            \item $A - B$型,适用"幂次最低"原则: 将$A, B$分别展开到他们的系数不相等的$x$的最低次幂为止
        \end{itemize}
        \item \textbf{两个重要极限:}
        \begin{itemize}
            \item $\displaystyle \lim_{x\to 0} \frac{\sin x}{x} = 1
            \Rightarrow \lim_{\text{狗}\to 0} \frac{\sin \text{狗}}{\text{狗}} = 1$
            \item $\displaystyle \lim_{x\to \infty} \left(1 + \frac{1}{x}\right)^x = e
            \Rightarrow \lim_{\text{狗}\to \infty} \left(1 + \frac{1}{\text{狗}}\right)^\text{狗} = e$
            \item $\displaystyle \text{狗} = \frac{1}{x} \Rightarrow \lim_{x\to 0} (1+x)^{\displaystyle\frac{1}{x}} = e$
        \end{itemize}
        \item \textbf{夹逼准则:} 如果函数$f(x), g(x)$及$h(x)$满足下列条件:
        \begin{itemize}
            \item $h(x) \leq f(x) \leq g(x)$
            \item $\displaystyle \lim h(x) = \lim g(x) = A$
        \end{itemize}
        则 $\displaystyle \lim f(x)$ 存在,且 $\displaystyle \lim f(x) = A$
    \end{enumerate}

    \subsection{七种未定式的计算}

    \begin{enumerate}
        \item $\displaystyle\frac{0}{0}, \frac{\infty}{\infty}, 0\cdot \infty, \infty - \infty, \infty^0, 0^0, 1^\infty$
        \item 解题思路
        \begin{itemize}
            \item 化简先行
            \begin{enumerate}
                \item 提出极限不为$0$的因式
                \item 等价无穷小替换
                \item 恒等变形:提公因式、拆项、合并、分子分母同除变量的最高次幂 + 换元法(负代换、倒代换)
            \end{enumerate}
            \item 判断类型(运算类型)
            \item 选择方法:泰勒公式、洛必达法则、夹逼准则
        \end{itemize}
        \item $\displaystyle\frac{0}{0}, \frac{\infty}{\infty}$
        \begin{itemize}
            \item 泰勒公式
            \item 洛必达法则
            \item \[
                      \lim_{x \to \infty}
                      \frac{a_nx^n + a_{n-1}x^{n-1} + \dots + a_1x + a_0}
                      {b_mx^m + b_{m-1}x^{m-1} + \dots + b_1x + b_0}
                      =
                      \begin{cases}
                          \dfrac{a_n}{b_m}, & n = m,\\[1.5mm]
                          \infty, & n > m,\\[1mm]
                          0, & n < m.
                      \end{cases}
            \]

            \textbf{抓大头法:} \\
            当 $\color{red}{x \to \infty}$,则应抓分子和分母中关于 $x$ 的{\color{red}{最高次项}}即可判断极限; \\
            当 $\color{red}{x \to 0}$,则应抓分子和分母中关于 $x$ 的{\color{red}{最低次项}}
            \item 脱帽法: 若$\lim\limits_{x\to 0} f(x) = A$,则$f(x) = A + \alpha(x)$,其中$\lim\limits_{x\to 0} \alpha(x) = 0$
        \end{itemize}
        \item $0\cdot \infty$: 设置分母,简单因式下放
        \begin{itemize}
            \item $\displaystyle \frac{0}{\displaystyle\frac{1}{\infty}} = \frac{0}{0}$
            \item $\displaystyle \frac{\infty}{\displaystyle\frac{1}{0}} = \frac{\infty}{\infty}$
            \item 简单因式: $x^\alpha, e^{\betax}, \sin rx$
            \item 复杂因式: $\arctan x, \arcsin x, \ln x$
        \end{itemize}
        \item $\infty - \infty$
        \begin{itemize}
            \item 如果函数是有分母,则通分,将加减法变形为乘除法,以便使用其他计算工具($\displaystyle\frac{0}{0}, \frac{\infty}{\infty}$)
            \item 如果函数中没有分母,则可以通过{\color{red}{提取公因式}}或者做{\color{red}{倒代换}}({\color{red}{创造分母再通分}}),出现分母后,再利用通分等恒等变形的方法,将加减法变形为乘除法
        \end{itemize}
        \item $\infty^0, 0^0$: $\lim u^v = e^{\lim v\ln u}$
        \item $1^\infty$: $\lim u^v = e^{\lim v(u - 1)}$
        \begin{analysisbox}[Example]
            \[
                \lim u^v = \lim \{[1 + (u - 1)]^{\frac{1}{u-1}}\}^{(u-1)v} = e^{\lim(u - 1)v}
            \]
        \end{analysisbox}
        \item 泰勒公式
    \end{enumerate}

    \subsection{结论}

    \begin{enumerate}
        \item 若$\lim \displaystyle\frac{f(x)}{g(x)} = A$,且$\lim g(x) = 0$,则$\lim f(x) = 0$
        \item 若$\lim \displaystyle\frac{f(x)}{g(x)} = A \neq 0$,且$\lim f(x) = 0$,则$\lim g(x) = 0$
        \item (连续型)当 $x\to+\infty$ 时,
        \[
            \ln^{\alpha} x \ll x^{\beta} \ll a^{x},\qquad \alpha,\beta>0,\ a>1
        \]

        \item (离散型)当 $n\to\infty$ 时,
        \[
            \ln^{\alpha} n \ll n^{\beta} \ll a^{n} \ll n! \ll n^{n},\qquad \alpha,\beta>0,\ a>1
        \]
        \item 当 $\alpha > 0$ 时,
        \[
            \lim_{x \to 0^+} x^\alpha \ln x
            = \lim_{x \to 0^+} \frac{\ln x}{x^{-\alpha}}
            \overset{\text{洛必达求导}}{=}
            \lim_{x \to 0^+}
            \frac{\dfrac{1}{x}}{-\alpha x^{-\alpha-1}}
            = -\frac{1}{\alpha}
            \lim_{x \to 0^+} x^\alpha
            = 0.
        \]
        一般形式: $\lim\limits_{x\to 0^+} x^\alpha\ln^\beta x = 0$,其中$\alpha, \beta > 0$
        \item $\text{狗} - 1 < [\text{狗}] \leq \text{狗}$
    \end{enumerate}

    \subsection{定理}

    \begin{enumerate}
    \end{enumerate}

    \subsection{运算}

    \begin{enumerate}

    \end{enumerate}

    \subsection{公式}

    \begin{enumerate}

    \end{enumerate}

    \subsection{方法总结}

    \begin{enumerate}
        \item 遇到幂指函数,用$e$括起来,$u(x)^{v(x)} = e^{v(x)\ln u(x)}$
        \item $f(x) = (1 + x)^{\displaystyle\frac{1}{x}}$在$x > 0$时有以下性质
        \begin{itemize}
            \item $f(x)$单调减少
            \item $\lim\limits_{x\to 0^+} (1 + x)^{\displaystyle\frac{1}{x}} = e$
            \item $\lim\limits_{x\to +\infty}(1 + x)^{\displaystyle\frac{1}{x}} = 1$
            \begin{tikzpicture}
                \begin{axis}
                    [
                    xshift=-1.8cm,
                    axis lines=middle,
                    xlabel={$x$},
                    ylabel={$y$},
                    domain=-3:3,
                    samples=200,
                    width=11cm,
                    height=9cm,
                    xtick=\empty,
                    ytick=\empty,
                    xmin=-1, xmax=25,
                    ymin=-0.5, ymax=4,      % y 轴范围,保证 y=1 和 y=e 在视野内
                    legend style={
                        at={(0.02,0.98)},
                        anchor=north west,
                        xshift=240pt,
                        yshift=-56pt,
                        draw=none
                    }
                    ]

                    % 红色曲线: (1+x)^(1/x), x>0
                    \addplot[red, thick, domain=0.01:20, samples=5000] {(1 + x)^(1/x)};
                    \addlegendentry{$y=(1+x)^{\displaystyle\frac{1}{x}}$}

                    % 红色空心圆表示未定义,不加入图例
                    \addplot[red, mark=o, only marks, forget plot] coordinates {(0,2.71828)};

                    % 绿色曲线: (1+1/x)^x, x>0
                    \addplot[green, thick, domain=0.001:20, samples=5000] {(1 + 1/x)^x};
                    \addlegendentry{$y=(1+\displaystyle\frac{1}{x})^x$}

                    % 绿色空心圆表示未定义,不加入图例
                    \addplot[green, mark=o, only marks, forget plot] coordinates {(0,1)};

                    % 虚线 y=1 和 y=e
                    \addplot[dashed, black, domain=0:20] {1};
                    \addlegendentry{$y=1$}

                    \addplot[dashed, black, domain=0:20] {exp(1)};
                    \addlegendentry{$y=e$}
                \end{axis}
            \end{tikzpicture}
        \end{itemize}
    \end{enumerate}

    \subsection{条件转换思路}

    \begin{enumerate}

    \end{enumerate}

    \subsection{理解}

    \begin{enumerate}
        \item 对于
        \[
            \lim_{x\to a} \frac{f(x)}{F(x)}
            =
            \lim_{x\to a} \frac{f'(x)}{F'(x)},
        \]
        \textbf{右极限存在,则左极限存在;但左极限存在,并不意味着右极限一定存在。} \emph{e.g.} $\lim\limits_{x\to 0} \displaystyle\frac{x^2 \sin \displaystyle\frac{1}{x}}{x}$
    \end{enumerate}
\end{document}
