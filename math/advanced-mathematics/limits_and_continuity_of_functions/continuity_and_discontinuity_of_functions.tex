\documentclass[a4paper,12pt]{article}
\usepackage{xeCJK}          % 中文支持
\usepackage{fontspec}       % 英文/数学字体
\usepackage{amsmath, amssymb} % 数学公式
\usepackage{graphicx}       % 插入图片
\usepackage{hyperref}       % 目录超链接
\usepackage{geometry}       % 页面布局
\usepackage{bm}             % 粗体
\usepackage{xcolor}         % 颜色
\usepackage{tabularx}       % 表格环境
\usepackage{tikz}           % TikZ 绘制主对角线斜线
\usepackage{tcolorbox}
\usepackage{xstring}
\usepackage{pgfplots}
\pgfplotsset{compat=1.18}
\geometry{left=3cm,right=3cm,top=3cm,bottom=3cm}

% 抽离颜色和尺寸参数
\newcommand{\analysisTitleColor}{green!50!black}
\newcommand{\analysisBackColor}{white}
\newcommand{\analysisBoxRule}{0.8pt}
\newcommand{\analysisArc}{3pt}
\newcommand{\analysisPadding}{6pt}

% 定义 tcolorbox
\newtcolorbox{analysisbox}[1][]{
    title=\IfStrEq{#1}{}{\textbf{解析}}{#1}, % 如果传参为空则使用“解析”
    colback=\analysisBackColor,
    colframe=\analysisTitleColor,
    boxrule=\analysisBoxRule,
    arc=\analysisArc,
    left=\analysisPadding,
    right=\analysisPadding,
    top=4pt,
    bottom=4pt
}

% =========================
% 字体设置
% =========================
\setmainfont{Times New Roman}
\setsansfont{Helvetica Neue}
\setmonofont{Menlo}
\setCJKmainfont{PingFang SC}

% =========================
% 图形路径(可调整)
% =========================
\graphicspath{{./assets/}}

% =========================
% 文档开始
% =========================
\begin{document}

%    \title{Template}
%    \author{Bowen}
%    \date{\today}
%    \maketitle
%    \tableofcontents
%    \newpage


% =========================

    \section{函数的连续与间断}

    \subsection{连续点的定义}

    \begin{enumerate}
        \item 设函数$f(x)$在点$x_0$的某一邻域内有定义,且有$\lim\limits_{x\to x_0}f(x) = f(x_0)$,则称函数$f(x)$在点$x_0$处连续
        \begin{itemize}
            \item $\lim\limits_{x\to x_0^-}f(x) = \lim\limits_{x\to x_0^+}f(x) = f(x_0) \Leftrightarrow f(x)\text{在点}x_0\text{处连续}$
            \item 设$f(x)$在点$x=x_0$处连续,且$f(x_0) > 0$(或$f(x_0) < 0$),则存在$\delta > 0$,使得当$|x - x_0| < \delta$时,$f(x) > 0$(或$f(x) < 0$)
        \end{itemize}
        \item 连续性运算法则
        \begin{itemize}
            \item 两个函数如果在同一点$x_0$处连续,则它们的和差商积在这点也是连续的
            \item 复合函数的连续性: 设$u=\varphi(x)$在点$x = x_0$处连续,$y = f(u)$在点$u = u_0$处连续,且$u_0 = \varphi(x_0)$,则$f[\varphi(x)]$在点$x = x_0$处连续
        \end{itemize}
    \end{enumerate}

    \subsection{间断点的定义与分类}

    \begin{enumerate}
        \item 讨论间断点只看\textbf{无定义点}和\textbf{分段点}
        \item 第一类间断点 = 可去间断点 + 跳跃间断点
        \begin{itemize}
            \item 可去间断点: 若$\lim\limits_{x\to x_0} f(x) = A \neq f(x_0)(f(x_0)\text{甚至可以无定义})$,则$x = x_0$称为\textbf{可去间断点},或称\textbf{可补间断点}
            \item 跳跃间断点: 若$\lim\limits_{x\to x_0^-}f(x)$与$\lim\limits_{x\to x_0^+}f(x)$都存在,但$\lim\limits_{x\to x_0^-}f(x) \neq \lim\limits_{x\to x_0^+}f(x)$,则$x = x_0$称为\textbf{跳跃间断点}
        \end{itemize}
        \item 第二类间断点: 无穷间断点和震荡间断点都属于第二类间断点
        \begin{itemize}
            \item 无穷间断点: 若$\lim\limits_{x\to x_0}f(x) = \infty$或$\lim\limits_{x\to x_0^+}f(x) = \infty$或$\lim\limits_{x\to x_0^-}f(x) = \infty$,则$x = x_0$称为\textbf{无穷间断点}(左右极限至少有一个无穷大)
            \item 震荡间断点: 若$\lim\limits_{x\to x_0}f(x)$震荡不存在,则$x = x_0$称为\textbf{震荡间断点}
        \end{itemize}
    \end{enumerate}

    \subsection{结论}

    \begin{enumerate}
    \end{enumerate}

    \subsection{定理}

    \begin{enumerate}
    \end{enumerate}

    \subsection{运算}

    \begin{enumerate}

    \end{enumerate}

    \subsection{公式}

    \begin{enumerate}

    \end{enumerate}

    \subsection{方法总结}

    \begin{enumerate}

    \end{enumerate}

    \subsection{条件转换思路}

    \begin{enumerate}

    \end{enumerate}

    \subsection{理解}

    \begin{enumerate}
    \end{enumerate}

\end{document}
