\documentclass[a4paper,12pt]{article}
\usepackage{xeCJK}          % 中文支持
\usepackage{fontspec}       % 英文/数学字体
\usepackage{amsmath, amssymb} % 数学公式
\usepackage{graphicx}       % 插入图片
\usepackage{hyperref}       % 目录超链接
\usepackage{geometry}       % 页面布局
\usepackage{bm}             % 粗体
\usepackage{xcolor}         % 颜色
\usepackage{tabularx}       % 表格环境
\usepackage{tikz}           % TikZ 绘制主对角线斜线
\usepackage{tcolorbox}
\usepackage{xstring}
\usepackage{pgfplots}
\usepackage{enumitem}
\usepackage{pifont}
\usepackage{ulem}
\pgfplotsset{compat=1.18}
\geometry{left=3cm,right=3cm,top=3cm,bottom=3cm}

% 抽离颜色和尺寸参数
\newcommand{\analysisTitleColor}{green!50!black}
\newcommand{\analysisBackColor}{white}
\newcommand{\analysisBoxRule}{0.8pt}
\newcommand{\analysisArc}{3pt}
\newcommand{\analysisPadding}{6pt}

% 定义 tcolorbox
\newtcolorbox{analysisbox}[1][]{
    title=\IfStrEq{#1}{}{\textbf{解析}}{#1}, % 如果传参为空则使用“解析”
    colback=\analysisBackColor,
    colframe=\analysisTitleColor,
    boxrule=\analysisBoxRule,
    arc=\analysisArc,
    left=\analysisPadding,
    right=\analysisPadding,
    top=4pt,
    bottom=4pt
}

\newlist{circlenum}{enumerate}{1}
\setlist[circlenum]{
    label=\ding{\numexpr171+\arabic*},
    leftmargin=2.2em,
    itemsep=0.6em,      % item 之间的距离(主要)
    topsep=0.6em,       % 列表与上下正文的距离
    parsep=0.3em,       % item 内段落的间距
    partopsep=0.3em     % 列表前后额外间距
}

\newcommand{\blueuline}[1]{{\color{blue}\uline{\color{black}{#1}}}}

% =========================
% 字体设置
% =========================
\setmainfont{Times New Roman}
\setsansfont{Helvetica Neue}
\setmonofont{Menlo}
\setCJKmainfont{PingFang SC}

% =========================
% 图形路径(可调整)
% =========================
\graphicspath{{./assets/}}

% =========================
% 文档开始
% =========================
\begin{document}

%    \title{Template}
%    \author{Bowen}
%    \date{\today}
%    \maketitle
%    \tableofcontents
%    \newpage


% =========================

    \section{一元函数积分学的概念与性质}

    \subsection{不定积分}

    \subsubsection{原函数与不定积分}

    设函数$f(x)$定义在某区间$I$上,若存在可导函数$F(x)$,对于该区间上任意一点都有$F'(x) = f(x)$成立,则称$F(x)$是$f(x)$在区间$I$上的一个\textbf{原函数}。称$\int f(x)\mathrm{d}x = F(x) + C$为$f(x)$在区间$I$上的\textbf{不定积分(全体原函数)}

    \medskip
    \noindent\textbf{注:}
    \begin{enumerate}
        \item $F'(x) = f(x)$。由$f(x)$处处有定义得$F(x)${\color{red}{处处可导}},即$F(x)${\color{red}{处处连续}}
    \end{enumerate}

    \subsubsection{原函数(不定积分)存在定理}

    \begin{enumerate}
        \item 连续函数$f(x)$必有原函数$F(x)$
        \[
                {\color{red}{\bigstar}}f(x)\text{连续} \Rightarrow
            \begin{cases}
                \int f(x)\,\mathrm{d}x = \int_a^x f(t)\,\mathrm{d}t + C, \\
                [\int_a^x f(t)\,\mathrm{d}t]' = f(x)
            \end{cases}
        \]
        \item 含有第一类间断点和无穷间断点的函数$f(x)$在包含该间断点的区间内必没有原函数$F(x)$
    \end{enumerate}

    \subsection{定积分}

    \subsubsection{定义}

    \subsubsection{存在定理}

    \subsubsection{性质(假设以下积分均存在)}

    \subsection{变限积分}

    \subsubsection{概念}

    \subsubsection{性质}

    \subsection{反常积分}

    \subsubsection{概念}

    \subsubsection{敛散性的判别法}

    \subsection{基础概念}

    \begin{enumerate}
    \end{enumerate}

    \subsection{结论}

    \begin{enumerate}
    \end{enumerate}

    \subsection{定理}

    \begin{enumerate}
        \item 积分中值定理:若函数$f(x)$在区间$[a, b]$上连续,则至少存在一点$\xi \in [a, b]$,使$\int_a^b f(x)\,\mathrm{d}x = f(\xi)(b - a)$
        \item 介值定理:
    \end{enumerate}

    \subsection{运算}

    \begin{enumerate}

    \end{enumerate}

    \subsection{公式}

    \begin{enumerate}

    \end{enumerate}

    \subsection{方法总结}

    \begin{enumerate}

    \end{enumerate}

    \subsection{条件转换思路}

    \begin{enumerate}

    \end{enumerate}

    \subsection{理解}

    \begin{enumerate}
        \item 证明:如果函数$f(x)$在$[a, b]$上连续,则函数$F(x) = \int_a^x f(t)\,\mathrm{d}t$在$[a, b]$上可导,且$F'(x) = f(x)$,即$\color{red}{\int f(x)\,\mathrm{d}x = \int_a^x f(t)\,\mathrm{d}t + C}$
        \begin{analysisbox}[证]
            若 $x\in(a,b)$,取 $\Delta x$ 使 $x+\Delta x\in(a,b)$,则
            \[
                \begin{aligned}
                    \Delta F
                    &= F(x+\Delta x)-F(x) \\
                    &= \int_a^{x+\Delta x} f(t)\,\mathrm{d}t - \int_a^{x} f(t)\,\mathrm{d}t \\
                    &= \int_a^{x} f(t)\,\mathrm{d}t + \int_x^{x+\Delta x} f(t)\,\mathrm{d}t - \int_a^{x} f(t)\,\mathrm{d}t \\
                    &= \int_x^{x+\Delta x} f(t)\,\mathrm{d}t
                \end{aligned}
            \]

            由积分中值定理,有$\int_x^{x + \Delta x} f(t)\,\mathrm{d}t = f(\xi)\Delta x$,其中 $\xi$ 介于 $x$ 与 $x+\Delta x$ 之间,当$\Delta x\to 0$时,$\xi \to x$,故
            \[
                F'(x) = \lim_{\Delta x\to 0}\frac{\Delta F}{\Delta x}  = \lim_{\Delta x\to 0} f(\xi) = \lim_{\xi\to x} f(\xi) = f(x)
            \]

            当 $x=a$ 时,取 $\Delta x>0$,同理可证 $F'(a)=f(a)$;
            当 $x=b$ 时,取 $\Delta x<0$,同理可证 $F'(b)=f(b)$。

            综上,$F(x)$ 在 $[a,b]$ 上可导,且 $F'(x)=f(x)$。
        \end{analysisbox}
        \item $\int f(x)\,\mathrm{d}x$称为不定积分,表示全体原函数
        \item $\int_a^b f(x)\,\mathrm{d}x$称为定积分,表示面积
        \item $F(x) = \int_a^x f(t)\,\mathrm{d}t$称为变上限积分,表示动态的面积
    \end{enumerate}

\end{document}
