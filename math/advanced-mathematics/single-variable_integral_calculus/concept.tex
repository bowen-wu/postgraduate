\documentclass[a4paper,12pt]{article}
\usepackage{xeCJK}          % 中文支持
\usepackage{fontspec}       % 英文/数学字体
\usepackage{amsmath, amssymb} % 数学公式
\usepackage{graphicx}       % 插入图片
\usepackage{hyperref}       % 目录超链接
\usepackage{geometry}       % 页面布局
\usepackage{bm}             % 粗体
\usepackage{xcolor}         % 颜色
\usepackage{tabularx}       % 表格环境
\usepackage{tikz}           % TikZ 绘制主对角线斜线
\usepackage{tcolorbox}
\usepackage{xstring}
\usepackage{pgfplots}
\usepackage{enumitem}
\usepackage{pifont}
\usepackage{ulem}
\pgfplotsset{compat=1.18}
\geometry{left=3cm,right=3cm,top=3cm,bottom=3cm}

% 抽离颜色和尺寸参数
\newcommand{\analysisTitleColor}{green!50!black}
\newcommand{\analysisBackColor}{white}
\newcommand{\analysisBoxRule}{0.8pt}
\newcommand{\analysisArc}{3pt}
\newcommand{\analysisPadding}{6pt}

% 定义 tcolorbox
\newtcolorbox{analysisbox}[1][]{
    title=\IfStrEq{#1}{}{\textbf{解析}}{#1}, % 如果传参为空则使用“解析”
    colback=\analysisBackColor,
    colframe=\analysisTitleColor,
    boxrule=\analysisBoxRule,
    arc=\analysisArc,
    left=\analysisPadding,
    right=\analysisPadding,
    top=4pt,
    bottom=4pt
}

\newlist{circlenum}{enumerate}{1}
\setlist[circlenum]{
    label=\ding{\numexpr171+\arabic*},
    leftmargin=2.2em,
    itemsep=0.6em,      % item 之间的距离(主要)
    topsep=0.6em,       % 列表与上下正文的距离
    parsep=0.3em,       % item 内段落的间距
    partopsep=0.3em     % 列表前后额外间距
}

\newcommand{\blueuline}[1]{{\color{blue}\uline{\color{black}{#1}}}}

% =========================
% 字体设置
% =========================
\setmainfont{Times New Roman}
\setsansfont{Helvetica Neue}
\setmonofont{Menlo}
\setCJKmainfont{PingFang SC}

% =========================
% 图形路径(可调整)
% =========================
\graphicspath{{./assets/}}

% =========================
% 文档开始
% =========================
\begin{document}

%    \title{Template}
%    \author{Bowen}
%    \date{\today}
%    \maketitle
%    \tableofcontents
%    \newpage


% =========================

    \section{一元函数积分学的概念与性质}

    \subsection{不定积分}

    \subsubsection{原函数与不定积分}

    设函数$f(x)$定义在某区间$I$上,若存在可导函数$F(x)$,对于该区间上任意一点都有$F'(x) = f(x)$成立,则称$F(x)$是$f(x)$在区间$I$上的一个\textbf{原函数}。称$\int f(x)\mathrm{d}x = F(x) + C$为$f(x)$在区间$I$上的\textbf{不定积分(全体原函数)}

    \medskip
    \noindent\textbf{注:}
    \begin{enumerate}
        \item $F'(x) = f(x)$。由$f(x)$处处有定义得$F(x)${\color{red}{处处可导}},即$F(x)${\color{red}{处处连续}}
    \end{enumerate}

    \subsubsection{原函数(不定积分)存在定理}

    \begin{enumerate}
        \item 连续函数$f(x)$必有原函数$F(x)$
        \[
                {\color{red}{\bigstar}}f(x)\text{连续} \Rightarrow
            \begin{cases}
                \int f(x)\,\mathrm{d}x = \int_a^x f(t)\,\mathrm{d}t + C, \\
                [\int_a^x f(t)\,\mathrm{d}t]' = f(x)
            \end{cases}
        \]
        \item 含有第一类间断点和无穷间断点的函数$f(x)$在包含该间断点的区间内必没有原函数$F(x)$
        \item 可导函数$F(x)$求导后的函数$F'(x) = f(x)$不一定是连续函数,也可能有震荡间断点
        \[
            \text{若}F(x)\text{处处可导} \Rightarrow F'(x)\begin{cases}
                                                              \text{连续函数或含有震荡间断点的函数} \\
                                                              \text{有介值性(达布性质)} \\
                                                              \lim\limits_{x\to x_0}F'(x)\ \text{存在}
                                                              \ \Longleftrightarrow\
                                                              F'(x)\ \text{在 }x_0\text{ 处连续;} \\
                                                              \neq 0 \ \Longrightarrow\ F(x)\ \text{单调}.
            \end{cases}
        \]
        \item 导函数$f'(x)$性质
        \begin{circlenum}
            \item 有介值性
            \item 如果导函数在一点存在,则这一点一定不会是导函数的第一类间断点
            \item 如果导函数$f'(x)$存在,当导函数在一点极限存在时,导函数在这一点必连续
            \item 若$f(x)$可导,则$f'(x)$可能连续,也可能含有震荡间断点
        \end{circlenum}
    \end{enumerate}

    \subsection{定积分}

    \subsubsection{定义}

    \begin{enumerate}
        \item 概念:设函数 $f(x)$ 在区间 $[a,b]$ 上有界,在 $(a,b)$ 上任取 $n-1$ 个分点
        $x_i \ (i = 1,2,3,\cdots,n-1)$,
        定义 $x_0 = a$ 和 $x_n = b$,
        且
        \[
            a = x_0 < x_1 < x_2 < \cdots < x_{n-1} < x_n = b,
        \]
        记
        \[
            \Delta x_k = x_k - x_{k-1}, \quad k = 1,2,3,\cdots,n,
        \]
        并任取一点
        \[
            \xi_k \in [x_{k-1}, x_k],
        \]
        记
        \[
            \lambda = \max_{1 \leq k \leq n} \{ \Delta x_k \},
        \]
        若当 $\lambda \to 0$ 时,极限
        \[
            \lim_{\lambda \to 0} \sum_{k=1}^{n} f(\xi_k)\,\Delta x_k
        \]
        存在且与分点 $x_i$ 及点 $\xi_k$ 的取法无关,
        则称函数 $f(x)$ 在区间 $[a,b]$ 上\textbf{可积},即
        \[
            \int_a^b f(x)\,\mathrm{d}x
            =
            \lim_{\lambda \to 0} \sum_{k=1}^{n} f(\xi_k)\,\Delta x_k.
        \]
        \begin{circlenum}
            \item 分割: 分割方法不唯一,只要分成$n$份即可
            \item 近似: $f(\xi_k)$
            \item 求和
            \item 取极限
        \end{circlenum}
        \item 几何意义
        \begin{itemize}
            \item 若$f(x) \geq 0$,定积分$\int_a^b f(x)\,\mathrm{d}x$表示面积
            \item 若$f(x) \leq 0$,定积分$\int_a^b f(x)\,\mathrm{d}x$表示面积的{\color{red}{负值}}
            \item 若$f(x)$有正有负,定积分$\int_a^b f(x)\,\mathrm{d}x$表示$x$轴上方图形的面积减去$x$轴下方图形的面积
        \end{itemize}
        \item 定积分的精确定义:当定积分存在时,存在两个 {\blueuline{任取}}:
        分点 $x_i$ 任取,一点 $\xi_i \in (x_{i-1}, x_i)$ 任取,
        故可做两个 {\color{red}{特取}},
        即将 $[a,b]$ $n$ 等分且取每个小区间的右端点为 $\xi_i$,即
        \[
            \int_a^b f(x)\,\mathrm{d}x
            =
            \lim_{n \to \infty}
            \sum_{i=1}^{n}
            f\!\left(a + \frac{b-a}{n} i\right)
            \frac{b-a}{n}
        \]
        若将式子中的 $a,b$ 特殊化为 $0,1$ 这两个数,即
        \[
                {\color{red}{\bigstar}}\int_0^1 f(x)\,\mathrm{d}x
            =
            \lim_{n \to \infty}
            \sum_{i=1}^{n}
            f\!\left(\frac{i}{n}\right)
            \frac{1}{n}
            =
            \lim_{n \to \infty}
            \sum_{i=1}^{n}
            f\!\left(\frac{2i - 1}{2n}\right)
            \frac{1}{n}
        \]
        \item 定积分的值与字母无关:定积分的值只与被积函数及积分区间有关,而与积分变量的记法无关
        \[
            \int_a^b f(x)\,\mathrm{d}x = \int_a^b f(t)\,\mathrm{d}t = \int_a^b f(u)\,\mathrm{d}u
        \]
    \end{enumerate}

    \subsubsection{存在定理}

    \begin{enumerate}
        \item 定积分存在的充分条件
        \begin{circlenum}
            \item 若$f(x)$在$[a, b]$上连续,则$\int_a^b f(x)\,\mathrm{d}x$存在
            \item 若$f(x)$在$[a, b]$上单调({\color[rgb]{0.2, 0.6, 0.3}{$f(a), f(b)$即为函数的界,对任意$x\in (a, b)$,$f(x)$一定介于$f(a), f(b)$之间}}),则$\int_a^b f(x)\,\mathrm{d}x$存在
            \item 若$f(x)$在$[a, b]$上有界,且只有有限个间断点({\color[rgb]{0.2, 0.6, 0.3}{不包含无穷间断点}}),则$\int_a^b f(x)\,\mathrm{d}x$存在
            \item 若$f(x)$在$[a, b]$上有有限个第一类间断点,则$\int_a^b f(x)\,\mathrm{d}x$存在
        \end{circlenum}
        \item 定积分存在的必要条件: 可积函数必有界。即若定积分$\int_a^b f(x)\,\mathrm{d}x$存在,则$f(x)$在$[a, b]$上必有界
    \end{enumerate}

    \subsubsection{性质(假设以下积分均存在)}

    \begin{enumerate}
        \item 当$b = a$时,$\int_a^b f(x)\,\mathrm{d}x = 0$
        \item 当$a > b$时,$\int_a^b f(x)\,\mathrm{d}x = -\int_b^a f(x)\,\mathrm{d}x$
        \item 求区间长度:假设$a<b$,则$\int_a^b \mathrm{d}x = b-a = L$,其中$L$为区间$[a, b]$的长度
        \item 积分的线性性质:设$k_1, k_2$为常数,则$\int_a^b [k_{1}f(x)\pm k_{2}g(x)]\,\mathrm{d}x = k_{1}\int_a^b f(x)\,\mathrm{d}x \pm k_{2}\int_a^b g(x)\,\mathrm{d}x$
        \item 积分的可加(拆)性:无论$a, b, c$的大小如何,总有$\int_a^b f(x)\,\mathrm{d}x = \int_a^c f(x)\,\mathrm{d}x + \int_c^b f(x)\,\mathrm{d}x$
        \item 积分的保号性:若在区间$[a, b]$上$f(x) \leq g(x)$,则有$\int_a^b f(x)\,\mathrm{d}x \leq \int_a^b g(x)\,\mathrm{d}x$。特殊地,有
        \[
            \left| \int_a^b f(x)\,\mathrm{d}x \right|
            \le
            \int_a^b |f(x)|\,\mathrm{d}x.
        \]
        \item 设$f(x)$是$[a, b]$上非负的连续函数,只要$f(x)${\color{red}{不恒等于零}},则必有
        \[
            \int_a^b f(x)\,\mathrm{d}x > 0
        \]
        \begin{itemize}
            \item 推论:若连续函数$f(x), g(x)$满足$f(x) \geq g(x)$,且$f(x)$不恒等于$g(x)$,有$a < b$,则必有
            \[
                \int_a^b f(x)\,\mathrm{d}x > \int_a^b g(x)\,\mathrm{d}x
            \]
        \end{itemize}
        \item 估值定理:设$M, m$分别是$f(x)$在$[a, b]$上的最大值和最小值,$L$为区间$[a, b]$的长度,则有
        \[
            \int_a^b m\,\mathrm{d}x = mL \leq \int_a^b f(x)\,\mathrm{d}x \leq ML = \int_a^b M\,\mathrm{d}x
        \]
        \item $\color{red}{\bigstar}$积分中值定理:若函数$f(x)$在区间$[a, b]$上连续,则至少存在一点$\xi \in [a, b]$,使$\int_a^b f(x)\,\mathrm{d}x = f(\xi)(b - a)$
    \end{enumerate}

    \subsection{变限积分}

    本质是一个由定积分定义的函数

    \subsubsection{概念}

    当 $x$ 在区间 $[a,b]$ 上变动时,对应于每一个 $x$ 值,积分$\int_a^x f(t)\,\mathrm{d}t$都有一个确定的值,因此$\int_a^x f(t)\,\mathrm{d}t$是一个关于$x$的函数,记作
    \[
        F(x)=\int_a^x f(t)\,\mathrm{d}t \quad (a \le x \le b)
    \]
    称函数 $F(x)$ 为变上限的定积分。同理,可以定义变下限的定积分以及上、下限都变化的定积分,这些统称为\textbf{变限积分}。事实上,变限积分就是定积分的推广。

    \subsubsection{性质}

    \begin{enumerate}
        \item 函数$f(x)$在$I$上\textbf{可积},则函数$F(x) = \int_a^x f(t)\,\mathrm{d}t$在$I$上\textbf{连续}($F(x)$若存在,则其一定连续。与$f(x)$作区分:$f(x)$存在但其不一定连续)
        \item $\color{red}{\bigstar}$函数$f(x)$在$I$上\textbf{连续},则函数$F(x) = \int_a^x f(t)\,\mathrm{d}t$在$I$上可导且$F'(x) = f(x)$
        \item 设 $F(x)=\int_a^x f(t)\,\mathrm{d}t$,则有:
        \begin{itemize}
            \item 若 $x=x_0\in I$ 是 $f(x)$ {\blueuline{唯一的跳跃间断点}},
            则函数 $F(x)$ 在 $x_0$ 处{\color[rgb]{0.2,0.6,0.3}{不可导}},且
            \[
                \begin{cases}
                    F'_-(x_0)=\lim\limits_{x\to x_0^-} f(x),\\
                    F'_+(x_0)=\lim\limits_{x\to x_0^+} f(x).
                \end{cases}
            \]

            \item 若 $x=x_0\in I$ 是 $f(x)$ {\blueuline{唯一的可去间断点}},
            则函数 $F(x)$ 在 $x_0$ 处{\color[rgb]{0.2,0.6,0.3}{可导}},且
            \[
                F'(x_0)=\lim\limits_{x\to x_0} f(x)\ne f(x_0).
            \]

            \item 上述两种情形中,$F(x)$ 不是 $f(x)$ 的原函数,
            而是变限积分,
            因为第一类间断点的函数不存在原函数
        \end{itemize}
    \end{enumerate}

    \subsection{反常积分}

    \subsubsection{概念}

    定积分有两个必要条件:一是积分区间有限,二是被积函数有界。如果破坏了积分区间的有限性,就引出\textbf{无穷区间}上的反常积分;如果破坏了被积函数的有界性,就引出了\textbf{无界函数}的反常积分

    \begin{enumerate}
        \item \textbf{无穷区间}上反常积分的概念与敛散性:设$F(x)$是$f(x)$在相应区间上的一个原函数
        \begin{circlenum}
            \item \[
                      \int_a^{+\infty} f(x)\,\mathrm{d}x = \lim_{x\to +\infty} F(x) - F(a)
            \]
            若上述极限存在,则称反常积分收敛,否则称发散
            \item \[
                      \int_{-\infty}^{b} f(x)\,\mathrm{d}x = F(b) - \lim_{x\to -\infty} F(x)
            \]
            若上述极限存在,则称反常积分收敛,否则称发散
            \item \[
                      \int_{-\infty}^{+\infty} f(x)\,\mathrm{d}x = \int_{-\infty}^{x_0} f(x)\,\mathrm{d}x + \int_{x_0}^{+\infty} f(x)\,\mathrm{d}x
            \]
            若右端两个积分都收敛,则称反常积分收敛,\textbf{否则}称发散
        \end{circlenum}
        \item 瑕点:使$f(x)$在$x_0$的邻域内\textbf{无界}的点即为\textbf{瑕点}。如:
        \begin{itemize}
            \item $\lim\limits_{x\to 0} \displaystyle\frac{1}{x} = \infty$,$x = 0$为瑕点
            \item $\lim\limits_{x\to 0} \displaystyle\frac{1}{x} \sin \displaystyle\frac{1}{x} = \text{无界震荡}$,$x = 0$为瑕点
        \end{itemize}
        \item \textbf{无界函数}的反常积分的概念与敛散性:设$F(x)$是$f(x)$在相应区间上的一个原函数,$x_0$为$f(x)$的瑕点
        \begin{circlenum}
            \item 若$x = a$是唯一瑕点,则
            \[
                \int_a^b f(x)\,\mathrm{d}x = F(b) - \lim_{x\to a^+} F(x)
            \]
            若上述极限存在,则称反常积分收敛,否则称发散
            \item 若$x = b$是唯一瑕点,则
            \[
                \int_a^b f(x)\,\mathrm{d}x = \lim_{x\to b^-} F(x) - F(a)
            \]
            若上述极限存在,则称反常积分收敛,否则称发散
            \item 若$x = c \in (a, b)$是唯一瑕点,则
            \[
                \int_a^b f(x)\,\mathrm{d}x = \int_a^c f(x)\,\mathrm{d}x + \int_c^b f(x)\,\mathrm{d}x
            \]
            若右端两个积分都收敛,则称反常积分收敛,否则称发散
        \end{circlenum}
        \item 在反常积分中,一般把$\mathbf{\infty}$和\textbf{瑕点}统称为\textbf{奇点}
        \item 在判别积分敛散性时,{\blueuline{一个积分中只能有一个奇点}},若出现两个及以上奇点,需拆分
    \end{enumerate}

    \subsubsection{敛散性的判别法}

    \begin{enumerate}
        \item 无穷区间
        \begin{itemize}
            \item 比较判别法({\color{red}{放缩法}}):设函数$f(x), g(x)$在区间$[a, +\infty)$上连续,并且${\color{red}{0 \leq f(x) \leq g(x)}}(a \leq x \leq +\infty)$,则
            \begin{circlenum}
                \item 当$\int_a^{+\infty} g(x)\,\mathrm{d}x$收敛时,$\int_a^{+\infty} f(x)\,\mathrm{d}x$收敛
                \item 当$\int_a^{+\infty} f(x)\,\mathrm{d}x$发散时,$\int_a^{+\infty} g(x)\,\mathrm{d}x$发散
            \end{circlenum}
            \item $\color{red}{\bigstar}$比较判别法的极限形式({\color{red}{计算法$\displaystyle\frac{0}{0}$}}):设函数$f(x), g(x)$在区间$[a, +\infty)$上连续,且$f(x) \geq 0, g(x) > 0, {\color{red}{\lim\limits_{x\to +\infty} \displaystyle\frac{f(x)}{g(x)} = \lambda}}$(有限或$\infty$)($\displaystyle\frac{0}{0}$型),则
            \begin{circlenum}
                \item 当$\lambda \neq 0$且$\lambda \neq \infty$时(\textbf{$f(x), g(x)$同阶无穷小}),$\int_a^{+\infty} f(x)\,\mathrm{d}x$与$\int_a^{+\infty} g(x)\,\mathrm{d}x$有相同的敛散性
                \item 当$\lambda = 0$时(\textbf{$f(x)$趋于$0$的速度比$g(x)$更快}),若$\int_a^{+\infty} g(x)\,\mathrm{d}x$收敛,则$\int_a^{+\infty} f(x)\,\mathrm{d}x$收敛
                \item 当$\lambda = \infty$时(\textbf{$g(x)$趋于$0$的速度比$f(x)$更快}),若$\int_a^{+\infty} g(x)\,\mathrm{d}x$发散,则$\int_a^{+\infty} f(x)\,\mathrm{d}x$发散
            \end{circlenum}
        \end{itemize}
        \item 无界函数
        \begin{itemize}
            \item 比较判别法({\color{red}{放缩法}}):设$f(x), g(x)$在$(a, b]$上连续,瑕点同为$x = a$,并且${\color{red}{0 \leq f(x) \leq g(x)}}(a < x \leq b)$,则
            \begin{circlenum}
                \item 当$\int_a^{b} g(x)\,\mathrm{d}x$收敛时,$\int_a^{b} f(x)\,\mathrm{d}x$收敛
                \item 当$\int_a^{b} f(x)\,\mathrm{d}x$发散时,$\int_a^{b} g(x)\,\mathrm{d}x$发散
            \end{circlenum}
            \item 比较判别法的极限形式({\color{red}{计算法$\displaystyle\frac{\infty}{\infty}$}}):设$f(x), g(x)$在$(a, b]$上连续,瑕点同为$x = a$,并且$f(x) \geq 0, g(x) > 0(a < x \leq b), {\color{red}{\lim\limits_{x\to a^+} \displaystyle\frac{f(x)}{g(x)} = \lambda}}$(有限或$\infty$)($\displaystyle\frac{\infty}{\infty}$型),则
            \begin{circlenum}
                \item 当$\lambda \neq 0$且$\lambda \neq \infty$时(\textbf{$f(x), g(x)$同阶无穷大}),$\int_a^{b} f(x)\,\mathrm{d}x$与$\int_a^{b} g(x)\,\mathrm{d}x$有相同的敛散性
                \item 当$\lambda = 0$时(\textbf{$g(x)$趋于$\infty$的速度比$f(x)$更快}),若$\int_a^{b} g(x)\,\mathrm{d}x$收敛,则$\int_a^{b} f(x)\,\mathrm{d}x$收敛
                \item 当$\lambda = \infty$时(\textbf{$f(x)$趋于$\infty$的速度比$g(x)$更快}),若$\int_a^{b} g(x)\,\mathrm{d}x$发散,则$\int_a^{b} f(x)\,\mathrm{d}x$发散
            \end{circlenum}
        \end{itemize}
        \item 两个重要结论($p$ 型广义积分)
        \begin{circlenum}
            \item 在 $(0,1]$ 上,$p$越小越收敛
            \[
                \int_{{\color{red}{0}}}^1 \frac{1}{x^p}\,\mathrm{d}x, \int_0^1 \frac{\ln x}{x^p}\,\mathrm{d}x \begin{cases}
                                                                                                                  \text{收敛}, \qquad \color{red}{0 < p < 1}, \\
                                                                                                                  \text{发散}, \qquad p \geq 1
                \end{cases}

            \]
            盯着$x\to 0^+$看
            \item 在 $[1,+\infty)$ 上,$p$越大越收敛
            \[
                \int_1^{{\color{red}{+\infty}}} \frac{1}{x^p}\,\mathrm{d}x, \int_1^{+\infty} \frac{\ln x}{x^p}\,\mathrm{d}x \begin{cases}
                                                                                                                                \text{收敛}, \qquad \color{red}{p > 1}, \\
                                                                                                                                \text{发散}, \qquad p \leq 1
                \end{cases}

            \]
            盯着$x\to +\infty$看
            \item 当$x\to 0^+$时,$\sin x \sim x$,即$\lim\limits_{x\to 0^+} \displaystyle\frac{\sin^p x}{x^p} = 1$,故$\int_0^1 \displaystyle\frac{1}{\sin^p x}\,\mathrm{d}x$依然满足\ding{172}
            \item 若$f(x)\sim x^\alpha \quad (x\to 0^+ \text{ 或 } +\infty)$,则$\int \displaystyle\frac{1}{f^p(x)}\,\mathrm{d}x \sim \int \displaystyle\frac{1}{x^{\alpha p}}\,\mathrm{d}x$
        \end{circlenum}
    \end{enumerate}

    \subsection{基础概念}

    \begin{enumerate}
    \end{enumerate}

    \subsection{结论}

    \begin{enumerate}
        \item $f(x)$可导 $\Rightarrow$ 连续 $\Rightarrow$ 可积 $\Rightarrow$ 有界
        \item $\sin x$和$\cos x$ 一拱面积为2
        \item $\int_0^{\frac{\pi}{2}} f(\sin x)\,\mathrm{d}x == \int_0^{\frac{\pi}{2}} f(\cos x)\,\mathrm{d}x$
        \item 当$f(x)$是偶函数且$\int_0^{+\infty} f(x)\,\mathrm{d}x$收敛时,$\int_{-\infty}^{+\infty} f(x)\,\mathrm{d}x = 2\int_0^{+\infty} f(x)\,\mathrm{d}x$
        \item 当$f(x)$是奇函数且$\int_0^{+\infty} f(x)\,\mathrm{d}x$收敛时,$\int_{-\infty}^{+\infty} f(x)\,\mathrm{d}x = 0$
        \item 设$f(x)$连续,且$\int_a^{+\infty} |f(x)|\,\mathrm{d}x$收敛,则$\int_a^{+\infty} |f(x)|\,\mathrm{d}x$收敛
        \begin{analysisbox}[证:]
            因为
            \[
                0 \le f(x)+|f(x)| \le 2|f(x)|.
            \]

            又
            \[
                \int_a^{+\infty} |f(x)|\,\mathrm{d}x
            \]
            收敛,根据比较判别法,
            \[
                \int_a^{+\infty} \bigl(f(x)+|f(x)|\bigr)\,\mathrm{d}x
            \]
            也收敛。

            又
            \[
                f(x)=\bigl(f(x)+|f(x)|\bigr)-|f(x)|,
            \]
            因此
            \[
                \int_a^{+\infty} f(x)\,\mathrm{d}x
                =
                \int_a^{+\infty} \bigl(f(x)+|f(x)|\bigr)\,\mathrm{d}x
                -
                \int_a^{+\infty} |f(x)|\,\mathrm{d}x.
            \]

            右端为两个收敛积分之差,故
            \[
                \int_a^{+\infty} f(x)\,\mathrm{d}x
            \]
            收敛。
        \end{analysisbox}
    \end{enumerate}

    \subsection{定理}

    \begin{enumerate}
    \end{enumerate}

    \subsection{运算}

    \begin{enumerate}

    \end{enumerate}

    \subsection{公式}

    \begin{enumerate}

    \end{enumerate}

    \subsection{方法总结}

    \begin{enumerate}
        \item 凑定积分定义:分子分母若为$h(n, i)$,则$h(n, i)$是关于$n, i$的齐次式,即可凑成$f(\displaystyle\frac{i}{n})\cdot \displaystyle\frac{1}{n}$,若不是齐次式,考虑夹逼准则。
        \begin{circlenum}
            \item 先提出$\displaystyle\frac{1}{n}$
            \item 再凑出$\displaystyle\frac{i}{n}$
            \item $\displaystyle\int_0^1 f(x)\,\mathrm{d}x
            =
            \lim\limits_{n \to \infty}
            \displaystyle\sum_{i=1}^{n}
            f\!\left(\displaystyle\frac{i}{n}\right)
            \displaystyle\frac{1}{n}
            =
            \lim\limits_{n \to \infty}
            \displaystyle\sum_{i=1}^{n}
            f\!\left(\displaystyle\frac{2i - 1}{2n}\right)
            \displaystyle\frac{1}{n}$
        \end{circlenum}
    \end{enumerate}

    \subsection{条件转换思路}

    \begin{enumerate}
        \item 见到 $\int_a^b f(x)\,\mathrm{d}x$
        \[
            \begin{cases}
                \text{用积分中值定理:}
                & \displaystyle \int_a^b f(x)\,\mathrm{d}x = f(\xi)(b-a), \\[6pt]
                \text{改成辅助函数:}
                & \displaystyle \int_a^x f(t)\,\mathrm{d}t.
            \end{cases}
        \]
    \end{enumerate}

    \subsection{经典}

    \begin{enumerate}
        \item $f(x) = 2x\sin \displaystyle\frac{1}{x^2} - \displaystyle\frac{2}{x}\cos \displaystyle\frac{1}{x^2}$,$x = 0$是其震荡间断点,$\lim\limits_{x\to 0} f(x)$是\textbf{无界震荡},其原函数是$F(x) = x^2 \sin \displaystyle\frac{1}{x^2}$
        \item $f(x) = 2x\cos \displaystyle\frac{1}{x} + \sin \displaystyle\frac{1}{x}$,$x = 0$是其震荡间断点,$\lim\limits_{x\to 0} f(x)$是\textbf{有界震荡},其原函数是$F(x) = x^2\cos \displaystyle\frac{1}{x}$
        \item $\int_0^1 \ln x\,\mathrm{d}x$ 是收敛的($\color{red}{\lim\limits_{x\to 0^+} x^{\epsilon}\ln x = 0, \epsilon > 0}$)
    \end{enumerate}

    \subsection{理解}

    \begin{enumerate}
        \item 证明:如果函数$f(x)$在$[a, b]$上连续,则函数$F(x) = \int_a^x f(t)\,\mathrm{d}t$在$[a, b]$上可导,且$F'(x) = f(x)$,即$\color{red}{\int f(x)\,\mathrm{d}x = \int_a^x f(t)\,\mathrm{d}t + C}$
        \begin{analysisbox}[证]
            若 $x\in(a,b)$,取 $\Delta x$ 使 $x+\Delta x\in(a,b)$,则
            \[
                \begin{aligned}
                    \Delta F
                    &= F(x+\Delta x)-F(x) \\
                    &= \int_a^{x+\Delta x} f(t)\,\mathrm{d}t - \int_a^{x} f(t)\,\mathrm{d}t \\
                    &= \int_a^{x} f(t)\,\mathrm{d}t + \int_x^{x+\Delta x} f(t)\,\mathrm{d}t - \int_a^{x} f(t)\,\mathrm{d}t \\
                    &= \int_x^{x+\Delta x} f(t)\,\mathrm{d}t
                \end{aligned}
            \]

            由积分中值定理,有$\int_x^{x + \Delta x} f(t)\,\mathrm{d}t = f(\xi)\Delta x$,其中 $\xi$ 介于 $x$ 与 $x+\Delta x$ 之间,当$\Delta x\to 0$时,$\xi \to x$,故
            \[
                F'(x) = \lim_{\Delta x\to 0}\frac{\Delta F}{\Delta x}  = \lim_{\Delta x\to 0} f(\xi) = \lim_{\xi\to x} f(\xi) = f(x)
            \]

            当 $x=a$ 时,取 $\Delta x>0$,同理可证 $F'(a)=f(a)$;
            当 $x=b$ 时,取 $\Delta x<0$,同理可证 $F'(b)=f(b)$。

            综上,$F(x)$ 在 $[a,b]$ 上可导,且 $F'(x)=f(x)$。
        \end{analysisbox}
        \item $\int f(x)\,\mathrm{d}x$称为不定积分,表示全体原函数
        \item $\int_a^b f(x)\,\mathrm{d}x$称为定积分,表示面积
        \item $F(x) = \int_a^x f(t)\,\mathrm{d}t$称为变上限积分,表示动态的面积
        \item 证明:函数$f(x)$在$I$上\textbf{可积},则函数$F(x) = \int_a^x f(t)\,\mathrm{d}t$在$I$上连续($F(x)$若存在,则其一定连续。与$f(x)$作区分:$f(x)$存在但其不一定连续)
        \begin{analysisbox}[证]
            对任意 $x$,若 $x+\Delta x \in I$,有
            \[
                F(x+\Delta x)-F(x)
                =
                \int_x^{x+\Delta x} f(t)\,\mathrm{d}t.
            \]

            由可积的必要条件可知,存在 $M>0$,使得在 $I$ 上有
            \[
                |f(x)| \le M.
            \]
            因此,
            \[
                0 \le \bigl| F(x+\Delta x)-F(x) \bigr|
                \le
                M\,|\Delta x|.
            \]

            当 $\Delta x \to 0$ 时,
            \[
                \lim_{\Delta x \to 0}
                \bigl[ F(x+\Delta x)-F(x) \bigr]
                =
                0,
            \]
            即
            \[
                \lim_{\Delta x \to 0} F(x+\Delta x) = F(x).
            \]

            根据函数在 $x$ 点连续的充要条件,
            $F(x)$ 在 $x$ 点连续。

            由此可见,对于变限积分
            \[
                F(x)=\int_a^x f(t)\,\mathrm{d}t
            \]
            只要它存在,就必然是连续的
        \end{analysisbox}
        \item 证明:函数$f(x)$在$I$上\textbf{连续},则函数$F(x) = \int_a^x f(t)\,\mathrm{d}t$在$I$上可导且$F'(x) = f(x)$
        \begin{analysisbox}[证]
            对任意的 $x$,若 $x+\Delta x \in I$,
            由于函数 $f(x)$ 在区间 $I$ 上连续,因此
            \[
                \begin{aligned}
                    \lim_{\Delta x \to 0}\frac{F(x+\Delta x)-F(x)}{\Delta x}
                    &= \lim_{\Delta x \to 0}
                    \frac{\left(
                        \displaystyle\int_a^{x+\Delta x} f(t)\,\mathrm{d}t
                        -
                        \displaystyle\int_a^{x} f(t)\,\mathrm{d}t
                        \right)}{\Delta x} \\
                    &= \lim_{\Delta x \to 0}
                    \frac{\displaystyle\int_x^{x+\Delta x} f(t)\,\mathrm{d}t}{\Delta x}
                \end{aligned}
            \]

            由积分中值定理,存在
            \[
                \xi \in (x,\,x+\Delta x),
            \]
            使得
            \[
                \int_x^{x+\Delta x} f(t)\,\mathrm{d}t
                =
                f(\xi)\,\Delta x.
            \]
            因此,
            \[
                \lim_{\Delta x \to 0}
                \frac{F(x+\Delta x)-F(x)}{\Delta x}
                =
                \lim_{\Delta x \to 0} f(\xi).
            \]

            由于 $\xi \to x$ 且 $f(x)$ 连续,故
            \[
                \lim_{\Delta x \to 0} f(\xi) = f(x).
            \]

            由此可见,如果函数 $f(x)$ 在区间 $I$ 上连续,
            则
            \[
                F(x)=\int_a^x f(t)\,\mathrm{d}t
            \]
            是 $f(x)$ 在区间 $I$ 上的一个原函数。

        \end{analysisbox}
        \item 设$f(x)$是$[a, b]$上非负的连续函数,且$f(x)${\color{red}{不恒等于零}},证明必有$\int_a^b f(x)\,\mathrm{d}x > 0$
        \begin{analysisbox}[证]
            因函数 $f(x)$ 在 $[a,b]$ 上不恒等于零,且 $f(x)\ge 0$,
            故至少存在一点 $x_0\in(a,b)$,使得
            \[
                f(x_0)>0.
            \]

            由连续函数的局部保号性知,
            存在 $\delta>0$ 与 $\eta>0$,
            使得当
            \[
                x\in[x_0-\delta,\;x_0+\delta]\subset[a,b]
            \]
            时,恒有
            \[
                f(x)\ge \eta>0.
            \]

            根据定积分的不等式性质,有
            \[
                \int_a^b f(x)\,dx
                \ge
                \int_{x_0-\delta}^{x_0+\delta} f(x)\,dx
                \ge
                \int_{x_0-\delta}^{x_0+\delta} \eta\,dx
                =
                2\delta\eta
                >0.
            \]
        \end{analysisbox}

    \end{enumerate}

\end{document}
