\documentclass[a4paper,12pt]{article}
\usepackage{xeCJK}          % 中文支持
\usepackage{fontspec}       % 英文/数学字体
\usepackage{amsmath, amssymb} % 数学公式
\usepackage{graphicx}       % 插入图片
\usepackage{hyperref}       % 目录超链接
\usepackage{geometry}       % 页面布局
\usepackage{bm}             % 粗体
\usepackage{xcolor}         % 颜色
\usepackage{tabularx}       % 表格环境
\usepackage{tikz}           % TikZ 绘制主对角线斜线
\usepackage{tcolorbox}
\usepackage{xstring}
\usepackage{pgfplots}
\usepackage{enumitem}
\usepackage{pifont}
\usepackage{ulem}
\usepackage{amssymb}
\usepackage{mathtools}
\usepackage{dsfont}
\usepackage{extarrows}
\pgfplotsset{compat=1.18}
\geometry{left=3cm,right=3cm,top=3cm,bottom=3cm}

% 抽离颜色和尺寸参数
\newcommand{\analysisTitleColor}{green!50!black}
\newcommand{\analysisBackColor}{white}
\newcommand{\analysisBoxRule}{0.8pt}
\newcommand{\analysisArc}{3pt}
\newcommand{\analysisPadding}{6pt}

% 定义 tcolorbox
\newtcolorbox{analysisbox}[1][]{
    title=\IfStrEq{#1}{}{\textbf{解析}}{#1}, % 如果传参为空则使用“解析”
    colback=\analysisBackColor,
    colframe=\analysisTitleColor,
    boxrule=\analysisBoxRule,
    arc=\analysisArc,
    left=\analysisPadding,
    right=\analysisPadding,
    top=4pt,
    bottom=4pt
}

\newlist{circlenum}{enumerate}{1}
\setlist[circlenum]{
    label=\ding{\numexpr171+\arabic*},
    leftmargin=2.2em,
    itemsep=0.6em,      % item 之间的距离(主要)
    topsep=0.6em,       % 列表与上下正文的距离
    parsep=0.3em,       % item 内段落的间距
    partopsep=0.3em     % 列表前后额外间距
}

\newcommand{\blueuline}[1]{{\color{blue}\uline{\color{black}{#1}}}}

% =========================
% 字体设置
% =========================
\setmainfont{Times New Roman}
\setsansfont{Helvetica Neue}
\setmonofont{Menlo}
\setCJKmainfont{PingFang SC}

% =========================
% 图形路径(可调整)
% =========================
\graphicspath{{./assets/}}

% =========================
% 文档开始
% =========================
\begin{document}

%    \title{Template}
%    \author{Bowen}
%    \date{\today}
%    \maketitle
%    \tableofcontents
%    \newpage


% =========================

    \section{一元函数积分学的计算}

    \subsection{基本积分公式}

    \begin{enumerate}
        \item $\int x^k\,\mathrm{d}x = \displaystyle\frac{1}{k+1}x^{k+1} + C, k \neq -1$
        \item $\int \displaystyle\frac{1}{x}\,\mathrm{d}x = \ln |x| + C$
        \item 指数函数的积分
        \begin{itemize}
            \item $\int e^x\,\mathrm{d}x = e^x + C$
            \item $\int a^x\,\mathrm{d}x = \displaystyle\frac{a^x}{\ln a} + C, \,a > 0 \text{且} a \neq 1$
        \end{itemize}
        \item 三角函数的积分
        \begin{itemize}
            \item $\int \sin x\,\mathrm{d}x = -\cos x + C$
            \item $\int \cos x\,\mathrm{d}x = \sin x + C$
            \item $\int \tan x\,\mathrm{d}x = -\ln |\cos x| + C$
            \item $\int \cot x\,\mathrm{d}x = \ln |\sin x| + C$
            \item ${\color{red}{\bigstar}}\int \displaystyle\frac{1}{\cos x}\,\mathrm{d}x = \int \sec x\,\mathrm{d}x = \ln |\sec x + \tan x| + C$
            \item $\int \displaystyle\frac{1}{\sin x}\,\mathrm{d}x = \int \csc x\,\mathrm{d}x = \ln |\csc x - \cot x| + C$
            \item $\int \sec^2 x\,\mathrm{d}x = \tan x + C$
            \item $\int \csc^2 x\,\mathrm{d}x = -\cot x + C$
            \item $\int \sec x \tan x\,\mathrm{d}x = \sec x + C$
            \item $\int \csc x \cot x\,\mathrm{d}x = -\csc x + C$
        \end{itemize}
        \item $\int \displaystyle\frac{1}{a^2 + x^2}\,\mathrm{d}x = \displaystyle\frac{1}{a}\arctan \displaystyle\frac{x}{a} + C, a > 0$
        \begin{itemize}
            \item $\int \displaystyle\frac{1}{1 + x^2}\,\mathrm{d}x = \arctan x + C$
        \end{itemize}
        \item $\int \displaystyle\frac{1}{\sqrt {a^2 - x^2}}\,\mathrm{d}x = \arcsin \displaystyle\frac{x}{a} + C, a > 0$
        \begin{itemize}
            \item $\int \displaystyle\frac{1}{\sqrt {1 - x^2}}\,\mathrm{d}x = \arcsin x + C$
        \end{itemize}
        \item 对数型(二次根式型)
        \begin{itemize}
            \item $\int \displaystyle\frac{1}{\sqrt {x^2 + a^2}}\,\mathrm{d}x = \ln (x + \sqrt {x^2 + a^2}) + C$
            \begin{circlenum}
                \item ${\color{red}{\bigstar}}\int \displaystyle\frac{1}{\sqrt {x^2 + 1}}\,\mathrm{d}x = \ln (x + \sqrt {x^2 + 1}) + C$
            \end{circlenum}
            \item $\int \displaystyle\frac{1}{\sqrt {x^2 - a^2}}\,\mathrm{d}x = \ln |x + \sqrt {x^2 - a^2}| + C, |x| > |a|$
            \begin{circlenum}
                \item $\int \displaystyle\frac{1}{\sqrt {x^2 - 1}}\,\mathrm{d}x = \ln |x + \sqrt {x^2 - 1}| + C, |x| > |a|$
            \end{circlenum}
        \end{itemize}
        \item 部分分式分解型
        \begin{itemize}
            \item $\int \displaystyle\frac{1}{x^2 - a^2}\,\mathrm{d}x = \displaystyle\frac{1}{2a}\ln \left|\displaystyle\frac{x-a}{x+a}\right| + C$
            \item $\int \displaystyle\frac{1}{a^2 - x^2}\,\mathrm{d}x = \displaystyle\frac{1}{2a}\ln \left|\displaystyle\frac{x+a}{x-a}\right| + C$
        \end{itemize}
        \item ${\color{red}{\bigstar}}\int \sqrt {a^2 - x^2}\,\mathrm{d}x = \displaystyle\frac{a^2}{2} \arcsin \displaystyle\frac{x}{a} + \displaystyle\frac{x}{2}\sqrt {a^2 - x^2} + C, a > |x| \geq 0$
        \begin{itemize}
            \item $\int \sqrt {1 - x^2}\,\mathrm{d}x = \displaystyle\frac{1}{2} \arcsin x + \displaystyle\frac{x}{2}\sqrt {1 - x^2} + C$
        \end{itemize}
        \item 三角函数降幂公式型
        \begin{itemize}
            \item $\int \sin^2 x\,\mathrm{d}x = \displaystyle\frac{x}{2} - \displaystyle\frac{\sin 2x}{4} + C, \,\sin^2 x = \displaystyle\frac{1-\cos 2x}{2}$
            \item $\int \cos^2 x\,\mathrm{d}x = \displaystyle\frac{x}{2} + \displaystyle\frac{\sin 2x}{4} + C, \,\cos^2 x = \displaystyle\frac{1+\cos 2x}{2}$
            \item $\int \tan^2 x\,\mathrm{d}x = \tan x - x + C, \,\tan^2 x = \sec^2 x - 1$
            \item $\int \cot^2 x\,\mathrm{d}x = -\cot x - x + C, \,\cot^2 x = \csc^2 x - 1$
        \end{itemize}
    \end{enumerate}

    \subsection{不定积分的积分法}

    \subsubsection{凑微分法}

    \[
        \int f[g(x)]g'(x)\,\mathrm{d}x = \int f[g(x)]\,\mathrm{d}[g(x)] = \int f(u)\,\mathrm{d}u
    \]

    常用的凑微分公式
    \begin{enumerate}
        \item 由于 $x\,\mathrm{d}x = \dfrac{1}{2}\mathrm{d}(x^2)$,
        因此
        \[
            \int x f(x^2)\,\mathrm{d}x
            = \frac{1}{2}\int f(x^2)\,\mathrm{d}(x^2)
            = \frac{1}{2}\int f(u)\,\mathrm{d}u
        \]

        \item 由于 $\sqrt{x}\,\mathrm{d}x = \dfrac{2}{3}\mathrm{d}(x^{\frac{3}{2}})$,
        因此
        \[
            \int \sqrt{x}\, f(x^{\frac{3}{2}})\,\mathrm{d}x
            = \frac{2}{3}\int f(x^{\frac{3}{2}})\,\mathrm{d}(x^{\frac{3}{2}})
            = \frac{2}{3}\int f(u)\,\mathrm{d}u
        \]

        \item 由于 $\dfrac{\mathrm{d}x}{\sqrt{x}} = 2\,\mathrm{d}(\sqrt{x})$,故
        \[
            \int \frac{f(\sqrt{x})}{\sqrt{x}}\,\mathrm{d}x
            = 2\int f(\sqrt{x})\,\mathrm{d}(\sqrt{x})
            = 2\int f(u)\,\mathrm{d}u.
        \]

        \item 由于 $\dfrac{\mathrm{d}x}{x^2} = \mathrm{d}\!\left(-\dfrac{1}{x}\right)$,故
        \[
            \int \frac{f\!\left(-\dfrac{1}{x}\right)}{x^2}\,\mathrm{d}x
            = \int f\!\left(-\dfrac{1}{x}\right)\,\mathrm{d}\!\left(-\dfrac{1}{x}\right)
            = \int f(u)\,\mathrm{d}u.
        \]

        \item 当 $x>0$ 时,$\dfrac{1}{x}\,\mathrm{d}x = \mathrm{d}(\ln x)$,故
        \[
            \int \frac{f(\ln x)}{x}\,\mathrm{d}x
            = \int f(\ln x)\,\mathrm{d}(\ln x)
            = \int f(u)\,\mathrm{d}u.
        \]

        \item 由于 $\mathrm{e}^x\,\mathrm{d}x = \mathrm{d}(\mathrm{e}^x)$,故
        \[
            \int \mathrm{e}^x f(\mathrm{e}^x)\,\mathrm{d}x
            = \int f(\mathrm{e}^x)\,\mathrm{d}(\mathrm{e}^x)
            = \int f(u)\,\mathrm{d}u.
        \]

        \item 由于 $a^x\,\mathrm{d}x = \dfrac{1}{\ln a}\,\mathrm{d}(a^x)$,
        $a>0,\; a\ne1$,故
        \[
            \int a^x f(a^x)\,\mathrm{d}x
            = \frac{1}{\ln a}\int f(a^x)\,\mathrm{d}(a^x)
            = \frac{1}{\ln a}\int f(u)\,\mathrm{d}u.
        \]

        \item 由于 $\sin x\,\mathrm{d}x = \mathrm{d}(-\cos x)$,故
        \[
            \int \sin x\,f(-\cos x)\,\mathrm{d}x
            = \int f(-\cos x)\,\mathrm{d}(-\cos x)
            = \int f(u)\,\mathrm{d}u.
        \]

        \item 由于 $\cos x\,\mathrm{d}x = \mathrm{d}(\sin x)$,故
        \[
            \int \cos x\,f(\sin x)\,\mathrm{d}x
            = \int f(\sin x)\,\mathrm{d}(\sin x)
            = \int f(u)\,\mathrm{d}u.
        \]

        \item 由于 $\displaystyle\frac{\mathrm{d}x}{\cos^2 x} = \sec^2 x\,\mathrm{d}x = \mathrm{d}(\tan x)$,故
        \[
            \int \frac{f(\tan x)}{\cos^2 x}\,\mathrm{d}x
            = \int f(\tan x)\,\mathrm{d}(\tan x)
            = \int f(u)\,\mathrm{d}u.
        \]

        \item 由于 $\csc^2 x\,\mathrm{d}x = \mathrm{d}(-\cot x)$,故
        \[
            \int \frac{f(-\cot x)}{\sin^2 x}\,\mathrm{d}x
            = \int f(-\cot x)\,\mathrm{d}(-\cot x)
            = \int f(u)\,\mathrm{d}u.
        \]

        \item 由于 $\dfrac{1}{1+x^2}\,\mathrm{d}x = \mathrm{d}(\arctan x)$,故
        \[
            \int \frac{f(\arctan x)}{1+x^2}\,\mathrm{d}x
            = \int f(\arctan x)\,\mathrm{d}(\arctan x)
            = \int f(u)\,\mathrm{d}u.
        \]

        \item 由于 $\dfrac{1}{\sqrt{1-x^2}}\,\mathrm{d}x = \mathrm{d}(\arcsin x)$,故
        \[
            \int \frac{f(\arcsin x)}{\sqrt{1-x^2}}\,\mathrm{d}x
            = \int f(\arcsin x)\,\mathrm{d}(\arcsin x)
            = \int f(u)\,\mathrm{d}u.
        \]

    \end{enumerate}

    \subsubsection{换元法}

    \[
        \int f(x)\,\mathrm{d}x \xlongequal{x = g(x)} \int f[g(u)]\,\mathrm{d}[g(u)] = \int f[g(u)]g'(u)\,\mathrm{d}u
    \]



    \subsubsection{分部积分法}

    \subsubsection{有理函数的积分}

    \subsection{定积分的计算}

    \subsection{变限积分的计算}

    \subsubsection{求导公式}

    \subsubsection{重要结论}

    \subsection{反常积分的计算}

    \subsection{基础概念}

    \begin{enumerate}
    \end{enumerate}

    \subsection{结论}

    \begin{enumerate}
    \end{enumerate}

    \subsection{定理}

    \begin{enumerate}
    \end{enumerate}

    \subsection{运算}

    \begin{enumerate}

    \end{enumerate}

    \subsection{公式}

    \begin{enumerate}

    \end{enumerate}

    \subsection{方法总结}

    \begin{enumerate}

    \end{enumerate}

    \subsection{条件转换思路}

    \begin{enumerate}

    \end{enumerate}

    \subsection{理解}

    \begin{enumerate}
    \end{enumerate}

\end{document}
